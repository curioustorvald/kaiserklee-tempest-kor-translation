

\chapter[외전1. 고급 식사][외전 1\hspace*{.5em}고급 식사]{외전 1 \ 고급 식사}



\begin{quote}

\small 11장과 관련된 이야기입니다. 엘사와 사울 사이의 관계를 엿볼 수 있습니다.\sourceatright{역자}

\end{quote} %force indent

``뭐가 이렇게 많아?''

엘사는 탁자에 늘어져 있는 수많은 식기를 당황한 채 바라보고 있었다. 모두 옅게 도금이 되어 있었고, 의심할 것 없이 가장 좋은 재질로 만들어져 있었다. 엘사에게는 보는 것만으로도 놀라웠다. 쓰는 것은 제외하고 말이다. 그래도 정말 너무 많았다. 빛을 반사하는 모습을 보는 것만으로도 어지러워졌다.

``좀 많긴 하지''

사울이 말했다. 어깨를 으쓱이고 있지도 않았다. 엘사는 이 모습에서 예절 수업을 떠올렸다. 절대 어깨를 으쓱이지 마라. 눈을 자주 깜빡이지 마라. 몸을 꼿꼿이 하고 걸어라. 어깨를 낮추고 고개를 꼿꼿이 하라. 조용히 걸어라. 알맞은 속도로 걸어라. 너무 느리거나 너무 빨라서도 안 된다. 팔을 크게 흔들지 마라.

이 모든 것을 기억하는 것만으로도 엘사는 벌써 머리가 어지러웠다. 갑자기 식사 예절을 배우는 것은 말할 것도 없이.

``자리에 앉아.''

사울이 말했다. 품위 있는 모습으로 손짓하고 있었다. 엘사는 부러워할 수밖에 없었다. 똑같이 따라 하자 엘사는 야만인처럼 보일 뿐이었다. 최소한 엘사는 며칠 연습한 끝에 앉는 법을 터득했다. 탁자 반대편에는 사울이 앉았다.

``좋아. 준비됐어.''

엘사가 속삭였다.

``진정해, 그냥 나라고.''

사울은 웃으며 말했다.

엘사는 살짝 웃어 보였다. 정말로 사울은 이곳에서 가장 친한 친구였다. 그러나 어느 정도는 이 사실에 기분이 좋지 않았다. 엘사는 일을 망친다면 창피해 죽으리라고 생각했다. 특히 키루스가 며칠 전에 저녁때 엘사의 끔찍한 식사 예절을 지적한 다음에는. 키루스는 마르쿠스에게 식사 예절을 교육해 주라고 재촉했다. 마르쿠스는 키루스를 꾸짖었지만, 정말로 교육을 받게 해 자기 아들의 말이 맞는다는 것을 보였다.

``그러다 웃음보 터지겠네.''

엘사가 중얼거렸다. 소리가 난 다음에야 중얼거리지 않아야 한다는 것이 떠올랐다. 그리고 `웃음보 터지다'와 같은 입말은 욕설과 같이 금기시되는 것이었다.

``그러겠지. 하지만 너랑 웃는 거지, 널 비웃는 게 아니란 걸 알 거야.''

사울이 말했다. 여느 때와 같이 매우 진지한 말투였다. 사울의 말을 믿지 않기란 어려운 일이었다.

``이게 낯선 일이란 건 이해해. 그래도 재밌는 연습이라고 생각하자고. 최소한 먹을 건 나오잖아.''

바로 이때에 하인 몇 명이 음식을 가져왔다. 엘사는 서둘러 사울을 따라 무릎 위에 냅킨을 올려놓았다.

``두 발 다 바닥에 붙이는 거 기억해. 다리 꼬지 말고, 등받이에 기대지도 말고. 팔꿈치는 몸 옆에 두고.''

사울이 말했다.

``어\ldots''

엘사는 입을 다물고 헛기침을 했다. 이렇게 자신감 없는 소리를 내는 것도 하지 말아야 할 행동이다.

``이게 맞는 건가?''

사울은 미소를 지었지만, 손으로 입을 가렸다. 엘사가 벌써 따라 하기 시작한 습관이다.

``긴장하지 마. 자연스럽게 하라고.''

엘사는 기계적으로 앉아 있는 상태에서 어떻게 자연스럽게 있을 수 있는지 전혀 알 수 없었다. 그래도 사울이 고개를 끄덕일 때까지 최선을 다했다.

``포도주도 마시는 거야?''

엘사는 탁자 위에 있는 잔들을 보며 물었다.

``어\ldots\,아니. 잔들은 뭐에 쓰는 건지 보라고 놓은 거야. 그래도 천천히 알아 가야지.''

사울은 왼쪽 위에 있는 잔을 들었다. 그러고는 아래에 있는 잔, 오른쪽 위에 있는 잔, 오른쪽 아래에 있는 잔을 차례로 들었다.

``적포도주, 백포도주, 샴페인, 물''

``알았어. 지금은 물만 마시는 거네.''

사울은 하인들에게 손짓을 보냈다. 이들은 덮개로 덮인 접시를 앞에 놓은 다음 덮개를 들었다. 엘사는 전투라도 나가는 듯 샐러드를—그냥 샐러드였다. 아직은 거창한 것이 나오지 않았다. 엘사는 안심이 되었다—유심히 살펴보았다.

``자, 이젠 식사 도구 차례야. 대략적인 건 바깥쪽 거부터 쓰는 거야. 항상 한 번에 조금씩만 먹고.''

사울이 말했다.

엘사는 고개를 끄덕이고 가장 왼쪽에 있는 포크를 집었다. 사울은 이 모습에 헛기침했다. 엘사는 독이라도 든 것처럼 포크를 내려놓고 꺼림칙하게 고개를 들었다.

``주인을 먼저 기다려야지.''

사울이 부드럽게 말했다. 자신의 포크를 집어 들고 샐러드를 약간 입에 넣고 나서, 사울은 엘사에게 손짓을 했다.

``이제 나한테 경의를 보인 거야. 난 독이 없다는 걸 보여서 거기에 답례한 거고.''

``뭐, 독이라고?''

``그런 게 있어.''

사울은 포크를 접시에 내려놓았다. 휘어진 것이 위로 향하게 가운데에 세로로 놓았다.

``이대로 따라 해봐.''

엘사는 서둘러 따라 했다. 엘사는 하인들이 샐러드 접시를 치우는 것에 놀랐다.

``난 하나도 못 먹었다고!''

``식사 예절이 부족한 데 대한 벌이야.''

사울이 말했다. 이번에는 미소를 감추지 않았다. 엘사는 사울이 즐기고 있는 것을 알 수 있었다. 엘사는 자신의 이른바 `친구'에게 성을 내었다.

`진정해.'

머릿속으로 되뇐 다음, 엘사는 깊게 숨을 들이마시고 똑같이 부드러운 미소를 지었다. 사울의 미소는 점점 커져 거의 이가 드러날 듯했다.

``이제 식사 도구를 접시에 놓아야 하는 걸 배운 거야. 절대로 도로 탁자에 놓지 말고. 다 안 먹었으면 비스듬히 놓고, 다 먹었으면 세로로.''

``알았어.''

엘사가 중얼거렸다.

``다음은 수프야.''

이들 앞에 그릇이 놓였다. 엘사는 그릇이라기보다는 움푹한 접시에 가깝다고 생각했다. 수프는 많이 담겨있지도 않았다. 그래도 허여멀건 국물은 매우 맛있어 보이고 김이 날 정도로 뜨겁고 신선한 해산물로 가득했다.

``수프는 우묵한 숟가락을 쓰는 거야. 봐봐.''

엘사는 이번에는 먼저 먹는 것을 기다리며 사울이 어떻게 하는지를 지켜보았다. 사울이 한 술을 끝내자, 엘사는 고상한 모습으로 숟가락을 들어 수프를 떴다. 그릇은 탁자 위에, 뜨는 것은 가운데부터 바깥쪽으로. 사울이 아무런 말이 없자, 엘사는 미소를 짓고 몸을 숙여 수프를 먹으려—

``으흠.''

엘사는 거의 숟가락을 떨어뜨릴 뻔했다. 사울은 계속 말했다.

``숟가락을 입에 가져가는 거야. 다른 건 안 되고. 미안해, 엘사야.''

그리고 엘사의 수프는 바로 치워졌다. 엘사는 얼굴을 찡그렸다. 하인들마저도 재미있어하는 것을 보고 최대한 투덜대지 않으려 했다.

``이러다 오늘 밤에 밥은 먹을 수나 있나 몰라.''

``그건 너한테 달렸지.''

사울이 말했다.

``잠깐만 기다려 봐. 이 수프 맛있네. 이거 다 먹고 계속하자.''

엘사는 사울이 일부러 꾸물거리고 있는 것을 바라보았다. 사울은 이따금 엘사의 눈치를 보았다. 엘사가 보기에는 눈치를 보면서도 으스대고 있었다. 물론 일부러 으스대는 모습을 보이는 것은 아니었지만, 엘사에게는 그렇게 보였다. 엘사는 최대한 애를 써서 팔짱을 끼지 않으려 했다. 자리에 앉아 예절대로 예의를 갖춘 채 기다렸다. 마침내 사울은 그릇을 가져가게 했고, 이들 앞에는 새로운 별미가 차려졌다.

``후식이다!''

``후식 아냐. 푸딩이지.''

사울이 말했다. 하지만 엘사는 행복해져서 듣고 있지 않았다.

``둘은 다른 거야. 이건 숟가락하고 포크를 둘 다 쓰는 거야. 어쩔 땐 푸딩 먹는 숟가락이 접시에 있을 때도 있는데, 그런 게 아니면 안쪽에 놓인 걸 쓰면 돼.''

엘사는 사울을 따라 왼손에 숟가락을 들고 오른손에 포크를 들었다. 그러고는 사울이 푸딩을 포크로 떠서 숟가락에 담는 것을 지켜보았다. 쉬워 보였다. 엘사는 사울이 한 대로 따라 했다. 이번에는 몸을 숙이지 않았다. 숟가락에 담긴 푸딩을 쏟지 않는 것은 신경이 곤두서는 일이었다. 그래도 결국은 입에 가져갈 수 있었다. 달콤한 맛이 입안을 가득 채웠다. 엘사는 겨우 소리를 내지 않았다. 아무래도 조금 많이 뜬 듯했다. 입가에 푸딩이 약간 묻어있었다. 엘사는 냅킨을 들어 입을 슥 닦았—

하인들이 푸딩을 치워버렸다.

``이번엔 뭘 잘못한 건데?''

엘사가 외쳤다.

``크게 떠먹는 건 넘어가 줬지만, 냅킨을 쓸 땐 두드려야지, 슥 닦는 게 아니라고.''

사울이 말했다. 엘사는 사울의 얼굴에 냅킨을 거의 던질 뻔했다.

``다음은 주요리야.''

스테이크와 감자. 엘사는 스테이크의 모습에 침이 고였다. 덮개가 치워지자, 아주 먹음직스러운 모양으로 김이 피어올랐고, 곧 지글거리는 스테이크가 모습을 드러냈다. 밤 동안 음식으로 장난을 당한 다음에 보는 이 스테이크는 터무니없이 유혹적이었다.

``포크는 왼손에, 칼은 오른손에.''

사울이 말했다.

``포크로는 고정하고, 칼로는 써는 거야. 끝에서 시작해서 가장자리부터 썰어야 해. 그리고 잊지 마. 한입 크기로 써는 거.''

엘사는 식사를 시작했다. 스테이크를 마구 난도질할 작정이었지만, 곧 식사 예절이 떠올랐다. 또 뺏기기는 싫으니, 엘사는 스테이크를 천천히 썰었다. 사울이 재미있어하며 바라보는 것은 무시하고 있었다. 사울은 엘사의 집중을 흩트리려 하고 있을 뿐이었다. 성공이다. 작은 한입 크기를 썰어내었다. 엘사는 식사하는 데 오른손을 썼다. 손을 바꾸는 게 위험한 행동일까? 하지만 왼손을 썼다가는 스테이크에 얼굴을 박을지도\ldots

엘사는 이를 악물고 칼을 내려놓았다. 손을 바꿀 준비를 하고 있었다. 그러자\ldots

``엘사야.''

사울은 한숨을 쉬었다. 엘사는 얼어붙었다.

``칼 탁자에 놓지 마라니까. 그리고 항상 쥐고 있어야 해. 그러니까 스테이크는 이제—''

``재밌을 거라고 했잖아!''

엘사는 자리를 박차고 일어났다. 눈에는 눈물이 맺혀있었다. 화가 나서 나는 눈물. 엘사는 생각했다. 하지만 실은 배신감에 차 있었다. 사울은 내내 엘사를 놀리기만 했다. 엘사는 소매로 눈물을 닦아내었다. 하지만 볼을 타고 흐르는 것을 막기에는 늦었다.

``미\ldots\,미안해, 이럴 생각은—엘사야!''

엘사는 식당을 뛰쳐나갔다.

엘사는 자신이 어디로 가는지 거의 모르고 있었다. 그저 어디든 가 버리고 싶었다. 엘사는 정상적이고 평범한 생활을 그리워했다. 혹은 최소한 마법이 나타나기 전의 생활을. 엘사는 답답한 규칙을 따르지 않고, 따를 일도 없던 때가 그리웠다. 엘사는—모든 일이 잘못되기 전, 아주 끔찍하게 잘못되기 전에 자신의 부모와 살던—과거의 생활이 그리웠지만—

``엘사야, 기다려!''

엘사는 뒤를 돌아보고는 사울이 눈에 눈물이 고인 채 쫓아오는 것을 보았다. 으스대는 품은 모두 사라져 있었다. 꽤 당황한 듯했다. 엘사를 따라잡았을 때, 사울은 숨을 허덕이고 있었고 얼굴은 붉어져 있었다. 엘사는 이때서야 생각보다 꽤 멀리 왔다는 것을 알아챘다. 발밑에 있던 얼음이 엘사를 밀어주고 있었다.

``저\ldots\,정말 미안해.''

사울은 헐떡이며 말했다.

``뭐, 당연히 그래야지.''

엘사는 코웃음을 치며 말했다.

``진짜야! 그냥 장난치던 거였어. 일부러 그런 건\ldots\,그러니까, 좀 심할 수도 있다고 생각했지만, 정말로 네 음식을 치워버리게 하려던 건 아니었고, 그리고, 그리고—''

엘사는 사울이 매우 허둥대는 모습에 웃음을 터트리지 않을 수 없었다. 사울은 항상 깐깐하고, 고상하고, 엘사 자신은 결코 갖출 수 없는 몸가짐을 갖추고 있었다. 사울이 이러는 모습을 보니\ldots\,기운이 났다.

``정말로 미안해.''

사울이 다시 한 번 말했다. 엘사는 진심으로 하는 소리임을 알았다. 사울은 단순히 미안하다고만 하는 것이 아니었다. 사과해야 하는 상황이어서 사과한 것이 아니었다. 사울은 진심이었다.

``알았어. 용서해 줄게.''

엘사가 말했다. 미소를 참을 수가 없어 화 난 말투를 유지하기가 힘들었다.

``이제 밥 먹으러만 가면.''

사울은 씁쓸하게 미소를 지었다.

``네가 도망칠 때, 그\ldots\,모든 걸 얼려버렸어. 이젠 못 먹게 됐을 거 같은데.''

``\ldots아.''

``그래도 이건 챙겨뒀지!''

사울은 주머니에서 긴 끈으로 묶인 작은 봉지를 꺼냈다. 주저하며 건네주고 나서는 잘못이라도 한 듯 시선을 돌렸다.

엘사는 끈을 풀었다. 한편으로는 당혹스럽기도 했고, 다른 한편으로는 그저 배가 고팠다. 완전히 열기도 전에 친숙한 향기가 새어나왔다. 엘사는 그대로 멈추고는 믿을 수 없다는 눈으로 사울을 바라보았다.

``열어 봐!''

사울이 말했다.

엘사는 지체하지는 않았지만, 봉지를 찢어버렸다. 초콜릿 수십 개가 바닥으로 떨어졌다. 모두 서로 다른 것들이었다. 몇 개는 트뤼프 초콜릿이었고, 몇 개는 프랄린 초콜릿이었고, 몇 개는 초콜릿을 입힌 과일이었다.

``이런 거 먹으면 안 되는 줄 알았는데.''

엘사가 속삭였다.

``그냥\ldots\,아무한테도 말하지 마.''

사울은 어깨를 으쓱였다. 정말로 어깨를 으쓱였다. 엘사는 웃음을 터트렸다. 엘사는 손으로 입을 가려 웃음은 품위 없는 코웃음이 되었지만, 사울은 규칙을 깬 것에 매우 당황해 뭐라 말할 수도 없었다.

``좋아할 줄 알고 가져왔는데.''

``맞아!''

엘사는 초콜릿을 하나 집어 입안에 넣었다. 초콜릿을 깨물어 안에 있는 신선하고 맛있는 딸기를 맛보고 있었다.

``이거 정말 맛있다.''

엘사는 입안에 초콜릿을 가득 문 채 말했다. 초콜릿을 삼키고 나서 엘사가 말했다.

``먹어볼래?''

``나\ldots\,난 먹으면 안 될 것 같아.''

사울이 말했다. 마르쿠스는 이들이 단것을 먹지 못하게 했다. 엘사는 눈앞에 초콜릿을 흔들어 보였고, 결국 사울은 호기심에 지고 말았다. 트뤼프 초콜릿을 깨물자, 사울은 기쁨에 `오오' 하며 입술을 둥글게 말고 있었다.

``잠깐만, 식사 예절은?''

``손으로!''



\chapter[외전2. 체크메이트][외전 2\hspace*{.5em}체크메이트]{외전 2 \ 체크메이트}



\begin{quote}

\small 18장과 관련된 이야기입니다. 구스타프의 극단이라는 물음에 관한 엘사의 답을 알 수 있습니다.\sourceatright{역자}

\end{quote}

\begin{quote}

\small 엘사는 열 살, 에드문드는 여덟 살. 구스타프는 이십 대 후반입니다.\sourceatright{작가}

\end{quote}구스타프를 처음 만났을 때, 엘사는 구스타프를 좋아했다.

뭐, 정확히 처음은 아니었다. 첫날의 급한 자기소개에서 모두를 만나보았고, 저녁 식사 후에도 만나보았으니 말이다. 토비아스와 파비안을 조심하게 되는 데에는 시간이 약간 걸렸다. 토비아스는 항상 엘사에게 냉소를 보냈고, 파비안은 엘사가 어울리지 않는다는 것으로 들쑤시는 데에서 재미를 보는 모양이었다. 그러나 구스타프는 당연히 앉아야 할 마르쿠스 옆자리—이제는 엘사가 이곳에 앉아있었다—가 아닌 탁자 맨 끝에 앉은 채 저녁 내내 정중하게 가만히 있었다.

사실 구스타프를 제대로 만난 것은 이 년 뒤였다. 그렇다고 얼굴도 본 적이 없었다는 것은 아니다. 지나가다 서로 마주치면 정중히 인사를 주고받았다. 물론 매일 저녁때에도 구스타프를 보았다. 그렇지만 같이 수업을 듣는 이들과 비교하면\ldots

뭐, 잘해 봐야 낯선 사이였다.

그러니 엘사가 돌아다닐 때 우연히 만나게 된 것이다. 엘사는 구스타프가 라운지에서 끈기 있게 훨씬, 훨씬 어린 동생에게 본인이 좋아하는 취미를 자세히 가르치는 것을 보았다. 아주 드문 일은 아니었다. 구스타프가 있다는 것은 키루스가 같이 있다는 것이고, 키루스가 있다는 것은 파비안도 같이 있다는 뜻이었지만 말이다. 엘사는 파비안을 피해 다녔다. 두려운 것은 아니고—절대로 두려워하지 마라. 항상 자랑스러워해라. 마르쿠스의 가르침이었다—그저 비열하고, 가끔은 외설적이기도 한 사람을 적당히 조심하는 것이었다. 키루스는 어느 선에서 자제해야 할지 알았다. 파비안에게 그런 것은 없었다.

우연히도, 이날은 오직 구스타프와 에드문드뿐이었다. 엘사는 편안하게 적당히 멀리서 지켜보았다. 에드문드는 탁자에 매우 가까이 앉아 소파 가장자리에 불안정하게 매달려 있었다. 체스판을 매우 집중해서 바라보는 통에 얼굴은 온통 일그러져 있었다. 반면에 구스타프는 매우 편안히 있었다.

``체크메이트.''

구스타프가 말했다.

``제대로 둔 건데! 어떻게 또 졌지?''

에드문드가 외쳤다.

``너무 생각에 빠져서 다른 수를 생각하지 못하고 있잖아. 가끔은 한 걸음 물러서서 큰 그림을 보는 게 좋아.''

구스타프는 엘사를 돌아보았다. 바라보는 동안 엘사는 자신이 이들과 동떨어져 있다는 생각이 들었다.

``너도 안 해보겠느냐, 엘사? 거기 오래 서 있는 것 같은데.''

``아\ldots\,아, 죄송해요. 괜히 참견하고 싶진—''

에드문드는 바로 자리에서 일어나 엘사의 손을 잡아 소파로 끌고 가 자신이 앉은 자리에 앉혔다. 매우 빨라서 엘사는 놀란 소리밖에 낼 수 없었다. 엘사를 데려오고 나서 에드문드는 다른 소파에 주저앉아 성을 내었다.

``나 대신 계속 지고 있으면 돼.''

에드문드가 말했다. 눈에 비친 환희 때문에 매우 천진난만해 보였다. 그러고는 갑자기 찡그린 채 관자놀이를 문질렀다. 구스타프는 에드문드에게 다가가려 했지만, 에드문드는 고개를 저었다.

``그냥 머리 아픈 거예요.''

``또?''

엘사가 물었다. 엘사는 수업 때 에드문드가 계속 이러는 것을 보아 왔다.

``오다 사라지다 하지. 이 체스하고 머리 쓰는 거 때문에 그래.''

에드문드는 어깨를 으쓱이며 말했다. 그러고는 다시 눈이 밝아졌다.

``아! 사실 궁금한 게 있는데, 누나로 불러도 돼? 이렇게 부르는 게 맞잖아, 안 그래?''

``애 놀래지 마라. 에드문드는 너무 열정적일 때가 있지. 대신 사과하지.''

구스타프는 한숨을 쉬며 말했다. 자기 쪽 말을 다시 놓고 있었다. 엘사는 서둘러 구스타프를 따라 자기 말을 다시 놓았다.

``잘했어. 아주 효율적이야.''

``고\ldots\,고마워요. 그리고 괜찮아, 에드문드.''

엘사가 말했다. 그러나 목소리는 불쌍하게 빽빽거리는 소리가 되었고, 구스타프는 눈썹을 추켜세웠다.

에드문드는 미소를 지었다.

``제가 뭐랬어요, 이제 가족이라고요. 누나로 부르는 게 큰일은 아니라고요.''

``꼭 그렇게 활기찬 목소리로 말하는데 거절할 수 있는 것처럼 그러네.''

구스타프는 무뚝뚝하게 말했다.

왕가의 일원이 되는 것은 약간 이상한 일이었다. 그리고 환영받지 못하는 생태에 익숙할 때에 이토록 따뜻하게 환영받는 것은 더욱 이상했다. 물론 대부분은 사울과 에드문드에게서 왔다. 다른 이들—스테판, 라파엘, 알바르가 주변에 없는 한 알렉—은 적당히 거리를 두었다. 한스는 모두와 떨어져 있었으니 셈에 포함하지 않는다. 한 번은 한스와 대화를 하려 했지만 제대로 퇴짜맞았다. 엘사는 더 다가가지 않기로 했다.

에드문드는 처음부터 엘사를 완전히 받아준 두 사람 중 하나였다. 그래서 에드문드의 요청을 받았을 때, 엘사는 개의치 않았다.

``정말 괜찮아.''

엘사가 다시 말했다. 에드문드는 환호했다. 구스타프는 어깨를 으쓱이고 엘사의 주의를 체스판으로 돌렸다.

엘사는 구스타프가 각각의 말이 움직이는 법과 어떻게 잡는지 설명하는 것을 주의 깊게 들었다. 다행히도 말 각각은 금방 머릿속에 들어왔다.

``각각은 간단하지만 다 모이면 경우의 수는 끝이 없다.''

구스타프가 경고했다.

``자, 그러면 대국을 시작하지.''

``바로바로 결정을 내리는 걸 가르치기 좋아하셔.''

엘사가 바로 시작하는 것에 놀란 것을 보고 에드문드가 말했다.

``충고를 주긴 할 거야\ldots\,다 끝난 다음에.''

엘사는 천천히 고개를 끄덕였다.

``사실 그렇게 배우는 게 더 좋아.''

``아, 에드문드한테 듣기론 꽤 출중하게 잘 배우고 있다면서?''

``출중하진 않아요. 그래도 따라가고는 있어요.''

``보통은 엘사 누나가 가장 잘해요. 사울 형 보다도요! 사울 형이 시험 결과에 아주 신경 써서 진짜로 공부하는 거 잘 알잖아요.''

에드문드가 말했다.

``대부분이 너보단 더 신경을 쓰지.''

에드문드가 웃음을 터뜨리자 구스타프는 고개를 저었다. 엘사는 손안에 미소를 감추었다. 다시 탁자를 바라보며, 구스타프는 체스판을 반대로 돌렸다.

``평소엔 내가 백으로 하는데 지금은 네가 먼저 두게 하지. 어서 둬라, 엘사.''

다행히도 구스타프는 엘사가 악수를 두는 것으로 승기를 잡지 않고 천천히 판세를 알아차리게 했다—엘사는 실수를 많이 한 것을 알고 있었다. 처음에는 어떻게 이를 아는지 알지 못했지만. 더 나은 수를 둘 수 있었다는 직관만이 있었다. 대국이 끝날 때마다 구스타프는 당연히 어떻게 더 나은 수를 두는지 알려주었다. 엘사는 천천히, 그리고 확실히 자기 전략의 맹점을 파악하고 적절히 맞추었다.

``빨리 배우네.''

구스타프가 말했다.

``고마워요.''

엘사가 대답했다. 예상치 못한 칭찬에 고개를 약간 숙이고 있었다.

``나보다 낫네.''

에드문드가 말했다. 들뜬 채 체스판을 눈여겨보고 있었다. 말 몇 개를 거의 넘어뜨릴 뻔했다.

``이 말들 다 기억하는 데 엄청 오래 걸렸지.''

``금방 기억할 수 있는 거 아니었어?''

엘사가 물었다.

에드문드는 어깨를 으쓱였다.

``뭐, 맞는 말야—''

구스타프가 헛기침했다. 에드문드는 눈알을 굴리며 말을 바로잡았다.

``뭐, 맞는 말이야. 그 뭐냐—''

이번에는 엘사가 헛기침했다. 에드문드는 이번에는 이를 악물고 말했다.

``하지만 그게 쉽지가 않아. 구스타프 형이 말하길 큰 그림을 보라고 했지. 누나하고 형하고 진짜 끔찍하네.''

``미안해. 하지만 지적을 많이 받고 나니까\ldots''

엘사는 활짝 미소를 지으며 말했다.

``알았어, 누나. 그건 인정할게.''

``옆길로 새는 말이지만, 체스는 평생 하는 공부인데 엘사 넌 아주 잘 배우고 있어.''

구스타프는 새 대국에서 엘사가 킹 앞에 있는 폰을 앞으로 두 칸 옮겨 시작하는 것을 보고 소리 없이 웃었다.

``아, 대니시 갬비트인가. 적극적이네.''

E4, E5.

D4, ExD4. 구스타프가 엘사의 폰을 잡았다.

엘사는 고개를 끄덕이며 말했다.

``마르쿠스께선 선공이 최선의 방어라고 말씀하셨죠.''

C3, 일시 정지.

놀랍게도, 구스타프는 얼굴을 찡그렸다.

``아버지께선 항상 무작정 나가셨지만\ldots''

D5. 엘사의 미끼를 무는 대신, 구스타프는 자신의 폰 뒤에 있는 폰을 보냈다. 엘사가 희생한 폰의 자리에 있는 검정 폰 뒤에 있는 것을 말이다.

``자, 안전하게 네 폰을 잡았고, 이제 수는 몇 개 없지. 아버지의 말을 맹목적으로 따랐다면 이게 그 결과다. 이게 무슨 말인지 난 잘 알고 있지.''

대국이 끝나는 데에는 오래 걸리지 않았다. 꽤나 처참했다. 구스타프의 자제심이 갑자기 사라지고, 체스 고수가 자리했다. 구스타프는 세 수만에 엘사를 궁지로 몰아넣었다.

``구스타프 형하고 아버지께선 이젠 잘 안 어울리셔.''

에드문드가 갑자기 말했다.

성 안에서 민감한 말을 직설적으로 듣는 일을 아주 드물었다. 모든 왕자는 태어날 때부터 자신을 너무 드러내지 않고 말하는 법을 배우니 말이다. 사울마저도 가끔은 본인도 모른 채 엘사 앞에서 무언가를 숨겼다. 오랜 시간이 지나자 엘사는 빠르게 익숙해져 이것이 두 번째 언어가 되다시피 했다. 말해지지 않은 것이 말로 나온 것보다 더 중요했고, 엘사는 아주 적은 단서에서도 속뜻을 알아내는 법을 연습해 왔다.

그렇지만 사실 구스타프와 마르쿠스의 일은 엘사가 오랫동안 궁금해 온 것이었다. 구스타프는 당연히 만찬 때 엘사의 자리에 있어야 한다. 맨 끝이 아니라 바로 오른쪽 자리 말이다. 그리고 장남으로서, 마르쿠스가 임명하지 않아도 계승자가 되어야 한다. 다른 일에도 단서를 얻을 수 있었다. 부자가 서로 마주칠 때, 이들의 인사는 엘사가 조절하는 법을 배우고 있는 얼음보다도 더 차가웠다.

``에드문드.''

구스타프가 나무랐지만, 에드문드는 혀를 내밀어 보이고 체스 말을 다시 놓아주었다. 엘사는 생각에 빠져 가만히 있었다. 구스타프도 크게 관심을 보이지는 않는 듯했다.

``그냥 마음을 잘 안 쓰시는 거겠지. 난 그렇게 생각해.''

엘사가 말했다. 머릿속에서도 어색하고, 말로 하니 더욱 어색했다. 평소에 말하는 버릇은 아니었다. 차라리 묻는 것에 가까웠다—사실상 답을 강하게 요구하는 것이었다—엘사는 구스타프가 대답하거나 정중히 거절할 수 있는 표현으로 말했다.

아주 길고 긴장되는 시간 동안 엘사는 구스타프가 후자를 택할 것으로 생각했다.

``의견 차이가 좀 있지.''

구스타프는 마침내 인정했다. 그러나 처음으로 가면이 벗겨져 험악한 빛이 눈에 감돌았다. 평소에는 냉담한 얼굴이 갑작스러운 활기에 밝아져 그 어느 때보다도 마르쿠스처럼 보였다.

``그리고 당신께서 갚아야 할 빚이 아직 많지.''

에드문드는 무슨 말이냐는 듯 고개를 갸우뚱거렸지만, 엘사는 더 묻지 않았다.

``아\ldots\,알았어요.''

이들은 계속 체스를 두었지만, 분위기는 현저히 달라졌다. 구스타프는 생각에 잠겨 충고는 건성에다 도움이 되지 않았고, 엘사도 마찬가지로 말없이 계속 두었다. 대국은 점점 부주의해졌다. 에드문드가 지루해 죽기 전에 다른 것을 해야겠다고 말하고 떠나자 상황은 더 나빠졌다. 엘사는 이대로 불편히 있다가는 얼음이 튀어나오겠다고 생각했다. 약 오 분이 지나 에드문드는 스케치북을 들고 돌아와 앉던 자리에 다시 앉았다. 에드문드가 돌아와 엘사는 매우 기뻤다. 에드문드는 의미 없는 말로 침묵을 채웠다. 이러는 동안 열심히 무언가를 그리고 있었다.

엘사는 에드문드가 무엇에 이렇게 빠져 있는지 궁금해하지 않을 수 없었다.

하늘에는 어두운 적란운이 있었다. 한쪽에는 용이 입에서 불을 뿜어내고 있었다. 반대편에는 고귀한 모습을 한 빛을 발하는 사람이 황금 창을 들고 있었다. 자세히 보자 엘사는 에드문드가 사람을 그리는 것이 아닌 것을 알았다. 결이 있고 가죽과 같은 질감에 박쥐 날개 모양인 용의 날개와 크게 대비되는 커다랗고 흰 날개가 달려 있었다. 바로 천사였다.

``팍툼 에스트 실렌티움 인 카일로. 미카일 우트 프로엘리아레투르 쿰 드라코네.\footnote{Factum est silentium in caelo, Michael ut proeliaretur cum dracone.}''

에드문드가 읊조렸다. 엘사가 호기심에 바라보는 동안 술술 그리고 있었다.

``하늘에는 침묵이 흘렀습니다. 천사 미카엘이 그 용과 싸우게 된 것입니다.''

엘사가 천천히 말했다.

``정말 멋진 그림이야.''

``고마워, 누나.''

에드문드는 평소대로 이를 드러내며 미소를 지었지만, 올려다보지는 않았다. 계속 그림을 그리고 있었다. 이제는 천사의 날개에 명암을 주려고 연필을 비스듬히 쥐고 있었다.

``그래도 스테판 형 그림을 봐봐. 난 다른 걸 따라 그리지만, 스테판 형은 진짜 자기만의 그림을 그리잖아.''

``그럼 왜 이걸 그리는 거야?''

``내가 예전부터 좋아하던 거지. 그리고 에드문드가 찾아온 것이기도 하고.''

구스타프가 말했다.

``반대되는 두 사이의 대결을 좋아했지. 양 극단, 선과 악, 정의와 불의, 영원한 전투에 몰두해 있는 용과 천사.''

``그냥 멋있어 보여서요.''

에드문드가 느닷없이 말했다.

``지금은 안 좋아하시죠?''

엘사가 물었다.

``이상주의가 항상 실현 가능한 건 아니지. 천사는 항상 천사가 아니고, 용이 괴물이 되는 건 가끔일 뿐이야. 세상사가 이렇게 간단한 게 아닌 걸 알게 됐지. 선과 악으로 나뉘는 게 아니란 걸.''

구스타프는 체스판을 보며 고개를 끄덕였다. 색 대비가 선명하고, 말은 완전히 반대 위치에 놓여있었다.

``너는 그렇게 생각하느냐, 엘사?''

엘사는 잠깐 멈추었다. 답을 생각해내고 있었다.

``그러기를 바라죠.''

``아마 언젠간 답을 찾게 되겠지.''

구스타프가 말했다. 쓴웃음을 짓고 있었다.

``난 확실히 안 그런 거 같고.''

\textbreak

오래지 않아, 마르쿠스는 엘사에게 구스타프가 반역을 저지른 것을 말해주었다. 엘사는 용과 천사 미카엘이 나온 양 극단 이야기를 머릿속에서 지워버렸다. 구스타프가 준 악영향은 모두 없애기로 했다.

차라리 잘된 일이다.

안 그래도 복잡한 세상이 더 복잡해질 필요는 없었다.



\chapter[외전3. 불운의 수 십삼][외전 3\hspace*{.5em}불운의 수 십삼]{외전 3 \ 불운의 수 십삼}



\begin{quote}

\small 24장의 앞 이야기입니다. 한스의 배경과 동기를 다루고 있습니다.\sourceatright{역자}

\end{quote} %force indent

``왜 다들 우리를 싫어하죠, 어머니?''

한스는 고작 여섯 살이었지만, 이미 주변의 모든 사람이 보내는 경멸의 시선을 느낄 수 있었다. 자신이 기억하는 한 한스와 어머니는 이 관용의 벼랑 끝에서 살아왔다. 많은 사건은 한스에게 냉혹함을 가르쳐주었다. 자기 형제의 멸시를 견디는 것, 자기 아버지에게 무정하게 무시당하는 것, 심지어는 이들에게 아무런 힘도 없는 것을 아는 하인들에게도 냉대받는 것도. 여섯 살이 되었을 때, 한스는 서던 제도가 자신이 있어야 할 곳이 아님을 알았다.

``그런 식으로 생각하지 마렴.''

아냐가 중얼거리듯 말했다. 항상 이런 식으로, 부드러우면서도 주저하는 목소리로 말하곤 했다. 자신에게 시선이 쏠리지 않게 더는 목소리를 높이지 않으려는 듯이.

``저도 다 안다고요. 다들 제가 쓸모없다고 생각해요. 하지만 전 그 사람들보다 낫다고요.''

한스가 말했다. 신랄한 말투를 감추지는 못하고 있었다. 한스는 자기 어머니의 조용한 모습에 화가 났다. 겨우 이런 신세가 될 이유가 없다. 겨우 이런 신세가 전혀 아니었다.

``네가 쓸모없다고 생각하는 사람은 없단다.''

아냐는 계속 말했다. 침대 옆자리를 톡톡 두드리고 있었다. 한스는 움직이려 하지도 않았다.

``네 형들은 그냥 놀리는 것뿐이야. 다 지나갈 거야, 정말로.''

``전 형들 말한 적 없어요. 엄마가 말한 거죠.''

한스가 말했다. 아냐는 이 말에 움찔했다.

``아무도 절 안 좋아하는 거 다 알아요. 제가 아버지 혈통이 아니래요. 저 보고 서자래요. 이런 취급 이상을 바라는 게 그렇게 잘못된 건가요?''

아냐는 무릎을 꿇고 앉아 자기 아들과 눈높이를 맞추었다. 이 꾸밈 없이 애정 어린 눈에 한스는 바로 자신의 응어리가 부끄러워졌다. 이런 때에 한스는 자신이 자신의 어머니와 다르다는 것을 알았다. 순진한 어린아이의 흔적이 남아있어도, 한스는 이와 반대되는 모습을 볼 수 있었다. 자신의 어머니는 자신이 보기에 가장 다정하고 이기심 없는 사람이었지만, 한스 자신은?

``한스야, 사랑한다.''

아냐가 말했다.

``저도요, 어머니.''

한스가 나지막이 말했다.

아냐는 미소를 짓고 한스의 머리를 쓸어주었다. 이마에 떨어진 머리카락을 정리해주고 있었다.

``아, 한스야. 다시 엄마로 불러주면 안 되겠니? 예전엔 그러더니.''

``하지만 아버지께서 허락 안 하실 거예요. 예의 없는 거잖아요.''

``어머니란 말은 좀 쌀쌀하잖니. 마지막으로 엄마란 말을 듣고 싶구나. 부탁 좀 들어줄 수 있겠니?''

아냐가 속삭였다.

``사랑해요, 엄마.''

한스는 천천히 말했다.

``무슨 일이에요—''

``쓸모없는 게 아니란 거 잊지 마렴. 넌 아주 쓸모 많은 사람이야. ''

아냐는 말을 마치고 바로 한스를 껴안았다. 한스는 뭐라 말을 꺼낼 수 없었다. 아냐는 떨고 있었다. 자기 아들을 꼭 껴안고 있어도 한스는 떠는 것을 느낄 수 있었다. 그리고 무섭게도, 자신의 어머니가 말하면서 흐느끼는 것을 들었다.

``넌 왕이 될 수 있어. 꼭 될 수 있단다. 내가 짐을 지우지만 않으면.''

``대체 무슨 일이에요?''

한스는 다시 물었다. 자신의 어머니를 껴안고 어깨에 얼굴을 파묻고 있었다. 한스의 어머니는 지금까지 자신의 곁에 있어 주었다. 이제는 자신이 어머니의 곁에 있을 때다.

``아무 일도 없어. 이번엔 진짜 아무 일도 없어.''

아냐가 말했다. 말을 끝낸 다음 다시 자세를 바로 하고 손등으로 눈물을 닦고 있었다. 그러고는 한스의 머리를 톡톡 두드리고 엷게 미소를 지어 보였다.

``이 엄마는 더 많은 걸 해줄 수 있단다, 한스야.''

이날 이후로 한스는 자신의 어머니를 볼 수 없었다.

\textbreak

서던 제도의 아냐는 외곽 지역 농부 집안의 딸로 태어났다. 최하층 집안은 아니었지만, 최하층에 매우, 매우 가까웠다. 아냐의 가족은 가난했지만, 집이 없는 정도는 아니었다. 아냐는 어린 두 자매와 부모와 함께 허름한 집에서 자랐다. 이런 환경에서도 집안은 밝은 분위기와 웃음으로 가득했다. 그리고 물려 입은 해진 옷은 따뜻하지는 않을지라도, 아냐의 사랑으로 가득한 마음은 그 어떤 매서운 겨울도 이겨낼 수 있었다. 삶은 단순했다. 사치스럽지도 않고 편안했다.

이 모든 것은 열여섯의 나이에 서던 제도 성의 시녀가 되었을 때 바뀌었다. 아냐는 어렸지만 야심적이지 않았다. 이 반짝이는 성의 비천한 시녀 이상이 되는 꿈을 즐기지도 않았다. 그저 더 많은 돈을 벌어 자신의 늙은 부모가 편안하게 휴식을 취하고 자신의 동생들이 더 나은 삶을 살 수 있게 할 생각이었다. 손쉽게 받아들여질 수 있었다.

한 번 바라보는 것으로 충분했다.

아냐는 겉으로는 아름답지 않았다. 이국적인 아름다움이나 명문 귀족의 황홀함과는 거리가 멀었다. 아냐의 머리는 휘날리는 금발이나 반들반들한 적발이 아닌 흔한 갈색 머리였다.\footnote{역주: 서양에서는 갈색 머리가 매우 흔함. 흔히 생각하는 금발에 파란 눈은 북유럽 외에는 흔치 않아 `이국적'으로 보는 경우가 있음.} 흘끗 보아서는 평범하다고도 할 수 있을 것이다. 그리고 주근깨까지도 있었다. 하지만 아냐는 자세히 볼수록 더욱 아름다워 보이는 부류였다. 아냐는 곱살한 데가 있었다. 표정은 상냥하고 부드러워 보기 좋았다. 바라보면 안심이 되기까지도 했다. 하지만 아냐는 항상 분위기가 있었다. 말씨는 부드럽고, 미소를 지을 때에는 항상 조심스러워하며 부끄러워했다. 동그란 눈은 어떠한 비밀도 감추지 않았다. 매우 간단히 다른 이들의 믿음을 샀다.

그러니 성 안에서 가장 중요한 이의 눈길을 끌게 되어 모두의 부러움을 산 것은 놀라운 일이 아니었다.

일솜씨가 좋고 머리도 좋아 아냐는 성에 들어오고 오래지 않아 왕비와, 심지어는 왕의 소실들에게도 귀여움을 받았다. 하지만 마르쿠스 왕이 좋아하게 되자 아냐는 바로 경쟁 상대가 되었고, 같이 있기 꺼려지게 되었다. 조용한 태도는 똑같은 전략을 사용한 여인들과 똑같이 보였고, 부드러운 말씨에는 감미로운 유혹이라는 수식이 붙었다. 마르쿠스는 아냐가 계속 자신의 시중을 들게 했고, 아냐는 마르쿠스가 업무를 보는 동안 차를 우려주었다. 모든 뜬소문에도 이들 사이에는 아무런 일도 벌어지지 않았다.

불가피하게 일이 벌어지기 전까지는.

아냐는 한편으로는 왕의 관심을 오랫동안 끌 수 없다는 것을 알고 있었지만, 더 어리고 순진한 다른 한편은 늦은 밤까지 탄원서를 읽고 제멋대로 퍼져 나가는 본인의 왕국을 다스리려 애쓰는, 이 일에 열심인 남자를 좋아했다. 아냐는 몇 주동안 지켜보아 왔다. 얼마나 열심히 일하는지 직접 보아왔다. 그리고 마르쿠스는 멋진 남자였다. 중년의 나이지만 여전히 한창때였고, 강하고 잘생겼다. 그러니 자신에게 온 호의를 보이자, 아냐는 사랑을 동반한 불합리한 생각에 시달렸다.

아냐는 어떠한 직함도 받지 못하는 것을 개의치 않았다. 아냐는 하인도 주인도 아닌 겨우 시녀일 뿐이었다. 다른 시녀들보다 지위는 높았지만, 정식인 것은 아니었다. 오직 더는 소홀히 대해지지 않도록 암묵적인 동의만이 있을 뿐이었다. 아냐는 자신이 시중을 드는 여인들에게 미움을 받았고 동료들에게도 따돌림을 받았다. 전자는 아냐의 뻔뻔함을 경멸했고, 후자는 두려움과 새 지위에 대한 부러움에 아냐 앞에 고개를 숙였다. 그래도 아냐는 여전히 위협이 되지 않아 아무런 직위 없이 홀로 남았다.

직위가 생기기 전까지는.

아냐의 임신은 온 성을 아연실색게 했다. 해산일이 가까워져서야—소실들의 갖은 노력에도 애석하게도 유산하지 않은 것이 분명해져서야—마르쿠스는 아냐와 결혼해주었다. 하지만 다른 일들이 혼례를 서둘러 끝내고, 사소한 일이 되게 했다. 황태자 구스타프의 반역이 겨우 평정되어 구스타프의 아내와 태어나지 않은 아들은 바즈로 사라졌고, 에드문드 왕자는 태어난 지 얼마 안 된 데다가 이름 없는 산모는 먼 곳으로 보내졌고, 게다가 이틀 뒤에는 한스 왕자가 열두 번째 왕자보다 덜한 축하를 받으며 태어났다. 열세 번째 왕자 한스가 치여 살리라는 것은 누가 보아도 뻔하디뻔한 일이었다.

``이게 내 아이인가?''

마르쿠스가 물었다. 충격적인 빨간머리의 모습에 얼굴을 찡그리고 있었다.

아냐는 마르쿠스가 믿을 만한 대답을 할 수 없었다. 그리고 이 순간에 아냐는 마르쿠스의 진정한 모습을 보았다. 아냐는 자신이 마르쿠스를 사랑했다고 생각했다. 갑작스레 이 군벌의 심판이 서린 고압적인 용모를 바라보니, 아냐는 다른 사람을 보고 있다는 생각이 들었다.

\textbreak

한스는 당연히도 존재가 희미해졌다. 어떻게 이러지 않을 수가 있을까. 위로만 열두 명에, 평민 혈통이라는 낙인에, 적출이 아닐 가능성까지도 있는데 말이다. 그리고 한스의 형제들은 자신들이 더 우위라는 것을 보이기를 매우 좋아했다. 지나가며 마주칠 때마다, 굳이 한스와 말을 섞으려고 하는 왕자들은 자신들이 한스를 어떻게 보는지를 일깨워 주곤 했다.

``남들이 네 어머니 그렇게 기억 못 하는 게 참으로 유감이네. 아니면 뭔가라도 될 기회가 있었을 텐데, 한스.''

토비아스가 말했다. 자신의 머리카락을 어깨 뒤로 넘기고 있었다.

한스는 얼굴을 찡그렸지만 어떠한 반박도 하지 않았다. 고개를 숙인 채 자신의 어머니와 같이 쓰는 방으로 돌아갈 뿐이었다. 하지만 짧은 다리를 아무리 움직여도 벗어날 수는 없었다. 열두 살 많은 것으로 으스대는 것에 보폭도 더 크니, 토비아스는 한스를 손쉽게 당해낼 수 있었고, 계속 높은 콧소리로 말을 이었다.

``난 네가 좋아, 정말로.''

토비아스가 말했다. 한스의 걸음이 빨라지는 것을 보고 작게 웃고 있었다.

``그러니까 충고 하나 하지. 네 어머니와 의절해.''

한스는 그대로 멈추어 선 채 토비아스를 올려다보았다. 인상을 쓰는 것을 참을 수 없었다. 자기 형제의 말이 얼마나 큰 영향을 주고 있는지를 얼굴에 드러내는 것은 실수지만, 평정심을 유지하는 것은 나중에야 배운 것이었다. 여섯 살 때는 여전히 생각을 숨기지 않고 있었다.

``뭐래.''

나지막하고 효과도 없는 대답이었지만, 한스는 자신의 어머니와 의절한다는 것이 얼마나 믿을 수 없는 말인지 말로 설명할 수 없었다. 사울을 제외한 모든 다른 왕자들은 보모의 손에 자랐다. 한스는 생모의 손에 자랐고, 다른 양육 방식은 상상할 수 없었다. 겨우 신체적으로만 편안하게 하는 이 끔찍한 이들의 차갑고 비인간적인 손길을 생각만 해도 한스는 욕지기가 날 지경이었다. 이들이 어머니의 사랑을 대신할 수 있다는 것처럼. 한스는 자신의 형제가 거의 가엾어 보이기까지 했다.

``말조심해. 또 알바르한테 예의범절 배우고 싶진 않잖아.''

토비아스가 말했다. 한스가 움찔하자 코웃음을 치고 있었다.

``꼭 발작이라도 하는 것 같네.''

아주 틀린 말은 아니었다. 알바르 생각에 한스는 숨고 싶어졌다. 알바르는 한스를 괴롭히는 습관이 있었다. 적출이 아니라는 소문을 확신하고 있고, 평민의 기운은 무엇이든 바로잡을 요량으로 항상 진정한 남자는 어떻게 행동해야 하는지 말하곤 했다. 다른 왕자들이 한스를—본인의 말로는—`괴롭히는' 것을 참고 넘어가지는 않았지만, 본인은 자신도 모르게 한스를 무서워하게 했다.

``앞으로 조심할게.''

한스가 중얼거렸다.

``이제 그냥 가게 해주면—''

``야, 봐봐. 우리의 인기인이 나타났어.''

토비아스가 말했다. 막 복도에 나타난 익숙한 얼굴을 보고 고개를 끄덕이고 있었다. 토비아스는 높은 목소리로 외쳤다.

``사울! 이거 반가운 얼굴이구만. 아직도 아버지의 총애를 받고 다니나? 아니면 이젠 에드문드를 좋아하시나?''

사울은 이들을 지나쳤다. 평소처럼 리드와 올리버가 따라붙었다. 이 삼인조는 멈추어서 인사했다. 사울은 평소대로 정중하게 질문을 회피했다.

``아버지께서는 우리가 지닌 장점으로 우리 모두를 좋아하십니다. 좋은 아침입니다, 토비아스 형님.''

올리버와 리드도 마찬가지로 인사했다. 다만 토비아스에게만 인사하고 있었다. 셋 다 한스에게는 눈길조차 주지 않았다.

``봤지, 한스? 저게 바로 예절 바르단 거야. 그리고 몇 살이지, 사울?''

사울이 대답하려 입을 열자 토비아스는 손을 올리고 나이를 맞히려 했다.

``일곱? 여덟?''

``\ldots열하나입니다.''

사울이 말했다.

토비아스는 사울을 조심스럽게 바라본 다음, 다들 들을 수는 있지만 들으라고 하지 않은 것으로 생각할 정도로 조용한 목소리로 중얼거렸다.

``그러면 밥 좀 더 먹고 다녀야겠네.''

그러고는 더 큰 목소리로 말했다.

``열하나! 잘 크고 있네. 수업은 어떤가? 한스는 잘 따라가고 있고?''

``잘하고 있어.''

한스가 말했다. 그리고 조금이나마 기대에 찬 목소리로 덧붙였다.

``사울 형이 가장 잘하지.''

사울은 눈 하나 깜빡이지 않고 계속 토비아스를 바라보았다. 올리버가 한스를 바라보려는 듯 움찔거리자, 사울은 올리버에게 손짓했고, 올리버는 가만히 있었다. 계속 친절한 미소를 띤 채 사울이 대답했다.

``수업은 잘 듣고 있습니다. 식사 때 보죠, 형님.''

한스는 말없이 있었다. 이 삼인조는 바로 떠나갔다. 마치 한스를 전혀 보지 못한 듯했다. 이들은 한스에 관한 말 한마디도 없이, 혹은 한스가 존재하지도 않는다는 듯 갈 길을 갔다. 한스는 이들이 가는 것을 지켜보았다. 마음이 동요하고 있었다. 토비아스가 괴롭히는 것이나 알바르가 못살게 구는 것은 참을 수 있었다. 가장 괴로운 것은 솔직한 불호가 아니라 무관심이었다.

``꼭 투명인간인 것처럼 한다니까. 사울이 요리사나 마구간 일손들한테도 인사하고 다니는 거 알아? 그리고 너한테는, 자기 형제한테는—진짠진 모르지만—아주 조용히 있네.''

토비아스는 상처에 소금을 뿌리는 재주가 있었다. 널리 소문난 성숙함에도 사울은 여전히 아이였고, 사울의 어머니인 왕비는 건강에 해가 될 정도로 아냐를 싫어했다. 물론, 아냐는 한때 왕비의 시녀였다. 그리고 장남 구스타프가 왕의 총애를 잃자 왕비의 모든 희망은 차남 사울에게 옮겨갔다. 한스는 경쟁 상대였다. 한술 더 떠 자신의 발을 씻겨주고 닦아주던 이와의 경쟁이었다. 용납할 수 없는 일이었다. 사울은 자신의 어머니를 보고 배웠을 뿐이었다.

``꼭 나보다 형을 더 좋아하는 것처럼.''

한스가 중얼거렸다. 토비아스가 잠깐이라도 굳은 채 서 있는 것을 보자 기분이 좋아졌다.

``하지만 네가 더 만만하지.''

토비아스가 말했다.

더는 말없이 있기 힘들었지만, 한스는 자신의 어머니가 대립각을 세우는 것을 싫어하는 것을 알기에 고개를 숙인 채 지나갔다. 다행히도 토비아스는 가는 동안 조용히 있었지만, 한스는 방 앞에 도착해도 토비아스가 여전히 옆에 있는 것에 화가 났다. 토비아스가 문고리를 향해 고개를 돌리자, 한스는 토비아스를 쏘아보며 말했다.

``어머니께서 안 보고 싶어 할 거야.''

``하지만 지난번엔 정말 멋진 대화를 나눴는걸.''

토비아스가 말했다.

한스는 토비아스가 자신의 어머니를 울게 한 이야기를 꺼내지 않기로 했다. 구스타프는 이런 옹졸한 짓을 하지 않았지만, 파비안과 토비아스는 이 평민 출신 여인의 천한 배경을 헐뜯는 이야기를 하며 매우 큰 즐거움을 느꼈다. 토비아스는 특히 한스에 관해 보고하는 것을 매우 좋아했다. 형제들이 피해 다닌다는 둥, 아냐 자신은 초대받지 못하고 한스는 형식적으로만 초대받는 만찬 때에 자신의 아버지가 약점을 찌른다는 둥 이야기를 했다.

``딴 사람이나 귀찮게 해, 형이랑 참고 얘기할 사람이나 있으면.''

한스가 말했다. 도서실에나 가서 박혀 있으라, 한스는 생각했다. 자신의 어머니는 토비아스의 모진 말에 괴롭힘을 받을 이유가 없었다.

``알았어.''

토비아스가 말했다. 항복의 표시로 손을 들어 보이고 있었다.

``하지만 내 말 생각해 보라고, 한스. 아냐는 너한텐 짐일 뿐이야. 그리고 본인도 잘 알고 있을 테고.''

``내 어머니라고.''

한스가 말했다. 말투는 점점 거칠어졌지만, 토비아스는 어깨를 으쓱일 뿐이었다.

``내가 필요하면 어디로 오면 되는지 알지?''

토비아스는 말을 마치고 떠나갔다.

토비아스가 모퉁이를 돌아 사라진 후에야 한스는 방문을 열었다. 여전히 낮이었으니 방 안은 밝아야겠지만, 아침에 일어났을 때처럼 모든 커튼이 쳐져 있었다. 밖에 나간 것일까. 아냐는 밖에 나가는 일이 잘 없었다. 온 성이 보내는 야유를 피해 안에 있는 것을 선호했다. 한스는 커튼을 걷고 밖을 바라보았다. 아름다운 날이었다. 서던 제도의 날씨가 완벽하지 않은 날은 없지마는 이날은 특히 화창하고 맑은 하늘에 구름 몇 조각만이 뙤약볕만을 적당히 가리고 있었다.

뭐, 날씨를 즐기러 간 것일지도 모르겠다. 곧 돌아오리라.

한스는 그대로 기다렸다. 한 시간, 두 시간, 세 시간. 그리고 자다 일어나서도 계속. 그러나 밤이 되어서도 한스의 어머니는 돌아오지 않았다.

하인이 생색내는 듯하게 만찬 초대를 알리자 한스는 애가 타기 시작했다. 방을 거의 떠나지 않은 것은 차치하고서라도, 한스의 어머니는 한스를 만찬에 보내기 전에 항상 단정히 해주곤 했다. 지금쯤이면 이미 돌아와 있어야 할 것이다. 한스는 식사를 거르겠다고 하며 하인을 돌려보냈다. 자신의 무례함으로 벌을 받을 것을 알고 있었지만, 깨어 기다려야 했다.

다시 시간이 지났다. 한 시간, 두 시간, 그러고는 세 시간.

한스는 계속 깨어 있으려 하며 소리 없이 밤새 방 안에 있었다. 그러고는 불청객처럼 잠이 찾아왔다.

\textbreak

한스는 다음 날 아침에 빈방에서 일어났다. 담요를 걷어치우고, 한스는 밖으로 뛰쳐나갔다. 좌우를 둘러보아도 아무도 없자, 한스는 아무도 오지 않는 성의 이 먼 구석으로 쫓겨난 것으로 욕을 내뱉었다. 삼 분 동안 복도를 달려가다가 처음 보는 사람에게 한스는 말을 걸었다. 전에 한스의 어머니와 같이 일했던 사람이었다.

``제 어머니 보셨나요?''

한스가 물었다.

``누굴 말하는지 모르겠구나.''

``제 어머니를 모른다는—''

이 하녀가 물러나려 하자 한스는 말문이 막혔다. 하지만 충격이 가시자, 한스는 달려가 앞길을 막은 다음 말했다.

``거짓말 마요! 모를 리가 없잖아요—''

``네 어머니는 여기서 산 적이 없어. 알아듣겠니?''

이 비밀스러운 말을 끝으로, 이 하녀는 떠나갔다.

누구를 만나든 모두 똑같은 반응이었다. 모두 한스의 어머니가 누구인지 모른다고 대답했다. 한스가 자신의 방에서 혼자 살아왔다고 말하고, 어머니가 누구든 한스를 직접 기른 적이 없다고 말했다. 같은 대답을 들을 때마다, 한스는 점점 미칠 것만 같았다. 물론 한스의 어머니는 이곳에 살았다. 대체 무슨 장난인가. 파비안의 짓이 틀림없다. 유괴까지 한 것일지도 모른다.

한스는 도서실을 지나쳐 가려다 멈추어 섰다. 돌아선 다음 문을 열어젖히자, 예상대로 토비아스가 먼지 쌓인 커다란 책에 빠져있는 모습이 보였다. 한스는 토비아스를 향해 돌진하고는 손에 들고 있는 책을 쳐서 떨어뜨렸다. 책이 책상에 부딪혀 빈방 안에 울려 퍼졌다. 책꽂이에 꽂힌 책들이 흔들렸다.

토비아스는 한숨을 내쉬었다.

``대체 왜 그러는데?''

``우리 엄마 어딨어?''

한스가 물었다.

무슨 이유인지, 토비아스는 자리에서 바로 일어나 빈 도서실을 둘러본 다음 한스에게 시선을 돌렸다.

``다른 사람한텐 물어봤어?''

``하녀들한테만—''

``다행이네. 다른 사람 말고 나한테 오길 잘한 거야. 아버지 귀에 들어갔으면 큰일 났을걸.''

토비아스가 말했다. 다시 의자에 주저앉고 있었다. 하지만 토비아스의 말과는 달리, 한스는 형제 중 누구를 봐도 단박에 가서 물어보았을 것이다.

``엄마 어딨는지 알고 싶다고!''

한스가 외쳤다.

``이젠 아무도 그 사람 얘기 안 할 거야. 너도 똑같이 하는 게 좋을걸.''

토비아스가 말했다. 한스가 질문을 반복하자, 토비아스는 한숨을 쉬고 고개를 저었다.

``알아봤자 좋을 거 없다고.''

``알고 싶다고.''

한스가 서둘러 말했다. 토비아스는 일부러 천천히 책을 덮고 책에 쌓인 먼지를 털어내고 있었다. 한스가 초조하게 기다리는 동안, 토비아스는 책을 책꽂이에 다시 꽂고 있었다. 마침내 한스가 말했다.

``말해!''

토비아스는 자리에 앉아 한스의 눈높이에서 한스를 바라보고 있었다.

``정말로?''

``응!''

``알았어. 따라와.''

한스가 따라잡기 힘들 정도로 빠르게 토비아스는 도서실을 나왔다. 한스는 토비아스를 따라갔다. 반은 의심에 차 있었지만, 반은 희망에 차 있었다. 토비아스를 믿어도 되는지 의심이 들었지만 달리 방법이 없었다. 성을 나와 마구간에 이르는 길에 들어서자 한스는 더욱 화가 날 뿐이었지만, 토비아스는 아무런 말도 꺼내지 않았고, 한스가 부르자 발걸음을 더욱 빠르게 했다.

이들은 마구간을 지나 숲으로 갔다. 아무도 오지 않은 듯한 곳으로 더욱 깊이 들어가더니 곧 나무가 햇빛을 가려 무성한 잎의 틈새로만 빛이 조금씩 들어왔다. 풀들이 크게 자라 있어 한스는 풀의 가시에 베이지 않으려 조심하고 있었지만, 서두르는 탓에 그럴 수 없었다. 다리에 따끔한 느낌이 여러 번 왔지만 어렴풋한 느낌일 뿐이었다.

마침내 이들은 멈추어 섰다.

``다 왔어.''

토비아스가 말했다.

그저 공터일 뿐이었다.

``아무것도 없잖아!''

한스는 주변을 둘러보았다. 장난질에 화가 났지만, 더 시간을 낭비하고 싶지 않았다. 한스가 떠나기 전에, 토비아스는 한스의 손목을 잡아 공터로 다시 돌려세웠다.

``봐봐!''

토비아스가 말했다. 나무 한 그루를 가리키고 있었다.

그저 평범한 나무였다.

하지만 한스는 자세히 들여다보고 있었다. 토비아스가 정확히 가리키는 것을 보자 입 안이 마르고 숨이 멈추고 심장도 멎는 듯했다. 한스가 보고 있는 것은 공터가 아니었다. 나무도 아니었다. 이 아무도 없는 숲 속에 있는 그 무엇도 아니었다.

가지에는 올가미가 걸려있었다.

``말도 안 돼.''

한스가 말했다. 눈을 감고 있었지만, 방금 본 광경은 이미 머릿속에 새겨져 버렸다. 더욱 선명히 보일 뿐이었다. 한스는 다시 눈을 뜨고 토비아스에게 분노에 찬 시선을 돌렸다. 토비아스—같이 있고, 두려움 없이 화를 쏟아낼 수 있는 이 말이다.

``말도 안 된다고!''

한스는 토비아스에게 달려들었지만, 옆으로 내팽개쳐질 뿐이었다. 등이 땅에 부딪혔다. 팔로 몸을 받쳐, 토비아스가 일으켜 주려 하는 참에 한스는 바닥에서 일어섰다.

``네 어머니는 너 때문에 돌아가신 거라고.''

토비아스는 빈정거리는 말투로 말했다.

`이런 취급 이상을 바라는 게 그렇게 잘못된 건가요?'

`이 엄마는 더 많은 걸 해줄 수 있단다, 한스야.'

한스는 바닥에 무릎을 꿇고 밧줄을 바라보았다. 자신의 어머니가—

정말로 한스 자신 때문에, 자신이 한 말 때문일까? 하지만 한스는 정말로 더 큰 것을 원하지 않았다. 한스가 필요한 것은 자신의 어머니뿐이었다. 왜 한스를 위해 이런 선택을 한 것일까\ldots? 한스는 결코 자신의 어머니를 두고 다른 것을 선택할 이유가 없었다. 한스의 어머니는 이를 알았어야 했다. 그래야만 했다.

하지만 한스의 잘못은 아니다.

그리고 한스의 어머니의 잘못도 아니다.

``들어 봐.''

한스가 생각에 잠겨있자 토비아스가 말했다. 

``아버지가 다시는 네 어머니 얘기 안 하도록 명령했어. 그리고 네가\ldots\,적자인지 아닌지 얘기도. 널 인정해줬다 생각하고 넘어가자고. 괜히 헛짓하지 말고. 알아들었어?''

``\ldots인정해주셨다고?''

한스가 중얼거렸다.

``인정해주셨어.''

토비아스가 말했다.

한스는 일어서서 눈을 감았다. 이제는 올가미의 모습이 아른거리지 않았다. 이후에 이 일을 생각할 때, 한스는 얼마나 이러고 서 있었는지 떠올릴 수가 없었다. 다시 눈을 뜨자, 한스는 토비아스를 바라보고는 자신의 장소인 성으로 시선을 옮겼다. 그리고 이 희생으로 얻어내야 할 왕좌를.

``난 열세 번째 왕자야. 그리고 아버지의 아들이고.''

한스가 말했다.

불운의 수 십삼.

한스의 잘못도, 어머니의 잘못도 아니다. 열두 왕자를 탓할 일이다. 한스에게는 형제 열둘이 있다. 한스는 한 명씩 한 명씩, 이들에게 쓴맛을 보여줄 것이다.

한 명씩 한 명씩, 아무도 남지 않을 때까지.



\chapter[외전4. 이루어질 뻔한 만남][외전 4\hspace*{.5em}이루어질 뻔한 만남]{외전 4 \ 이루어질 뻔한 만남}



\begin{quote}

\small 34장의 앞 이야기입니다. 엘사와 아그다르 왕, 그리고 누군가와의 첫 만남을 다루고 있습니다.\sourceatright{역자}

\end{quote} %force indent

``안나야, 밤도 되기 전에 드레스 더럽히지 마려무나.''

아그다르 왕이 나무라며 말했다.

``말했다시피, 대신들이 많이 올 거라고. 그리고—''

``알아요. 기다릴 수가 없다고요. 정말 신날 거예요!''

드레스를 꽉 쥐고 자락을 들어 올린 채, 여덟 살 난 아렌델의 안나 공주는 자신의 머리를 만져주려는 안쓰러운 하녀들 주변을 돌아다니고 있었다. 드레스를 제대로 입힐 수 있게 가만히 서 있게 한 것부터도 운에 따른 일이었다. 이런 왈가닥을 가만히 앉혀 머리를 똑바로 만질 수 있게 하는 것도 기적이 필요한 일이었다. 이러는 동안 아그다르 왕은 방의 구석에서 팔짱을 낀 채 지켜보고 있었다. 꾸중하고 있지마는 입꼬리는 살짝 올라가 있었고 목소리도 부드러웠다.

하녀들의 애원과 안나의 고집이 한바탕 더 이어지고 나서야 아그다르 왕은 방 안을 여전히 뛰어다니는 안나에게 다가가 팔로 안아 들었다. 왕이 눈알을 굴리고 거울 앞에 내려놓아도 안나는 웃을 뿐이었다. 안나는 이 작은 대결을 계속할까도 했지만, 아그다르 왕은 계속 안나를 주시하고 있었다.

``시작할 때까지 얼마 안 남았단다. 손님들이 곧 도착할 거야.''

``그리고 드디어 새로운 사람을 만나게 되는 거고요!''

안나는 기쁜 목소리로 소리쳤다. 허공에 주먹을 지르며 미용사 한 명을 거의 칠 뻔했다. 안나는 바로 주먹을 거두고 사과했지만, 미소는 얼굴을 떠날 줄을 몰랐다.

``왕자나 공주 중에 재수 없는 사람이 없어야 할 텐데요. 제 또래들이에요?''

``그런 왕자나 공주도 있을 거야.''

안나의 미소는 더욱더 커졌다.

친구를 사귈 수도 있을 것이다.

\textbreak

``드러내.''

엘사는 등을 똑바로 편 채 배 갑판의 정중앙에서 완벽히 균형을 잡으며 서 있었다. 천천히 숨을 들이마시며 사방의 바다 냄새를 맡고 있었다. 눈을 감은 채 배의 흔들림과 부딪히는 파도 소리를 느끼고 있었다. 이 느낌은 점점 커지더니 온몸을 흔들고 귀를 때렸다. 엘사는 전혀 신경을 쓰지 않았다. 방해할 수 있는 것은 없었다.

``자유롭게 해.''

엘사는 눈을 팍 떴다. 얼음과 같은 파란색 눈에는 단호한 빛이 비쳤다.

자연적이지 않은 겨울 폭풍이 주변에 터져 나왔다. 매 초가 지날 때마다 힘이 쌓이더니 곧 눈 태풍의 눈에 엘사가 서 있는 모습이 되었다. 엘사는 앞으로 팔을 뻗었다. 눈과 얼음이 거대한 하나의 덩어리가 되어 앞으로 돌진했다. 돛대에 부딪히기 직전에 엘사는 손을 펴고 손바닥을 위로 올렸다. 이 한기의 물결은 위로 솟아올라 돛대를 따라 올라가 꼭대기에 닿고는 둘로 갈라져 밖으로 터져 나갔다. 엘사는 주먹을 쥐었다. 바다를 향해 곤두박질치며 엘사의 눈은 수면을 때리고 거의 느려지지도 않은 채 깊이 빠져 들어갔다.

집중하며 엘사는 몸에 힘을 빼고 손을 폈다. 자신의 마법이 깊은 물 속에 흩어지면서 넘실거리는 바다를 확인하고 있었다. 엘사가 하려는 것은 어리석은 짓이었다. 절대 성공하지 못할 일이다. 그 어떤 `인간'도 이런 일을 해낼 수는—

``한계는 없어.''

마지막 남은 것까지 쥐어짜 내 자신의 마법이 자유로이 날뛰게 하며 엘사는 성공하기를 바라고 있었다. 다시 폭풍이 주변을 에워쌌지만, 이번에는 달랐다. 바람은 금방이라도 돛을 찢을 듯 마구 불어 닥쳤고, 파도는 더욱 거세어졌다. 배를 집어삼키려 하고 있었다. 엘사는 전혀 신경을 쓰지 않았다. 눈을 감고는 힘이 점점 쌓이게 했다. 취할 듯한 힘이 혈관을 메우는 것을 제하고는 모든 감각을 차단하고 있었다. 그리고 자신도 모르게 팔을 벌리고 어두워지는 하늘을 향해 고개를 치켜들었다.

얼음이 바다를 메웠다.

자신의 마법이 이미 퍼져있는 바다 위에 서서 엘사는 한기가 퍼지도록, 바다가 얼도록, 이 불가능을 극복하도록 마음을 먹으며 이 바다를 정복해 자신의 것으로 만들려 했다. 끝없는 분전이었다. 한기가 펴져도 혹독한 바다는 엘사의 영역과 싸우며 얼음을 부수고 눈을 녹였다. 하지만 엘사는 계속 자신의 힘을 물 깊숙이 쏟아 부었다. 얼음이 생겨나고, 깨어지고, 다시 생겨나고 깨어지는 것을 반복했다, 깊은 곳에 압력이 쌓이고 쌓일 때까지.

제어력을 잃을 때까지.

모든 것이 한 번에 무너졌다. 바닷속에서의 폭발에 두 물 벽이 위로 솟아올랐다. 온 배가 작아 보일 정도였다. 그러고 나서 바다는 곧 잔잔해졌다. 엘사는 남은 힘을 풀어주었다. 하늘은 개고 파도는 사그라졌다. 남은 것은 비처럼 후드득 떨어지는 바닷물이었다.

``정말 놀라웠다. 하지만 지금까지의 것 중 가장 대담한 목표였다. 바다를 가르는 것은 너에게도 너무 과한 일일지도 모르겠구나.''

엘사는 몸을 돌리고 고개를 숙였다. 다시 고개를 들자 빤히 즐거워하는 빛으로 미소를 짓는 마르쿠스가 보였다. 엘사는 이 모습에 자신을 방어할 수밖에 없었다.

``얼리려 했습니다. 장차 해낼 수 있을 것으로 생각합니다.''

``마음에 품은 일은 무엇이든 해낼 수 있을 것이다, 엘사야.''

마르쿠스는 가까이 다가와 팔을 벌려 보였다. 엘사의 표정은 밝아졌다. 주저 없이 바로 허리를 껴안았다. 마르쿠스가 애정을 보이는 일은 드물었다. 엘사에게만 보이는 애정이었다.

``자랑스럽게 해 드릴 거예요.''

마르쿠스는 물러나 다시 다정한 미소를 보였지만, 말을 꺼내기 전에 얼굴이 일그러졌다. 몸을 돌려 마르쿠스는 몸을 굽히며 기침을 하기 시작했다.

``괜찮아, 괜찮단다.''

마르쿠스는 엘사를 안심시켰지만, 엘사는 걱정하는 얼굴을 하는 모양이었다. 마르쿠스는 기침을 내뱉고 덧붙였다.

``걱정하지 마라. 곧 도착할 테니 또 놀라운 일은 벌이지 말아 주겠니?''

마르쿠스는 휴식을 위해 자신의 방으로 들어갔다. 엘사는 갑판에 홀로 남았다. 난간으로 걸어가 엘사는 조용히 바다를 바라보았다. 달리 할 일이 없으니 아렌델에 가기 싫다는 생각만이 가득했다. 바다를 얼리려 한 이유 중 하나도 발목을 잡기 위한 것이었다. 마르쿠스가 다른 왕자를 대신 데려가거나, 사울도 같이 왔기를 바랐다. 하지만 엘사는 생각에 잠긴 채 혼자 있었다.

아렌델에서는 매년 낙성제\footnote{落星祭}가 열렸다. 이때에는 이를 기념하고 동맹을 강화하기 위해 특별한 행사가 열렸다. 엘사는 낙성제를 싫어했다. 엘사는 바로 이날에 태어났다. 축제가 벌어지는 동안 지옥과도 같던 집에서 사랑과 관심을 바라던 자신이 떠오를 뿐이었다.

그저 이를 안고 살아가야 할 뿐이었다.

\textbreak

\forceindent``엘사를 소개해 주었던가요?''

엘사는 앞으로 다가가 미소를 지었다. 마르쿠스에게와는 다르게 묵례를 하고 있었다.

``만나서 영광입니다, 폐하.''

``내가 다 기쁘다는 말을 하고 싶구나.''

아그다르 왕이 말했다.

연회가 무르익어 가는 시끌벅적한 모습에 대한 말이었다. 연회장에서 대신들이 춤추는 동안 한쪽에서 곡을 연주하는 연주자들과, 은 쟁반에서 조심스럽게 고른 간식들을 권하며 이리저리 돌아다니는 하인들의 모습에 관한 말이었다. 엘사는 거의 눈길도 주지 않았다. 위험을 피할 정도로만—일단 위험이 다가온다면은—주의를 기울일 뿐이었다. 도착한 때부터 엘사의 눈길은 아그다르 왕에게만 가 있었다.

``하지만 마르쿠스, 당황스러운 것이, 구스타프는 어디에 있습니까?''

아그다르가 물었다.

``불행히도 구스타프는 제 기대와는 어긋나게 됐습니다. 이제는 엘사가 제 후계자입니다. 엘사는 차차 알아가게 될 겁니다.''

``아, 모\ldots\,몰랐군요. 구스타프에게는 수치스러운 일이겠지만, 이유가 있어서 그러셨겠죠.''

아그다르는 엘사를 더 찬찬히 살펴보았다. 엘사는 조금도 물러서지 않고 차분히 눈을 마주쳤다.

``몇 살이니, 엘사야?''

``열한 살입니다, 폐하.''

엘사는 예를 보이며 고개를 숙였다.

``미래에도 폐하의 왕국과 제 왕국이 좋은 관계를 유지하기를 바랍니다.''

아그다르는 미소를 지었다. 엘사는 아그다르의 미소가 진심이 실린 다정한 미소인 것을 알아채고 놀랐다. 단지 아주 뛰어난 거짓말쟁이여서 그러리라.

``나도 그러기를 바라는구나. 어린 것치고 꽤 조숙하구나, 엘사야. 정말이지 놀랍다.''

``엘사는 저의 더없는 위업이 될 겁니다. 틀림없이 서던 제도를 새로운 시대로 이끌 겁니다.''

마르쿠스가 말했다.

``안나한테도 똑같은 말을 할 수만 있다면. 정말 걱정됩니다.''

아그다르는 고개를 젓고 웃음을 터뜨렸다.

``이미 여기 와 있어야 하지만, 꽤나 난장판을 만들어서 좀 늦을 겁니다. 모든 손님이 도착할 때쯤엔 준비돼 있기를 바라야죠.''

``그렇게 되겠죠.''

마르쿠스는 엘사를 보며 말했다.

``가서 좀 섞여 보려무나. 모든 게 낯설 테지만, 좋은 경험이란다.''

``물론이죠. 만나서 반가웠습니다, 폐하.''

엘사는 무엇 때문에 나가 있으라는 것인지 알아듣고 자기 할 일을 하러 갔다. 대신들과 섞이는 일은 필요치 않았다. 자신을 이들 수준으로 낮출 이유도 없거니와 마르쿠스도 도착하기 전에 이를 분명히 했다. 엘사는 단지 지켜보기만 했다. 이곳의 사람들은 자신들을 터놓고 있었다. 이들은 큰 목소리로 자유롭게 말을 나누었고 이곳의 부산한 분위기는 서던 제도의 효율적인 모습과는 극명히 대비되었다.

엘사는 물잔을 들어 조심스럽게 들이켰다. 계속 안을 바라보고 있었다. 물론 거의 모두가 자신보다 나이가 많았다. 하지만 대부분은 마르쿠스보다 나이가 적었다. 모두 뚱뚱하고 오만했고, 당연히도 의지가 탄탄하지 못하고 마르쿠스의 권위적인 모습도 없었다. 엘사는 이들도 같은 왕족이라는 것을 좀체 믿을 수 없었다. 이따금 자기 나잇대의 사람을 마주쳤다. 코로나는 몇 년 전에 공주를 잃었지만, 다른 많은 왕국이 참석해 있었다. 엘사는 새로운 사람이었으니 이들은 셈을 하며 엘사를 바라보았다. 누구인지 궁금해하며 시간을 쓸 가치가 있는지, 동맹인지 적인지 따지고 있었다. 엘사는 이들을 무시했다.

트럼펫 소리가 들리자 엘사는 앞쪽을 향해 바라보았다.

``아렌델의 아그다르 왕!''

아그다르는 안으로 들어와 왕좌 앞에 섰다. 침착했고, 엘사가 보아 온 다른 사람들과는 많이 달랐다. 마지못해 엘사는 아렌델의 왕이 최소한 바보가 아니라는 것을 인정할 수밖에—

``아렌델의 안나 공주입니다!''

엘사는 눈을 끔뻑였다. 넘쳐흐르는 열광으로 빨간 머리 소녀가 안으로 들이닥쳤다. 힘껏 손을 흔들며 완전히 잘못된 자리에 멈추어 섰다. 엘사는 이 안나가 예절을 더 배워야겠다는 생각이 들었다. 누군가가 안나를 자신의 아버지 옆으로 데려갔다. 그러고 나서 아그다르는 한바탕 연설을 늘어놓았다. 엘사는 안나의 크게 뜬 눈과 너무나도 뻔한 호기심이 어린 눈빛에 정신을 쏙 빼앗겼다. 엘사는 사람들 사이에 끼어 있었으니 이 공주가 찡그린 자신의 얼굴을 보지 못했으리라 생각했다. 그래도 왕족은 자기 생각을 그렇게 쉽게 드러내서는 안 된다고 말해주고 싶었다. 솔직히 엘사는 그렇게 크게 웃는 것과는 거리가 멀었다. 게다가 특히나 지루한 부분에서 하품을 한 후에 안나는 연설이 계속되는 와중에 정말로 사람들에게 손을 흔들어 보였다. 엘사는 헛기침을 하고 시선을 돌렸다. 입꼬리가 올라가려는 것을 참고 있었다.

이렇게 호의적인 모습을 보는 것이 정말로 조금은 기운이 났다.

교육을 받기 전의 자신의 모습이 살짝 떠올랐다.

``\ldots그리고 평화가 이어지기를 빕니다. 그리고 우리의 아이들이 위협과 불화에 방해받지 않고 삶을 이어가도록 관용을 빕니다.''

좋던 기분은 한순간에 날아갔다.

엘사는 연회장에서 바로 자신의 마법을 폭발시키려는 충동을 참아내었다. 아그다르가 정말로 평화와 관용을 설파하려는 것이라면 이곳보다는 자신의 왕국에부터 이를 퍼뜨려야 할 것이다. 아렌델의 누구도 자신을 받아주지 않았고, 자신의 고통을 덜어주지도 않았다. 분명히 `누군가'는 무슨 일이 일어나고 있는지 알았을 것이다. 그러면서도 자신을 구해준 사람은 마르쿠스였다. 엘사는 깊게 숨을 들이마셨다. 떠나도 좋은 때가 되자마자 사람들을 헤치고 나와 마당으로 나갔다. 엘사는 마지막으로 안나를 바라보았지만, 이 빨간머리가 알아채기 전에 엘사는 밖으로 나갔다.

아렌델에는 이미 진절머리가 났다. 그리고 이 썩어빠진 곳에 있는 모든 사람도.

\textbreak

연회가 끝나고 꽤 시간이 지난 후에 안나는 언짢은 기분을 간직한 채 무거운 발걸음으로 자신의 방으로 돌아갔다. 다른 왕자나 공주와 이야기를 나누기는 했지만, 안나는 이들을 거의 좋아하지 않았다. 모두 성품이 못나고, 안나 자신의 관점으로는 자기들 생각으로 가득했다. 전체적으로 이날 밤은 매우 실망이었다. 물론 음악 연주는 재미있었고 초콜릿은 매우 맛있었지만, 안나는 사람을 만나고 싶었다. 

안나는 계단을 올라가 곧 자신의 방으로 가는 복도를 느긋하게 걸어갔지만, 공교롭게도 창밖을 바라보았다. 관심을 끄는 것을 딱히 없었다. 다른 때처럼 그냥 지나칠 수도 있었지만, 무언가에 이끌린 듯 창밖을 내다보았다. 처음 눈이 간 곳은 쏟아질 듯 밝게 빛나는 보름달도 아니고 구름 없는 하늘에 박힌 별도 아니었다. 안나는 위가 아닌 아래를 보았다. 분수대 앞에 금발 머리 소녀가 앉아 있었다.

너무도 슬퍼 보였다.

안나는 이렇게 슬퍼하는 사람을 본 적이 없었다. 외로워 보였다. 홀로 앉아 물속에 손가락을 넣어 휘젓고는 물결이 이는 물에 일렁이는 자신의 모습을 바라보았다. 어린 나이였지마는 안나는 자신의 도움이 필요한 것을 느꼈다. 안나는 최대한 빠르게 계단을 내려갔다. 갑옷 장식에 넘어질 뻔하면서도 힘껏 달려 마당으로 향하는 문밖으로 나왔다.

``얘! 난 안나라고—''

아무도 없었다.

안나는 그 아이의 이름도 알지 못했다.

\textbreak

떠다니는 말로는 전쟁이 하루가 채 되지 않아 끝났다고 했다.

