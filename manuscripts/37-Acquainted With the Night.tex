

\chapter[37장  어둠과의 동침][37장\hspace*{.5em}어둠과의 동침]{37장 \ 어둠과의 동침}



엘사는 비틀거리며 서던 제도 성으로 걸어갔다.

홀로.

성이 얼마나 불길해 보이는지 엘사는 전혀 알지 못했다. 십여 년 전, 멀리서 어렴풋이 보이는 정말 높고 장대한 모습을 보았을 때 엘사는 경외감이 들었다. 지금은, 바깥에 서서 톱니와 같은 선과 뾰족탑을 올려다보는 엘사의 눈에 예전의 장엄함은 전혀 비치지 않았다. 꼭대기에 놓인 금강석과 같이 빛나는 얼음판에도 잿빛 석재는 이 얼음판과 똑같이 인간미 없이 메말라 보였다. 최고의 예술 작품이 놓여 있다 해도 이 외곬스러운 군국주의적인 모습에 묻혀 찾지도 못하게 될 것이다.

차라리 요새의 모습이었다.

엘사는 안으로 걸어 들어갔다. 뒤에서 문이 닫히자 눈을 감았다.

잠깐의 시간이라도 지났을까. 엘사는 바로 안나가 있기 전의 삶이 떠올랐다. 단순한 차가움을 넘어 메마르고 고독하고, 닫힌 문과 같았다. 침묵과 거짓에 둘러싸여 자신의 목소리도 얼굴도 낯설고 알아볼 수 없는 지경까지 이르렀다. 그러면서도 여전히 이 성이 자신의 집이라고는 할 수 있었다. 이와 다른 삶을 살 수 있는지는 거의 알지 못했다.

이제는 무엇이 빠져 있는지를 알 수 있었다.

성 안 사람들은 자신의 도착을 경비병을 통해 안 듯하다. 엘사는 왕자들이 지내는 곁채에서 다가오는 발소리를 들었다. 몸을 똑바로 하고 상처에서 손을 떼자마자 같이 있을 일 없는 두 사람—알렉과 사울—이 도착했다. 둘 다 옆문을 지나 다가왔다. 사울은 의료품 가방을 들고 나타났다. 알렉은 보기 드문 미소를 띠고 있었지만, 엘사를 보고는 곧 평소의 인상을 쓰는 모습으로 돌아갔다.

``안나 공주님은 어딨습니까?''

알렉이 물었다.

엘사는 웃음이 나오려는 것이 느껴졌다. 억눌린 무언가가 비집어 나오려 하고 있었다. 그러면서도 냉담한 눈빛으로 엘사는 알렉을 쏘아보며 어떠한 반론도 받지 않겠다는 투로 말했다.

``안나는 돌아오지 않을 것이다.''

``\ldots뭐라고요?''

알렉이 가까이 다가서려 하자 사울은 어깨에 손을 올려 알렉을 붙들었다. 하지만 시선은 엘사를 떠나지 않았다. 알렉이 빠져나오려 몸부림치는 중에도 사울은 계속 눈길을 주며 얼굴을 살피고 있었다. 엘사는 눈을 떼지 않고 사울을 바라보았다.

``진정하게.''

사울이 퉁명스럽게 말했다.

``이래라저래라 말게.''

알렉은 사울을 뿌리치고 물었다.

``왭니까? 공주님은 돌아오겠다고 말했습니다. 무슨 일입니까?''

``아무 일도 아니다.''

엘사가 말했다. 왜 계속 웃음이 나오는지는 알지 못했지만, 자신이 아무것도 하지 않은 것은 정말로 정말로 사실이었다. 그리고 아주 아주 재미있는 상황이었다.

``좋아할 것으로 생각했다.''

알렉은 얼굴을 찡그렸다.

``좋아한다고요?''

``자네야말로 내가 안나의 의지에 반해 계속 데리고 있다고 하지 않았는가. 같이 지낸 때가 너무도 좋아서 전부 잊어버린 것인가?''

``그런 게 아니라\ldots\,제가 틀렸습니다. 공주님은 여기 있고 싶어 했습니다.''

다른 때면 알렉의 후회에 찬 얼굴을 즐겼을지도 모르겠다. 지금은 이 얼굴을 짜부라뜨려 말을 하지 못하게, 또다시 상처를 주는 이런 말이 다시는 나오지 않게 하고 싶었다.

``자네의 말이 맞았다. 안나는 아렌델에 남는 것을 선택했다.''

엘사는 이 형제들에게서 눈을 떼었다. 자신의 고백에 갑자기 속 깊은 곳에서부터 피로가 몰려왔다.

``도착 소식을 전파하라. 그리고 사울, 따라오너라.''

알렉은 매우 놀라 아무런 대꾸도 못 하는 듯 보였다. 엘사는 말이 없는 틈을 타 계단을 올라갔다. 사울이 뒤따랐다. 같이 계단을 올라 엘사는 자신의 곁채로 갔다. 짧은 거리였지만 속이 드러나 보였다. 평소의 사울은 존대의 행동으로 두 걸음 뒤로 물러서 있었지만, 이번만큼은 엘사는 사울이 거리를 유지하려 애쓰는 모습을 눈치챘다. 앞으로 튀어나오다 다시 위치를 바로잡기를 되풀이하고 있었다.

집무실에 도착하자 사울이 물었다.

``무엇이 필요하십니까, 엘사 여왕님?''

엘사는 단 위 책상에 놓인 의자가 아닌 소파에 앉았다. 엘사의 선택에 놀랄 법했지만, 사울은 놀라움을 잘 숨겼다.

``이미 잘 준비되어있는 것 같다.''

엘사는 사울의 가방을 향해 고개를 끄덕이며 말했다.

``그러니 치료를 요구한다.''

아렌델을 떠나고 몇 시간이 지난 뒤, 엘사의 몸은 다시 평형을 되찾아 독에 버티고 있었다. 엘사는 독미나리로 생각했다. 짙은 농도 때문에 빨리 작용한 듯했지만, 독이 문제가 아니었다. 힘이 돌아오자마자 마법은 계속 화살에 맞은 상처를 치유하려 했다. 감염된 부위가 커지는 것은 무시하고 말이다. 며칠간의 뱃길 동안 상처는 붙고 떨어지는 것을 반복했다. 엘사는 드레스의 몸통 부분을 찢어내었다. 옆구리의 반쯤 치유된 상처가 드러났다.

이제는 괴사한 살덩이가 되어 있었다. 거의 이십 센티 정도로 허리춤을 지나 등까지 이어져 있었다. 붉은색과 누런색으로 얼룩덜룩했다. 잔뜩 부은 피부 가운데에는 고동치는 상처가 있었다. 피와 고름이 흘러나오고 있었다.

``흉터도 안 남게 하겠습니다.''

사울이 중얼거렸다.

``그것이 가능하기는 한가?''

``제대로 치유되게 하려면 일단 살을 잘라내야 합니다.''

사울이 말했다. 옆에 무릎을 꿇은 채 가방을 열어 엘사는 볼 생각 없는 물건들을 꺼내고 있었다.

``하지만 불가능할 수도 있습니다. 여왕님의 자동적 방어 때문에 다가가지 못하면—''

``이제는 조절할 수 있다.''

엘사가 말했다.

``그렇다면 깨어 있어야 하는 것을 알고 계시겠죠. 여왕님께 효과를 줄 수 있을 정도로 강한 마취제는 없습니다.''

사울은 올려다보며 팔을 뻗었다. 얼굴을 만지려는 듯했다. 엘사는 고개를 돌렸다. 사울은 대신 빠져나온 머리카락을 귀 뒤로 넘겨주었다.

``참지 못할 고통은 없다.''

엘사는 자신의 신체적인 고통을 고마워하기 시작했다. 자신이 살아있다는 것을 알려주는 증표였다. 최소한 이것은 여전히 느낄 수 있었다. 온갖 일을 겪고 엘사는 자신이 살아있다는 데에 더러 회의를 느끼곤 했다. 거의 다른 모든 느낌이 빼앗긴 반쯤 죽은 삶을 사는 데에 말이다.

``그렇다면 시작하죠.''

소독약은 살갗에 붓는 쇳물과 같이 느껴졌지만, 엘사는 조용히 고통을 참으며 피부가 벗겨지는 것을 지켜보았다. 사울은 엘사의 맨 배에 손을 올렸다. 다른 손에는 메스가 빛나고 있었다. 옆구리에 날이 들어가 살을 째자 엘사는 이를 악물며 계속 바라보았다. 눈을 감을 수 없었다. 소리도 내지 않았다. 이를 똑똑히 기억할 참이었다.

상처 각각은 자신의 나약함을 알리는 또 다른 증표였다.

``무슨 일입니까?''

사울이 물었다. 엘사가 계속 말없이 있자 덧붙였다.

``말하는 것이 주의를 돌리는 데 좋을 겁니다.''

``아그다르가 암살을 꾸몄다.''

엘사는 겨우 말을 꺼냈다. 전혀 주의가 돌려지지 않았다. 외려 고통이 더 심해지는 듯했다. 옆구리에서만 오는 고통이 아니었다.

``하지만 미리 채비한 듯하다.''

``한스를 살피고 있었습니다. 때마침 말이죠.''

정말 `때마침'이다.

하지만 엘사는 미리 채비한 것을 부정하지는 않는 것을 알아챘다.

곧 고통이 끝났다. 사울은 죽은 살을 모두 없앤 상처에 붕대를 두르고 있었다. 엘사는 벌써 달라진 것을 느낄 수 있었다. 치유를 방해하는 것이 사라져 자신의 마법이 빠르게 상해를 복구하고 있었다. 한기는 불타는 것 같은 고통을 무디게 했다. 이마저도 곧 가실 것이다. 사울이 자리에서 일어나 가까이 몸을 기울여 허리춤으로 손을 뻗자 엘사는 사울의 멱살을 잡아 가까이 끌어당겼다. 거의 눈앞에서 엘사는 속눈썹 하나하나를 셀 수 있었다. 사울은 여전히 차분했다.

``왜 이런 짓을 했느냐?''

사울은 잠깐 엘사의 손을 내려다본 다음 엘사를 바로 바라보았다.

``무슨 말인지 잘 못알아듣겠습니다.''

이런 사울의 말에도 자신을 바라보는 검푸른 눈은 이미 필요한 것을 알려주었다.

``무언가로 책망하려면 더 명확히 말씀하셔야 합니다, 폐하.''

``아그다르는 안나가 나의 약점을 자세히 적어주었다고 말했다. 나의 마법이 어떤 것인지 자세히 알고 있는 사람은 많지 않다.''

엘사가 말했다. 얼음이 비밀이라는 것이 아니라 모든 것—자동적 방어, 번개—이 어떻게 생겨나는지 말이다. 이를 알고 있는 사람을 엘사는 한 손에 꼽을 수 있었다.

``그리고 무엇보다 안나는 나의 약점을 그렇게 완벽히 추정해낼 수 없다.''

``맞습니다.''

사울이 말했다.

``솔직히 말하면, 안나 공주는 너무 단순하니까요.''

``그리고 단 한 사람만이 내게 듣는 독을 알 수 있다.''

엘사는 사울의 입꼬리가 조용한 만족감에 올라가는 것을 바라보았다.

``몇 년 동안 나의 건강을 살핀 후에야 말이다.''

``저였습니다. 제가 아그다르에게 그 편지를 보냈습니다. 하지만 안나 공주의 손글씨였죠.''

``내가 알아챌 것을 알고 있었을 것이다.''

``그렇습니다.''

``그리고 내가 돌아오면 내 손에 죽을 생각은 않았느냐? 혹은 내가 죽을 것으로 생각했느냐?''

사울은 아무런 말도 하지 않았다. 한동안 이들은 말없이 있었다. 둘 다 결단을 내릴락 말락 상황을 재고 있었다. 침묵은 소리 없는 말로 가득했다. 자신이 멱살을 쥐고 있지마는 엘사는 주도권이 없는 듯한 이상한 느낌이 들었다. 마침내 엘사는 멱살을 놓아주었다. 사울은 붕대질을 끝내고 물러났다. 무릎은 여전히 꿇려 있었다. 판결을 기다리고 있었다.

``여왕님이 훨씬 우위이니 성공할 수가 없었죠.''

``그렇기는 했다.''

``다만 정신이 팔려서 그랬을 뿐이죠.''

``그래서 요점이 무엇인가?''

``이미 알고 계신 걸 보여주려는 겁니다.''

사울이 말했다. 조심스럽게 손을 뻗었다. 사울이 손을 잡아도 엘사는 몸을 피하지 않았다.

``안나 공주는 여왕님을 선택하지 않을 겁니다. 우리의 일원이 아니라고요. 결코 완전히 여왕님의 것이 될 수 없다고요. 하지만\ldots\,전 항상 같은 편이 되어줄 수가 있다고요. 항상 그래 왔죠.''

``그저 안나 대신으로 있겠다. 원하는 것이 이게 다인가?''

``그렇습니다.''

사울이 말했다. 진실된 광희로 목소리는 숨소리가 섞여들었다. 지금껏 계속 담아온 듯 간절한 희망으로 가득했다.

``상관없습니다. 그저 같이 있고만 싶습니다.''

사울은 엘사의 손을 자신의 볼로 가져갔다. 내내 엘사의 눈을 바라보고 있었다. 부드럽고 소박한 미소는 어릴 때 본 것과 똑같았다. 엘사는 이 쾌활하고 정중한 소년과 보낸 나날이 떠올랐다. 예절 수업 중간중간에 초콜릿을 몰래 가져다주고 자신이 가르치는 수업 외의 시간에도 따로 찾아와 가르쳐준 이 소년이 말이다. 사울의 피부는 따뜻했다—엘사는 생각했다. 마치 안나처럼.

 ``정말로 개의치 않는다면, 말하지.''

엘사가 말했다. 사울이 움찔하며 물러날 정도로만 손바닥에 한기를 냈다.

``자네는 나를 사랑한 적이 결코 없었다.''

``그렇지 않습니다.''

``그랬다면 대신으로 있는 거로 만족하지 않았겠지.''

엘사는 사울의 미소가 서서히 사라지는 것을 바라보았다. 수수한 모습은 잠깐 모질고 험악한 모습으로 뒤틀렸다. 그러고 사울은 곧 감정을 제어해 표정은 부자연스러운 무표정으로 바뀌었다. 하지만 거짓 모습을 보이려는 것은 아니었다. 엘사는 사울을 오랫동안 알아왔다, 잘 알아왔다. 사울의 가면은 그저 본인의 일부였다. 항상 표정을 잘 숨겨 왔다.

``안나 공주가 저의 대신이 아니었다는 말입니까?''

사울이 물었다.

``안나 공주는 여왕님을 두려워하지 않아서 여왕님이 흥미를 보인 것뿐입니다. 하지만 안나 공주 전에는, 전 항상 곁에 있었고 두려워하지도 않았단—''

``겨우 그것 때문이 아니다.''

``그렇다면 대답해주시죠. 안나 공주는 여왕님을 전혀 알지 못한 멍청한 여자아이란 말입니다. 자격이 없다고요.''

사울은 고개를 흔들었다.

``그리고 어떻게 되갚았는지를 보라고요. 안나 공주는 여왕님을 선택하지 않았단 말입니다.''

``자네 말이 맞는다. 나를 이해하지 못했지, 우리가 생각하는 식으로는.''

엘사가 말했다. 고개를 돌리고 있었다. 생각과 달리 눈이 감겼다.

``가거라, 사울. 그것으로 이미 됐다.''

``됐을 리가요. 왜 알지를 못하시는 겁니까?''

``가라고 했을 것이다.''

밖에서는 지나간 폭풍이 다시 몰아치고 있었다. 창이 흔들릴 정도로 우박이 불어 닥쳤다. 거센 바람은 눈을 휩쓸어 마구 흩날렸다. 모든 것을 흰 베일로 가리고 있었다. 태양도 폭풍 구름에 가려졌다. 빛이 조금도 들지 못했다. 엘사는 소파를 꽉 쥐었다. 자신의 바람과는 달리 소파의 흰 잇에 얼음이 덮였다.

``이 차이를 이해하셨으면 합니다. 전 항상, 항상 여왕님을 선택할—''

``그게 정말로 날 위한 것이겠는가?''

엘사는 홱 돌아서서 다시 사울의 멱살을 낚아챘다.

``내가 왜 안나를 선택했는지 아느냐? 자기는 이해 못 하겠지만 날 사랑했기 때문이다. 다른 이유 없이 오직 내가\ldots\,나 자신이라는 것으로.''

엘사가 가까이 끌어당기자 사울은 눈을 동그랗게 떴다.

사울의 입술에 엘사는 입을 맞추었다.

안나라는 비교 대상이 생기기 전까지는 이해하지도, 알아차리지도 못한 것이었다. 사울은 항상 자신을 바라보았지마는 단 한 번도 바라보아주지는 않았다. 엘사는 이것으로 사울을 낮게 보지는 않았다. 여전히 처음으로 자신에게 처음 눈길을 준 때를 기억하고 있었다. 사울이 이때는 잠깐 자신을 사랑했다고 생각했다. 하지만 자신이 그저 겁에 질린 여자아이에서 벗어나자, 유용하고 가치 있는 아이가 되자 사울의 눈에서도 똑같은 것이 보였다. 다른 모두와 마찬가지로 말이다. 엘사도 똑같이 다른 사람에게서 열등함을 보았다.

우월해야만 했다. 그렇지 않으면 그 누구도 자신을 다시 바라보아주지 않을 것이니 말이다.

안나를 만나기 전까지는. 그리고 처음으로 누군가가 자신을 바라보아주었다.

이것이 안나를 사랑한 이유였다.

그리고 이것이 사울이 자신을 사랑했다고 생각한 이유였다. 사랑하고 있다고 가장하고, 연애라는 꿈으로 자신을 속이고 싶었기 때문이었다. 엘사는 이해했다. 꿈을 꾸는 것으로 싫어할 수는 없었다.

자신의 입맞춤에 반응하지 않아도 엘사는 놀라지 않았다. 둘 다 냉랭했다. 엘사는 물러섰고 사울은 시선을 돌렸다. 매우 멍해 보여 엘사는 사울이 이해했다고 생각했다.

``뭔가를 느낄 수 있느냐?''

엘사가 물었다. 답은 이미 알고 있었다.

``전 제 생각을 보였고, 여왕님도 여왕님의 생각을 보였습니다.''

사울이 말했다. 자리에서 일어선 다음 사울은 자신의 물건을 챙겨 가방에 쑤셔 넣었다. 난생처음으로 걸쇠를 더듬고 있었다.

``자네는 안나와는 전혀 같지 않네.''

엘사가 나지막이 말했다.

``아니어서 유감입니다.''

사울은 문을 향해 걸어갔다. 문고리를 반쯤 돌리고는 멈추었다. 놋쇠 손잡이를 두드리고 있었다. 그러고는 뒤를 돌아보며 말했다.

``\ldots안나 공주가 무슨 선택을 할지는 알고 있었습니까?''

``그렇다.''

엘사가 말했다. 이미 알고 있었다. 안나에게 직접 듣는 것만이 필요했을 뿐이다.

``같이 갈 만큼 같이 있었다. 지금은\ldots\,지금은 우리의 길이 갈라진 것이지.''

``시간이 지나면 다시 바뀌겠지요.''

``인정 못 한 줄 알았다.''

``우리의 길도 똑같이 바뀌리라고 생각합니다.''

사울은 미소를 지었다. 자신감이 약간은 돌아왔다. 은근한 침착함은 그저 허식이 아니었다. 훨씬 확신에 차 있었다.

``안나 공주도 저도 알지 못하지만, 우리는 같은 자리에 있습니다. 여왕님을 사랑하는 법을 배우겠습니다. 언젠가는 저 자신을 모두 내어줄 수 있겠죠.''

엘사는 미소를 짓지 않았지만, 한 번은 미소를 지은 것을 알고 있었다.

``그런다 해도 어느 것도 장담은 못 하네.''

``희망을 품게는 해 주십시오.''

사울이 말했다. 엘사는 고개를 끄덕였다. 이 정도는 해줄 수 있었다.

``제가 말해 봐야 소용은 없겠지만, 이렇게 된 게 정말로 유감입니다. 여왕님과 안나 공주 사이의 일 말입니다.''

``자네를 탓하지는 않네. 알고 있다고 말했잖는가.''

`그래도 날 선택하기를 바랐어.'

사울은 떠났다. 엘사는 홀로 남겨졌다. 정말 피곤했다. 뒤로 몸을 기대었지만, 잠은 전혀 오지 않았다. 잠 없는 밤이 이제 돌아오려는 참이었다. 눈을 감아도 보이는 것은 안나 뿐이었다. 같이 지낸 마지막 몇 분이 눈꺼풀에 새겨졌다. 이 마지막 순간 동안 엘사는 안나가 더는 성물을 품고 있지 않은 것을 알아챘다. 완벽한 기억력으로 그때를 다시 불러오니, 흰 머리칼이 사라진 것이 보였다. 심장의 조각이 어디로 사라졌는지 엘사는 알지 못했다. 마침내 허물어졌을 수도 있다는 생각이 들었다. 구스타프가 말한 대로였다.

엘사는 웃음을 터뜨렸다. 그러고는 울음을.

안나는 자신이 느끼는 것을 느낄 수 없었다.

`그러면 이젠 나 혼자만 상처받고 있다는 거야?'

\textbreak

\forceindent``안나야, 저\ldots\,저녁밥은 문밖에 두마. 뭐라도 좀 먹으렴, 제발.''

안나는 아버지의 발걸음 소리가 멀어지는 것을 들을 수 있었지만, 조금도 움직이지 않았다. 안나는 무릎을 끌어안고 침대의 머리판에 기대어 앉은 채 반대편 벽을 바라보고 있었다. 사흘 내내 이러고 있었다. 올라프는 침대 아래에서 기어 나와 문을 향해 천천히 걸어가고는 문 아래 틈으로 아무도 없는 것을 확인했다. 그러고는 문을 열어 안나의 아버지가 둔 접시를 가져왔다. 안나는 곁눈으로 아무 일도 없는 것을 확인했다.

``저기, 안나!''

올라프가 말했다. 발을 끌며 침대로 돌아와 머리 위로 접시를 안나 곁으로 밀어주었다. 안나는 올라프를 흘겨보았다. 진심 어린 미소를 띠고 있었다.

``이거 정말 좋아할 거야.''

``배 안 고파, 올라프.''

끈질기게 물고 늘어지는 통에 이미 한 번 먹은 터이다. 아그다르가 점심을 두고 가자, 올라프는 안나를 보채 몇 입 먹게 했다. 항상 좋아하는 것들이었다. 그리고 항상 밍밍하고 재를 집어먹는 듯한 맛이었다.

``그래도 계속 힘내야지. 밥도 제대로 안 먹고 힘낼 수 있겠어?''

올라프가 말했다. 담요를 잡고 침대 위로 올라왔다.

``그냥 먹고 싶은 생각이 없어서 그래. 나중에 먹을 거야.''

``알았어. 근데 나중에가 언젠데?''

``내일.''

``어제도 그랬잖아. 그때도 안 배고프다 그랬고\ldots''

때때로 안나는 올라프를 내쫓아버리고 싶었다. 조용히 있는 것이 간절했다. 하지만 이 눈사람은 밥을 먹어야 하네, 일어나서 좀 걸어야 하네, 밖에 나가서 햇빛을 좀 보아야 하네 하며 한시도 들볶지 않는 때가 없었다. 가끔은 아예 올라프를 만들지 말아야 했다는 생각도 들었다. 올라프를 보면 엘사가 떠오를 뿐이었다. 집에 와 홀로 있으니 지난 여섯 달이 꿈만 같이 느껴졌다. 거의 그랬다, 올라프 자체가 이제는 없어진 것에 관한 부정할 수 없는 증거였다.

하지만 원한이 사라지자 피할 수 없는 죄책감이 자리 잡았다.

안나는 한 입 거리를 집어 억지로 입에 넣었다. 우물거리는 움직임, 올라갔다 내려가는 턱—모두 기계적으로 움직이는 느낌이었다. 삼킬 때는 재가 목구멍을 긁는 느낌이 들었다. 상상이겠지만, 속이 울렁거리는 느낌까지도 들었다. 뱃속까지 내려가기도 한참 전에 말이다.

``좀 어때?''

올라프가 물었다.

``차라리 아예 안 깨어나고 싶어.''

안나가 말했다. 그러고는 엘사가 약속한 말이 생각났다. 깨어나면 곁에 있겠다는 것 말이다. 지금 깨어나야 소용없다. 둘 사이의 약속은 전부 사라졌다.

``아.''

올라프는 잠깐만 풀이 죽을 뿐이었다.

``뭐, 그래도 괜찮아. 얼마 되지도 않았잖아. 다시 박차고 일어나면—''

``왜 계속 그런 소리만 하는 건데? 힘내라는 둥, 박차고 일어나야 한다는 둥.''

안나는 담요를 바싹 끌어안았다. 접시가 떨어져 양탄자 위로 음식이 쏟아지는 것은 무시하고 있었다.

``끝난 게 아니라고!''

``뭐가 안 끝났단 건데?''

``엘사! 아직 사랑하고 있잖아.''

안나는 움찔했다. 갑작스러운 움직임에 올라프는 침대에서 튕겨 나갔다. 올라프가 나가떨어진 것을 보고 안나는 사과하려 입을 열고는\ldots\,도로 다물었다. 몸을 돌리고는 담요를 바싹 쥐었다. 손톱이 담요를 뚫고 손바닥에 파고들려는 것이 느껴졌다. 제대로 숨을 쉴 수도 없었다. 안나는 몸을 웅크리고 눈을 질끈 감았다. 멀리 벗어나기를, 엘사 생각에서 멀리 떠나기를 바라고 있었다. 엘사를 생각할 때마다 가슴이 미어지는 것만 같았다.

``똑같이 마음 아파하고 있다고''

올라프가 말했다.

``지금은 아냐, 올라프.''

안나가 말했다. 자기도 모르게 흰 머리칼이 있던 곳을 만지작거리고 있었다.

``지금은 말고. 좀, 나 좀 내버려 둬.''

``여기서도 느낄 수 있다고.''

``좀, 저리 가, 제발.''

엘사 생각을 멈추기에는 이미 늦었다, 사흘 동안 조심스레 억눌렀다 마침내 이 길을 택한 후에는. 엘사는 전혀 상처받지 않았다—안나는 생각했다, 적어도 자기만큼은. 괴로움이 목구멍을 막는 타르와 같이 생겨났다. 진하고 걸쭉한, 말할 수 없이 큰 괴로움이었다. 한편으로는 엘사의 행동이 몹시 싫었다. 안나는 깨달았다. 항상 엘사는 자신을 밀쳐 놓을 방법을 생각해 냈다. 기분이 상할 자격이나 되었을까? 엘사야말로 이 불공평한 선택을 강요하게 한 장본인이었다. 엘사는 항상 무언가를 선택하게 했다.

``어떻게 아빠를 죽게 뒀겠냐고?''

안나가 내뱉었다.

``그걸 바란 건 아니었을걸.''

``진짜?''

안나는 아직도 자신의 목을 겨누던 칼을 잊을 수 없었다. 그러면서도 자신의 얼굴을 어루만지던 애정 넘치는 손길 또한 기억하고 있었다. 모질고 거친 목소리도 기억하고 있었지만, 속삭이는 약속의 말과 노랫소리 또한 잊히지 않았다.

``그러면 왜 아빠를 죽이게 하지 않았는데도 가버린 거냐고?''

``정말 이해 못 하네.''

올라프가 중얼거렸다.

또다시 알 필요는 없었다.

``됐어! 올라프 너 같은 건 필요 없어!''

고함이 입 밖으로 나오자 안나는 울림이 사그라질 때까지 숨을 몰아쉬었다. 자신의 목소리라고는 믿을 수가 없었다. 안나는 이러지 않았다. 무슨 일일까? 언제부터 이토록 증오에 찬 것일까? 안나는 누군가를 내치지 않는 사람이었다, 그들에게 자신이 필요할 때에는, 심지어는 자신이 필요한 이유도 모를 때도. 안나는 다시 이런 사람이 되고 싶었다.

안나는 올라프를 돌아보았다. 올라프는 바닥에 앉아있었다. 소리 없이, 상상 이상으로 의기소침한 모습이었다. 그리고 움직이지 않고 있었다. 사실, 완전히 움직임이 멎어 있었다. 안나는 침대에서 뛰쳐나와 이 눈사람을 향해 달려갔다. 이 엘사의 마지막 일부가 사라져버렸을지도 모른다는 생각에 심장이 멎는 듯했다. 올라프의 눈이 감겨 있었다. 마법이 남김없이 빠져나간 듯했다.

``올라프?''

안나는 손을 뻗었지만, 떨리는 손은 바로 앞에서 멈추었다.

``올라프, 제발. 정말 미안해. 진심으로 그런 건 아냐. 내가 뭘 이해 못 한다는 건데?''

반응이 없었다.

안나는 진정으로 홀로 남겨졌다.

\textbreak

온몸을 타고 흐르는 벼락에 엘사는 자신이 무적처럼 느껴졌다.

엘사는 그 어느 때보다도 자신을 밀어붙이고 있었다. 한기보다 열기가 느껴질 정도였다. 거세게 밀려나오는 번개의 날카로운 소리는 다른 소리를 모두 묻어버리고 있었다. 얼음 조각이 공중으로 떠올라 사방을 감쌌다. 조각 사이사이로 번갯불이 얽어 들어가 뚫을 수 없는 막이 생겨났다. 이처럼 마법이 강력한 적은 없었다.

혹은 한때는 그랬다고 생각했다.

엘사는 방어막을 풀어주었다. 번개와 얼음이 산산이 흩어졌다. 곧바로 몸을 누르는 압력이 풀어졌다. 아무런 절제 없이 온 힘을 눈앞에 쏟아부을 수 있을 정도로 무모한 적이 있었나 하는 생각이 들 정도로 훨씬 가벼워진 느낌이 들었다. 전에는 마법이 주는 무게감 자체가 온 자신을 얽매었다. 지금은 생각을 흐리는 뜨거운 분노도 없이 눈앞이 선명했다.

``정말 너무 자신만만해졌다니까.''

엘사가 중얼거렸다.

엘사는 마법사가 얼마나 손쉽게 자신을 앞질렀는지 기억하고 있었다. 오직 힘의 차이로 이길 수 있던 것이었다. 어릴 때는 자신이 더 쉽게 움직일 수 있던 것을, 더 직관적으로 움직일 수 있던 것을 알고 있었다. 성물을 만들기 전에는 폭넓은 선택권이 있었다. 떨어져 가는 제어력을 가리는 강제력이 자신의 힘이라고 생각하게 되고 만 것이다.

이것이 이제는 칼을 가지고 연습하는 이유이다.

``그 뭐냐, `퓽, 퓽, 퓽'하는 마법 말고 그걸 쓰는 게 좀 이상하긴 해요.''

엘사는 소리가 난 곳을 돌아보았다. 에드문드가 이상하다는 듯 고개를 기울인 채 바라보고 있었다.

``이것저것 할 줄 아는 게 좋지.''

엘사가 말했다. 손에 쥔 얼음 칼을 깨트렸다. 조각이 떨어져 바닥의 눈과 만나고 있었다.

``모든 면에서.''

``뭐, 그렇긴 하죠.''

에드문드는 스카프를 고쳐매고 있었지만, 안으로 들어가려는 움직임은 전혀 없었다. 추운 밖에서 말을 나누는 것에 만족하는 듯했다.

``전 그냥 한 가지만이라도 정말 정말 잘했으면 하지만요.''

엘사는 할 말이 떠오르지 않았다. 구스타프 일 이후로 에드문드와 말을 나눈 적이 없는 것이 겨우 떠올랐다. 에드문드의 목숨을 위협하고 거의 죽어가는 모습을 지켜본 것만을 기억하고 있었다.

``춥지는 않은가?''

``익숙해진 지가 몇 년 전이죠.''

``어, 그렇지.''

``좀 더 자주 말을 나눠야겠어요.''

에드문드가 말했다. 마당의 분수를 가리키며 미소를 지어 보였다. 기다림 없이 에드문드는 눈을 쓸어내고 가장자리에 앉았다.

``그럴지도.''

엘사는 잠깐 주저하고는 옆에 앉았다. 손가락을 가볍게 튕겨 쌓인 눈을 밀어내었다.

``다만 말재주가 좋지는 않아서.''

``말은 할수록 잘하게 되는 법이죠.''

엘사는 흠 하는 소리를 내어 그렇다고 했다. 손을 깍지낀 채 무릎에 놓고 엘사는 에드문드를 바라보고 있었지만, 눈길은 새로 일어난 겨울 폭풍에 가 있었다. 무엇을 해도 이 세찬 바람을 가라앉힐 수가 없었다.

``전의 일은 미안하네.''

``전의\ldots\,일이라뇨?''

``마법사로 오해하고 거의 죽이려 든 거.''

``뭐, 의심받을 만도 했죠.''

에드문드가 말했다. 코웃음을 치고 있었다.

``미안해할 건 없어요, 정말로.''

``그리고 구스타프가 죽어야만 한 것도.''

에드문드는 몸을 떨었다. 오른손은 꽉 주먹을 쥐고 있었다. 숨도 멎어 있었다. 엘사는 숨을 내쉬는 에드문드의 모습을 바라보았다. 눈은 감긴 채 주먹 쥔 손은 겨우겨우 펴는 듯 천천히 움직였다. 눈을 뜨자 에드문드의 얼굴에 비친 웃음이 거짓인 것을 엘사는 바로 알 수 있었다.

``그래야만 했겠죠.''

``둘이 가까웠지.''

엘사가 말했다. 무슨 이유인지 모르겠지마는 사과를 해야겠다는, 무언가를 해주어야 해야겠다는 알 수 없는 느낌이 들었다.

``그리고 나야말로 이런 상황을 만든—''

``아니, 그건\ldots''

에드문드가 말했다. 목소리가 잠깐 공감할 수 있는, 자신이 깊이 알고 있는 것으로 인해 거칠어졌다. 죄책감, 후회—엘사는 생각했다. 에드문드는 고개를 저었다.

``여왕님이 아니었죠.''

``소용 있는 말인지는 모르겠지만, 그래도 정말 미안하네.''

``소용없을 리가요\ldots\,누나.''

에드문드는 이제 부드럽게 미소를 짓고 있었다. 조금 전의 끔찍하도록 억지로 짜낸 미소보다 훨씬 진심이 담긴 작은 미소였다.

``지금도 이렇게 불러도 되나?''

``그러고 싶다면.''

둘 다 분수대에 조용히 앉아있었다. 각자의 생각에 잠겨 있었다. 엘사는 손을 벌려 눈송이를 잡았다. 손바닥에 잠깐 머물다 바람에 실려 날아갔다. 바다를 건너 멀리 아렌델에 닿을지도 모르겠다—엘사는 생각했다. 그리고 안나는 이 눈송이를 보고 자신을 떠올리리라고. 좋은 기억이 떠오를까? 가장 행복한 기억에도 언짢아할까? 에드문드가 헛기침을 했다. 엘사는 에드문드를 바라보았다. 다 안다는 눈으로 자신을 바라보고 있었다.

``이 얘기 하면 안 되는 거 아는데, 안나가 그리운 거야?''

``항상.''

안나 생각 없이 한시도 지나간 적이 없었다. 아무리 시간이 지난다한들 잊히지 않는 것이 있는 법이다. 그리고 한 주도 채 지나지 않았다. 태연한 체하기에도 너무 짧은 시간이다. 몸의 상처는 아물었을지라도 몸 안은, 안나가 떨어져 나간 몸 안은 아직도 찢긴 상처 그대로 벌어져 있었다.

``왜 가버리게 둔 건데?''

에드문드가 물었다.

``내가 너무 많은 걸 바랐어. 안나는 너무 남을 위하고.''

엘사가 말했다. 엘사는 자신이 결점이 있는 것을 알고 있었다. 마음이 없어진 것뿐만이 아니다. 인간을 넘어서기로 정했을 때 엘사는 또 다른 것이 없어진 것을 알았다. 바로 정체성이다. 이제는 자신이 무엇이였는지도 거의 이해하지 못하고 있었다.

안나가 자신의 마음이 되어준다고 했을 때, 엘사는 마음 깊은 곳에서 이미 알고 있었다. 실패할 수밖에 없는 것이었다. 안나는 안나 자신일 뿐이다. 자신의 마음이 되어줄 수가 없다. 하지만 안나는 자신이 필요한 무엇이든 되어주려 했다. 자신이 행복할 수 있다면 자기 자신의 행복도 집어 던질 수 있었다. 엘사는 이 정도로 이기적이 되고 싶지 않았다. 자신이 망가지는 것은 자기만의 책임이다. 엘사는 자신을 위한다고 안나가 똑같은 일을 하게 둘 수 없었다.

``안나는 그런 식이지. 정말 강한 애야.''

에드문드가 말했다. 부츠로 눈 쌓인 바닥을 비비고 있었다.

``솔직히 말하면 난\ldots\,내가 그렇게 강했으면 해. 안나처럼 되고 싶다고. 누나도 그래?''

``그래.''

``뭐 때문에 주저하는 건데?''

``나 자신을 제대로 알지 못하겠어.''

엘사가 말했다. 마법을 지니고 있지마는 안나와 비교하면 정말로 나약하게 느껴졌다. 안나는 믿음이 있었다. 자기 자신과 선함, 신념을 믿고 있었다. 결코 흔들리지 않은 믿음이었다. 그리고 자신은 목적과 감정 없이 어둠을 헤치고 있었다.

``안 그런 사람이 어딨겠어?''

엘사는 굳었다.

딱 한 사람은 그렇지 않을 것이다.

``뭐, 난 들어가 봐야겠어. 곰곰이 잘 생각해 봐.''

에드문드는 일어서서 바지를 털고는 손인사를 하고 사라졌다. 엘사는 겨우 고개를 끄덕일 뿐이었다. 보는 사람 없이 성 안에 남겨질 때까지 기다리고는 홱 돌아 반대편을 바라보았다.

딱 한 사람만이 자신을 도와줄 수 있을 것이다. 엘사는 자신의 심장이 필요했다. 다시금 감정을 느낄 수 있게 되어야 했다. 자신이 누구인지, 무엇을 해야 하는지, 그래서 짐이 되지 않고 안나를 사랑할 수 있도록 말이다.

엘사는 탑을 향해 바라보았다.

마르쿠스를 깨울 때였다.

