

\chapter[외전4. 이루어질 뻔한 만남][외전 4\hspace*{.5em}이루어질 뻔한 만남]{외전 4 \ 이루어질 뻔한 만남}



\begin{quote}

\small 34장의 앞 이야기입니다. 엘사와 아그다르 왕, 그리고 누군가와의 첫 만남을 다루고 있습니다.\sourceatright{역자}

\end{quote} %force indent

``안나야, 밤도 되기 전에 드레스 더럽히지 마려무나.''

아그다르 왕이 나무라며 말했다.

``말했다시피, 대신들이 많이 올 거라고. 그리고—''

``알아요. 기다릴 수가 없다고요. 정말 신날 거예요!''

드레스를 꽉 쥐고 자락을 들어 올린 채, 여덟 살 난 아렌델의 안나 공주는 자신의 머리를 만져주려는 안쓰러운 하녀들 주변을 돌아다니고 있었다. 드레스를 제대로 입힐 수 있게 가만히 서 있게 한 것부터도 운에 따른 일이었다. 이런 왈가닥을 가만히 앉혀 머리를 똑바로 만질 수 있게 하는 것도 기적이 필요한 일이었다. 이러는 동안 아그다르 왕은 방의 구석에서 팔짱을 낀 채 지켜보고 있었다. 꾸중하고 있지마는 입꼬리는 살짝 올라가 있었고 목소리도 부드러웠다.

하녀들의 애원과 안나의 고집이 한바탕 더 이어지고 나서야 아그다르 왕은 방 안을 여전히 뛰어다니는 안나에게 다가가 팔로 안아 들었다. 왕이 눈알을 굴리고 거울 앞에 내려놓아도 안나는 웃을 뿐이었다. 안나는 이 작은 대결을 계속할까도 했지만, 아그다르 왕은 계속 안나를 주시하고 있었다.

``시작할 때까지 얼마 안 남았단다. 손님들이 곧 도착할 거야.''

``그리고 드디어 새로운 사람을 만나게 되는 거고요!''

안나는 기쁜 목소리로 소리쳤다. 허공에 주먹을 지르며 미용사 한 명을 거의 칠 뻔했다. 안나는 바로 주먹을 거두고 사과했지만, 미소는 얼굴을 떠날 줄을 몰랐다.

``왕자나 공주 중에 재수 없는 사람이 없어야 할 텐데요. 제 또래들이에요?''

``그런 왕자나 공주도 있을 거야.''

안나의 미소는 더욱더 커졌다.

친구를 사귈 수도 있을 것이다.

\textbreak

``드러내.''

엘사는 등을 똑바로 편 채 배 갑판의 정중앙에서 완벽히 균형을 잡으며 서 있었다. 천천히 숨을 들이마시며 사방의 바다 냄새를 맡고 있었다. 눈을 감은 채 배의 흔들림과 부딪히는 파도 소리를 느끼고 있었다. 이 느낌은 점점 커지더니 온몸을 흔들고 귀를 때렸다. 엘사는 전혀 신경을 쓰지 않았다. 방해할 수 있는 것은 없었다.

``자유롭게 해.''

엘사는 눈을 팍 떴다. 얼음과 같은 파란색 눈에는 단호한 빛이 비쳤다.

자연적이지 않은 겨울 폭풍이 주변에 터져 나왔다. 매 초가 지날 때마다 힘이 쌓이더니 곧 눈 태풍의 눈에 엘사가 서 있는 모습이 되었다. 엘사는 앞으로 팔을 뻗었다. 눈과 얼음이 거대한 하나의 덩어리가 되어 앞으로 돌진했다. 돛대에 부딪히기 직전에 엘사는 손을 펴고 손바닥을 위로 올렸다. 이 한기의 물결은 위로 솟아올라 돛대를 따라 올라가 꼭대기에 닿고는 둘로 갈라져 밖으로 터져 나갔다. 엘사는 주먹을 쥐었다. 바다를 향해 곤두박질치며 엘사의 눈은 수면을 때리고 거의 느려지지도 않은 채 깊이 빠져 들어갔다.

집중하며 엘사는 몸에 힘을 빼고 손을 폈다. 자신의 마법이 깊은 물 속에 흩어지면서 넘실거리는 바다를 확인하고 있었다. 엘사가 하려는 것은 어리석은 짓이었다. 절대 성공하지 못할 일이다. 그 어떤 `인간'도 이런 일을 해낼 수는—

``한계는 없어.''

마지막 남은 것까지 쥐어짜 내 자신의 마법이 자유로이 날뛰게 하며 엘사는 성공하기를 바라고 있었다. 다시 폭풍이 주변을 에워쌌지만, 이번에는 달랐다. 바람은 금방이라도 돛을 찢을 듯 마구 불어 닥쳤고, 파도는 더욱 거세어졌다. 배를 집어삼키려 하고 있었다. 엘사는 전혀 신경을 쓰지 않았다. 눈을 감고는 힘이 점점 쌓이게 했다. 취할 듯한 힘이 혈관을 메우는 것을 제하고는 모든 감각을 차단하고 있었다. 그리고 자신도 모르게 팔을 벌리고 어두워지는 하늘을 향해 고개를 치켜들었다.

얼음이 바다를 메웠다.

자신의 마법이 이미 퍼져있는 바다 위에 서서 엘사는 한기가 퍼지도록, 바다가 얼도록, 이 불가능을 극복하도록 마음을 먹으며 이 바다를 정복해 자신의 것으로 만들려 했다. 끝없는 분전이었다. 한기가 펴져도 혹독한 바다는 엘사의 영역과 싸우며 얼음을 부수고 눈을 녹였다. 하지만 엘사는 계속 자신의 힘을 물 깊숙이 쏟아 부었다. 얼음이 생겨나고, 깨어지고, 다시 생겨나고 깨어지는 것을 반복했다, 깊은 곳에 압력이 쌓이고 쌓일 때까지.

제어력을 잃을 때까지.

모든 것이 한 번에 무너졌다. 바닷속에서의 폭발에 두 물 벽이 위로 솟아올랐다. 온 배가 작아 보일 정도였다. 그러고 나서 바다는 곧 잔잔해졌다. 엘사는 남은 힘을 풀어주었다. 하늘은 개고 파도는 사그라졌다. 남은 것은 비처럼 후드득 떨어지는 바닷물이었다.

``정말 놀라웠다. 하지만 지금까지의 것 중 가장 대담한 목표였다. 바다를 가르는 것은 너에게도 너무 과한 일일지도 모르겠구나.''

엘사는 몸을 돌리고 고개를 숙였다. 다시 고개를 들자 빤히 즐거워하는 빛으로 미소를 짓는 마르쿠스가 보였다. 엘사는 이 모습에 자신을 방어할 수밖에 없었다.

``얼리려 했습니다. 장차 해낼 수 있을 것으로 생각합니다.''

``마음에 품은 일은 무엇이든 해낼 수 있을 것이다, 엘사야.''

마르쿠스는 가까이 다가와 팔을 벌려 보였다. 엘사의 표정은 밝아졌다. 주저 없이 바로 허리를 껴안았다. 마르쿠스가 애정을 보이는 일은 드물었다. 엘사에게만 보이는 애정이었다.

``자랑스럽게 해 드릴 거예요.''

마르쿠스는 물러나 다시 다정한 미소를 보였지만, 말을 꺼내기 전에 얼굴이 일그러졌다. 몸을 돌려 마르쿠스는 몸을 굽히며 기침을 하기 시작했다.

``괜찮아, 괜찮단다.''

마르쿠스는 엘사를 안심시켰지만, 엘사는 걱정하는 얼굴을 하는 모양이었다. 마르쿠스는 기침을 내뱉고 덧붙였다.

``걱정하지 마라. 곧 도착할 테니 또 놀라운 일은 벌이지 말아 주겠니?''

마르쿠스는 휴식을 위해 자신의 방으로 들어갔다. 엘사는 갑판에 홀로 남았다. 난간으로 걸어가 엘사는 조용히 바다를 바라보았다. 달리 할 일이 없으니 아렌델에 가기 싫다는 생각만이 가득했다. 바다를 얼리려 한 이유 중 하나도 발목을 잡기 위한 것이었다. 마르쿠스가 다른 왕자를 대신 데려가거나, 사울도 같이 왔기를 바랐다. 하지만 엘사는 생각에 잠긴 채 혼자 있었다.

아렌델에서는 매년 낙성제\footnote{落星祭}가 열렸다. 이때에는 이를 기념하고 동맹을 강화하기 위해 특별한 행사가 열렸다. 엘사는 낙성제를 싫어했다. 엘사는 바로 이날에 태어났다. 축제가 벌어지는 동안 지옥과도 같던 집에서 사랑과 관심을 바라던 자신이 떠오를 뿐이었다.

그저 이를 안고 살아가야 할 뿐이었다.

\textbreak

\forceindent``엘사를 소개해 주었던가요?''

엘사는 앞으로 다가가 미소를 지었다. 마르쿠스에게와는 다르게 묵례를 하고 있었다.

``만나서 영광입니다, 폐하.''

``내가 다 기쁘다는 말을 하고 싶구나.''

아그다르 왕이 말했다.

연회가 무르익어 가는 시끌벅적한 모습에 대한 말이었다. 연회장에서 대신들이 춤추는 동안 한쪽에서 곡을 연주하는 연주자들과, 은 쟁반에서 조심스럽게 고른 간식들을 권하며 이리저리 돌아다니는 하인들의 모습에 관한 말이었다. 엘사는 거의 눈길도 주지 않았다. 위험을 피할 정도로만—일단 위험이 다가온다면은—주의를 기울일 뿐이었다. 도착한 때부터 엘사의 눈길은 아그다르 왕에게만 가 있었다.

``하지만 마르쿠스, 당황스러운 것이, 구스타프는 어디에 있습니까?''

아그다르가 물었다.

``불행히도 구스타프는 제 기대와는 어긋나게 됐습니다. 이제는 엘사가 제 후계자입니다. 엘사는 차차 알아가게 될 겁니다.''

``아, 모\ldots\,몰랐군요. 구스타프에게는 수치스러운 일이겠지만, 이유가 있어서 그러셨겠죠.''

아그다르는 엘사를 더 찬찬히 살펴보았다. 엘사는 조금도 물러서지 않고 차분히 눈을 마주쳤다.

``몇 살이니, 엘사야?''

``열한 살입니다, 폐하.''

엘사는 예를 보이며 고개를 숙였다.

``미래에도 폐하의 왕국과 제 왕국이 좋은 관계를 유지하기를 바랍니다.''

아그다르는 미소를 지었다. 엘사는 아그다르의 미소가 진심이 실린 다정한 미소인 것을 알아채고 놀랐다. 단지 아주 뛰어난 거짓말쟁이여서 그러리라.

``나도 그러기를 바라는구나. 어린 것치고 꽤 조숙하구나, 엘사야. 정말이지 놀랍다.''

``엘사는 저의 더없는 위업이 될 겁니다. 틀림없이 서던 제도를 새로운 시대로 이끌 겁니다.''

마르쿠스가 말했다.

``안나한테도 똑같은 말을 할 수만 있다면. 정말 걱정됩니다.''

아그다르는 고개를 젓고 웃음을 터뜨렸다.

``이미 여기 와 있어야 하지만, 꽤나 난장판을 만들어서 좀 늦을 겁니다. 모든 손님이 도착할 때쯤엔 준비돼 있기를 바라야죠.''

``그렇게 되겠죠.''

마르쿠스는 엘사를 보며 말했다.

``가서 좀 섞여 보려무나. 모든 게 낯설 테지만, 좋은 경험이란다.''

``물론이죠. 만나서 반가웠습니다, 폐하.''

엘사는 무엇 때문에 나가 있으라는 것인지 알아듣고 자기 할 일을 하러 갔다. 대신들과 섞이는 일은 필요치 않았다. 자신을 이들 수준으로 낮출 이유도 없거니와 마르쿠스도 도착하기 전에 이를 분명히 했다. 엘사는 단지 지켜보기만 했다. 이곳의 사람들은 자신들을 터놓고 있었다. 이들은 큰 목소리로 자유롭게 말을 나누었고 이곳의 부산한 분위기는 서던 제도의 효율적인 모습과는 극명히 대비되었다.

엘사는 물잔을 들어 조심스럽게 들이켰다. 계속 안을 바라보고 있었다. 물론 거의 모두가 자신보다 나이가 많았다. 하지만 대부분은 마르쿠스보다 나이가 적었다. 모두 뚱뚱하고 오만했고, 당연히도 의지가 탄탄하지 못하고 마르쿠스의 권위적인 모습도 없었다. 엘사는 이들도 같은 왕족이라는 것을 좀체 믿을 수 없었다. 이따금 자기 나잇대의 사람을 마주쳤다. 코로나는 몇 년 전에 공주를 잃었지만, 다른 많은 왕국이 참석해 있었다. 엘사는 새로운 사람이었으니 이들은 셈을 하며 엘사를 바라보았다. 누구인지 궁금해하며 시간을 쓸 가치가 있는지, 동맹인지 적인지 따지고 있었다. 엘사는 이들을 무시했다.

트럼펫 소리가 들리자 엘사는 앞쪽을 향해 바라보았다.

``아렌델의 아그다르 왕!''

아그다르는 안으로 들어와 왕좌 앞에 섰다. 침착했고, 엘사가 보아 온 다른 사람들과는 많이 달랐다. 마지못해 엘사는 아렌델의 왕이 최소한 바보가 아니라는 것을 인정할 수밖에—

``아렌델의 안나 공주입니다!''

엘사는 눈을 끔뻑였다. 넘쳐흐르는 열광으로 빨간 머리 소녀가 안으로 들이닥쳤다. 힘껏 손을 흔들며 완전히 잘못된 자리에 멈추어 섰다. 엘사는 이 안나가 예절을 더 배워야겠다는 생각이 들었다. 누군가가 안나를 자신의 아버지 옆으로 데려갔다. 그러고 나서 아그다르는 한바탕 연설을 늘어놓았다. 엘사는 안나의 크게 뜬 눈과 너무나도 뻔한 호기심이 어린 눈빛에 정신을 쏙 빼앗겼다. 엘사는 사람들 사이에 끼어 있었으니 이 공주가 찡그린 자신의 얼굴을 보지 못했으리라 생각했다. 그래도 왕족은 자기 생각을 그렇게 쉽게 드러내서는 안 된다고 말해주고 싶었다. 솔직히 엘사는 그렇게 크게 웃는 것과는 거리가 멀었다. 게다가 특히나 지루한 부분에서 하품을 한 후에 안나는 연설이 계속되는 와중에 정말로 사람들에게 손을 흔들어 보였다. 엘사는 헛기침을 하고 시선을 돌렸다. 입꼬리가 올라가려는 것을 참고 있었다.

이렇게 호의적인 모습을 보는 것이 정말로 조금은 기운이 났다.

교육을 받기 전의 자신의 모습이 살짝 떠올랐다.

``\ldots그리고 평화가 이어지기를 빕니다. 그리고 우리의 아이들이 위협과 불화에 방해받지 않고 삶을 이어가도록 관용을 빕니다.''

좋던 기분은 한순간에 날아갔다.

엘사는 연회장에서 바로 자신의 마법을 폭발시키려는 충동을 참아내었다. 아그다르가 정말로 평화와 관용을 설파하려는 것이라면 이곳보다는 자신의 왕국에부터 이를 퍼뜨려야 할 것이다. 아렌델의 누구도 자신을 받아주지 않았고, 자신의 고통을 덜어주지도 않았다. 분명히 `누군가'는 무슨 일이 일어나고 있는지 알았을 것이다. 그러면서도 자신을 구해준 사람은 마르쿠스였다. 엘사는 깊게 숨을 들이마셨다. 떠나도 좋은 때가 되자마자 사람들을 헤치고 나와 마당으로 나갔다. 엘사는 마지막으로 안나를 바라보았지만, 이 빨간머리가 알아채기 전에 엘사는 밖으로 나갔다.

아렌델에는 이미 진절머리가 났다. 그리고 이 썩어빠진 곳에 있는 모든 사람도.

\textbreak

연회가 끝나고 꽤 시간이 지난 후에 안나는 언짢은 기분을 간직한 채 무거운 발걸음으로 자신의 방으로 돌아갔다. 다른 왕자나 공주와 이야기를 나누기는 했지만, 안나는 이들을 거의 좋아하지 않았다. 모두 성품이 못나고, 안나 자신의 관점으로는 자기들 생각으로 가득했다. 전체적으로 이날 밤은 매우 실망이었다. 물론 음악 연주는 재미있었고 초콜릿은 매우 맛있었지만, 안나는 사람을 만나고 싶었다. 

안나는 계단을 올라가 곧 자신의 방으로 가는 복도를 느긋하게 걸어갔지만, 공교롭게도 창밖을 바라보았다. 관심을 끄는 것을 딱히 없었다. 다른 때처럼 그냥 지나칠 수도 있었지만, 무언가에 이끌린 듯 창밖을 내다보았다. 처음 눈이 간 곳은 쏟아질 듯 밝게 빛나는 보름달도 아니고 구름 없는 하늘에 박힌 별도 아니었다. 안나는 위가 아닌 아래를 보았다. 분수대 앞에 금발 머리 소녀가 앉아 있었다.

너무도 슬퍼 보였다.

안나는 이렇게 슬퍼하는 사람을 본 적이 없었다. 외로워 보였다. 홀로 앉아 물속에 손가락을 넣어 휘젓고는 물결이 이는 물에 일렁이는 자신의 모습을 바라보았다. 어린 나이였지마는 안나는 자신의 도움이 필요한 것을 느꼈다. 안나는 최대한 빠르게 계단을 내려갔다. 갑옷 장식에 넘어질 뻔하면서도 힘껏 달려 마당으로 향하는 문밖으로 나왔다.

``얘! 난 안나라고—''

아무도 없었다.

안나는 그 아이의 이름도 알지 못했다.

\textbreak

떠다니는 말로는 전쟁이 하루가 채 되지 않아 끝났다고 했다.

