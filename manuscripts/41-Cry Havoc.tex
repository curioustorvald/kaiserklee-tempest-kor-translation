<FocusWriter 사용 바람>


\chapter[41장  ~~][41장\hspace*{.5em}~~]{41장 \ ~~}



전쟁이 다가오고 있었다.

온 해군이 소집되었고 모든 군인이 준비되었다. 모든 노예가 한데 모였고 모든 건장한 사람들이 징집되었다. 모든 왕자는 갑옷을 차려 입었고 엘사도 이들 중에 서 있었다. 이런 날이 올 줄은 생각지도 못했다. 마르쿠스가 이전 자신의 자리를 차지하고 있었다. 엘사는 기꺼이 지휘권을 넘겼다. 이제는 다른 이와 같이 병사일 뿐이었다, 항상 그래야 했던 대로.

마르쿠스는 연설을 시작했다.

``나의 아들들이여, 우리는 위대한 승리를 쟁취하였다. 전쟁터에서의 승리가 아닌 우리 마음속이라는 진정한 전쟁터에서의 승리다. 너무도 오랫동안 우리 백성들의 마음은 나약해지고 퇴폐해졌다. 이것도 이제는 끝이다, 그리고 내가 살아있는 동안은, 너희가 합심하여 서 있는 동안은 다시는 되풀이되지 않을 것이다.''

갑옷을 입은 채 마르쿠스는 <struck an imposing figure.> 투구의 Y자 모양 구멍 너머로 기대에 차 빛나는 파란 눈이 보였다. 자신의 아들들 앞에서 열정적으로 연설하는 중에도 엘사는 벌써 전투를 바라는 눈빛을 볼 수 있었다. 채울 수 없이 큰 <살육/carnage>을 향한 욕망이었다. 그렇지만 다른 면도 볼 수가 있었다. 급한 마음이었다. 자신만의 것뿐만 아니라 에드문드를 향한 마음이었다.

거울의 마지막 조각을 되찾아올 때였다.

``몇몇은 주저하고 있을 것이다. 몇몇은 자비의 마음을 품고 있을 것이다. 흔들리지 마라, 가책하지 마라. 너희의 신념이 흔들리게 하지 마라. 이것이 자연의 법칙이 아니냐? 강자가 약자를 지배하는 것이야말로, 우리의 영광이 부활의 불길 위에 빛나는 것이야말로 섭리에 따른 행동일 뿐이다. 세상은 항상 이렇게 진화해왔다. <disservice>, <겁을 내는 것이/cowardice> 우리 왕국의, 우리 백성들의 성장을 방해하는 것이 될 것이다. 힘은 취하라고 존재하는 것이다. 인간이 질서를 바로잡지 않는다면 그 무엇이 인간과 짐승을 가를 수 있으랴!''

``반드시 우리의 손으로 이루어져야 한다. 우리는 정복자다, 우리는 구원자다, 곧 전쟁은 시작될 것이다. 그래도 나는 한마디 덧붙이려 한다. 우리야말로 영광의 짐을 지고 버틸 꿋꿋함이 있는 자들이요, 마땅히 우리의 것을 취할 대담함이 있는 자들이요, 더 큰 선을 위해 버티어 싸울 수 있는 힘이 있는 자들이다.''

``이제 어둠이 드리워지려 한다. 세상이 잠에 빠지며 장막으로 덮이려 한다. 이 잔잔한 어둠 속에 세상은 퇴보한다. 지금까지 우리는 눈꼽만큼의 욕망도 피어오르는 것을 보지 못했다. 평화는 침체를 가져올 뿐이다. 우리는 싸움을 통해 우리 자신을 확립할 것이다! 평화 따위처럼 서서히 좀먹는 것들 앞에서 내게 `정도껏'이란 건 없다. 내게 굴복이란 없다. 내게 애매한 언사란 건 없다. 내게 변명이란 건 없다. 나는 한 걸음이라도 물러서지 않는다. 이제 우리는 승리를 향해 전진한다!''

마르쿠스는 알현실을 걸어 나갔다. 투구의 흰색과 붉은색이 섞인 술은 걸음에 맞추어 흔들리고 있었다. 칼 두 자루가 허리춤에 꽂혀 있었다. 엘사는 따라 나갔다. 마지막 여정을 착수할 준비가 되어 있었다.

모든 것이 끝나면 안나와 같이 있을 것이다.

\textbreak

트롤들에게 `놀랐다'는 말로는 충분치 않았다.

안나는 이들에 관한 옛이야기를 들으며 자라왔다. 이들이 살아 움직이는 것은 꿈에서나 볼 법한 일이었다. <It was told to children that ...> 배게 밑에 빠진 이를 두면 착한 트롤들이 이를 가져가고 선물을 준다는 둥, 착하게 지내면 동지절에 선물과 새해 복을 준다는 둥 하는 이야기들이었다. 물론 나쁜 트롤들은 나쁜 아이를 잡아가 엄지발가락을 잡아먹는다고 했다.

``진짜로 엄지발가락을 잡아먹는 건 아니죠?''

``그럴 리가.''

패비가 말했다. 껄껄 웃고 있었다. 그러고는 다시 말을 꺼냈다.

``음\ldots\,우리가 무얼 먹고 지내는지는 잘 모르겠구나. 사실 뭘 먹기나 하는지도 모르겠구나. 정말 희한한 일일세\ldots''

무례한 말을 했다고 열댓 번을 사과하고 그냥 많이 놀라서 그랬다 말하고 `당연히' 만나게 되어서 아주 반갑다고 주절댄 뒤에야 안나는 어느 정도는 이 트롤들에게 익숙해졌다. 대부분은 자기만큼이나 호기심에 찬 모습이었다. 안나는 마음이 놓였다.

껄껄 웃으며 패비는 다는 트롤들을 소개했다. 안나는 꿇어 앉아 트롤들 하나 하나를 살펴보며 각각의 특징들을 머리에 넣으려 했다. 패비는 망토를 두르고 머리에 무언가를 쓰고 있었고, 불다는 분홍색 수정이 있는 목걸이를 했다. 다른 트롤들은 <각기둥/prismatic> 같은<substitute:Strange> 것이 튀어 나와 있었다. 개중 하나는 밝은 붉은색 수정을 품고 있었다. 다들 똑같이 큰 미소와 똑같이 작은 키에 똑같이 단단한 체구였지만 이들의 차이는 눈을 떼지 못할 만큼 분명했다. 안나는 모두를 바라보며 이들의 묘한 점들을 놓치지 않으려 했다.

``모든 게 정말 신기해요.''

안나는 웃으며 말했다.

<"It's our pleasure!" Bulda said, holding onto Anna's hands and examining her with a rather keen eye. It had taken some smooth talking to get Bulda to stop looking at her teeth. Anna couldn't decide if she was flattered. "We don't get visitors often, and none are as polite as you!">

``안나 좀 내버려 두게.''

패비가 말했다. 또다시 맑은 눈을 가지고 말을 쏟아내기 전에 불다를 끌어내었다.

``긴히 할 말이 있어.''

자신의 이름을 말하는 것에 안나는 패비를 향해 뒤돌아섰다. 생각해보면, 전에도 자신의 이름을 부르며 인사했다.

``어떻게 제 이름을 알죠?''

``거울의 아이와 같이 그 아이의 집에 왔을 때 우리는 모든 것을 보고 있었단다.''

패비가 말했다. 왜 `거울의 아이'라고 하는지 안나는 알지 못했지만, 누구를 일컫는지는 알 수 있었다. 엘사 이야기에 몇몇 트롤들이 중얼거리는 것을 안나는 눈치채지 않을 수 없었다.

``오랫동안 그곳을 지켜보고 있었지.''

뭐, 집이 어찌 그대로인지에 관한 수수께끼가 풀렸다.

``잠깐만요, 왜 지켜보시는 건데요?''

<"It's something of a memorial," Pabbie said. "Who was it that led you here?">

``그러면 그 아이가 진짜였다는 거야?''

안나는 고개를 가로저었다.

``그건 말도 안 돼요. 여왕님은 서던 제도에 있고 여덟 살도 아니라고요. 진짜예요. 잠깐만요! 절 데려오려고 할아버지께서 만들어낸 건가요?''

``아니, 내가 한 게 아니다. <Memories are very real things,>''

진중한 목소리에 안나는 자기도 모르게 진정이 되었다.

``커다란 비극이 생긴 곳에는 대개\ldots\,자국이 남는다. 고통의 메아리 같은 것이지. 아무나 볼 수 있는 게 아니지만, 안나 너는 매우 중요하다.''

``저\ldots\,전 잘 모르겠어요. 죄송해요. 이해하기엔 너무 벅차서요.''

미소를 지으며 웃는 엘사의 모습에 안나는 그 무너진 집에 무슨 일이 있었는지 거의 잊을 뻔했다. 왜 이런 엘사의 기억이 남아있는지를 알아낸 것만도 겨울 비에 흠뻑 젖는 듯 혹독한 깨달음이었다. 거기에다 트롤들이 진짜라는 것과 엘사의 집 터를 지켜보고 있다는 것은? 왜 굳이 그러는 것일까? 왜 엘사를 신경쓰는 것일까?

``괜찮다, 이해한단다.''

안나를 자리에 앉히며 패비가 말했다. 안나는 바닥에 완전히 주저앉았다. 다리를 감싸 안고 있었다.

``그래도 우리가 도움을 줄 수는 있을 거다.''

``알았어요.''

안나는 숨을 들이마셨다.

``왜 여왕님을 신경쓰는 건가요?''

``우린 거울과 같이 태어났다. 거울이 무엇인지는 알겠지.''

고개를 끄덕이는 것을 보고 패비는 말을 계속했다.

``그리고 우리는 엘사를 거울의 아이라고 부른단다. 그 거울은 오로지 그 아이의 것이고 그 아이의 마음만이 제어할 수 있단다. 그 아이에 관한 일은 우리의 일이기도 하지.''

``그러면 눈사람 만들고 있을 때 내려오셔도 됐잖아요?''

``우린 이미 만났다.''

패비가 말했다. 언제냐고 묻기 전에 패비는 말을 이었다.

``너희 둘이 그 아이의 옛 집에 오기 며칠 전이었다. 우리가 지키는 거울 조각을 가지러 온 게지.''

시간이 조금 지나고서야 안나는 완전히 이해했다. 처음 들었을 때는 아주 오래 전에 만난 걸로 생각했다, 어릴 때라든지. 자신의 말대로 이곳으로 온 때로는, 일주일도 안 됐다고는 생각지 못했다. 안나는 입술을 깨물었다. 자기 때문에 엘사가 아렌델로 온 게 아니라는 것일까? 거울 조각 때문에 온 것일까? 그렇다면 자기 때문이 전혀 아니었다. 바보같은 자신의 모습에 안나는 거의 웃음이 나올 지경이었다. 다른 이유가 있으리라고는 생각도 못한 자신의 모습이.

``그랬겠죠.''

안나가 말했다.

``한 마디 덧붙이자면, 안나 네가 생각하는 대로는 아주 아닐 거다.''

패비는 미소를 보이며 어깨를 으쓱했다. 안나는 기분을 낫게 해주려는 걸로 생각했다.

``거울 때문에 온 것만은 아니란다. 너를 위해 온 것이기도 하단다.''

``그 아이 같은 사람을 우리는 망령이라고 부른단다. 더는 심장이 없으니까.''

불다가 말했다.

``여왕님이 성물을 만들었으니까요.''

안나가 말했다.

엘사가 망령으로 불리는 것은 생각하기도 힘들었다. 무슨 마귀라도 되는 양 말이다. 망령 이야기에 안나는 사람들의 영혼을 빨아들이는 망토 쓴 괴물이 떠올랐지, 아름답고 우아한 엘사의 모습은 떠오르지 않았다. 마법사야말로 망령에 훨씬 더 어울렸다.

그러면서도 안나는 왜 트롤들이 엘사를 두고 그렇게 말하는지 이해할 수 있었다.

얼굴은 비인간적인 혐오로 뒤틀리고 피칠갑이 되어있던 엘사의 모습은 지금도 머릿속에서 떠나지 않았다. 안나는 이것을 싫어했다. 아주 잠깐은.

이를 싫어했던 것인지 무서워했던 것인지 안나는 알 수 없었다.

``사랑하는 사람이 완벽하지 않다는 걸 인정하는 건 쉽지 않단다. 그리고 자기 자신이 완벽하지 않다는 건 더욱이 그렇지.''

불다는 고개를 끄덕이며 말했다.

``쉬운 결심은 아니었을 거야. 그 아이는 완전해지기를 원했어, 너를 위해서. 다시 되돌릴 방법을 물었다니까. <Not really healthy, but sort of admirable too.>''

``그러니까\ldots\,다시 심장을 되찾을 방법을 물었다고요? 저만을 위해서요\ldots\,?''

엘사가 떠나간 자리를 억울함과 원통함으로 채우는 것은 쉬운 일이었다. 하지만 마음 속 깊은 곳에서는 엘사가 자신을 신경쓰는 것을 알고 있었다. 엘사는 자신의 이러한 면을 싫어했다. 자신의 마음을 약점이라 하며 심장을 파내버릴 정도로 그 기억 자체를 혐오했다. 하지만 안나 자신을 위해서는 다시금 기꺼이 받아들이려 하고 있었다.

``이제 의심이 가시느냐, 안나야?''

``솔직히\ldots\,의문만 더 늘어나요.''

``의문은 좋은 거지. 더 듣기를 바란다는 표시니까.''

패비가 말했다. 안나는 미소를 지었다.

``여기 왔던 다른 아이도 똑같았단다. 너도 아는 사람일 거다. 서던 제도의 구스타프였단다.''

안나의 미소가 가셨다. <the compliment having turned into something much more like an insult.> <안나에게\append?> 구스타프는 결코 좋아하고 싶지 않은 사람이었다.

``네, 알아요. 아주 정신 나간 사람이었죠.''

패비는 얼굴을 찡그렸다.

``그새 무슨 일이 있었는지는 모르겠지만, 내가 본 구스타프는 전혀 그렇지 않았다. 더 말해줄 수 있겠느냐?''

``뭐, 자기 아빠의 자리를 빼앗으려 하고—''

``실패했지.''

패비가 말했다. 말이 이어지자 안나의 눈이 휘둥그레졌다.

``맞아. 더 큰 해가 생기기 전에 아버지의 미친 짓을 멈추게 하라고 했지.''

``죄송한데 제가 잘못 들은 것 같아서요.''

안나가 말했다. 분노가 신물처럼 목구멍을 타고 올라왔다.

``할아버지께서 자기 아빠한테 반역을 일으키라고 말했다는 거예요?''

``모든 아버지가 네 아버지와 같지는 않단다, 안나야. 꼭 막아야만 하는 사람이 있는 법이란다. 구스타프는 그걸 깨닫고 내게 조언을 구하러 왔단다.''

패비는 부드러운 말씨로 말했다.

``그게 어떻게 그런데요?''

패비는 잠깐 안나를 바라보았다. 천천히, 아무런 의심이 들지 않게 명확히 패비는 말했다.

``구스타프가 여기 왔을때 마르쿠스는 절벽 아래에 있었다.''

``절벽\ldots\,아래—!''

안나는 말을 끊었다. 그랬다기보다는 밀려오는 깨달음에 말하는 것은 고사하고 숨도 쉴 수 없었다. 따라갈 수도 없을 정도로 머리가 돌기 시작했다. 끊긴 사건들을 짜맞추고 있었다. 어지럽고 혼란스러운 깨달음에 안나는 공포에 질렸다.

절벽 아래, 패비가 말했다.

절벽 아래에는 엘사가 살고 있었다.

구스타프는 이십 년 전에 반역을 일으켰다. 이때 엘사는 만 한 살을 겨우 넘었다. 엘사의 기억대로면 마르쿠스는 여덟 살 때 엘사를 찾아왔다. 그렇다면 이 칠 년 동안은\ldots\,?

``절벽 아래에서 뭘 하고 있었죠?''

쉰 목소리로 안나는 물었다.

``그 답은 이미 알고 있을 거다.''

패비는 눈을 감고 말했다.

``그냥 찾아올 리가 없었다니까.''

안나는 자리를 박차며 외쳤다. 목소리는 높아져만 갔다.

``다 계획된 거야. 다 지켜보고 있던 거라고! 여지껏 뭔 짓을 했는지 알아? 이젠 무슨 계획을 짜고 있을 거고? 말해줘야 해, 가야만 해—!''

``안 돼!''

패비는 안나의 손을 움켜쥐었다. 이제서야 안나는 패비의 눈에 비치는 좌절이 보였다. 실망한 눈빛이었다.

``엘사에 관해 얼마나 더 알았다고 생각하고 엘사의 마음을 바꾸려 드느냐? 네 몸 속에 있던 엘사의 심장 없이는 가면 뒤 모습은 전혀 보지도 못하잖느냐?''

안나는 멈추어 섰다. 패비가 손을 놓자 안나는 뒤로 주춤했다. 팔은 축 늘어지고 다리는 겨우 몸을 지탱하고 있었다. 사실이었다. 모든 것은 자신이 아는 게 거의 없다는 것으로 귀결되었다. 성물 없이는 엘사가 무슨 생각을 하는지 알지 못했다. 자신이 볼 수 있는 것은 눈앞에 있는 것뿐이었다. 엘사이거나 여왕이거나. 항상 둘 중 하나일 뿐이었다.

``네 열의를 꺾으려는 것은 아니지만, 아직은 안 된다, 안나야.''

패비가 말했다. 목소리가 부드러워졌다.

``전 그냥 도와주고 싶어요.''

안나가 나지막이 말했다.

``스스로 찾아야 하는 답이란 게 있는 것이란다. 스스로 내려야 하는 결정도 있는 법이지. 그렇지 않으면 아무 소용이 없게 돼.''

``이\ldots\,이해한 것 같아요.''

``그렇지는 않다. 그래도 이해하려 하고 있지. 이게 중요한 거다.''

``이젠 어쩌죠? 아직도 아무것도 모르고, 뭐가 잘못됐는지도 모르고, 모든 걸 죄다 망쳐놓고 있다고요. 어떻게\ldots\,어떻게 되돌릴 지를\ldots\,''

더 말을 이을 수가 없었다.

누구를 향한 말인지는 알았지만, 인정할 수가 없었다.

어떻게 되돌릴 지를—엘사를.

``그게 네 문제란다. 하지만 곧 답을 알아낼 거다, 안나야. 말을 듣는 것보단 네가 직접 이해하는 게 항상 더 나은 법이다.''

``잠깐만요!''

안나는 손을 뻗었지만 패비는 이미 바위로 돌아간 뒤였다. 깊은 잠에 들고 있었다. 하나씩 다른 트롤들도 뒤따랐다. 모두 동정 어린 눈빛을 보냈다. 안나는 어쩔 수 없이 돌아섰다. 아무나 그저 잠깐 같이 있어만 주기를 바랐다.

``미안하다, 얘.''

불다는 가까이 가며 말했다. 안나는 몹시 마음이 놓였다.

``누구를 바꿀 수는 없다는 걸 잊으면 안 돼\ldots\,남의 마음을 바꾸는 건 너무도 어려운 일이란다.''

``네? 그럼 아무것도 되돌리지 못한단 거예요?''

``내 말은 그게 아냐. 사람들은 잘못 선택한단 거야. 그리고\ldots\,여기 왔던 여자아이한텐 조금 심하게 대한것 같기도 해.''

불다는 미소를 지어 보이며 안나의 얼굴을 어루만졌다.

``너무 그렇게 생각하진 마, 너답지 않아. 나중에 또 오렴, 모든 걸 알아냈을 때 말이야.''

불다도 안나를 떠났다.

안나는 홀로 남겨졌다.

\textbreak

뱃머리에 서서 엘사는 팔을 뻗어 거울이 돌아오게 했다.

처음에는 잘 되지 않았지만 이제는 익숙한 일이 되었다. 거울 조각들의 속삭임은 엘사의 생각을 묻어버릴 기세였지만, 엘사는 이들의 소리를 죽이는 법을 배웠다. 이들을 억누르라, 마르쿠스의 가르침이었다. 그러면 이들의 힘을 제어할 수 있었다. <자연스러운 일이었다.> 거울 조각들을 불러오고 <돌아오는 것을 스스로 바라는 것처럼 이들을 동화시키는 일>은 세상에서 가장 쉬운 것이었다. 물론 트롤들의 말은 잘 기억하고 있었다. 조각들이 정말로 돌아오는 것을 스스로 바라는 것일는지도 모르겠다. % TODO rephrase?

엘사는 지난 몇 일을 마르쿠스와 같이 보냈다. 둘은 서던 제도 주변의 작은 왕국들을 <학살?\slaughter>해왔다. 이전까지는 이들을 건너뛰고 손쉽게 느낄 수 있는 큰 조각들에 집중해 왔다. 이 작은 왕국들이 가지고 있던 것들은 대개 티끌 정도였고 큰 것도 겨우 조각으로 부를 수 있을 정도였지만, 이들은 모여 작은 진전을 만들어 내었다.

이제는 손바닥만한 크기가 된 거울조각이 손에 들려 있었다. 엘사는 거울의 깨끗한 표면을 바라보았다. 자신의 눈을 바라보자 눈앞에는—

`엘사야! 거기 숨어서 뭐 하는 거야?'

`엄마 아빠를 다치게 하고 싶지 않아요.'

`그럴 일은 절대 없을 거야. 우린 가족이잖니, 엘사야. 우리가 같이 있어 줄 거야. 다 괜찮을 거란다.'

엘사는 움찔거리며 뒷걸음질했다. 남는 손으로 눈앞을 가렸지만 몸의 떨림은 그칠 줄을 몰랐다. <It felt like a spike was being driven through the back of her skull, slowly drilling through to her eye socket.>

``괜찮느냐, 얘야?''

마르쿠스가 앞으로 다가가며 물었다.

``괜찮습니다.''

엘사는 거울 조각을 밀쳐냈다. 하지만 조각을 불러오지 않아도 자신의 손에 묶인 듯 돌아왔다. 거울의 저항을 이겨내고 엘사는 기다리고 있는 마르쿠스의 손으로 조각이 떠 움직이게 했다.

``왜 이 조각들이 제 가족을 보여주는지 모르겠습니다. 단 하나도 말이 되는 모습이 아닙니다.''

``\ldots흐음?''

``제 기억과는 전혀 다른 것들입니다.''

엘사가 말했다. 목소리에서 좌절감을 숨길 수가 없었다. 이런 <tantalizing>한 모습들은 필요 없었다. 진짜 모습도 아니었다.

``거울이 욕망을 비추어준다고 하지 않으셨습니까? 하. 아직도 가족들의 인정을 바라는 것같습니다.''

``이제는 우리가 네 가족이잖느냐.''

마르쿠스가 말했다. 손을 뻗어 엘사의 얼굴을 어루만지고 있었다. 엘사는 감사히 마르쿠스의 손길에 기대었다.

``우리는 항상 너를 환영한다, 엘사야.''

``\ldots감사합니다.''

엘사는 바로 섰다. 마르쿠스의 미소에 기운이 생겼다.

``하지만 말로는 다 표현할 수 없습니다. 모든 것이 너무도 감사합니다. 저를 구해주셨고, 저를 가르쳐주셨고 집 없는 제게 집을 주셨습니다. % improvise?

``그리고 완벽히 되갚아주었지.'' % (Spoiler warning!) You know what he's up to and you should be noticing it; it's fourty-bloody-first chapter. — Translator

마르쿠스는 돌아서 뱃전으로 걸어가 난간을 짚은 채 바다를 향해 기대었다.

``정말로 나는 너를 내 딸처럼 사랑한다. 그 누구도 너만큼 충절을 다하지 못했다. 너는 내 기대를 저버리지 않은 내 유일한 자식이다.''

``그리고 저는 결코 기대를 저버리지 않을 것입니다.''

``\ldots나는 너를 의심해본 적이 없다.''

다시 엘사를 바라보며 마르쿠스는 말을 꺼내려 입을 열었지만, 더 나은 생각이 있는지 그저 고개를 끄덕여 보였다.

``그렇다면 나머지는 네게 맡기마. 해야 할 일이 매우 많다.''

마르쿠스는 선실을 향해 갔다. 엘사는 홀로 남아 바다의 모습을 음미했다. 엘사는 눈을 감았다. 흔들리는 배를 느끼며 부딪히는 파도 소리를 들으며 비린 바다 냄새를 맡으며 짠 물안개를 맛보았다. 자신의 몸이 같이 흔들리고 다른 소리는 전혀 들리지 않을 때까지 감각은 점점 커졌다. 즐기기에는 매우 혼란스럽지만, 엘사는 자신을 잠깐 떼 놓을 수 있는 것이, 완전히 잠기는 느낌이 좋았다. 시각을 제한 온 감각이 뒤덮였다. % You actually enjoy these kind of chaos when you are... hugely stressed. — Translator

이것이라도 없으면 엘사는 가끔 자신이 <drift away/멀어져갈?/??!?!?!!??> 것이 두려웠다.

\textbreak

안나는 엘사와 같이 도착한 바닷가에 앉아있었다.

바닷가에 나와있기에는 매우 추운 날씨였지만, 안나는 이것이 괜찮았다. 안나는 망토를 둘러메고 바다를 바라보았다. 수평선에 걸린 해는 바다를 주황빛과 보랏빛으로 물들였다. 교향곡의 마무리를 짓듯 서정적이고 아름다운 빛깔로 죽어갔다.

안나는 엘사와 춤추던 것이 떠올랐다.

게다가 엘사는 매우 잘 추었다. 춤추는 것을 즐기는 것 아니냐고 생각할 정도였다. 서던 제도에 간 후 사교를 위해 배운 것일수도 있지만, 원래 춤추는 것을 좋아했던 것일수도 있었다. 솔직히 잘 어울렸다. 밝은 눈빛의 아이에게는 잘 어울렸다. 구속 없는 움직임, 감정에 몸을 맡기는 춤사위 말이다.

정말로 물어봐야 했다.

\textbreak

철의 장막 사이를 비집으며 엘사는 적들을 향해 벼락을 내리꽂았다.

온 방향에서 날이 날아들었지만 엘사는 잠깐이라도 멈추지 않았다. 이들 사이를 뚫고 들어갈 뿐이었다. 손에서는 번개를 쏘며 다리는 항상 중심을 잡고 여기저기로 몸을 돌리고 있었다. 무기는 쥐고 있지 않았다. 전투 중에 잃어버렸다. 하지만 벼락으로 충분하고도 남았다. 벼락은 시선에 선 불행한 이를 가르며 갑옷과 살을 찢었다. 하지만 엘사는 이 살육에서 어떠한 즐거움도 느끼지 못했다.

엘사는 춤사위에 집중했다. 발을 차고 몸을 뒤로 굽혀 칼날을 피하고는 바로 벼락으로 갚아주었다. 몸뚱이가 바닥에 떨어지기 전에 엘사는 이미 움직이고 있었다. 적들을 피하고 넘기고 헤쳐 나가고 있었다. 한 손은 바닥을 향한 채 엘사는 계속해서 찬 공기를 쏘며 빠르게 전장을 누볐다. 그리고 거미줄과 같은 번개에 둘러싸여 그 누구도 손댈 수 없었다. 발끝으로 몸을 돌리며 나아가면서 번개는 가까이 다가오는 이들을 향해 날아갔다.

집안에서 어머니가 지켜보고 있었다. 엘사는 어머니의 자랑스러운 미소를 떠올렸다.

엘사는 아버지와 마당에서 춤을 추었다. 들어보기만 하고 본 적 없는 화려한 춤을 서툴게 흉내 내고 있었다. 엘사는 즐거운 한 때를 보내고 있었다. 아버지는 간간이 맞잡은 손을 들어올렸다. 한 바퀴 돌 때마다 땋은 머리는 아버지의 배를 때리곤 했고, 아버지가 아픈 척을 할 때마다 엘사는 웃음을 터뜨렸다.

``우리 딸 정말 잘 추네.''

``나중에 크면 진짜 춤을 배울 거예요!''

엘사는 다시 춤을 추었다. 이제는 번개와 얼음이 춤 상대가 돼 주었다.

\textbreak

안나는 시장으로 향해 점점 커지는 북적임에 섞여 들었다.

안나는 초콜릿 냄새를 맡을 수 있었다. 과일향과 다른 단 것들 사이에 섞여 있었다. 이 단 것들의 향은 지나가는 사람은 모두 취할 정도였다. 온갖 것들이 모여 있었다. 계피 섞인 사탕의 달큰함, 박하의 강한 향, 졸인 꿀의 단 내, 바닐라의 부드러움, 초콜릿의 진함.

안나는 엘사가 초콜릿을 좋아하는 것을 떠올렸다.

이 차가운 여왕이 자신만큼이나 단 것을 좋아한다는 것이 안나는 매우 놀라웠다. 어느 정도는 자신보다도 더 심했다. 엘사는 자주 이제는 유치한 것에 손을 뗀 양 했지만, 초콜릿을 입에 넣자 더는 못 보는 것을 대하듯 먹어치웠다. 가끔 안나는 거의 끌어내다시피 말린 때도 있었다. 안나는 웃음을 참으며 생각했다—다른 사람에게 이 말을 해도 엘사가 이런 모습을 믿지 못할 것이다.

왜 그만두었을까, 안나는 궁금했다.

\textbreak

사방에서 피비린내가 났다.

주변 공기도 피로 가득해 입에서도 피 맛이 났다. 엘사는 비린내를 개의치 않았다. 죽음의 맛이 아니었다. 피는 따뜻하고 생기 있는 것이었다. 피 냄새와 맛에는 오래전부터 익숙했다. 자기 부모의 몸에서 자기 손으로 붉게 물들이며 흐르는 피를 본 때부터였다. 엘사를 가장 무섭게 한 것은 모든 것이 멈춘 것이었다.

엘사는 공중에서 고드름을 불러왔다. 팔뚝만큼 두꺼운 고드름들은 드릴처럼 회전하며 한 지점을 향해 날아갔다. 갈라진 땅에서 터져 나오듯 얼음으로 된 나무가 자라났다. 생명을 만들어내는 듯했다. 난도질된 시체가 공중에 매달렸다. 가시에 박혀 꿰어진 것이다. 각각은 희생자를 품고 있었다. 흘러 내려오는 피로 붉어졌다.  % dropped one phrase: untranslatable word "penitente (snow formation)"

``우리 딸 깜짝 선물 준비했단다!''

엘사는 눈을 감았다. 아버지가 손으로 눈을 가리자 소리를 질렀다. 아버지는 엘사가 엿볼 것을 알고 있었다. 어머니가 생일을 위해 준비한 것이 있었다. 볼 수는 없어도 엘사는 알아맞출 수 있었다. 아주 멀리서부터 냄새가 났다. 그래도 엘사는 놀란 모습을 보였다. 작은 케이크가 눈 앞에 나타났다. 아주 소박하고 네 입, 잘해봐야 다섯 입이면 없어질 것이었다. 엘사는 어머니와 아버지에게 행복한 모습을 보여주고 싶었다.

``같이 먹어요.''

엘사가 말했다.

엘사는 케이크를 파고들었다. 눈과 같은 설탕옷을 갈라 케이크 조각을 나누었다. 엘사는 초콜릿을 베어물었다. 달콤한 냄새와 맛이 온 입 안을 뒤덮었다. 엘사는 미소를 지었다. 초콜릿이 흰 이에 묻어났다.

이제는 어느 것도 달콤하지 않았다. 쇠비린내만이 느껴질 뿐이었다.

\textbreak
