<FocusWriter 사용 바람>


\chapter[41장  ~~][41장\hspace*{.5em}~~]{41장 \ ~~}



전쟁이 다가오고 있었다.

온 해군이 소집되었고 모든 군인이 준비되었다. 모든 노예가 한데 모였고 모든 건장한 사람들이 징집되었다. 모든 왕자는 갑옷을 차려 입었고 엘사도 이들 중에 서 있었다. 이런 날이 올 줄은 생각지도 못했다. 마르쿠스가 이전 자신의 자리를 차지하고 있었다. 엘사는 기꺼이 지휘권을 넘겼다. 이제는 다른 이와 같이 병사일 뿐이었다, 항상 그래야 했던 대로.

마르쿠스는 연설을 시작했다.

``나의 아들들이여, 우리는 위대한 승리를 쟁취하였다. 전쟁터에서의 승리가 아닌 우리 마음속이라는 진정한 전쟁터에서의 승리다. 너무도 오랫동안 우리 백성들의 마음은 나약해지고 퇴폐해졌다. 이것도 이제는 끝이다, 그리고 내가 살아있는 동안은, 너희가 합심하여 서 있는 동안은 다시는 되풀이되지 않을 것이다.''

갑옷을 입은 채 마르쿠스는 <struck an imposing figure.> 투구의 Y자 모양 구멍 너머로 기대에 차 빛나는 파란 눈이 보였다. 자신의 아들들 앞에서 열정적으로 연설하는 중에도 엘사는 벌써 전투를 바라는 눈빛을 볼 수 있었다. 채울 수 없이 큰 <살육/carnage>을 향한 욕망이었다. 그렇지만 다른 면도 볼 수가 있었다. 급한 마음이었다. 자신만의 것뿐만 아니라 에드문드를 향한 마음이었다.

거울의 마지막 조각을 되찾아올 때였다.

``몇몇은 주저하고 있을 것이다. 몇몇은 자비의 마음을 품고 있을 것이다. 흔들리지 마라, 가책하지 마라. 너희의 신념이 흔들리게 하지 마라. 이것이 자연의 법칙이 아니냐? 강자가 약자를 지배하는 것이야말로, 우리의 영광이 부활의 불길 위에 빛나는 것이야말로 섭리에 따른 행동일 뿐이다. 세상은 항상 이렇게 진화해왔다. <disservice>, <겁을 내는 것이/cowardice> 우리 왕국의, 우리 백성들의 성장을 방해하는 것이 될 것이다. 힘은 취하라고 존재하는 것이다. 인간이 질서를 바로잡지 않는다면 그 무엇이 인간과 짐승을 가를 수 있으랴!''

``반드시 우리의 손으로 이루어져야 한다. 우리는 정복자다, 우리는 구원자다, 곧 전쟁은 시작될 것이다. 그래도 나는 한마디 덧붙이려 한다. 우리야말로 영광의 짐을 지고 버틸 꿋꿋함이 있는 자들이요, 마땅히 우리의 것을 취할 대담함이 있는 자들이요, 더 큰 선을 위해 버티어 싸울 수 있는 힘이 있는 자들이다.''

``이제 어둠이 드리워지려 한다. 세상이 잠에 빠지며 장막으로 덮이려 한다. 이 잔잔한 어둠 속에 세상은 퇴보한다. 지금까지 우리는 눈꼽만큼의 욕망도 피어오르는 것을 보지 못했다. 평화는 침체를 가져올 뿐이다. 우리는 싸움을 통해 우리 자신을 확립할 것이다! 평화 따위처럼 서서히 좀먹는 것들 앞에서 내게 `정도껏'이란 건 없다. 내게 굴복이란 없다. 내게 애매한 언사란 건 없다. 내게 변명이란 건 없다. 나는 한 걸음이라도 물러서지 않는다. 이제 우리는 승리를 향해 전진한다!''

마르쿠스는 알현실을 걸어 나갔다. 투구의 흰색과 붉은색이 섞인 술은 걸음에 맞추어 흔들리고 있었다. 칼 두 자루가 허리춤에 꽂혀 있었다. 엘사는 따라 나갔다. 마지막 여정을 착수할 준비가 되어 있었다.

모든 것이 끝나면 안나와 같이 있을 것이다.

\textbreak

트롤들에게 `놀랐다'는 말로는 충분치 않았다.

안나는 이들에 관한 옛이야기를 들으며 자라왔다. 이들이 살아 움직이는 것은 꿈에서나 볼 법한 일이었다. <It was told to children that ...> 배게 밑에 빠진 이를 두면 착한 트롤들이 이를 가져가고 선물을 준다는 둥, 착하게 지내면 동지절에 선물과 새해 복을 준다는 둥 하는 이야기들이었다. 물론 나쁜 트롤들은 나쁜 아이를 잡아가 엄지발가락을 잡아먹는다고 했다.

``진짜로 엄지발가락을 잡아먹는 건 아니죠?''

``그럴 리가.''

패비가 말했다. 껄껄 웃고 있었다. 그러고는 다시 말을 꺼냈다.

``음\ldots\,우리가 무얼 먹고 지내는지는 잘 모르겠구나. 사실 뭘 먹기나 하는지도 모르겠구나. 정말 희한한 일일세\ldots''

무례한 말을 했다고 열댓 번을 사과하고 그냥 많이 놀라서 그랬다 말하고 `당연히' 만나게 되어서 아주 반갑다고 주절댄 뒤에야 안나는 어느 정도는 이 트롤들에게 익숙해졌다. 대부분은 자기만큼이나 호기심에 찬 모습이었다. 안나는 마음이 놓였다.

껄껄 웃으며 패비는 다는 트롤들을 소개했다. 안나는 꿇어 앉아 트롤들 하나 하나를 살펴보며 각각의 특징들을 머리에 넣으려 했다. 패비는 망토를 두르고 머리에 무언가를 쓰고 있었고, 불다는 분홍색 수정이 있는 목걸이를 했다. 다른 트롤들은 <각기둥/prismatic> 같은<substitute:Strange> 것이 튀어 나와 있었다. 개중 하나는 밝은 붉은색 수정을 품고 있었다. 다들 똑같이 큰 미소와 똑같이 작은 키에 똑같이 단단한 체구였지만 이들의 차이는 눈을 떼지 못할 만큼 분명했다. 안나는 모두를 바라보며 이들의 묘한 점들을 놓치지 않으려 했다.

``모든 게 정말 신기해요.''

안나는 웃으며 말했다.

<"It's our pleasure!" Bulda said, holding onto Anna's hands and examining her with a rather keen eye. It had taken some smooth talking to get Bulda to stop looking at her teeth. Anna couldn't decide if she was flattered. "We don't get visitors often, and none are as polite as you!">

``안나 좀 내버려 두게.''

패비가 말했다. 또다시 맑은 눈을 가지고 말을 쏟아내기 전에 불다를 끌어내었다.

``긴히 할 말이 있어.''

자신의 이름을 말하는 것에 안나는 패비를 향해 뒤돌아섰다. 생각해보면, 전에도 자신의 이름을 부르며 인사했다.

``어떻게 제 이름을 알죠?''

``거울의 아이와 같이 그 아이의 집에 왔을 때 우리는 모든 것을 보고 있었단다.''

패비가 말했다. 왜 `거울의 아이'라고 하는지 안나는 알지 못했지만, 누구를 일컫는지는 알 수 있었다. 엘사 이야기에 몇몇 트롤들이 중얼거리는 것을 안나는 눈치채지 않을 수 없었다.

``오랫동안 그곳을 지켜보고 있었지.''

뭐, 집이 어찌 그대로인지에 관한 수수께끼가 풀렸다.

``잠깐만요, 왜 지켜보시는 건데요?''

<"It's something of a memorial," Pabbie said. "Who was it that led you here?">

``그러면 그 아이가 진짜였다는 거야?''

안나는 고개를 가로저었다.

``그건 말도 안 돼요. 여왕님은 서던 제도에 있고 여덟 살도 아니라고요. 진짜예요. 잠깐만요! 절 데려오려고 할아버지께서 만들어낸 건가요?''

``아니, 내가 한 게 아니다. <Memories are very real things,>''

진중한 목소리에 안나는 자기도 모르게 진정이 되었다.

``커다란 비극이 생긴 곳에는 대개\ldots\,자국이 남는다. 고통의 메아리 같은 것이지. 아무나 볼 수 있는 게 아니지만, 안나 너는 매우 중요하다.''

``저\ldots\,전 잘 모르겠어요. 죄송해요. 이해하기엔 너무 벅차서요.''

미소를 지으며 웃는 엘사의 모습에 안나는 그 무너진 집에 무슨 일이 있었는지 거의 잊을 뻔했다. 왜 이런 엘사의 기억이 남아있는지를 알아낸 것만도 겨울 비에 흠뻑 젖는 듯 혹독한 깨달음이었다. 거기에다 트롤들이 진짜라는 것과 엘사의 집 터를 지켜보고 있다는 것은? 왜 굳이 그러는 것일까? 왜 엘사를 신경쓰는 것일까?

``괜찮다, 이해한단다.''

안나를 자리에 앉히며 패비가 말했다. 안나는 바닥에 완전히 주저앉았다. 다리를 감싸 안고 있었다.

``그래도 우리가 도움을 줄 수는 있을 거다.''

``알았어요.''

안나는 숨을 들이마셨다.

``왜 여왕님을 신경쓰는 건가요?''

``우린 거울과 같이 태어났다. 거울이 무엇인지는 알겠지.''

고개를 끄덕이는 것을 보고 패비는 말을 계속했다.

``그리고 우리는 엘사를 거울의 아이라고 부른단다. 그 거울은 오로지 그 아이의 것이고 그 아이의 마음만이 제어할 수 있단다. 그 아이에 관한 일은 우리의 일이기도 하지.''

``그러면 눈사람 만들고 있을 때 내려오셔도 됐잖아요?''

``우린 이미 만났다.''

패비가 말했다. 언제냐고 묻기 전에 패비는 말을 이었다.

``너희 둘이 그 아이의 옛 집에 오기 며칠 전이었다. 우리가 지키는 거울 조각을 가지러 온 게지.''

시간이 조금 지나고서야 안나는 완전히 이해했다. 처음 들었을 때는 아주 오래 전에 만난 걸로 생각했다, 어릴 때라든지. 자신의 말대로 이곳으로 온 때로는, 일주일도 안 됐다고는 생각지 못했다. 안나는 입술을 깨물었다. 자기 때문에 엘사가 아렌델로 온 게 아니라는 것일까? 거울 조각 때문에 온 것일까? 그렇다면 자기 때문이 전혀 아니었다. 바보같은 자신의 모습에 안나는 거의 웃음이 나올 지경이었다. 다른 이유가 있으리라고는 생각도 못한 자신의 모습이.

``그랬겠죠.''

안나가 말했다.

``한 마디 덧붙이자면, 안나 네가 생각하는 대로는 아주 아닐 거다.''

패비는 미소를 보이며 어깨를 으쓱했다. 안나는 기분을 낫게 해주려는 걸로 생각했다.

``거울 때문에 온 것만은 아니란다. 너를 위해 온 것이기도 하단다.''

``그 아이 같은 사람을 우리는 망령이라고 부른단다. 더는 심장이 없으니까.''

불다가 말했다.

``여왕님이 성물을 만들었으니까요.''

안나가 말했다.

엘사가 