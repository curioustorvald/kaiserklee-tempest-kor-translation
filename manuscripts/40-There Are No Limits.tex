

\chapter[40장  한계는 없어][40장\hspace*{.5em}한계는 없어]{40장 \ 한계는 없어}



\forceindent`드러내. 자유롭게 해.'

`한계는 없어.'

엘사는 이를 악물었다. 손가락을 갈고리 모양으로 굽힌 채 팔을 들어 올리자 얼음으로 된 벽 두 개가 동시에 솟았다. 벽은 오 미터 높이에 꼭대기에는 가시가 아무렇게나 튀어나와 있었다. 벽 사이에는 마르쿠스가 서 있었다. 이 가시로 덮인 벽이 자신의 퇴로를 가로막자 눈을 가늘게 뜨고 있었다. 엘사는 오른손을 앞으로 내질렀다. 발밑에서 눈이 파도 모양으로 앞으로 나아갔다. 이 왕 앞에 다다를 즈음에는 눈사태와 같이 커졌다. 이 거대하고 하얀 파도 뒤에서 마르쿠스의 모습은 전혀 보이지 않았다.

이때 회색 벽이 생겨났다. 아무것도 없던 데에 그냥 나타났다. 눈사태가 다가오자 표면에는 검은 불길이 일었다. 우레와 같은 굉음을 내며 눈은 벽에 부딪혔다. 한동안은 양쪽 다 물러섬이 없었다. 눈은 나아가려 하고 불길은 계속 퍼져 나갔다. 마침내 결국에는 불길이 완전히 퍼져 눈을 덮었다. 눈은 아무런 영향도 주지 못하고 조금씩 불길에 먹히더니 곧 아무것도 남지 않았다. 마르쿠스는 멀쩡히 서 있었다. 엘사는 손을 내리고는 고개를 숙였다.

``다시!''

마르쿠스가 외쳤다. 엘사는 다시 자세를 바로 했다.

``열성을 다하지 않고 있다. 건성이지 않느냐!''

얼음은 항상 안에서 왔다. 마음과 영혼에서 말이다. 엘사는 이 찬 마법을 바깥으로 보낼 뿐이었다. 자유로이 흘러나갔다, 온몸을 타고 흐르는 강처럼. 번개는 전혀 달랐다. 밖으로부터 왔고, 저항하며 팔에 충격과 저릿한 반동과 벼락에 맞은 것과 같은 끔찍한 고통을 주었다.

필요한 분노를 충전하는 데에 제격이었다.

엘사는 자신이 퇴보했다는 것으로 좌절감을 끌어모았다. 여태껏 약해졌고 거울 조각을 찾는 데에, 혹은 그 어느 것에도 진전이 없다는 것으로 말이다. 팔과 가슴에 고통이 퍼졌다. 커다란 압력이 자신을 짓누르는 것과 같았다. 그래도 마법은 생각대로 점점 쌓였다. 청백색 전기가 손끝에 모여들었다.

엘사는 전기를 앞으로 쏘아 보냈다. 눈으로는 궤적을 좇으며 각도를 계산하고 누구도 따라 할 수 없을 정도로 모든 것을 읽고 있었다. 번개가 벽을 맞히자 불길이 갈라져 그림자의 결정으로 된 내부가 드러나 깨어졌다. 마르쿠스는 옆으로 뛰어 피했다. 눈밭 위에 떨어졌지만 머리 위로 스쳐 간 벼락은 피했다.

작은 승리를 축하할 시간 따윈 없었다. 마르쿠스는 바로 일어나 두 손으로 그림자 덩어리를 쏘았다. 마법사의 것과 똑같았다. 아버지와 아들의 이 똑같은 보랏빛에 엘사는 반 초 늦게 반응했다. 피할 새도 없이 옆구리를 스쳤다. 불타는 기름이 피부를 지지는 듯한 느낌이었다. 벼락을 쏠 때마다 자동 방어는 내려갈 뿐이었다.

``집중하라!''

엘사는 이를 악물고 재빠르게 탄막을 피했다. 발밑에서 나오는 찬 공기가 밀어주고 있었다. 엘사는 마르쿠스에게 벼락을 쏘아 공격을 멈추게 했다. 벼락이 맞은 자리는 눈은 모두 녹고 흙은 검게 탄 채 깊게 파여 있었다. 엘사는 전기를 쏘아 보내며 거의 날아다니다시피 하고 있었지만, 마르쿠스는 너무도 눈치가 빨랐다. 마법사의 것과 같이 반쪽짜리 그림자 형태를 취해 좌우로 움직이는 어둠의 꼬리를 남기며 미끄러져 갔다. 움직임이 너무 빨라 엘사는 직선으로 공격을 내리꽂을 수 없었다.

한 번은 속았다. 두 번은 속지 않는다.

엘사는 온 주변에 얼음벽을 세웠다. 마르쿠스는 전혀 신경 쓰지 않았다. 엘사는 신경 쓰리라 생각하지 않았다. 마르쿠스는 급히 만들어진 이 벽을 깨부수며 계속 움직였지만, 엘사는 계속 벽을 세웠다, 부서지는 것보다도 더 빠르게 말이다. 엘사의 계책을 알아챈 듯했다. 마르쿠스는 곧바로 엘사를 향해 갔다. 하지만 엘사는 돌풍을 일으켜 벽에 계속 가까이 붙게 했다. 만족감이 솟는 것을 느끼고 엘사는 얼음을 향해 벼락을 돌렸다. 벼락은 벽을 맞고 반사되며 귀가 찢어질 듯한 폭발과 함께 벽을 깨부수었다. 이 한 방은 피했을지라도, 이를 피하느라 마르쿠스는 다른 벽에 가까이 붙었고 벼락은 이 벽을, 그리고 다음 벽을, 또 다음 벽을 스쳐 날며 마르쿠스가 가는 곳마다 따라잡았다. 매우 빠른 속도에 온 벽이 동시에 폭발하는 듯 보였다.

연기가 잦아들자 등을 대고 누운 마르쿠스가 보였다.

``내가 졌다.''

마르쿠스가 말했다. 이제야 엘사는 벼락을 흩트려 없었다.

이들이 있던 안마당의 바깥쪽에서 에드문드는 손뼉을 치며 뛰어나왔다. 이 시합 동안 이쪽으로 가까이 다가올 만큼 용기가 있는 이는 거의 없었다. 마르쿠스와 엘사가 같이 훈련하는 것은 거의 의식에 가까운 것이 되었다. 물론 한동안은 중단되었지만 말이다. 마르쿠스가 돌아왔으니 이제 훈련은 다시 시작되었다. 마르쿠스는 엘사의 능력이 떨어진 것이 매우 불쾌했다.

``다시는 뒤처지지 마라.''

마르쿠스가 꾸짖었다.

``물론입니다.''

``하지만 아버지, 엘사가 이겼잖아요.''

에드문드가 말했다. 크게 미소를 짓고 있었다. 마르쿠스가 노려보자 에드문드는 엘사에게 엄지를 들어 보였다.

``둘 다 잘했지만, 이건 누나 거예요.''

``마르쿠스께서 훈련을 잘 해주신 거야.''

엘사가 말했다.

``에드문드 너는 엘사에게 겸손해지는 법을 배워야 하겠다.''

마르쿠스는 고개를 저으며 말했다. 그러고는 이 둘에게서 돌아섰다. 에드문드에게 자신의 표정을 보이고 싶지 않은 듯했다, 엘사는 생각했다.

``그렇다만\ldots\,네 생각도 중요한 걸지도 모르겠다.''

``아, 그냥 돌아다니고 있던 거예요. 마법은 하나도 모르지만, 그래도 멋지잖아요.''

에드문드는 어깨를 으쓱했다. 이상하게도 마르쿠스는 이 무성의한 대답에 만족한 듯했다. 별말 없이 마르쿠스는 안으로 향했다. 엘사가 혼자 계속 훈련하도록 했다.

``넌 여기 있을 거야?''

엘사가 물었다. 에드문드가 가만히 있는 것에 나온 말이었다.

``딱히 할 일이 있는 것도 아니고. 뭘 알아갈 수도 있는 거잖아?''

에드문드는 미소를 지어 보였다. 손을 크게 젓고 있었다.

``그러니까, 누나, 내가 뭘 도와줄 수도—''

그러고는 갑작스레 조용해졌다. 온몸이 그대로 움직임을 멈추었다.

``갑자기 왜 그래—''

엘사는 말을 마칠 수 없었다. 목멘 숨을 헐떡이며 에드문드는 무릎을 꿇고 쓰러졌다. 갈고리처럼 구부린 손으로 심장에까지 파고 들어갈 기세로 가슴을 움켜쥐었다. 엘사는 바로 곁으로 가 무릎을 꿇어앉았지만, 에드문드가 무엇 때문에 이러는지 젼혀 알 수가 없었다. 에드문드의 얼굴은 고통으로 일그러졌다. 혈관이 모두 보일 정도로 눈은 휘둥그레졌다. 오래지 않아 에드문드는 완전히 쓰러졌다. 거칠게 신음을 내고 있었다.

``에드문드!''

엘사가 외쳤다. 하지만 에드문드는 답이 없었다. 이제 두 손은 머리를 감싸쥐고 있었다. 손마디가 하얘졌다. 에드문드의 비명은 끔찍했다. 엘사는 많은 비명을 들어 왔다. 고통, 공포, 분노의 비명, 모두 사람의 감정이 서린 것이었다. 하지만 이처럼 짐승의 울부짖음과 같은 것은 전혀 들어보지 못했다. 이 동물적인 비명은 매번 높은 콧소리로 끝났다. 마치 인간성이 없어져 가는 듯했다.

엄청난 노력으로 에드문드는 겨우 고개를 들어 엘사의 눈을 바라보았다. 엘사는 에드문드에게서 자신과 같은 모습을 본 듯했다. 엘사는 최대한 에드문드를 끌어안았다. 속삭이는 말을 겨우 들을 수 있었다. 매 초가 중요한 것을 느낄 수 있는 매우 절실한 소리였다.

``'살려줘—!''

그러고 이 절실한 목소리는 죽어버렸다. 다시 신음이 그 자리를 차지했다. 엘사는 아무것도 할 수 없었다. 에드문드를 움직이게 해도 되는지도 알 수 없었다. 도움을 바라며 소리칠 뿐이었다. 그러나 아무도 오지 않았다. 에드문드가 조용해지는 것을 엘사는 손 놓고 바라볼 수밖에 없었다. 에드문드는 기절했다.

그리고 숨을 쉬지 않고 있었다.

\textbreak

서던 제도에 다다르기란 불가능한 것이었다.

안나는 자신이 도착한 때를 잘 기억하고 있었다. 얼어붙은 부두 주변을 움직이려면 쇄빙선이 얼마나 필요한지도 잘 알고 있었다. 아렌델에도 이런 배는 있었지만, 대부분은 엘사의 침입 때 부서졌고, 그 후로 자원이란 자원은 모두 빼앗겼다. 쇄빙선이 잘 움직인다 해도 쓸모가 있을지는 다른 이야기였다. 이 침입 때도 역할을 거의 못했으니 서던 제도의 얼음을 깨어 나갈 것을 바라기란 어불성설이었다.

그래도 상관없었다. 안나는 다시 돌아갈 생각이 없었다.

어쨌든 아직은.

안나는 정말 그러고 싶었지만 아무것도 바뀐 것이 없었다. 엘사는 자신을 내쳤다. 선택이란 것을 할 수 있는 것이 아니었다. 엘사는 안나가 남기를 바랐지만, 그래도 엘사는 자신을 내쳤다. 거기다 안나는 좋은 이유가 꼭 있으리라고 엘사를 믿었다. 왜 그런지는 몰라도, 엘사 자신도 왜 그랬는지 몰랐을 것이다. 그저 그렇게 되어야만 했을 것이다, 아마도. 안나는 믿고 있었다.

하지만 지금 돌아가도 결국엔 이 모든 것이 되풀이될 것이다. 틀어진 일들과 둘 사이에 틀어졌던 것들도 아직 남아있었다. 암살 일만은 아니었다. 이 때문이 아니었다. 무어라고 안나는 말할 수 없었지만, 둘 다 항상 알아채 왔다—저 깊은 곳에서 무언가가 잘못되었다. 둘 사이의 무언가가 항상\ldots\,어긋나 있었다.

이들이 아렌델에 오기 직전부터 생긴 느낌이었다.

엘사가 자기 자신이 사라져 간다고 하며 자신을 현실에 매어놓을 것에는 모두 매달리려 한 때였다. 이때 안나는

``그런 일 없게 할 거예요.''

라고 말했다. 엘사가 눈물과 함께 무너져 내린 때에 느껴진 것이었다. 안나는 엘사의 약점 하나하나와 가능성을 보고는

``여왕님의 마음이 돼 줄게요.''

라고 했다. 안나는 말을 마치고 마음이 놓이기는커녕 불편해졌다. 죄책감이 들었다. 자신을 바라보는 눈빛은, 구원의 순간을 목격하는 사도와 같던 엘사의 눈빛은\ldots

이 순간 일이 매우 잘못되었다는 느낌이 왔다, 왜인지는 모르겠지만.

안나는 한숨을 내뱉었다.

``복잡해질 수밖에 없는 거겠지\ldots''

불평하는 것이 아니었다. 불평하나 마나 복잡한 일은 복잡한 일이다.

안나는 생각을 한쪽으로 밀어 넣고 눈앞의 길에 집중했다. 겨울이 다가오고 있었다. 익숙한 나무 태우는 냄새가 밀려왔다. 차갑고 건조한 공기가 얼얼하게 느껴졌다. 이른 아침 동안 조약돌 깔린 길은 밤새 앉은 서리로 매끈했다. 안나는 날이 저물 때쯤이면 한두 군데 멍이 들어있을 것으로 생각했다.

그래도 보람 있는 일이었다, 빛을 보는 게 말이다.

안나는 시장터에 서서 가게 하나하나를 꾸미며 죽 늘어서 있는 등불을 올려다보았다. 지금쯤이면 동지절 장식이 거의 다 달렸을 때다. 지나가면서 아침 댓바람부터 일하는 사람들은 조용히 마무리 장식을 하고 있었다. 어둑한 새벽녘의 등불은 밤에 보는 것처럼 밝게 빛났다. 보랏빛 하늘과 대비되는 주황빛과 노란빛을 깜빡이고 있었다. 엘사는 자신과 같이 이 모습을 보자고 약속했다. 그리고 지금 안나는 홀로 있었다.

엘사도 이 모습을 좋아할 것이다, 안나는 생각했다.

안나는 시장을 빠져나왔다. 시장이 목적지는 아니었다.

더욱 갑작스러운 생각에서 나온 순간적인 결정이었다. 머릿속에 떠오르자마자 안나는 이 생각을 무시할 수 없었다. 아무 데에도 도움이 안 될 것이었다. 그저 지나간 것에 아직도 매달리고 있어서일지도 모르겠다. 하지만 안나는 엘사와 있던 곳을 다시 가보기로 했다.

안나는 바로 엘사의 옛집 터로 향했다.

길을 아직도 생생히 기억하고 있다니 이상한 일이었다. 안나는 자신이 길을 잘 찾는다고는 생각지 않았지만, 딱 한 번 가본 것으로 다른 곳과는 동떨어진 이 산길을 놀라우리만치 정확히 찾아갔다. 그렇지마는 엘사와의 모든 기억도 마찬가지였다. 안나는 꺼낸 말 한마디, 매 순간의 느낌 하나하나까지 모조리 기억했다. 절대로 잊을 수가 없었다.

하물며 이 집쯤이야.

성 밖의 마을을 지나, 그리고 더욱 떨어진 곳에 있는 집들은 언덕 위의 개미와 같이 보였다. 엘사의 옛집은 산기슭에 땅 위에 남은 상처처럼 있었다. 멀리서도 안나는 흔적을 볼 수 있었다. 집뿐만이 아니라 주변 땅도 말이다. 아직도 탄 자국이 남아있을 정도로 번개는 강력했다. 집을 중심으로 모든 방향으로 뻗어 나와 눈송이 모양을 이루었다.

오랜 시간이 지나고도 이 폐허가 그래도 남아있는 것이 다소 놀라운 일이었다. 대자연이 마치 이 일을 보존해야겠다고 마음먹은 듯한 모양새였다. 안나는 가까이 다가갔다. 왠지 신성한 땅을 침범하는 듯한 느낌이 들었다. 엘사는 이곳에 있을 자격이 있지만 안나는 홀로는 그러지 못했다.

눈은 거의 다 녹아 있었다. 딱히 따뜻한 날씨도 아니었다. 겨울이 눈앞이었다. 엘사가 만들어 낸 눈밭은 조금씩만 남아 있었다. 평범한 눈이었을는지도 모르겠다. 햇볕에 닿은 눈이 으레 녹듯이 말이다. 하지만 올라프도 바로 이곳에서 태어났다. 똑같은 눈으로부터 말이다. 올라프는 움직임이 없어진 상태에서도 전혀 녹지 않았다. 올라프를 만들어 낸 마법이 특별한 것이었을는지도 모르겠다. 언젠간 올라프가 다시 깨어날 것이라는 희망이 안나에게는 생겨났다.

안나는 잔해 앞에 쭈그려 앉았다. 이 어마어마한 파괴의 현장을 여겨보고 있었다. 엘사가 고작 여덟 살 때 벌인 일이다. 얼마나 두려움에 떨었을까. 벼락으로 집을 무너뜨리고 나무를 태워 숯으로 만들고, 그 숯도 재가 되었다. 안나는 궁금해했다—다른 이웃이 있기는 했을까, 엘사가 이토록 학대받고 있던 것을 다른 사람은 알기나 했을까.

``여기 같이 있었으면 좋았을 거예요. 이야기를 더 많이 나눠봐야 했어요. 저한테 솔직해지는 게 그냥 안 됐던 거죠.''

안나는 중얼거였다.

바닥 널빤지 아래에서 나는 빛에 안나는 시선을 돌렸다. 칙칙한 갈색과 검은색 사이에 파란빛이 틈새를 뚫고 빛났다. 안나는 널빤지의 끝을 잡고 다시 일어서서 크게 한 번 힘껏 들어 올려 옆으로 내던졌다. 파란빛의 정체가 드러날 때까지 안나는 이를 반복했다.

난초였다.

천천히 안나는 꽃잎을 덮고 있는 검댕을 닦아내고, 흙을 파헤쳐 다른 꽃들을 꺼냈다. 막 자란 더 많은 싹이 잿더미를 뚫고 나와 있었다. 엘사의 가족들이 기르던 것이었을까, 혹은 자리가 정리되고 새로 뿌리를 내린 것들이었을까? 안나는 꽃잎 하나를 만져보았다. 매끈하고 보들보들한 꽃잎을 감탄하고 있었다. 결도 멍도 없이 완전한 파란색이었다. 이 세상 꽃이 아닌 것처럼 보였다. 얼음이 꽃 모양을 한 것처럼.

``그 꽃 좋지?''

안나는 몸을 홱 돌렸다.

눈앞의 불가능한 모습에 안나는 얼어붙었다. 자기 생각을 알고 있는 듯한 어린 여자아이가 있었다. 자리에 조용히 선 채 미소를 짓고 있을 뿐이었다. 안나는 무슨 일인지 이해하려 애를 썼지만 영문을 알 수 없었다. 파란 눈에 백금발을 하나로 땋아 등 뒤로 늘어뜨린 여자아이가 서 있었다. 기억 속에서 딱 한 번 본 적이 있는 얼굴이었다.

``나도 이 꽃들이 좋아.''

엘사가 말했다. 그러고는 안나의 말을 기다리며 흥얼거리기 시작했다.

``왜 여기에\ldots''

안나는 헛기침을 했다. 손을 내밀고 싶었지만 손이 너무도 떨렸다. 있는 힘을 쥐어짜내 안나는 겨우 멀쩡히 있었다. 어릴 적 엘사의 모습에—한 번도 본 적 없이 행복한 모습이었다. 같이 있을 때보다도—안나는 그 무엇보다도 가슴이 미어졌다. 어떻게든 엘사를 치유해줄 수 있을 것 같은 생각이 들었다. 그러기에 충분한 적이 없던 것을 떠올리고는 자신이 언짢아졌다.

``진짜가 아니지.''

``그럴지도.''

``그러면 왜 여깄는 거야?''

안나는 엘사의 심장 조각을 더는 갖고 있지 않은 것을 확실히 알고 있었다. 기억을 보여주고 있는 것은 없었다. 그렇다 할지라도 바라보는 것만이 가능했다.

``이런 장소는 특별하다고.''

엘사가 말했다.

``내가 미쳐가나 보다.''

엘사는 킬킬거렸다. 지금의 엘사가 그러듯 입을 가리고 있었다. 항상 이러던 것이 상심과 함께 떠올랐다. 나중에 배운 것이 아니었다. 예절로 배운 것이 아니었다. 엘사는 알지 못했다. 마르쿠스에게 배운 것이라고 둘러댈 뿐이었다.

``어차피 미쳐가는 거면\ldots''

엘사는 씩 웃었다. 장난기 그득한 눈이 밝아졌다. 전에 한두 번밖에는 본 적 없는 모습이었다. 그러고는 산을 향해 달려나갔다.

``따라와!''

``기다려!''

안나는 자리를 박차고 일어났다. 엘사는 벌써 비탈의 반을 올라가 있었다. 멈추어 서서 따라오라고 손짓하고 있었다. 온 상식이 좋지 않은 일이라고 말하고 있었다. 아이 모습을 한 환영을 따라가는 것은 문제를 스스로 만드는 일이니 말이다. 그렇지마는, 자신이 한 일은 대개 문제를 만드는 일들이었다. 안나는 끙하는 소리를 냈다.

``아오, 이런 멍청한 짓 좀 그만해야 하는데. 좀 기다려봐!''

가는 길은 처음에는 쉬웠지만, 나무들이 우거지기 시작하자 쉽다는 생각은 금세 사라졌다. 엘사는 가는 길이 거친 것은 전혀 신경 쓰지 않는 듯했다. 숲속 깊은 곳으로 곧장 달려갔지만 안나는 꾸물거리며 모든 길을 의심의 눈으로 바라보았다. 길이 있었어도 마구 뻗은 나뭇가지와 잡목으로 오래전에 덮였을 것이다. 나무들이 빽빽이 자라 있어 꽉 들어찬 상록수 사이로 빛이 거의 들지 않았다. 엘사는 등대와 같았다. 밝은 파란색 옷은 충분히 길을 밝혀주었다.

``안 올 거야?''

엘사가 물었다.

엘사가 계속 움직이자 안나는 자신의 호기심이 원망스러워졌다. 울퉁불퉁한 길을 지나고 가지를 돌아 피하는 모양새가 거의 춤을 추는 듯했다. 당연히 길은 잘만 찾아다녔다. 하지만 안나는 한 걸음 한 걸음이 힘겨웠다. 돌을 잘못 밟아 미끄러지거나 가지에 겉옷이 걸리기 일쑤였다. 숲치고는 아주 이상한 장소였다. 새 지저귀는 소리나 귀뚜라미 소리같이 다른 소리는 나지 않았고 풀냄새도 없었다. 소나무 냄새뿐이었다. 안나는 가는 길에 집중하며 신경을 쓰지 않았다.

엘사는 항상 앞서나가 안나를 기다렸다. 안나의 꼴사나운 모습에 웃음을 감추고 있었다.

``언제 도착하는 거야?''

``다 큰 어른이 왜 그래?''

엘사가 대답했다. 다시 출발하고 있었다.

올라간 시간만큼 다시 내려온 듯했다. 이 중 절반은 산양도 겁낼 법한 길이었다. 그래도 안나는 결심이 굳어 있었다. 가시가 팔다리를 긁고 옷을 뜯어내고 머리카락을 잡아채도 안나는 계속 갔다. 어디로 가는지도 몰랐지만, 엘사와 있을 때조차도 확신이라는 것을 가져봤던가. 처음 있는 일도 아니었다. 군데군데 멍이 들고 머리부터 발끝까지 온몸이 아팠지만 안나는 계속 걸음을 옮겼다.

몇 시간이 지나고서야 엘사는 마침내 멈추어 섰다.

안나는 공기가 따뜻해진 것을 알아채지 못했지만, 후들거리는 다리로 헉헉대며 멈추자 뿜어져 나오는 증기가 확실히 눈에 띄었다. 어떤 계곡에 온 듯싶었다. 층이 진 공터였다. 바위들이 흩어져 있는 것을 빼고는 이상한 것이 없었다. 그리 먼 곳도 아니었다, 온 길은 매우 고통스러웠지만. 지금처럼 오래 걸릴 일이 없었다.

``좀 돌아서 왔긴 했어.''

엘사가 말했다. 아래를 내려다보며 치맛자락을 만지작거리고 있었다. 안나가 아무 말이 없자 엘사는 미안해하는 모습은 버리고 씩 웃어 보였다.

``미안해. 같이 있는 게 정말 재밌었어.''

``넌 정말, 정말 나쁜 아이야.''

안나가 말했다. 그러고는 미소를 지으며 머리를 두드려주었다.

손은 뚫고 지나갔다.

안나는 움찔했지만, 곧 진정했다. 다른 것을 바랄 것이 없었다. 엘사는 이제는 이런 아이가 아니었다. 완전히 자라 서던 제도의 여왕이 되었고 과거를 완전히 지워 이런 모습은 남지 않았다.

``왜 여기로 날 데려온 거야?''

안나는 나지막이 물었다.

``금방 알 거야.''

엘사가 말했다. 뒤로 물러서고 있었다. 파란 눈은 계속 안나를 바라보고 있었다. 미소는 어울리지 않게 칙칙한 표정으로 바뀌었다. 갑자기 안나는 떠올렸다—다 큰 엘사의 모습을 닮았다.

``다신 날 잊지 말아줘.''

``다시라니?''

대답을 듣기도 전에 엘사는 사라졌다. 안나는 계곡에 홀로 남았다. 바위들만이 자신을 둘러싸고 있었다. 이유가 무엇이었을까. 이유가 있기는 했을까? 어떤 엘사와의 일이든 결과는 항상 같았다. 안나는 땅을 발로 차고는 머리를 손으로 감싼 채 바위 옆에 주저앉았다. 무엇을 기대했던 것일까. 이제는 궁금증이 들었다, 정말로 어린 모습의 엘사를 본 것일까, 혹은 그 정도로 그리워한 걸까—

우르릉거리는 소리가 들려왔다.

안나는 자리에서 일어섰다. 놀라 입을 벌린 채 바위가 제자리에서 떨다 공터 가운데로 굴러가는 모습을 바라보았다. 바위 하나는 거의 다리에 부딪히려 했다. 안나는 꺅하는 소리를 내며 옆으로 비키고는 나머지 바위들을 바라보았다. 백 개가 넘는 바위가 가운데에 모여들었다. 가운데에 안나를 둔 채 고리 모양으로 있었다. 안나는 이상한 시선이 느껴졌다\ldots

바위들에서.

``정말 이상한 하루야\ldots''

바로 이때 바위 하나가 망토를 갖춘 뚱뚱한 사람 모습으로 펼쳐졌다.

안나는 비명을 내질렀다. 다시 입을 다물었다. 더는 놀라고 싶지 않았다. 도망치고 싶었다. 짐을 챙겨 쉴 곳을 찾아가고 싶었다. 물론 이 모든 것이 감당하기에는 너무 컸다. 그래도 천천히 진정이 되자 이 바위로 된 사람이 얼굴을 찡그리는 것이 눈에 보였다. 안나는 미안한 마음이 들었다. 자신의 비명이 듣기 좋은 소리는 아니었을 것이다. 바위가 소리를 들을 수는 있었을까? 귀가 달려있기는 했다\ldots

``어, 미안해요.''

이 바위가 미소를 짓자 안나는 또다시 놀랐다. 진주와 같이 흰 이를 드러내며—어떻게 가능한 것인지 안나는 묻지 않기로 했다—말했다. 맞물려 갈리는 바위처럼 깊은 목소리였다.

``괜찮다. 하지만 정말로 내가 누군지 모르는 건 아니겠지, 안나야?''

``아, 죄, 죄송해요. 이런 걸\ldots\,본 적은 없는 거 같아요.''

그래도 안나는 찬찬히 바라보았다. 펼쳐지는 바위들의 모습을 보자 말끝을 흐렸다. 뚱뚱한 사람 모습에 밝은 미소를 짓고 있었다. 어릴 때 들은 옛이야기에 나올 법한 모습들이었다. 천진한 놀라움에 안나는 소리쳤다.

``트롤이잖아? 진짜로 있는 건 줄은 몰랐는데!''

``당연히 진짜지! 나는 패비라고 한단다.''

패비는 웃으며 주변을 가리키며 말했다.

``환영한다, 안나야, 살아있는 바위의 계곡에 온 것을.''

\textbreak

모든 왕자는 에드문드의 방 주변에 모였다. 더 큰 일을 위해 밤새 조용히 기다리고 있었다. 에드문드가 정신을 잃자 엘사는 사울에게 달려갔다. 마르쿠스는 어디에도 보이지 않았다. 일찍이 소식을 전했지만 아무런 말도 없었다. 기력 없이 체념한 듯한 움직임으로 물러가라 할 뿐이었다. 이제 사울은 마침내 에드문드의 방에서 나왔다. 굳은 표정에서 이미 진전이 거의 없는 것을 볼 수 있었다.

``어떻게 된 건가?''

엘사가 물었다.

``이런 모습은 전에 본 적이 없습니다.''

사울이 말했다. 매료된 듯하면서도 좌절감이 뒤섞인 목소리였다. 사울은 풀 수 없는 수수께끼를 좋아하던 적이 단 한 번도 없었다.

``독에 매우 천천히 중독됐던 것 같습니다. 최소 십 년 동안이요.''

독 이야기에 파비안은 늙은 하녀처럼 이러쿵저러쿵 물어보고 다녔지만, 엘사는 사나운 눈빛으로 입을 다물게 했다.

사울은 말을 계속했다.

``솔직히 전에도 똑같은 걸 본 적이 있습니다. 마법사가 한 짓 때문이라고 생각했었죠. 정확하게는 구스타프 형님이 한 거요.''

``그리고 그게 아니라는 거고?''

``맞습니다.''

사울이 말했다. 다시 얼굴을 찡그리고 있었다.

``이런 상태로 이렇게 오래 살아있던 게 이해가 되지 않는 정도입니다.''

``마르쿠스께 물려받은 것은 아니냐?''

``아버지의 병은 전투에서의 상처와 노환 때문입니다. 폐병을 더 악화시킨 거죠. 에드문드의 경우요? 에드문드는 온몸이 망가졌습니다. 이 상태로는\ldots\,오래전에 죽은 게 정상이죠.''

엘사의 시선은 문으로 갔다. 침대에 힘없이 늘어진 에드문드의 모습이 선했다. 평소의 활기는 침묵으로 바뀌었다. 아무것도 말이 되지 않았다. 고작 며칠 전까지는, 몇 시간 전까지는 활발하던 것이 말이다. 감기에 걸려도 이런 적은 전혀 없었는데, 이제는 죽은 게 정상이라는 말까지가 나왔다. 에드문드는 살아있을 것이다. 살아있어야 했다.

``할 수 있는 게 꼭 있을 것이다.''

엘사는 다시 말했다.

``무엇이 문제인지 알면 좀 나을 겁니다. 가장 비슷했던 건\ldots''

사울은 입을 다물었다. 갑자기 피가 빠져나간 듯 얼굴이 창백해졌다.

``뭔가?''

엘사가 물었다.

``아무것도 아닙니다.''

사울은 서둘러 답했다.

이토록 상태가 나쁜 것이었을까? 불치병 같은 것이리라, 아마도. 엘사는 다른 왕자들을 해산시켰다. 몇몇은 드디어 풀려났다는 모습이었고 몇몇은 진심으로 궁금해하는 모습이었다. 엘사는 문을 열고 들어가 자신을 가족으로 숨김없이 맞이해 준 이 형제에게 가까이 다가갔다. 에드문드는 생각한 대로 늘어져 있지 않았다. 계속되는 고통에 몸을 뒤틀고 있었다. 피부가 갈라지고 찢어졌다. 엘사가 오는 것을 알아채자 핏발이 서고 흐리멍덩한 눈을 팍 뜨고 눈을 마주쳤다.

``나한테서 떨어져!''

``에드문드\ldots?''

엘사는 가까이 가 어깨에 손을 올렸지만, 에드문드는 바로 손을 떼어내고 몸을 웅크렸다.

``괜찮은 거야\ldots?''

``나한테 떨어져, 떨어져, 떨어져, 떨어져, 떨어져, 떨어져, 떨어져, 떨어져\ldots''

에드문드는 중얼거렸다. 이때야 엘사는 알아챘다, 에드문드는 깨어있던 것이 아니었다.

``시간이 다 돼가, 시간이 더 필요해, 빛이 더 필요해!''

엘사는 뒤로 물러났다.

마르쿠스에게 말해야만 했다.

마르쿠스가 잘 있던 알현실로 가려 했다. 엘사는 이 곁채의 긴 복도를 걸어나갔다. 왕자들의 방을 하나씩 지나치고 있었다. 끝에 다다를 때쯤 첫 번째 방의 문이 살짝 열려 있는 것을 알아챘다. 여기에 들어가 있을 법한 사람이 없었다. 엘사는 조심스레 문을 열어 안을 엿보았다.

마르쿠스가 구스타프의 방에 서 있었다. 침대 기둥 하나를 꽉 쥐고 있었다. 엘사는 이해가 가지 않았다. 마르쿠스는 비탄에 잠긴 표정으로 허공을 바라보고 있었다. 아주 크게 몸을 기대며 전에 없던 약한 모습을 보였다. 온 세상이 무너져 내린 듯이. 엘사는 천천히 문을 밀어 열었다. 끼익하는 소리가 들리게 했다.

``에드문드가 깨어나지 않고 있습니다.''

엘사가 말했다.

``나도 안다. 하지만 할 수 있는 게 없지.''

``어떻게 아십니까?''

마르쿠스는 아무 말이 없었다.

``방법이 꼭 있을 것입니다.''

엘사가 말했다. 그러자 마르쿠스는 돌아섰다.

``거울이 저를 고칠 만큼 강력하다면 당신의 병도 고칠 수 있을 것이고, 그렇다면 에드문드도\ldots?''

``\ldots그렇다. 당연하지.''

마르쿠스는 나지막이 말했다.

마르쿠스는 바로 일어섰다. 잠깐, 아주 잠깐 시선이 첼리나의 초상으로 향했다. 거대한 혐오가 얼굴에 비쳐 엘사는 어렴풋이 그림을 불태울 것으로 생각했다. 하지만 그냥 지나갔다. 마르쿠스는 다시 빈 침대를 바라보고는 구스타프의 방에서 나왔다. 엘사에게 고개를 끄덕여 보였다.

``그렇다면 이제 시작할 때입니다.''

거울을 완성할 때였다.

