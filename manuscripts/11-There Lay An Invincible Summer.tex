

\chapter[11장  스러지지 않는 여름][11장\hspace*{.5em}스러지지 않는 여름]{11장 \ 스러지지 않는 여름}



봉인이 사라졌다.

이전에는 봉인이 풀리는 것을 느낄 수 없었어도, 자신이 서 있는 계단에서 엘사는 이제 알 수 있었다. 얼음이 녹아 방의 입구가 드러났지만, 웬일인지 직접 보게 되니 예상보다 더 좋지 않았다. 이제 엘사는 안나가 통과한 의심의 안개 너머를 알았다. 엘사는 자신이 보게 될 모습이 두려워 더 나아갈 수 없었다.

그러고는 비명이 들려왔다.

안나는 아니었다.

토비아스가 비틀거리며 방 안에서 나왔다. 검게 탄 손을 감싸 쥔 채 크게 떨며 비명을 지르고 있었다. 엘사는 토비아스가 자신의 성물을 강제로 가져가려 한 것을 알아챘다. 우둔한 자이다, 엘사의 능력을 탐내다니. 떨어져 있어도 엘사의 심장은 주인을 알고 있었다. 오직 엘사만이 접근할 수 있었다. 파괴될 수 없는 얼음으로 보호되고 있을 때에는. 엘사를 보자, 토비아스는 거의 넘어질 듯 잰 걸음으로 엘사에게 다가갔다.

``엘사 여왕님!''

토비아스는 무릎을 꿇고 주저앉았다. 동상이 팔로 올라오자 얼굴은 공포에 일그러졌다.

``어떻게 멈추는 것입니까? 말해 주십시오! 죽고 싶지 않습니다, 죽기 싫다고요, 제발—''

이럴 시간이 없었다. 엘사는 본인의 운명에 내버려두고 싶었다. 그러나 어찌 되었건, 토비아스는 여전히 마르쿠스의 자식이었다. 그리고 살려두는 것이 바로 토비아스 본인에 대한 징벌이었다.

손을 까딱하자, 얇은 얼음이 팔꿈치 아래를 베어냈다.

토비아스는 비명을 지르듯 입을 열었다. 흐리멍덩한 눈에 고통이 선명해도 아무런 소리가 나지 않았다. 남는 손은 잘린 곳을 감싸 쥐었다. 피 한 방울 떨어지지 않고 얼어붙었다. 토비아스는 온몸을 떨며 잘린 팔을 향해 조금씩 움직였다. 그러고는 극심한 고통이 찾아왔다. 눈이 뒤집혔고, 곧 쓰러졌다.

엘사는 뒤돌아보지도 않았다.

`안나야.'

엘사는 서둘러 방 안으로 들어섰다.

안나는 바닥에 쓰러져있었다. 의식을 잃었거나—엘사는 다른 것은 상상할 수도 없었다. 살아있어야 한다. 살아 있어야만 한다. 엘사는 바로 안나에게 달려가 옆에 주저앉았다. 그러나 막상 도착하니 무엇을 해야 할지 알 수 없었다. 엘사는 손을 뻗었다. 안나 곁에서만 느끼는 망설임으로 손이 떨렸다. 그러고는 손을 내려놓았다. 안나를 더욱 다치게 한다면?

눈앞에서 안나의 머리카락에 흰 줄기가 자라났다.

엘사가 속삭였다.

``안 돼. 안 돼, 안나야, 이렇게 날 버려둘 순 없어. 이제야\ldots\,이제야 감정을 느끼기 시작했는데, 제발, 제발\ldots''

외로움.

엘사는 이제 외로움이 자신을 괴롭히는 것임을 깨달았다. 마치 마음을 천천히 조르는 덩굴처럼. 엘사는 자신이 우위에 있다고 생각했다. 하지만 외로울 뿐이었다. 엘사는 자신이 무감정하다고 생각했다. 하지만 절망에 빠져있을 뿐이었다. 고통은 마음 안에 숨어 있었다. 매우 깊게 상처를 내 모든 빛깔이 빠져나갔다. 엘사의 삶은 빛바랜 회색 조로 되어 겨우 연명하고 있을 뿐이었다.

안나를 보기 전까지는. 안나의 세상은 빛깔로 가득했다.

``네가 있어야 해. 네가 필요해, 안나야. 일어나.''

안나는 몸을 거칠게 떨 뿐이었다. 엘사는 이 모습에 두려움을 내려놓고 안나를 팔로 안아 머리를 받쳤다. 하지만 지금까지 알던 안나의 느낌이 아니었다. 안나의 몸은 따뜻했다. 피부는 차갑고 머리카락은 희어진 이런 얼음장이 아니었다. 안나는 쾌활했다. 이렇게 창백하고 핏기없는 모습이 아니었다. 안나는 활기로 가득했었다.

``죽으면 안 돼.''

엘사가 말했다. 엘사는 안나를 가까이 끌어안았다. 이제는 자신의 성물이 안나의 심장에 맞추어 뛰는 것을 들을 수 있었다. 작은 조각이 튀어 나가 갈망하던 온기 안으로 박혀 들어간 것이다.

``대로 죽을 순 없어.''

엘사는 안나의 심장 위에 손을 올리고 은빛 얼음을 빼내려 했다. 그러나 엘사가 부르는 것에 응답하자—안나가 몸을 웅크렸다. 얼굴은 말로 표현할 수 없는 고통으로 일그러졌다. 엘사는 바로 멈추었다. 엘사는 조절할 수 없었다. 성물을 만들고 나서부터 엘사는 섬세한 조절을 할 수가 없었다. 억지로 빼내다가는 바로 안나를 죽게 할 수도 있었다. 엘사는 아무것도 할 수 없었다.

``어떡해야 해\ldots?''

얼음이 안나의 몸을 덮기 시작하자, 손끝이 파래지기 시작하자 엘사는 안나를 더욱 세게 끌어안았다. 안나는 빠르게 쇠약해져 갔다. 이대로라면 안나는\ldots

``안나야!''

손끝이 따끔거리는 것, 눈이 얼얼한 것. 이것이 상실감이라고 하는 것일까. 엘사는 차갑고 생기 없는 안나를 차마 볼 수 없었다. 하지만 눈을 뗄 수도 없었다.

엘사가 속삭였다.

``죽으면 안 돼. 내 허락 없이 죽을 순 없어. 일어나! 일어나, 당장!''

십삼 년 만에 처음으로, 엘사는 기적을 빌며 울음을 터뜨렸다.

뒤에서는 거울이 빛나고 있었다.

\textbreak

\forceindent``새로운 집에 온 걸 환영한다.''

서툴게 행동하고 위엄이 없어도 안나는 여전히 왕족이었고, 평생 아렌델 성에서 살아왔다. 서던 제도의 성을 이미 본 것은 말할 것도 없었다. 엘사가 완벽하게 만들어 놓은 것 말이다. 그래서 마르쿠스가 원래 성을 보여주었을 때, 안나는 흔히 보던 석재 건물과 구포를 잠깐 보고 나서 마르쿠스를 따라오고 있는 아이를 바라보았다.

안나는 왜 엘사가 감명을 받았는지 이해할 수 있었다. 왕국의 변두리에서 살아온 평민에게 이렇게 가까이에서 성을 바라보는 것은 꿈에서나 있을 일이다. 그리고 정말로 성 안에서 살게 되는 것은? 꿈이 아니라 환상이었다. 엘사는 입을 벌린 채 따라갔다. 아이다운 놀라움에 눈이 휘둥그레져 있었다. 이런 모습에서 나중에 될 차가운 여왕의 모습을 떠올리기란 매우 힘든 일이었다.

엘사가 물었다.

``저\ldots\,정말로요? 하지만 전 특별하지도 않은데요, 폐하—''

``더는 그런 식으로 말하지 말려무나, 엘사야.''

마르쿠스는 엄한 목소리로 말했다. 진지한 눈빛으로 엘사를 바라보고 있었다. 아주 겁에 질린 채로 고개를 끄덕이자 마르쿠스의 눈빛은 부드러워졌다.

``그리고 칭호를 붙이지 않아도 된단다. 이름으로 직접 불러도 돼.''

``그건 안 돼요! 무례한 거잖아요!''

엘사가 외쳤다.

``오늘부터 넌 원래 그래야 하던 대로, 왕족이야.''

더 말을 꺼내기 전에, 마르쿠스는 엘사를 안으로 데려갔다. 엘사는 자신은 상상도 할 수 없는 부에 말이 쑥 들어갔다.

안나는 엘사를 데려가는 마르쿠스를 따라 들어갔다. 마르쿠스는 알현실로 갔다. 안나는 엘사에게 과하게 잘 대해준다는 망상에 거의 빠져드는 듯했지마는 이 친절하고 자애로운 사람이 미심쩍었다. 엘사에게 감당할 수 없는 일이 일어났을 때 우연히도 나타나 수습해주었다니? 미래의 엘사는 항상 존경을 보이며 마르쿠스를 언급했다. 그러나 이에도—혹은 이 때문에—안나는 반신반의하며 상황을 지켜보았다.

대문이 열리자 왕자 열세 명이 이들을 맞았다. 엘사는 부끄럼을 타고 있었다. 사람들을 마주하는 데에 익숙하지 않았고, 해를 주는 것을 두려워하고 있었다. 하지만 마르쿠스는 당당하게 엘사와 함께 홀 앞으로 걸어갔다. 엘사가 단 위로 올라오도록 불리어 마르쿠스의 오른쪽에 바로 서자, 호기심 어린 눈 열세 쌍이 엘사를 바라보았다. 이들도 맛볼 수 없는 영광이었다.

``엘사를 소개하마. 오늘부터 이 아이는 서던 제도의 엘사 공주다.''

마르쿠스가 말했다. 왕자들이 소곤거리며 투덜대는 것은 무시하고 있었다. 마르쿠스는 말을 계속했다.

``그리고 너희는 집에 온 것처럼 환영해 주어라. 이제 자기소개해라.''

한 명씩, 안나는 이들의 자기소개를 지켜보았다. 구스타프는 이미 이십 대였고, 이 모든 일이 당혹스러운 듯했다. 그러나 몇몇은 그리 친절하지 않았다. 토비아스는 벌레라도 본 듯한 표정을 짓고 있었다. 안나는 마르쿠스가 꾸중하는 것을 보고 기분이 좋아졌다. 대부분은 대개 관심 없는 듯했다. 그래도 안나는 어린 에드문드가 매우 기운차게 자기소개하는 것을 재미있게 보았다. 엘사는 거의 떨면서 고개를 끄덕였다. 한스는 역시 차분했다.

사울은 본인의 말대로 보자마자 빠져들어 있었다. 안나는 사울이 말을 더듬으며 자기소개하는 것에 점점 얼굴을 찡그렸다. 엘사가 이 아이를 보고 미소를 짓자 안나는 더욱 불편해졌다. 엘사는 안나에게는 이렇게 웃어주지 않았다. 사울은 미소를 지어 보였다. 안도된 듯했다. 그러고는 열 번은 넘게 연습한 듯 점잖게 인사하며 물러났다.

``너 때문에 불편해하잖아, 사랑꾼아.''

안나가 중얼거렸다.

기억은 마르쿠스가 엘사에게 성을 구경하게 해 주라고 말하는 데에서 멈추었다. 안나는 성물이 다른 기억으로 데려갈 때까지 침착히 기다렸다. 장면이 돌아오자, 안나는 이 온 고난을 너무 편안해 하는 것이 아니냐고 생각했다.

많은 시간이 지나지는 않았다. 엘사는 전과 같은 나이인 듯했다. 교실 뒤에 앉아 있었다. 안경을 낀 노인이 가르치는 강의를 매우 지루해하는 듯했다. 안나는 엘사가 등을 꼿꼿이 하고 앉아있는 것을 알아챘다. 엘사는 다른 아이들이 할 법하게 턱을 괴고 있지 않았다. 그렇지 않고 고개를 들고 꼿꼿이 앉은 채 바로 바라보고 있었다. 엘사는 이미 왕족이 되도록 교육을 받고 있었다. 그래도 안나는 엘사의 눈이 침침한 것을 볼 수 있었다. 엘사가 허벅지를 꼬집으며 깨어 있으려고 하는 것을 보고 미소를 지었다.

쉬는 시간이 되자, 엘사 나잇대 되는 다른 아이들은—일곱 번째 왕자 라파엘부터 에드문드와 한스까지—바람을 쐬러 빠져나갔다. 엘사는 뒤에서 고개를 두어 번 젓고는 두꺼운 교과서를 읽어나갔다. 내내 한 손으로 관자놀이를 문지르고 있었다. 몇 분이 지나자 교과서를 밀어놓았다. 꺼림하게 주변을 둘러본 후, 엘사는 손을 천천히 움직였다. 파란 안개와 같은 것이 손가락 주변에 감돌았다.

그러고는 무언가를 만들어내었다.

책상에는 바로 성의 모형이 나타났다. 기단에서부터 점점 올라가 뾰족탑이 만들어졌고, 창문 하나하나와 문과 장식까지도 나타났다. 어린아이가 이토록 정교한 것을 만들어내는 것이 아무래도 더욱 인상적이었다. 그러다가 안나는 엘사가 더 나이가 들었을 때 이런 일을 하는 것을 전혀 보지 못한 것을 깨달았고, 엘사 본인이 말했듯이 더는 이런 것을 할 수 없는 것인가 하는 의문이 들었다.

안나는 매우 열중하고 있어서 누군가가 다시 들어오는 것을 알아채지 못했다. 엘사도 그런 듯했다. 목소리가 들려오자 엘사는 깜짝 놀랐다.

``그거 정말 예쁘네.''

엘사는 바로 멈추었다. 성이 만들어지는 것이 중단되었다. 아찔해 하며 겁에 질린 눈은 벌을 받는 것을 생각하고 있는 듯했지만, 엘사의 눈앞에는 솔직한 모습으로 감탄하며 바라보고 있는 사울이 있었다.

``뭐\ldots\,뭐라고?''

``건물 설계 좋아하는구나.''

사울은 놀란 채 말했다. 가까이 다가가 얼음 모형을 살펴보고 있었다. 엘사는 예상하지 못한 칭찬에 크게 미소를 짓다가 다시 무표정으로 돌아갔다. 그러나 아직 완전하지 못한 가면은 사울의 말에 깨어졌다.

``이거 끝부분까지 다 세세하게 돼 있네. 어떻게 한 거야?''

``그냥 머릿속으로 그리는 건데.''

엘사는 무심코 말했다.

``그러니까, 고정된 거로 시작하지 않는다고. 생각이 떠오르면—''

``스스로 설계되게 두는구나.''

사울이 말을 끝맺었다. 엘사와 똑같이 활짝 미소를 짓고 있었다.

``영감이 떠오르면 거침없이 나오는 거지.''

``맞아! 수학하고 기하학투성이이긴 해도, 규칙에 매일 순 없지!''

엘사는 신이 나 거의 달아오르는 듯했다. 목소리도 높게 올라가 있었다. 이를 알아채자 엘사는 손으로 입을 가렸다.

``어, 미안해. 나도 모르게 흥분해 있었네.''

``아니, 이해해. 나도 설계하는 거 좋아해. 물론 작은 거지. 너만큼 큰 건 아니고, 너만 한 아름다움은 아주 조금밖에 없고.''

사울이 말했다. 안나는 사울을 노려보았다. 사울은 어려서도 극적인 데가 있었다.

그러나 안나는 엘사의 반응에 매우 놀랐다. 엘사는 얼굴을 붉히며 고개를 돌렸다.

``네가 만든 거 몇 개 봤어. 정말 멋지던데. 특히 배는! 넌 정말로 손을 쓰잖아. 난 반칙 쓰는 거 같아.''

사울은 손을 뻗어 손톱으로 얼음 성을 톡톡 두드렸다. 음악과 같은 소리가 나는 것을 듣고 나서 다시 말을 꺼냈다.

``특별한 능력이 있는 건 반칙이 아냐. 쓰지 말아야 하면 그게 재능이게? 난 네 능력 반이라도 있으면 좋겠는데.''

``지금은 에드문드의 재능이 정말 부러워.''

엘사가 말했다. 교과서를 바라보며 씁쓸하게 미소를 짓고 있었다.

``난 정말 꼴찌라니까\ldots''

``완전기억력 말이야? 잠깐 보는 것만으로 기억하는 건 정말 좋은 능력이긴 하지.''

사울이 말했다.

``도서실에 있는 책은 다 봤을걸.''

안나는 눈썹을 추켜세웠다. 지금까지 알던 에드문드와는 전혀 달랐다. 중간에 머리를 다치기라도 한 것인가.

``그래도 다들 재능이 있지. 라파엘은 음악에, 알렉은 검술에. 그러니까 네 능력도 크게 다르진 않아.''

사울이 말했다. 교과서를 뒤에 슬쩍 밀어놓고 있었다.

``그럼 이젠 이거 정복하는 걸 열심히 도와줘야겠네. 차라리 수학이 낫지. 역사는 단순히 있던 일만 늘어놓은 게 아니면 덜 지루할 텐데\ldots''

기억이 회색으로 바뀌자 안나는 얼굴을 찡그렸다. 더는 사울이 싫지 않았다. 그러나 이대로는 다시 사울이 야비하다는 생각으로 돌아갈 수도 있었다. 자기 생각이 어디까지 이어지는지를 알아채고, 안나는 엘사에 대한\ldots\,이상한\ldots\,것들이 계속 떠오르자 크게 끙하는 소리를 냈다.

기억의 단편이 조금씩 보였다. 시간이 빠르게 갈수록 기억 각각은 일 분도 채 보이지 않았다. 그래도 엘사의 삶에 사울은 항상 한결같았다. 어릴 때, 사울은 식사 때에 수많은 식기를 바르게 쓰는 법을 가르쳐주었다. 더 나이가 들자, 사울은 설계도를 그리고 엘사는 설계를 현실로 만들어내었다. 둘 다 매우 들떠있었다. 안나는 십 대가 된 엘사가 그 누구도 아닌 사울과 `춤추는 법'을 배우는 것에 괴로워졌다. 엘사가 마침내, 마침내 한 번 발을 헛디디자 둘은 초조하게 웃으면서 약간 과하게 웃고 있었다.

안나는 여전히 엘사가 행복한 것을 보아서 좋았다.

그러고는 다음 기억이 시작되었다.

``드러내\ldots''

``자유롭게 해\ldots''

``한계는 없어.''

엘사는 말을 끝냈다.

엘사는 안마당에서 마르쿠스와 함께 있었다. 마르쿠스는 엘사 뒤에 서 있었다. 엘사는 깊게 숨을 들이마시면서 허수아비를 바라보고 있었다. 자신이 되뇌던 말과는 다르게 엘사는 주저하며 손을 벌렸고, 목표물은 맞힌 눈은 목표물을 거의 넘어뜨리지도 못했다. 안나는 마르쿠스가 매우 크게 얼굴을 찡그려 이마에 주름이 패는 것을 눈치챘다.

``너 자신을 자유롭게 하고 있지 않잖느냐.''

마르쿠스가 꾸짖고 있었다.

``힘을 두려워하지 마라. 받아들여라. 네 안에는 엄청난 힘이 있다, 엘사야. 네가 풀어주기만 한다면.''

``전 누구도 다치게 하기 싫어요.''

엘사가 말했다.

``그러면 틀림없이 남들이 널 다치게 할 거야. 이 세상은 모질고 매정하고 이해할 수 없는 건 뭐든지 파괴하려 들지. 이미 경험이 있잖니.''

안나는 이 우롱하는 듯한 말에 분노가 끓어올랐다. 어떻게 자신의 말을 증명하려고 엘사의 학대당한 기억을 끄집어내는가. 게다가 대체 무슨 소리를 하는 것인가. 그저 몇몇 사람이 그렇다는 것으로 온 세상이 모두 적대적인 것은—

하지만 효과가 있었다. 엘사는 허수아비를 향해 돌아섰다. 눈빛은 냉정해졌다. 손끝에서 나온 하얀 흐름은 이번에는 나무 받침대를 산산조각냈고, 지푸라기는 마구 흩뿌려졌다. 그러나 이에도 흐름은 멈추지 않았다. 앞으로 나아가면서 더욱 강력해져 마침내 우레와 같은 소리를 내며 성벽에 부딪쳤다. 눈이 가라앉자, 안나는 땅이 크게 팬 것을 보았다.

``이제 뭘 할 수 있는지를 똑똑히 보았겠지. 그리고 이건 시작에 불과해.''

마르쿠스가 말했다. 만족한 듯 미소를 짓고 있었다. 엘사는 깊게 숨을 들이마신 다음 자신이 벌인 일을 보고 다시 자신의 손을 바라보았다. 자신이 이렇게 큰 힘을 불러낼 수 있는 것이 믿기지 않는 듯했다.

흐릿한 장면들로 몇 해가 지났다. 각각은 짧은 순간 동안만 보였다.

이 중에 안나가 알아챌 수 있는 것은 엘사의 마법이 점점 강력해지는 것이었다. 첫 번째 기억에서부터 점점 강력해질수록, 엘사는 더욱더 많은 것들을 해내었다. 엘사는 마르쿠스에게서 자신의 온 힘을 쏟아내는 법을 배웠다. 안나는 눈보라와 거대한 눈사태와 수많은 얼음 창을 만들어내는 법을 배우는 것을 보았다. 그러던 어느 날\ldots

훨씬 더 위엄 있고 당당한 엘사가 서 있었다. 안나 정도 키에 두 배는 더 기품이 있었다. 엘사는 안마당 가운데에 홀로 서 있었다. 눈을 감은 채 침착하게 숨을 들이 내쉬고 있었다. 날카로운 호각 소리가 나자 사방에서 쇠뇌 화살이 쏟아졌다. 안나는 다치지 않는다는 것을 알고 있어도 화살을 막으려고 본능에 따라 손을 올렸다. 엘사는 눈을 팍 뜨고 호를 그리며 손을 휘둘렀다. 얼음을 뿜어 주변을 둘러싸 화살을 낚아챘다. 엘사는 발을 굴렀다. 화살은 반대 방향으로 발사되어 발사 장치로 다시 날아갔다. 화살 각각은 거뜬히 쇠뇌를 맞추었다.

``검사들 나오너라!''

마르쿠스의 명령에 무기를 든 병사 오십 명이 뛰쳐나왔다. 안나는 놀란 채 지켜보았다. 십 대가 된 엘사가 얼음으로 된 칼을 꺼내어 병사들과 맞서고 있었다. 늘어서 있는 병사들을 손쉽게 헤쳐 나아가며 베어버리고, 공격을 받아내고, 다시 찔렀다. 싸우기보다는 춤을 추는 것처럼 보였다. 엘사는 손 하나 까딱하지 않고 병사들을 이길 수 있었다. 엘사는 빠르게 움직였다. 사람이 움직이는 속도로는 보이지 않았다. 그 자리에서 만들어내는 센 바람을 타고, 따라 움직이는 얼음으로 된 바닥 위에서 그 누구도 따라잡을 수 없는 속도로 재빨리 돌고 있었다.

훈련된 병사 오십 명과 소녀 한 명 간의 싸움.

끔찍할 정도로 일방적이었다.

빙산이 발밑에서 솟아올랐다. 빙산은 여전히 서 있는 병사들 위로 엘사를 올려주었다. 이 모습에 안나는 이상한 생각이 스쳤다. 빛나는 태양 앞에서 평온한 모습으로 눈을 감은 채 빙산 위에 우뚝 서 있는 모습에, 엘사는 여신처럼 보였다.

잠깐의 모습이었다. 곧 엘사의 표정은 무자비한 분노로 일그러져 여신의 모습은 악마의 모습이 되었다. 뜨인 눈은 궁지에 몰린 동물과 같이 원초적인 격분을 내보이고 있었다. 맑은 하늘에 눈이 날리더니 곧 눈보라가 몰아쳤다. 안나는 슬픈 생각이 들었다. 지금과 같이 여전히 어릴 때에 마르쿠스는 엘사에게 혐오를 가르쳤다. 그것도 매우 잘 가르쳤다. 엘사의 손에서 전기가 흘러나왔다. 청백색 벼락이 창처럼 던져져 땅을 그슬리며 파고들었다. 번개와 얼음의 흐름이 큰 소용돌이가 되었다. 맹렬한 폭풍에 맞설 수 있는 이는 아무도 없었다.

엘사는 다시 땅으로 내려온 다음 모든 병사가 무력화되자 손에 쥔 칼을 깨트렸다. 마르쿠스는 바라보고 있던 발코니에서 천천히 손뼉을 쳤다. 엘사는 미소를 지으며 자부심을 띤 채 위를 올려다보았다. 미소는 양 귀에 걸릴 듯했다.

``하지만 한 명도 죽이지는 않았구나.''

마르쿠스가 말했다.

이것이야말로 놀라운 소리였다. 그러나 다행스러운 소리였다. 안나는 병사들을 자세히 살펴보았다. 다행히도 모두 여전히 숨이 붙어있었다. 몇몇은 정신을 잃었고, 몇몇은 부상에 신음하고 있었지만 살아있었다. 엘사에게 최소한 이 정도의 억제력은 아직 남아 있었다.

``꼭 그럴 필요는 없다고 생각했습니다.''

엘사가 말했다. 안나는 말투가 바뀐 것에 놀랐다. 매우 자신감에 찬 말투였다. 현재의 엘사와 같았다. 유순한 평민 여자아이의 모습은 사라졌다.

``다들 너보다 아래라는 걸 잊지 마라. 네 앞에 쓰러지는 건 능력이 없어서야.''

마르쿠스가 말했다.

``가르쳐 준 건 벌써 잊어버린 것이냐?''

엘사는 고개를 저었다.

``드러내. 자유롭게 해. 한계는 없어.''

또다시 장면이 바뀌었다.

이번에는 안나의 눈앞에 탑이 펼쳐졌다. 안나는 잠깐 현재로 돌아왔다고 생각했다. 하지만 아니었다. 엘사와 마르쿠스도 같이 있었다. 엘사는 약간 나이를 먹었다. 이제는 안나보다 키가 크고, 흠 잡을 데 없는 모습이었다. 마르쿠스가 거울로 힘겹게 발걸음을 옮기는 동안 엘사는 두 손을 가지런히 한 채 바라보며 서 있었다.

마르쿠스가 점점 늙어가는 것이 훤히 보였다. 짧은 머리카락은 잿빛이 되었고, 이마는 항상 보던 주름이 져 있었다. 눈은 움푹 꺼지기 시작했다. 눈가는 벌게져 있고, 눈동자는 빈 동굴과 같이 공허했다. 여전히 틀림없는 권력의 풍채를 지니고 있었지만, 세월의 흔적은 가릴 수 없었다.

``엘사야, 이것이 나를 위해 완수해야 할 일이다.''

마르쿠스가 말했다. 원래의 부드럽고 나긋나긋한 목소리는 온데간데없었다.

``이 거울, 이건 오래된 전설이다.''

``말씀을 듣고자 합니다.''

엘사가 말했다.

``옛날 옛적에 달빛이 떨어져, 그 빛을 받은 작은 연못을 얼렸다는 전설이 있다. 그게 거울이 되었지. 강력하고 불가사의한 것이지. 거울은\ldots\,마음속 깊은 소망을 비춘다는구나.''

마르쿠스가 말했다. 이것이 자신을 살아나게라도 하는 듯 매우 기뻐하는 말투였다.

``이게 바로 그 거울이란다.''

마르쿠스는 사랑스러운 손짓으로 가장자리를 훑었다. 그러고는 몸을 굽히며 피 섞인 기침을 했다.

``마르쿠스—''

마르쿠스는 자존심이 있었다. 손을 올리자 엘사는 말을 멈추었다.

``하지만 보다시피, 거울은 성하지가 않지. 그래도 파괴된 것도 아니야. 조각이 나 있을 뿐이지. 거울 조각은 세계 곳곳에 있단다. 더 작은 파편은 사람의 심장에 박혀 있기도 하지. 한때는 거울을 완성하려 했지만 난 너무 늙었구나, 엘사야. 네게 이 임무를 맡기마. 보렴! 네 힘에 반응하잖니.''

마르쿠스의 손짓에 엘사는 앞으로 손을 뻗었다. 거울의 가장자리에 새겨진 무늬에서 빛이 새어나왔고, 틀은 천천히 돌기 시작했다. 둔탁한 소리를 내며 돌다 멈추다 하고 있었다. 이들 앞에 완성된 거울의 모습이 나타났다. 모양이 계속 변하는 눈송이 문양이 중심에서 돌고 있었고, 파란색 불은 눈송이의 맥을 따라 타올랐다. 손을 내리자 거울에서 나는 빛은 어두워지고, 상은 사라졌다. 엘사는 이게 무엇이냐는 듯 마르쿠스를 바라보았다. 마르쿠스는 답을 주지 않았다.

``난 곧 죽게 될 거야. 그러니 네게 거울을 완성할 임무를 맡기마.''

``그렇지만 제가 살릴 방법이 있다고 하셨잖아요!''

엘사가 말했다.

마르쿠스는 고개를 끄덕였다. 부드럽게 미소를 짓고 있었다.

``지금 상태로도, 거울로 내 목숨을 부지할 순 있지. 네 힘을 합친다면. 내 몸을 얼려두면, 거울에 충분한 힘이 있을 때 날 살릴 수 있지. 하지만\ldots''

``할 수 있어요. 살려드릴 수 있습니다.''

엘사는 서둘러 말했다.

``지금 넌 이 일을 할 정도로 강하지가 않단다.''

마르쿠스가 말했다. 주저하며 몸을 조금 돌려 엘사의 시선을 피하면서, 마르쿠스가 중얼거렸다.

``다른 방법이 있기는 하지만\ldots''

엘사는 앞으로 나아갔다.

``무엇이든지 하겠습니다.''

마르쿠스는 쓴 미소를 지었다.

``아니다, 얘야. 내가 감히 바라기에는 과한 희생이야.''

``과한 희생 같은 건 없습니다.''

엘사가 말했다.

안나는 마르쿠스가 무슨 말을 하려는지 감이 왔다. 토비아스가 말한 대로, 성물은 마르쿠스를 살려두고 있었다. 안나는 마르쿠스의 잔인한 생각에 주먹을 꽉 쥐었다. 최악의 상황에 엘사를 가지고 놀면서 꺼리는 척을 하고 있었다. 그렇지 않으면 애초에 왜 이런 말을 꺼냈을까.

``성물을 만든다면 넌 강해지겠지. 마치 빛을 모으는 렌즈처럼, 거울에 날 살려 둘 힘을 줄 거야.''

마르쿠스는 고개를 저었다.

``하지만 내가 바라기에는 과한 일이지. 성물을 만든다는 게 무슨 소리인지는 알겠지.''

``넘어가면 안 돼요.''

안나가 중얼거렸다. 안나는 이 두 사람 사이에 서서 엘사의 어깨를 잡으려 했지만, 손은 엘사를 통과할 뿐이었다.

``넘어가면 안 돼! 속임수라고요—''

``하겠습니다.''

엘사는 바로 말했다.

``얘야—''

``저를 거두어주신 분의 목숨보다 소중한 것은 없습니다.''

엘사는 말을 끊고 말했다. 마르쿠스는 아주 거짓 없는 슬픈 눈으로 엘사를 바라보았다. 안나는 이 모습에 화가 났다.

``성물을 만드는 것에 결점은 없는 것 같습니다. 목숨을 살릴 수 있다면, 전 하겠습니다.''

마르쿠스는 고개를 저었다.

``넌 모르는 게 많단다, 엘사야.''

``전 알고 있습니다. 성물은 제 감정을 억누릅니다.''

엘사는 눈을 감았다. 입을 떼자, 엘사의 목소리는 약했다.

``그게 필요합니다. 과거의 기억을 잊으려면 그게 필요합니다. 아직도 그 생각이 납니다. 부디, 자유롭게 되려면 그게 필요합니다.''

안나는 이제야 왜 엘사가 이런 결정을 내렸는지를 이해했다. 엘사는 자신의 부모와 지낸 기억을 억누르고 싶었다. 학대당한 기억을. 나약할 때의 기억을. 멀리 치워놓으면 아무 일도 없던 체할 수 있다는 듯이. 이 욕망에서 안나가 알던 엘사가 태어났다. 자신을 사람들에게서 아주 멀리 떨어뜨려 놓아 인간성을 벗어버린 엘사 말이다.

``그럴 가치가 없다고요.''

안나가 속삭였다. 떨리는 손은 엘사의 볼 위에 떠 있었다.

``여왕님, 제발. 이러면 안 돼요, 후회할 거라고요—!''

마르쿠스는 고개를 끄덕였다. 주문을 외자 손에 검은 칼이 나타났다. 마르쿠스는 칼을 엘사에게 건네주고 나서 크게 한숨을 쉬며 시선을 피했다. 마치 비탄에 잠긴 듯이.

엘사는 칼을 받아 자신의 심장 위에 올려놓았다. 안나는 공포에 질린 채 꼼짝하지 못하고 있었다. 시선을 돌리고 싶지만 움직이지 않았다. 안나는 계속 되뇌었다.

``안 돼요, 여왕님. 안 돼, 안 돼\ldots''

``고맙습니다.''

엘사가 말했다.

``멈춰요!''

엘사는 칼을 몸 안 깊숙이 박았다.

\textbreak

엘사는 제어할 수 없었다. 마르쿠스를 살려두려고 엘사의 성물은 균형을 위해 온 서던 제도를 취했다. 그리고 영원한 겨울이 시작되었다.

그러나 엘사는 제어할 수 있다는 모습을 보여야 했다. 모든 것이 자신의 의지에 따른 것이라는 모습을 보여야 했다.

도도한 인물이 거룩한 자태로 아래를 내려다보고 있었다.

``오늘부터 나는 엘사 여왕이다.''

그러나 엘사의 눈에는 아무것도 없었다. 오직 거대한 겨울 폭풍만이 자리했다. 거룩하게 되는 것은 외롭게 되는 것이었다.

\textbreak

안나는 비명을 지르며 깨어났다.

``안나야!''

눈앞의 흐릿한 모습에서, 안나는 누군가가 자신을 흔들고 있는 것을 어렴풋이 알아챘다. 천천히 초점이 잡혔다. 안나는 엘사가 옆에서 가까이 끌어안고 있는 모습을 보았다. 얼굴은 공포에 질려있었다. 엘사 여왕이—겁에 질린 아이나 절망에 빠진 십대 소녀가 아닌 눈의 여왕 엘사가 안나를 껴안은 채 울고 있었다.

``\ldots여왕님?''

엘사는 미소를 지었다. 안나를 꼭 안으면서 울음과 웃음의 중간쯤 되는 소리가 입 밖으로 나왔다. 언제든 사라지기라도 할 듯 껴안고 있었다.

``아직 있구나, 아직 여기 있구나.''

안나는 엘사를 껴안아주었다. 엘사의 어깨에 얼굴을 파묻고 얼음 냄새를 맡고 있었다. 그러나 해야만 하는 말이 있었다. 마지막 남은 힘을 짜내 안나는 고개를 들었다.

``왜죠?''

안나는 조용히 물었다. 엘사는 그대로 움직임을 멈추었다. 안나는 이제 더는 진짜로 숨을 쉴 필요가 없다는 것을 알아챘다. 안나의 몸은 움직이지 않았다. 돌과 같이 굳어져 있었다.

``왜 그런 짓을 한 거예요?''

``\ldots봤구나.''

``전부 다요.''

안나는 엘사의 심장 위에 손을 올려놓았다. 엘사는 눈으로 안나의 손을 좇았다. 안나는 엘사가 느끼는 모든 것을 느낄 수 있다는 것을 깨달았다. 엘사는 확실히 하기를 필사적으로 원했다. 심장을 희생한 자신의 결정이 옳은지를. 엘사는 확신을 원했다. 자신의 선택이 잘못된 것이 아닌가 하고 두려워하고 있었다. 마음의 이면에 있는 생각이었다. 혹은 마음 깊숙한 곳에. 이것이 더 정확한 말일 것이다.

``자유로워져야 했어.''

엘사가 중얼거렸다.

``잊어야 했어. 그리고 효과가 있었다고, 안나야. 성물을 만드니 세상이 더 뚜렷이 보였다고. 생각을 누그러뜨리고 감정을 조절할 수 있었다고—''

``감정을 없애버리는 건 조절하는 게 아니라고요.''

안나가 말했다.

``여왕님은 여왕님 자신을 속이는 거예요.''

``감정을 안 느끼고 싶다면?''

엘사가 물었다. 그러나 흔들리는 눈은 생각을 훤히 드러내고 있었다. 눈보라는 점점 강해졌다.

``인간이 된다는 건 고통을 느낀다는 거야. 더는 나약한 아이가 되기 싫었다고. 다른 선택이 있으면, 차라리 무정해지는 걸 택하겠어.''

``그렇지만 행복하지가 않잖아요.''

엘사는 아무 말도 하지 않았다. 그러나 안나는 엘사가 피해 가게 두지 않았다. 안나는 계속 눈을 바라보며 말없이 있었다. 엘사의 대답을 기다리고 있었다. 엘사는 드디어 머리를 흔들었다.

``너무 늦었어, 너무 늦었다고, 안나야! 끝난 일은 이미 끝난 거야. 결정할 때 이미 알고 있었어. 절대 되돌릴 수 없는 거.''

``그럼 왜 그런 선택을 한 거예요? 왜 다른 생각을 해 보지도 않은 거냐고요?''

안나가 물었다.

``다른 선택을 할 수 있었다고요. 행복해질 수도 있었다고요. 이럴 필요는 없었—''

``더 일찍 만났다면.''

엘사가 말했다. 입꼬리가 올라가 엷고 씁쓸한 미소를 짓자, 안나는 가슴이 찢어지는 듯했다.

``널 더 일찍 만났으면 이런 거에 의지할 필요도 없었잖아.''

``\ldots하지만 이제 제가 여깄잖아요.''

안나가 말했다. 엘사는 눈을 감았다.

``망가지지 않았어요, 여왕님. 망가지지 않았다고요. 직접 말했잖아요. 다시 감정을 느끼기 시작했다고.''

``그걸 들었어?''

``틀렸나요?''

엘사는 다시 천천히 고개를 저었다. 바깥은 폭풍이 몰아치고 있었다. 바람은 얼음과 눈을 막을 수 없는 눈보라로 몰고 있었다. 엘사는 두려웠다. 다시 감정을 느끼는 것이 두려웠을까? 아니다. 안나는 눈에 비친 모습에서, 자신의 마음에서 느낄 수 있었다. 엘사는 이를 그 무엇보다 원했다. 그러면 대체 무엇 때문일까.

``이제 내가 어떤 사람인지를 아니까, 내가 한 일을 아니까\ldots''

엘사가 입을 떼었다. 목소리는 떨렸다. 안나는 말이 끝나기도 전에 무슨 말을 하려는지 알아챘다.

``내가 무섭니?''

안나는 눈을 깜빡이지도 않고 엘사를 바라보고 있었다. 아주 가까이에서, 이들은 말없이 서로를 바라보았다. 두 사람이 세상에 남은 마지막 사람처럼 보였다. 다른 사람 따위는 생각할 겨를도 없었다.

``아뇨, 안 무서워요.''

안나가 말했다.

우레와 같은 소리와 함께, 폭풍은 마침내 가라앉았다.

