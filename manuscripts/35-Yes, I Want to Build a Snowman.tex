

\chapter[35장  그래, 같이 눈사람 만들자][35장\hspace*{.5em}그래, 같이 눈사람 만들자]{35장 \ 그래, 같이 눈사람 만들자}



안나는 상쾌하게 아침을 맞았다. 하지만 무언가 이상한 것이\ldots

안나는 자신의 방에 있었다.

집에 돌아왔다.

그 어느 아침때보다도 활기차게 담요를 젖히고 안나는 양탄자 깔린 바닥에 발을 내딛고는 제자리에서 한 바퀴 돌았다. 남겨 두고 떠난 친숙한 집의 냄새에 숨이 멎을 정도로 놀라고 있었다. 시장에서 구한 오래된 물건들도 책상 위에 그대로 있었다—안나는 행운의 부적으로 쓰는, 식탁에서의 멋 부리기로 쓰이는 예스러운 쌍 유니콘 나이프 받침대, 붉은 새 문양으로 꾸며진 낡은 방패와 오로라 아래에 있는 친절한 트롤이 있는 못 그린 그림이었다. 아렌델에 전해지는 옛이야기에 관한 것이었다.

안나는 욕실로 가 기운을 차렸다. 자신의 아버지를 볼 생각에 들떠있으면서도 도착하기도 전에 잠이 든 것이 부끄러웠다. 생각해 보니 분명히 엘사가 자신을 여기로 데려왔을 것이다. 안나는 모두가 엘사에게 달라붙어 있던 자신의 모습을 봤을 생각에 얼굴을 붉혔지마는, 엘사 자신이 누군가를 업고 있는 모습을 남들에게 보인 것에 마음이 따뜻해졌다. 안나는 다시 진정이 될 때까지 얼굴에 물을 끼얹었다.

다시 방으로 돌아가자, 문 너머에서 수군대는 목소리가 들려왔다.

``폐하, 안나 공주는 아직 자고 있습니다.''

게르다가 말했다. 안나는 자신의 오랜 보모의 목소리에 꺅하는 소리를 냈다.

``적절한 일이 아닐—''

``안나를 위하는 것은 잘 알지만, 걱정은 필요치 않다.''

엘사의 목소리일 것이다—안나는 생각했다. 엘사 외에는 이렇게 권위적인 말투를 쓰는 사람이 없다. 문이 열리자 안나는 씩 미소를 지었다. 엘사는 큰 걸음으로, 게르다는 놀란 채 뒤처져 들어오고 있었다.

``깨어있다고 말했잖느냐.''

엘사가 말했다.

``좋은 아침, 게르다!''

안나는 손을 흔들어 보였다.

``조\ldots\,좋은 아침이에요, 안나 공주님.''

게르다는 눈을 끔뻑이며 엘사에게 돌아섰다. 매우 놀라 존칭을 쓰는 것도 잊었다.

``대체 어떻게—''

``별 상관 없잖아.''

안나는 말을 가로막고는 팔을 크게 벌려 게르다를 껴안았다. 게르다가 바로 껴안아주는 것에 미소를 짓고 있었다. 바로 어제 만났던 것처럼 생생했다.

``다시 봐서 정말 좋아! 그리고 진짜로 좋아 보이고. 머리 새로 한 거야?''

``그렇고 말고요!''

게르다의 미소는 숨김없었다. 이까지 드러나 있었다. 행복과 안도만이 섞인 미소였다. 속셈이나 자신을 안심시키려는 거짓 꾸밈도 없었다.

``못 알아챌 줄 알았어요. 얼굴이 좀 갸름해 보이나요?''

``진짜 그래 보여! 난 뭐 달라진 거 없어?''

안나가 물었다.

``좀 키가 컸나요?''

``게르다, 나 열여덟이야. 이제 더 클 일은 없다고.''

``그런 게 어딨어요.''

게르다가 말했다. 혀를 차고 있었다.

``사람은 계속 크는 법이라고요! 위로 크는 게 아니라 옆으로 크지만요. 그러니까 초콜릿 좀 줄이는 게\ldots''

``그게 누구 탓이더라?''

``뭐, 그 살이 다 공주님 원하는 데로 갔으면 좋겠네요.''

게르다가 말했다. 안나는 미간을 찡그렸다. 이해를 못 하는 중에 이 하녀가 거의 킬킬대다시피 하며 두 손을 배 위쪽으로 올리는 것이 보였다.

``게르다!''

안나는 얼굴을 붉혔다.

엘사는 헛기침을 했다.

``말을 자르기는 싫지만, 안나와 이야기를 하고자 한다. 좀 비켜주지 않겠는가, 게르다?''

게르다가 깜짝 놀라는 모습에 안나는 웃음을 참았다. 엘사도 같은 자리에 있는 것을 잠깐 잊은 듯했다. 공손히 인사를 해 보인 다음 게르다는 문밖으로 나갔다. 문을 살살 닫는 것으로 볼 때 안나는 게르다가 걱정하고 있는 것을 알 수 있었다. 처음 보는 모습은 아니었다. 안나는 어떻게 정확히 모두가 서던 제도에서의 자신의 처우를 생각하고 있는지가 궁금했다. 자신을 집에 데려다주고, 심지어는 업어다 주기까지 했으니 엘사가 그들이 생각하는 괴물이 아니라는 것이 충분히 인정되어야겠지만, 오랜 습관은 쉽게 사라지는 것이 아니다.

둘만이 남자 엘사는 안나에게 돌아섰다. 참고 있는 웃음에 입술이 실룩거렸다.

``둘이 정말로 가까운가 보네. 내가 방해하지 않았으면 종일\ldots\,그 얘기만 할 거 같았는데.''

``우리는 너무 정상이에요.''

안나는 볼멘소리를 했다.

``전 가족한테도 놀림을 받는데요.''

``정상인 거 좋지. 내 가족은 날 죽이려 들었는걸.''

``끔찍해라. 어쨌든, 업힌 채로 잠들어서 죄송해요. 정말 불편했죠.''

안나는 입술을 깨물었다. 부디 어깨 위에 침을 흘리지 않았기를.

``심심했어요?''

``관심거리를 찾았어.''

엘사가 말했다. 잠깐 눈에 초점이 풀리고 있었다. 안나는 거의 보지 못했지만, 심란해 하는 것은 알아챌 수 있었다. 묻기도 전에 엘사는 말을 덧붙였다.

``잊어버리기 전에 우리가\ldots\,뭘 할지를 얘기해 봐야겠어.''

``왜 갑자기 진지하게 들리죠?''

엘사는 자신의 손을 만지작거리며 어깨를 으쓱였다. 전혀 본인답지 않은 이 우물쭈물하는 모습에 안나는 입을 벌린 채 바라보고 있었다. 십 초가 지나서야 엘사는 말을 꺼냈다.

``네 아버지께 무슨 말을 해야 할지 모르겠어.''

``아, 그렇지. 아.''

안나는 손바닥을 얼굴에 올렸다. 볼을 밀어 올리며 끙하는 소리를 내고 있었다.

``어, 진짜로 어색한 대화가 되겠는데. 그냥 제가 얘기할게요.''

웬일인지 안나는 자신이 엘사와 `같이' 돌아온 것이 그냥 흐지부지되지는 않으리라고 생각했다. 예절은 차치하고, 엘사는 아렌델을 정복했다. 안나는 이를 오래도록 마음에 두지 않아 아렌델이 서던 제도의 속국인 것을 거의 잊을 정도였지만, 자신이 왕국 최악의 적과 연애 관계인 것을 자신의 아버지가 좋아할 것으로는 생각하지 않았다. 뭐, 안나는 생각했다. 용기를 끌어모아 시도해보아야 한다. 이 상황을 나아지게 무언가를 할 수도 있을 것이다. 자신의 아버지에게 엘사는 생각하는 것과 다른 사람인 것을 설득할 수도 있을 것이다.

``난 그 대화 자체를 피해야 한다고 생각해. 일을 더 어렵게 만들고 싶지는 않아.''

``뭐라고요?''

안나는 고개를 저었다.

``전 안 숨기고 싶어요. 여왕님한텐 불공평한 거라고요. 그러니까, 같이 있는 게 부끄러운 것도 아니고 제 아빠도 알았으면 하고—''

``무슨 말인진 알아. 그렇게 해 주면 고맙고. 하지만\ldots''

엘사는 얼굴을 찡그렸다.

``생각을 숨기는 게 나쁜 의도로 하는 행동이 아닐 때도 있어. 네가 누군가를 신경을 써서 그럴 수도 있는 거라고. 네 아버지를 슬프게 하고 싶지는 않잖아.''

``그렇게 이해가 되는 거 같진 않은데요.''

``나라면 차라리 전혀 모르고 있겠어. 걱정 없이 너랑 더 많은 시간을 보내게. 더 일찍 말하면 더 괴로워질 뿐이라고.''

``이젠 진짜로 이해가 안 되네요.''

``나도 알아. 하지만 지금은 이게 최선이라고 봐, 네 아버지랑 시간을 보내는 거. 날 갖고 상황을 어렵게 만들지도 않고. 이 기회를 그냥 버리지 말라고.''

여섯 달이 지났다. 안나는 여섯 달 동안 자신의 아버지를 보지 못했고 바로 지금 아버지를 볼 수 있는 작은 틈이 생겼다. 솔직히 말하자면 안나는 이들의 관계에 해를 줄 수도 있는 주제를 꺼내는 데에 주저하고 있었다. 자신의 아버지가 이에 기뻐하는 모습은 전혀 떠올릴 수 없었다. 다음에 말할 것이다—안나는 결정했다. 다음에, 혹은 더 나중에면 자신의 아버지에게 모든 것을 말할 준비가 되어 있을 것이다. 그저 지금은 그렇지 않을 뿐이다. 이것이 진정 최선일 터이다.

``난 괜찮아.''

엘사가 나지막이 말했다.

``\ldots알았어요. 정 그렇다면요.''

안나는 엘사를 껴안고는 미소를 지은 채 물러났다.

``이따가 마을 구경하러 가자고요! 그냥 보통 사람처럼 하고 다닐 거예요. 그 얼음 여왕으로만 안 있으면 사람들은 못 알아볼 거예요. 일단 제 옷장에서 좀 정상적인 걸 입어야 할 거예요.''

``그래야 한다면.''

안나는 웃음을 터뜨렸다.

``그럼 이따 봐요!''

안나는 엘사의 볼에 입을 맞추고 방을 나왔다. 아버지의 서재를 향해 발걸음을 서두르고 있었다. 아렌델 성은 차갑고 으스스한 서던 제도 성과는 다른, 눈에 띄게 따뜻한 분위기가 감돌았다. 더욱 밝고 생기가 있었다. 잎사귀는 색을 바꾸며 가을의 도착을 맞이하고 있었지만, 햇볕은 여전히 여름의 따뜻함을 가져오고 있었다. 안나의 발걸음은 서던 제도에서보다 가벼웠다. 서던 제도는 가혹하게 안나를 짓눌렀다. 오래지 않아 안나는 아버지의 문 앞에 도착했다.

안나는 손을 올려 문을 두드리려 했지만, 바로 앞에서 손이 멈추었다. 숨을 깊게 들이마셔도 보았지마는 머리가 어지러울 정도로 심장이 크게 뛰었다. 정말로 오래되었다. 한편으로는 자기나 아버지 중 하나가 너무 바뀌어 둘의 관계가 전과 같 않으면 어떡하나 하고 겁이 났다.

자신이 이제는 다르다는 것을 안나 자신은 잘 알고 있었다. 여태껏 안나의 아버지는 안나를 감싸며 있는 힘껏 안나의 세상을 바라보는 마음을 지켜 주었다. 언젠가 아그다르는 안나에게 동정심과 낙천적인 천성이 드문 재주라고 말한 적이 있었다. 그리고 쉽게 더럽혀질 수도 있는 것이라고. 하지만 이제는 이 성의 경계 밖의 것을 알고 있었다. 탐욕과 혐오와 절망을 겪었지마는 안나는 여전히 순수한 채 자신의 아버지가 원한 것보다 더 많은 것을 보았다. 바깥세상이 항상 완벽하고 아름다운 것은 아니라는 무서운 사실을 말이다. 안나 자신은 여전히 신념을 지닌 안나였을까?

하지만 이것이 상관은 있는 것일까.

아버지는 여전히 아버지일 뿐이다.

안나는 문을 밀어 열었다.

아그다르는 책상 앞에 앉아 있었다. 등을 굽힌 채 서류를 바라보고 있었다. 생각보다 더 나이가 들어 보였다. 잿빛 머리카락이 옆머리를 덮고 눈가의 주름도 깊어졌다. 한때는 매끈했을 피부에도 골이 파였다. 안나는 한동안 가만히 선 채 바라만 보고 있었다. 항상 그랬듯 일하는 모습을 바라보면서도 자신의 아버지가 정말로 눈앞에 있는 것이 믿기지 않았다.

숨기려 애를 썼지만, 소리 죽인 흐느낌이 입 밖으로 나왔다. 아그다르는 고개를 들어 안나의 모습을 보았다. 그대로 얼어붙은 채 눈만을 깜빡이고 있었다. 겨우 작게 미소를 짓는 안나를 보고는 자리에서 일어나 천천히, 그리고 조심스럽게 다가갔다. 안나는 남은 거리를 좁히며 품 안으로 뛰어들었다. 아버지는 예전과 같았다. 여전히 강하고 차분했다. 조금도 흔들리지 않았다.

``안나야.''

아그다르는 나직한 목소리로 말했다. 안나의 정수리에 입을 맞추고 있었다.

``다 괜찮아질 거야. 집에 왔잖니. 이 아빠 여깄다, 안나야. 바로 여깄어.''

``정말 그리웠어요.''

``나도 정말 그리웠다.''

아그다르는 깊게 숨을 들이마셨다. 안나는 눈물을 삼키며 훌쩍거리는 소리를 들었다고 확신했다.

``정말 미안하다, 네가 희생되어야 한 게 정말 미안하구나. 모두 내 잘못이다.''

``그렇진 않았어요. 아렌델을 위해 뭔갈 할 수 있어서 기뻤어요.''

안나는 나지막이 말했다.

``그리고 나쁘지만도 않았고요. 전 아빠 생각보다 강한걸요.''

``그래, 나도 안다.''

아그다르는 물러서고는 안나를 찬찬히 살펴보았다. 씁쓸한 미소를 띠고 있었다.

``떠나기 전이랑 똑같아 보이는구나. 하지만 난 많이 늙었지.''

``아녜요. 지금도 좋아 보여요, 아빠.''

아그다르는 웃음을 터뜨렸지만, 그냥 넘어갔다.

``어떻게 지냈는지 말 좀 해 주겠니?''

``그렇게 나쁘진 않았어요.''

안나가 말했다. 마법사 부분은 무시하고 있었다. `그렇게 나쁘진 않았어요'에서 바로 `정말 끔찍했어요'로 떨어지는 일이었으니 말이다. 엘사의 말마따나 안나는 자기 아버지에게 고통을 주지는 않을 참이었다.

``처음엔 힘들었는데 좀 익숙해지니까 꽤 잘 적응했어요. 제 생각으론요. 왕자들도 만났고요. 대부분은 별로였지만 몇몇이랑은 친구가 됐어요. 아빠도 좋아할 거예요.''

``엘사 여왕도 널 좋아하는 것 같구나.''

아그다르가 말했다.

``네\ldots\,네?''

안나는 몸이 확 달아오르는 것을 느꼈다. 얼굴이 붉어지지 않았기를 빌고 있었다.

``네가 잠들어 있는 동안 업어서 데리고 왔거든.''

안나는 자신의 아버지가 더\ldots\,큰 것을 알아채지 못한 것에 안도의 한숨을 내쉬었다. 그러면서도 한편으로는 이미 알아채 모든 것을 말할 수 있기를 바랐다. 안나는 말하지 않는 것이 아예 거짓말하는 것보다는 낫다고 생각하며 자신을 달랬지만, 정말로 알게 된다면 어떻게 반응할지 궁금했다. 안나는 지금까지 자신의 아버지에게 그 무엇도 숨기지 않았다. 하지만 엘사의 말도 일리가 있다. 지금은 허락하지 않을 것이다. 엘사가 다르다는 것을 먼저 설득해야 할 것이다.

``혹사당하지 않았다니 다행이구나.''

아그다르는 말을 이었다. 눈의 초점이 풀리면서 어둡고 겁에 질린 빛을 띠기 시작했다.

``난 네가 혹시나\ldots''

``아녜요. 전 정말 괜찮아요. 그리고 많은 걸 배웠어요. 특히 여왕님에 관해서요.''

아그다르는 미소를 지었다.

``그래, 안단다. 정말 잘했다, 안나야. 난 몸조심하라고만 했지. 더 큰 걸 바라진 않았고\ldots\,정말이지 자랑스럽구나.''

``어\ldots\,그렇죠.''

안나는 말뜻을 이해하지는 못했지만, 무언가 진전이 되는 것이 느껴졌다. 이 기회를 잡는 것이 좋으리라.

``그래서 말하려고 한 게—''

``다 잘 돌아가고 있단다.''

아그다르가 말했다. 안나의 어깨를 두드리며 진지한 눈빛으로 바라보고 있었다.

``걱정하지 마라, 안나야. 넌 네 역할 아주 잘하고 있단다. 나머지는 내게 남겨 둬라.''

``아뇨, 그게 아니라 엘사 여왕님—''

``나도 안다. 널 믿고 있지.''

뭐, 맞는 말이기는 하다. 안나는 아버지의 이상한 눈빛에 얼굴이 찡그려지는 것을 참고 있었다. 완고함과 후회가 반씩 섞인, 일종의 결의였다. 솔직히 안나는 무슨 영문인지 전혀 알 수 없었다. 그저 자신을 포기해야 하는 것에 아직도 슬퍼하는 것으로 생각했다. 지금은 좋은 때가 아닌 듯하다. 안나는 결정했다. 나중에 시도할 것이다.

``알았어요.''

안나는 문을 향해 손짓을 보였다.

``이제 여왕님한테 가 볼게요. 절 기다리고 있을 거예요. 나중에 또 얘기해요. 알았죠?''

``시간은 앞으로도 많이 남아 있을 거야.''

``얼마나 오래 있을진 몰라요. 며칠 안 있을지도 모른다고요.''

``그렇겠지.''

아그다르가 말했다. 정치적인 자리에서 보이는 사무적인 태도로 고개를 끄덕이고 있었다.

``금방 떠나게 되겠구나. 잊지 마라, 더는 걱정할 것 없다, 안나야. 집에 와 있잖니.''

``맞아요.''

안나는 한 번 더 자신의 아버지를 껴안았다.

``돌아와서 정말 좋아요, 아빠.''

``다시 보니 정말 좋구나.''

아그다르는 꼭 껴안아주고는 안나를 놓아주었다. 자랑스러워하는 미소를 보이고 있었다.

``자, 계속 기다리게 하면 안 되겠지.''

뭐, 안나는 생각했다. 엘사와 시간을 보내는 것을 싫어하지는 않는 것 같았다.

여기부터가 시작이다.

\textbreak

모든 사람은 엘사를 피해 다녔지만, 엘사 자신은 이에 익숙해 있었다. 최소한 이곳에서의 이런 대우는 정당했다. 엘사는 하인들이 서둘러 지나가며 자신을 흘끗 바라보는 모습을 즐겼다. 이들이 자신이 고른 드레스를 이상하게 바라보는 것을 엘사는 알아채지 못한 체했다. 아렌델 왕족을 위한 암청색 드레스였다. 안나의 것이었지만, 옷장 안 깊숙한 곳에 있었다. 안나는 밝은 빨간색과 파란색을 좋아하는 듯했지만, 엘사는 이런 색 옷을 입고 있는 자신을 상상할 수 없었다.

그리고 이제는 안나가 있었다. 이상하게도 낡은 가방을 멘 채로 흥에 겨워 달려오고 있었다. 엘사는 눈썹을 추켜세웠다. 안나는 껴안으려 하는 도중에 멈추었다. 팔은 이미 올라가 있었다. 엘사는 하녀들을 향해 고개를 돌렸다. 게르다도 같이 둘을 엿보고 있었다. 안나는 아무도 둘의 관계를 모른다는 것을 떠올리고는 기지개를 켜는 것처럼 팔을 휘둘렀다. 엘사는 이 뻔한 행동에 거의 얼굴을 찡그리려 했지만, 하녀들에게는 잘 통한 듯했다.

``자! 아렌델 구경할 준비 됐어요?''

``이걸로 정말 우리가 보통 사람으로 보일 수 있겠어?''

엘사는 둘의 화려한 옷차림에 인상을 받은 것은 아니었지만, 평민들이 입는 것보다는 훨씬 좋은 옷차림이었다. 조금이라도 관찰력이 있으면 알아챌 수 있을 정도였다.

``아뇨. 그래도 좋은 지적이에요! 그래서 이걸 가져왔거든요.''

안나가 말했다. 가방을 둘 사이에 놓고는 침착하게 거친 천으로 만든 갈색 망토 두 개를 꺼내 하나를 걸쳤다. 그러고는 남은 하나를 엘사에게 주었다.

엘사는 망토를 바라보며 코를 찡긋했다.

``효과 있다니까요.''

안나가 말했다.

``\ldots알았어. 하지만 아무도 들으면 안 돼.''

엘사는 이 참견쟁이 하녀들이 자신들의 일로 돌아갈 때까지 안을 죽 훑었다. 엘사는 머리 위로 망토를 뒤집어썼다. 얼음 드레스의 차갑고 공기와 같은 자유가 그리워졌다.

변장을 끝내고 이들은 길을 떠났다. 안나는 마당의 새끼 오리를 보고는 잠깐 이들과 놀아주었다. 특히 못생기고 바보 같은 하나를 엘사의 어깨 위에 올려주었다. 무엇을 하기도 전에 이 오리는 엘사를 횃대로 보고 머리 위에 앉았다. 그러고는 삼 분 내내 그대로 있었다. 떼어놓으려 하면 크게 깍깍거릴 뿐이었다. 엘사는 거칠게 상대하고 싶지는 않았다. 결국 이 새끼 오리는 지루해져 머리에서 뛰어 내려왔다. 엘사는 안나를 끌고 밖으로 나갔다. 안나는 줄곧 엘사가 살짝 즐기고 있었다고 말했다. 엘사는 계속 그러지 않았다고 했지만\ldots

엘사를 두려워할 정도로 똑똑한 것은 아닌 것이 다행일는지도 모르겠다.

성문을 지나며 이들은 마을로 향하는 조약돌 길을 따라갔다. 서던 제도의 왕가는 평민들과 꽤 거리를 두었다. 아렌델은 확실히 이들과 가까이 있었다. 나오자마자 바로 마을이 보이기 시작했다. 엘사는 사람들이 몰려드는 모습에 몸이 굳었다. 평범한 사람들 속의 보이지 않는 위험을 경계하는 본능적인 경고였다. 인파 속에는 암살자가 숨어들기 매우 쉬우니 말이다.

``긴장 풀어요.''

안나가 속삭였다. 안나는 엘사의 손을 잡고 사람들 속을 헤쳐나가 시장가의 가장자리로 이끌었다. 조금 더 넓은 공간에서 구경할 수 있게 해 주고 있었다.

불안감을 억누르며 엘사는 북적대는 시장의 모습을 바라보았다. 아주 많은 사람이 매우 가까이 있었지만, 그중에서 이들에게 관심을 보이는 사람은 아무도 없었다. 한두 번 보기 드문 백금발을 흘끗 보는 눈은 있었지마는 이들의 눈길은 바로 거두어졌다. 이렇게 무관심하게 물러나는 사람들의 모습은 당황스러웠다. 평소에는 겁이나 혐오가 담긴 눈을 볼 뿐이었다. 엘사는 이\ldots\,기분 좋은 변화를 인정하지 않을 수 없었다.

``자, 걱겅할 거 없다니까요. 여기서 여왕님은 지극히 평범하다고요.''

안나가 말했다. 한쪽 입꼬리를 올리며 덧붙였다.

``좋은 쪽으로요.''

엘사는 미소를 지었다.

``여기선 뭘 할 거니?''

``그냥 이것저것 볼 거예요!''

안나는 다시 엘사의 손을 잡아 사람들 속으로 이끌었다.

``늘 새로운 게 생긴다고요. 빨리 와요!''

이런저런 일이 많은 몇 시간이었다.

노천 시장은 매점 주변에 모인 사람들로 붐비었다. 이유는 매우 확실했다. 딸기향이 이 부드러운 빨간 열매가 담긴 나무 상자들에서 퍼져 나왔다. 엘사는 가뭄에 밀 작황이 큰 피해를 보았는데도 딸기가 풍작이라는 말을 엿들었다. 딸기 바람이 든 듯했다.

``딸기 좋아해요?''

안나가 물었다.

``초콜릿 묻힌 거.''

엘사가 말했다. 안나는 씩 미소를 지었다.

빵을 파는 곳으로 걸어가자 둘은 이 바람기가 든 간식들을 발견했다. 딸기 파이, 딸기 파르페, 딸기 퍼널 케이크가 모두 수레 안에 줄지어 있었다. 안나의 눈은 몹시 굶주린 모양으로 이 주전부리들을 훑고 있었다. 이 모습에 결국 엘사는 퍼널 케이크 하나를 집어 입에 넣었다. 그러고는 케이크 값을 낼 수 없는 것을 깨달았다. 안나는 이 큰 실수를 어찌어찌 잘 모면하고—빠져들 것 같은 활기에 엘사는 잠깐 안나가 매우 매우 위험해질 수도 있다는 생각이 들었다—케이크 값을 치렀다. 시간이 좀 지나서야 엘사는 자신보다 안나가 더 상식적인 것을 마지못해 인정했다.

공예품 매점은 형형색색의 천막이 드리워진 진열대로 가득했다. 진열대 각각에는 물건이 올려져 있었다. 안나는 여러 진열대를 왔다갔다했지만, 엘사는 찬찬히 물건을 살폈다. 포동포동한 가게 주인은 손톱을 뜯고만 있었다. 엘사는 잡동사니가 실려 있는 수레를 찬찬히 살폈다. 나무를 깎아 만든 작은 조각상을 만져보고 색칠된 종이 등에 새겨진 시를 감상했다. 모두 지금까지 익숙한 것보다는 못한 재료로 만들어져 있었지만, 빠져드는 멋이 있었다. 모두 매우 낯설었지마는\ldots\,익숙했다. 엘사는 여기에 온 적이 있었다. 거의 기억하지는 못하지만, 엘사의 부모는 어릴 때 이곳에 엘사를 데려온 적이 있었다. 이상하게도 이 기억은 아프지가 않았다.

``맘에 드는 거라도 있어요?''

안나가 물었다. 뒤에서 다시 나타나 엘사는 거의 놀라 움찔할 뻔했다.

``맘에 드는 거 있으면 돈 내는 거 잊지 마요—''

``말 안 해줘도 알아.''

엘사가 말했다. 가게 주인이 고개를 들자 점점 얼굴이 붉어졌다. 엘사는 안나의 손을 잡아끌어 가게 주인의 시선에서 벗어났다.

다른 탁자에는 무기가 줄지어 놓여있었다. 엘사는 이 모습에 잠깐 굳었지만, 장난감인 것을 알아채고는 바로 긴장을 풀었다. 철의 모습은 잿빛 칠이 된 매끈한 목재의 모습으로 바뀌었다. 안나는 코웃음을 쳤지만, 말을 꺼내지는 않았다. 엘사를 탁자로 끌고 갈 뿐이었다. 매우 기뻐하는 눈빛으로 안나는 진열된 장난감을 살폈다.

``이거야!''

안나가 말했다. 나무 방패와 칼을 들고 있었다.

``정말 기사 같지 않아요?''

``그렇게 드는 거 아니야.''

엘사가 말했다. 엘사는 안나 뒤에 서서 자세를 잡아주었다. 자루 아래쪽을, 코등이에서 먼 쪽을 잡게 했다.

``검지에 힘 빼야 해. 안 그러면 잘 못 휘둘러.''

``아\ldots''

``팔꿈치는 너무 굽히지도, 너무 펴지도 말고.''

엘사는 계속했다. 안나의 팔을 잡아 움직이고 있었다.

``위팔은 힘 빼고 아래팔에 힘을 줘.''

안나는 꺅하는 소리를 냈다.

안나를 흥분시키고 있는 것을 엘사는 시간이 지나서야 알아챘다. 온 정신을 집중한 채 저절로 몸을 움직이느라 둘이 얼마나 가까이 붙어있는지를 알아채지 못했다. 혹은 입술이 안나의 귓전 바로 앞에 떠 있던 것도. 엘사는 헛기침을 하며 물러났다.

``어떻게 쓰는지 보여 주는 게 낫겠니?''

엘사가 물었다.

``네\ldots\,네!''

안나는 칼을 넘겨주었다. 엘사는 무게와 균형에 적응하려 잠깐 칼을 휘둘러댔다. 칼을 써본 지는 꽤 오래되었지만, 몸은 절대 잊지 않는 법이다. 엘사는 간단한 것에서부터 시작했다. 칼을 위로 들어 바꾸어 쥐고는 웅크려 앉아 뒤 찌르기를 했다. 엘사는 천천히 움직였다. 다음으로 보여줄 것을 준비하며 숨을 고르고 있었다.

바로 다음 순간 엘사는 더욱 어지러운 상황을 훑고 지나갔다. 다리는 한순간에도 멈추지 않았다. 몸을 움직이는 축이 되어 보이지 않는 적을 베고 있는 엘사를 새 지점으로 데려가고 있었다. 정수에서 역수\footnote{역주: 正手·逆手. Standard grip/Reverse grip을 번역하기 위해 임의로 만든 단어임.}로 바꾸는 모습은 익지 않은 눈으로는 흐릿한 회색 형체만이 보였을 것이다. 선 자세와 받아치기 자세 사이를 바꾸며 공격에서 방어로 물 흐르듯 끊김 없이 넘어갔다. 정말이지 오래되었다. 칼을 쓸 때의 흥분감을 잊고 있었다. 엘사는 공중으로 칼을 던졌다. 다시 자루를 역수로 쥐고는 칼을 땅에 박아 마무리를 지었다.

박수갈채가 쏟아졌다.

사람들을 끌어모은 모양이다. 구경꾼들은 손뼉을 치며 놀란 표정으로 서 있었다. 가게 주인은 이 무료 광고에 꽤나 만족하는 듯 보였다. 물론 아이들은 이제 부모들에게 칼을 사 달라고 조르며 어디서 이런 것을 배울 수 있는지 묻고 있었다. 이목을 끌고 싶지는 않아 엘사는 칼을 돌려주고 안나의 손을 잡아 사람들 사이를 빠져나왔다.

``정말 멋졌어요!''

안나는 속삭이는 목소리로 말했다.

``여왕님이 칼 쓰던 모습을 거의 잊고 있었어요.''

``이젠 잘 안 써. 이렇게 직접 싸울 일도 전혀 없었고.''

``어디서 이런 걸 다 배울 거예요?''

``그게\ldots''

엘사는 어깨을 으쓱했다.

``한동안은 구스타프가 가르쳐 줬어. 그 반역 일을 알기 전까지는. 그 뒤로는 혼자 익혔어.''

``\ldots아.''

지금 이들은 장난감 무기 가게를 둘러싸고 있는 군중에게서 빠져나와 있었다. 엘사가 더 말을 꺼내기도 전에 바구니를 들고 있는 남자아이가 이들에게 다가왔다. 안나는 자세를 낮추고는 아이에게 웃어 보였다. 아이는 바구니를 내밀어 보이고는 말을 꺼냈다.

``어\ldots\,저기요, 꽃 사실래요?''

꺅하는 소리를 내며 머리를 두드려주는 안나의 모습을 보자 엘사는 비웃으려는 것을 참았다.

``물론이지!''

안나는 동전을 찾으려 여기저기 뒤적거렸다. 이 아이와 엘사 둘이 남겨져 서로를 바라보고 있었다. 엘사는 눈살을 찌푸렸고 아이는 고개를 숙였다. 귀가 빨개질 정도로 얼굴을 붉히고 있었다. 엘사는 얼굴을 찡그리고는 고개를 돌렸다. 안나가 한두 번 찌르고 나서야 이 아이는 고개를 들었다.

중얼거리는 목소리로 고맙다고 말하며 돈을 받고 이 아이는 땅을 쿡쿡 찧었다. 잠시 뒤 아이는 고개를 끄덕이고는 다시 엘사를 바라보며 불쑥 말을 꺼냈다.

``정말 예쁘신 거 같아요!''

누구도 반응하기 전에 이 아이는 쏜살같이 사라졌다.

엘사는 안나를 돌아보았지만, 이 빨간 머리는 갑자기 자지러지게 웃기 시작했다. 가능했으면 엘사는 빨갛게 달아올랐을 것이다. 엘사는 길옆으로 안나를 데려갔지만, 계속 웃어 대는 통에 빨리 진정시키려 망토 안으로 손을 집어넣었다.

``차가워!''

안나는 제자리에서 콩콩 뛰기 시작했다. 망토 안에서 얼음 조각이 떨어졌다. 안나는 얼음 조각을 밟아 부수며 등을 문질렀다.

``진짜, 그러는 게 어딨어요!''

``그럴 만했다고.''

``재밌었잖아요!''

안나가 말했다. 다시 미소를 짓고 있었다.

``진짜, 정말로 귀여웠잖아요. 분명히 한눈에 반해서 그랬을걸요.''

엘사는 입을 굳게 다문 채 아무런 말도 하지 않았다.

``알았어요. 웃어서 죄송해요. 자, 여왕님 거예요.''

안나는 자신이 산 꽃을 내밀어 보였다. 엘사는 이때서야 파란 난초가 들려있는 것을 알아챘다. 아름다운 꽃이었다. 결 없이 부드럽고 매끄러운 파란 꽃잎이 검은 줄기에서 피어나 있었다. 엘사는 눈을 깜빡이지도 않고 꽃을 뚫어지라 바라보았다. 안나는 긴장이 되었다.

``제가 뭘—''

``아냐, 꽃 좋아.''

엘사가 말했다. 꽃을 받고 있었다.

``그냥 그\ldots''

엘사는 `집'이라는 말을 꺼내고 싶었다. 엘사는 이 말이 거의 입 밖으로 나올 뻔한 것이 놀라웠다. 그곳이 전혀 집이 아니었지마는 말이다. 하지만 정말로 아닐 수가 있을까. 엘사는 자신의 아버지가 보살피던 꽃 정원이 떠올랐다. 자신의 어머니가 기르던 파란 난초가 떠올랐다. 한때는 집인 곳이었다. 악몽이 시작되기 전에 엘사는 이 집에서 지낸 순간순간을 사랑했다. 사랑을 느꼈다.

``아, 죄송해요. 그럴 줄은\ldots\,그러려고 한 게\ldots''

안나의 속삭임은 이내 사라졌다. 다시 꺼낸 말에는 놀란 기색이 있었다.

``그리운 거죠.''

정말로 묻는 말은 아니었다. 모든 감정을 똑같이 느낄 수 있는 데 부정하는 것은 소용없다. 무엇 때문에 가슴이 조여들고 입 안이 마르는지도 모를 때도 말이다. 얼마나 그리워했으면 안나도 느낄 수가 있을까. 이런 느낌의 양을 재고 이름을 붙이고 분류할 방법이나 있을까. 알지 못하는 감정에 시달려야만 하는 것일까. 감정을 느낀다는 것이 이런 것이라면, 엘사는 감정을 느끼지 못하는 것이 애초에 이토록 나쁜 일인가 하는 생각이 들었다.

``어\ldots떡하고 싶은 거예요?''

안나가 물었다.

``나\ldots\,난 모르겠어.''

엘사가 말했다. 이것이야말로 놀랄 일이었다. 항상 엘사는 자신이 무엇을 하는지 알고 있었다. 항상 통제하고 있었다. 통제하지 않는 것은 이도 저도 못하고 있는 것이고, 이것이야말로 받아들일 수 없는 일이었기 때문이다. 모르는 것을 인정하는 것에 한때 심장이 뛰던 곳에서 아픔이 왔다.

``그냥 돌아가자. 어두워지고 있어.''

``네, 그게 낫겠네요. 아빠가 저녁상에서 기다리고 있을 거예요.''

``난 괜찮아, 안나야. 걱정하지 마.''

엘사가 중얼거렸다. 엘사는 안나의 손을 잡았다. 바로 잡아주는 것이 느껴졌다.

``네가 있잖아.''

``항상 그럴 거예요.''

엘사는 고개를 끄덕이고 성으로 돌아갔지만, 마음은 자신을 기다리고 있는 진짜 집을 떠나지 않았다.

\textbreak

다음 며칠은 살짝 어색했지만, 안나는 그래도 성공한 것으로 쳤다.

첫날 밤 저녁은 정말이지 괴로웠다. 딱딱한 얼굴을 하고 있는 엘사와 누가 보아도 불편해하는 자신의 아버지와의 대화를 이어야 했기 때문이다. 죽이 척척 맞게 이들은 안나를 사이에 두고 말없이 반대편에 앉았다. 왜 아무것도 먹지 않느냐고 엘사에게 묻자 엘사는 눈을 가늘게 뜰 뿐이었고, 안나는 중간에 끼어들어 시장에서 한 군것질 때문이라고 둘러댔다.

``조금만이라도 먹어 봐요.''

안나가 말했다. 급한 눈으로 바라보며 발을 툭툭 치기까지 하고 있었다.

``어, 샐러드라도요. 몸에 좋잖아요.''

안나의 말대로 엘사는 샐러드를 입 앞에 놓고 잠깐 멈추어 이 녹빛과 아그다르를 둘 중 하나는 폭발할 것을 바라는 듯 쳐다보고는 마침내 입 안으로 가져갔다.

``정말 맛있네.''

엘사가 말했다.

``그렇다니 다행이네요.''

아그다르가 말했다. 말이 끝나고 저녁상에는 다시 침묵이 내려앉았다.

안나는 얼굴을 감싸고 끙하는 소리를 내지 않으려 무척 애를 썼다.

그래도 좋은 점은 나머지 때는 즐거웠다는 것이다.

안나는 매일 아버지와 단둘이 보낼 시간을 남겨놓았다. 그동안 이들은 딸기 바람이 얼마나 불지와 같은 가벼운 것에서 서던 제도에서 있던 일과 같이 무거운 이야기를 주고받았다. 안나의 아버지는 엘사에게 꽤 관심이 있는 듯했고, 안나는 엘사에 관해 더 말을 할 수 있어 좋았다. 안나가 아버지와 있는 동안 엘사는 내내 도서실에 있었다. 안나가 돌아오면 이들은 마을을 쏘다니거나 성 안을 구경했다.

둘째 날 안나는 엘사에게 성을 구경해 주었다. 석재가 어디에서 조달되었는가와 같은 답답하고 형식적인 것은 전혀 아니었다. 안나는 훔친 초콜릿을 들고 사라질 수 있는 최고의 탈출로를 모두 알려주었고 수많은 그림을 보여주었다.

``그림에 말 걸면서 얼마나 있는 거니?''

``그런 식으로 생각하지 마요.''

안나가 중얼거렸다. 잔 다르크 그림의 액자를 톡톡 두드리고 있었다.

그런 다음 아무도 쓰지 않는 무도회장에서부터 순록보다는 말을 매는 마구간까지 이들은 온갖 곳을 쏘다녔다. 안나가 말을 탈 수 있는 것에 놀라워하는 엘사를 두고 안나는 기분이 상한 듯 행동했지만, 대개는 질문을 피하려는 것이었다. 다행히도 엘사는 말하는 주제를 바꾸었고 안나는 걷는 속도보다 조금이라도 빠르게 달릴 수 있게 될 때까지 여러 번 넘어진 것을 말하지 않을 수 있게 되었다.

이날 저녁은 조금 더 편안했다. 안나가 없는 동안은 서너 마디 이상 말을 나누지는 않았지만, 엘사와 아그다르는 어색한 첫날과는 다르게 화기애애하게 있었다. 둘 사이를 오가는 것은 피곤했지만, 둘 다 해상 여행에 관심이 있는 것을 알아냈으니 도움은 되었다. 그러니 둘 다 동시에 대화를 이어나가고 있었다.

다음날 둘은 다시 밖으로 나갔다.

자유로이 오갈 수 있다는 것은 이상야릇한 느낌이었다. 한때 안나의 아버지는 절대 밖으로 나가지 못하게 했지만, 이제는 엘사와 더 많은 시간을 보내도록 해주고 있었다. 안나는 아버지가 한때의 자기 생각과 똑같은 생각을 하는 것으로 생각했다. 아렌델의 모습을 더 많이 보면 더욱 관용을 베풀 것을 말이다. 한편으로는 자신의 아버지가 엘사를 받아들이고 있는 것일지도 모르겠다고 생각했다.

엘사가 아렌델을 받아들이고 있는 것처럼 말이다.

안나의 말대로 항상 새로운 볼거리가 생겼다. 아렌델은 항상 이런저런 일로 부산했지만, 다음 며칠은 평소의 시장 물건들과 악사들을 제하고도 다가오는 축제를 준비하느라 특히나 부산했다. 시기도 절묘하게 엘사가 이 휴가를 즐기고 있을 때였다.

``좀만 늦게 왔으면 동지절\footnote{冬至節}을 보낼 수 있었는데요.''

안나가 말했다. 나무 상자에서 순록 장식을 꺼내는 사람들을 바라보고 있었다.

``왕국을 돌보러 돌아가야 하지만\ldots\,몇 달 뒤에 또 오면 되지.''

엘사가 말했다. 안나는 흥분한 듯한 목소리를 들은 것을 확신했다. 미소를 지어 보이자 엘사는 시선을 피했다.

``그럼 약속한 거예요! 같이 동지절 지내는 거예요. 여기서요.''

안나는 벌써 같이 있는 모습이 머릿속에 그려졌다. 초와 종이 등으로 밝혀진 길, 무르익어 가는 축제 모습을 말이다. 그렇지만 지금은 같이 이 드문 정상 상태를 즐기고 있었다.

시간이 흘렀다.

모든 일은 잘 풀렸다.

하지만 계속 신경이 쓰이는 것이 있었다. 그냥 넘어갈 수도 없었다. 때때로 아무 일도 없을 때, 보는 사람이 없는 것 같으면 엘사는 멍하니 있곤 했다. 안나는 무슨 생각을 하는지 알고 있었다.

바로 집이었다.

매우 강한 데다 고집스럽게 숨기고 있어 안나는 연결된 마음이 고동치는 것을 느낄 수 있었다. 날이 지날 때마다 점점 강해졌다. 서던 제도로 돌아가기 하루 전날의 엘사는 정말로 정신이 팔린 상태였다. 안나는 말을 꺼내야겠다고 생각했다.

``같이 가 봐요.''

안나가 말했다. 엘사는 잠깐 놀란 듯했지만, 둘 다 설명이 필요 없었다. 안나는 엘사의 어깨에 손을 올렸다.

``아니면 혼자 가 봐도 돼요. 기다리고 있을게—''

``아니, 그건\ldots\,난 혼자 있기 싫어.''

엘사는 눈을 감았다. 한동안은 말이 없었다. 무슨 생각을 하는지 안나는 매우 알고 싶었지마는 똑같이 조용히 있었다. 어느 정도 느낌은 있었지만, 엘사 스스로 결정해야 하는 일이었다. 과거를 대면하고 싶지 않다고 해도 안나는 강요할 처지가 되지 못했다. 마침내 엘사는 눈을 떴다.

``같이 있어주겠니, 안나야?''

질문은 필요 없었다.

``당연하죠. 같이 있고 싶다면 언제든 같이 있을게요.''

\textbreak

온 사방이 황량했다.

이곳에 오기까지 정말 많은 용기를 끌어모았다. 자신의 죽음과 재탄생이 일어난 이곳에서 엘사는 어떠한 답도 결심도 계시도 얻지 못할 것으로 생각했다. 여왕이 되기 위해 엘사는 이곳에서 죽음을 맞았다. 하지만 무슨 일인지 결정했다고 생각할 때마다 다시 생각하기 시작했다. 과거가 더는 자신에게 영향을 줄 수 없다고 자신에게 말할 때도 엘사는 돌아오고 싶어 했다.

그리고 정말도 돌아오니\ldots

자신이 무엇을 기대하고 있었는지는 몰랐지만, 자신의 옛집이 황폐해져 있는 모습에 매우 화가 치밀었다. 그렇지마는 자신이 벌인 일이었다. 그 누구도 탓할 수 없었다. 안나는 입을 가린 채 헉하는 소리를 내며 옆에 서 있었다. 집이 있던 곳에는 돌 부스러기와 온갖 방향으로 찢긴 널빤지가 땅 위에 흩뿌려져 있었다. 그날의 기억을 너무도 깊이 마음속에 묻어둔 탓에 거의 잊힌 느낌으로만 남아 있었다. 자신의 손에 닿는 불꽃의 열기, 점점 커지는 두려움이었다.

그러고는\ldots

``그들을 죽였지, 안나야.''

엘사는 가까이 걸어가 잔해 위에 무릎을 꿇었다. 자신의 기억일지도 모르겠지만, 피비린내가 느껴졌다. 어찌 되었든 잔해 위에 흩뿌려진 자국은 상상이 아니었다.

``둘 다.''

``그래야만 했던 거예요. 여왕님 자신을 지키고 있던 거니까요.''

엘사는 대꾸하지 않았다. 맹신적이고 실리적인 행동이 이곳을 엉망으로 만든 것이다. 엘사의 가족은 외곽에서도 외곽 지역에 살았다. 땅값이 싸고 사는 사람도 드문 곳 말이다. 엘사는 무슨 일이 있었는지 알아볼 정도로 신경이나 쓰는 사람이 있으리라고는 생각지 않았다. 멀리 사는 이웃들도 이 소름 끼치는 사건 뒤로는 이곳에 살려 하지는 않을 것이다.

``사람들은 정말 편협돼 있어.''

엘사가 중얼거렸다. 새카맣게 탄 목재를 만지작거리고 있었다. 그때 벼락도 같이 날린 줄은 알지 못했다.

``다 그런 건 아녜요.''

안나가 말했다.

``다 그런 건 아니지. 모두 내 마법은 저주고, 숨겨야 하고 무서워해야 하는 것으로 생각했어. 너만 빼고.''

``제 얘기 하려는 건 아녜요.''

안나가 말했다. 엘사 옆에 무릎을 꿇고 앉았지만, 엘사는 곁눈으로 안나의 눈길이 계속 잔해에 가 있는 것을 보았다.

``이 온 세상에서 저만 그런 게 아녜요.''

``마르쿠스.''

엘사가 말했다. 숨을 쉬듯 바로 나온 답이었다. 삶 대부분에서 마르쿠스는 항상 한결같았다. 안나가 있기 전 엘사에게는 마르쿠스만이 있었다.

``널 만나지 못했으면 지금쯤 어디서 뭘 하고 있을까.''

``그\ldots\,그 얘기하려던 것도 아녜요.''

엘사는 안나를 돌아보았다. 안나는 두 손을 내려다보며 말없이 손을 쥐어짜고 있었다. 엘사는 기다렸다. 곧 안나는 고개를 들었다. 입술을 깨물고 있었지만, 매우 확신하고 매우 옳다고 생각하는 모습이었다. 엘사는 곧 꺼낼 말이 거의 두려워지려 했다.

``이런 말 하고 싶지는 않았는데\ldots마르쿠스가 틀렸어요.''

안나는 팔을 뻗어 엘사의 손을 잡았다.

``그 사람이 한 일은 여왕님을 병기로 만든 것뿐이라고요. 여왕님은 겨우 그 정도가 아니잖아요.''

``내가 살아있을 이유를 주셨다고.''

``아녜요!''

엘사는 고개를 떨구었다. 안나의 손에 힘이 들어갔다.

``그 사람은 그냥 이래라저래라 한 거라고요. 여왕님은 겨우 그 정도가 아니라고요. 살아있을 이유도 고작 그 뭐냐, 그 사람의 무기 따위가 될 게 아니라고요.''

``그런 게 아니야. 그분은 날 지키는 법을 가르쳐주신 거라고.''

엘사가 말했다. 물론 안나는 이해하지 못했다. 그래도 괜찮았다. 이해할 것이다.

``이 세상에서 날 지켜주는 건 힘뿐이야. 정말 지긋지긋해, 안나야. 삶 자체가 모든 것을 망가뜨린다고. 난 힘이 필요해. 이젠 다칠 수 없어.''

``하지만 자신을 지키느라 다른 중요한 걸 잃어버렸잖아요. 알겠어요? 삶이 여왕님을 다치게 하겠지만, 이건 그냥\ldots\,모든 것의 일부예요. 다시는 다치고 싶지 않다고 어떤 것도 느끼지 못하게 한 거잖아요. 이건\ldots\,그냥 잘못된 거예요.''

``이게 왜 잘못된 건데?''

엘사가 물었다. 자리에서 일어나 머리카락을 쥐며 잔해에서 돌아서고 있었다. 자신을 지킨다는 것이 도대체 무엇이기에 모두가—안나와 트롤들이—이렇게도 잘못되었다고 하는 것일까?

``마르쿠스는 고통을 잊는 법을 가르쳐 주셨어. 그리고 그분이 맞아. 고통은 장애물이라고.''

``하지만 고통은 여왕님의 일부라고요.''

안나가 말했다.

``왜 약점이 내 일부여야 하는 건데?''

엘사가 말했다. 다시 뒤돌아서고 있었다. 어린 자신의 모습이, 없어진 자신의 행복과 가족의 사랑을 슬퍼하는 모습이 보였다. 엘사는 눈사람밖에는 만들 수 없었다. 의지할 곳도 없었다.

안나는 자리에서 일어섰다. 한시도 엘사에게서 눈을 떼지 않았다.

``이유 같은 건 없어요. 자신의 일부를 내버리고 나머지를 멀쩡히 갖고 있을 수는 없는 법이라고요. 모든 것을 무시한다고 행복한 것도 아니라고요. 여왕님도 아시잖아요. 이거 때문에 다시 감정을 느끼고 싶어하는 거 아닌가요?''

``너 때문이야.''

엘사가 말했다.

``너처럼 되고 싶어. 너처럼 감정을 느끼고 싶어. 사랑하고 싶고 행복해지고 싶어. 나 같은 인간이 되기 싫어.''

``전 안 완벽해요.''

안나가 말했다. 안나는 천천히 다가갔다. 엘사는 지켜보기만 할 뿐이었다. 가까워지자 안나는 손을 내밀어 엘사의 머리를 감쌌다. 엘사는 앞으로 몸을 숙여 이마를 맞대었다.

``나에게는 완벽한걸.''

엘사가 속삭였다.

안나는 미소를 지었다.

``안 그래요. 완벽이랑은 한참 떨어져 있어요. 왜 완벽하다고 하는지 전혀 모르겠다니까요. 정말 덜렁대는 데다\ldots\,어디 가서 부딪치는 게 아니라, 그러니까\ldots\,옳은 일을 하려고 하면 거의 더 악화시킨다고요. 그\ldots\,여왕님이랑 있을 때요. 어떻게 도와야 하는지는 모르겠지만, 그래도 계속 해 볼 거예요. 그러니까 여왕님도 계속해야 해요. 포기하지 말라고요.''

``그건 못 하겠어, 안나야. 다시 그 아이로 돌아갈 순 없어.''

엘사는 안나의 어깨너머 자신의 옛집을 바라보았다. 입꼬리가 작게 올라갔다.

``눈사람을 꿈꾸던 그 아이가 될 순 없다고.''

안나는 뒤로 물러났다. 팔을 붙잡은 채 진짜 미소로 맞받아치고 있었다.

``아직도 여기 있어요. 아직도 여왕님의 일부라고요. 제가 보여줄 거예요.''

``어떻게—''

``우리 눈사람 안 만들래요?''

엘사는 그대로 굳었다. 둘 다 지난번을 잊지 않고 있었다. 하지만 안나는 계속 미소를 지으며 매우 자신감에 찬 눈으로 바라보고 있었다. 정말로 눈사람을 만들어본 지 십삼 년이 지났다. 땅딸막하고 우스꽝스러운, 어린아이의 솜씨로 만든 조악한 눈사람이었지만, 엘사는 이 눈사람을 얼마나 아꼈는지 여전히 기억하고 있었다. 가족 대신 자신을 사랑해주는 존재였다. 유일한 벗이었다. 이들이 서 있는 바로 이곳에서, 옛집의 정원에서 엘사는 이 눈사람을 만들었다.

`그래, 같이 눈사람 만들자.'

천천히, 본능에 따라, 엘사는 손을 내밀고는 자신의 마법을 불렀다. 자주 부르던 폭풍이 아니라 오래도록 부정해온, 자신의 더 순한 면을 부르고 있었다. 찬 겨울 공기가 손바닥 가운데에서 합쳐지고 압축되어 눈덩이가 되었다. 엘사는 안나를 바라보았다. 안나는 작게 고개를 끄덕였다. 엘사는 용기를 모아 눈덩이를 위로 쏘았다. 폭풍이 아니었다. 부드럽게 눈이 내렸다. 깃털과 같이 사뿐히 내려앉았다.

자기 마법의 아름다움을 보는 일은 좀처럼 없었다. 장엄한 모습이기는 하다—눈사태나 고드름 폭포를 부르거나 바다를 가르거나 하늘을 어둡게 하는 것 말이다. 하지만 이런 모습에는 아름다움이 없다. 자신의 두려움이 내지르는 포효가 눈의 노래를 묻어버릴 때, 분노가 눈앞과 마법의 우아함을 가려 버릴 때는. 이들은 마법이 아니다. 죽음과 파괴일 뿐이다. 하지만 이 모습은, 눈송이가 땅으로 떨어져 순수하고도 하얀 평원을 만들어내는 이 모습이야말로 예술이었다. 쓰러진 집의 모습도 이를 망치지는 못했다.

엘사는 손바닥을 펴 눈송이를 받았다. 자신의 피부 위에서 속삭이는 것이 느껴졌다. 눈송이는 바람에 날리어 사라졌다.

안나는 미소를 지었다.

``이제 시작하자고요.''

안나는 쭈그려 앉아 눈을 뭉치고 있었지만, 엘사는 조심스럽게 손을 움직이고 있었다. 거의 움직이지도 않고 있었다. 손을 젓고 있지도 않았다. 손가락을 움직일 뿐이었다. 발밑에서는 눈이 나선 모양으로 천천히 올라갔다 아래로 떨어졌다. 엘사는 다시 안나를 바라보았다. 안나는 말없이 미소를 지으며 눈덩이를 만들고 있었다.

물론 논리적으로는 할 수 있는 것을 알고 있었다. 눈 파도를 만들어 마을까지 타고 가기도 했다. 어렵지 않게 얼음 드레스를 만들었다. 하지만 눈사람을 만든다는 것에 엘사는 굳었다. 눈덩이 세 개를 쌓기가 어려운 것이 아니었다. 눈사람에 깃든 과거의 기억—자신의 약점과 나약함—때문이었다. 그리고 엘사는 이런 기억을 억누르려 정말 애를 써 왔다.

``할 수 있을지 모르겠어.''

엘사가 중얼거렸다.

``할 수 있어요.''

안나가 말했다.

``어떻게 하는지 몰라.''

``그럴 리가 없죠. 아직 기억이 안 나는 것뿐이에요.''

안나는 발로 눈덩이를 굴렸다. 엘사는 눈덩이에 점점 눈이 묻어 아래층에 쓸 만큼 커지는 것을 바라보았다.

``그래도 같이 할 순 있어요.''

검술을 보여주려 자세를 잡아준 것처럼, 이제는 안나가 엘사의 허리춤에서 손을 잡아주었다. 항상 엘사는 추위에서 편안함과 안정감을 느꼈지만\ldots\,이 이상은 느끼지 못했다. 안나의 온기와는 달랐다. 가끔 엘사는 온기가 자신을 태워버릴 것을 두려워했지만, 은은한 따뜻함은 생소하면서도 진정이 되었다. 항상 이러했다—엘사는 떠올렸다. 자신의 마법이 이토록\ldots\,차가운 줄은 전혀 알지 못했다. 지금처럼 매우 자주 그러지는 않았다. 평소에는 따뜻함과 만족감이 느껴졌다. 안나 덕분에 엘사는 떠올릴 수 있었다.

``같이.''

같이 잡은 손에서 파란 빛이 났다. 둘은 같이 손을 뻗었다.

`그래, 눈사람 만들자.'

안나의 눈덩이 위에 눈송이가 제어된 돌풍을 타며 소용돌이쳤다. 자신의 손을 잡고 있는 안나가 이끄는 대로 엘사는 눈을 지휘하고 조각했다. 손에서 나던 빛이 다시 밝게 빛나더니 사그라졌다. 이들의 눈앞에 눈이 알아볼 수 있는 모양이 되기 시작했다. 공 모양 셋에, 꼭대기에 있는 것은 길쭉한 머리가 되었다. 엘사의 집에 있던 목재가 막대 모양으로 잘려 가운데 몸체의 양쪽에 꽂혔다. 머리카락 모습으로 꼭대기에도 놓였다. 벼락 맞은 자리에 있던 새카맣게 탄 조각도 단추 모양으로 몸체에 달라붙었다.

눈사람이 완성되었다.

오랫동안 엘사는 눈앞의 모습을 믿으려 하지 않았다. 어린아이일 때 만들었던 작은 눈사람이 그때와 똑같은 모습으로 앉아 있었다. 짤막한 다리, 땅딸막한 몸, 불완전한 생김새지만, 기억 속 모습 그대로였다. 엘사는 숨이 멎었다. 다시 보게 되니, 자신은 많이 변했지만, 변함없는 이 모습이\ldots

``얘 이름이 뭐예요?''

안나가 물었다. 엘사는 마침내 알아챘다.

정말로 해냈다.

엘사는 눈사람 뒤로 가 쪼그려 앉은 다음 안나를 향해 눈사람의 팔을 들어 올렸다.

``안녕, 난 올라프야! 난 끌어안기를 좋아해.''

안나는 기쁨에 찬 웃음을 터뜨리고는 앞으로 달려나갔다. 눈사람과 엘사를 한꺼번에 끌어안았다. 밀려오는 행복감에 엘사는 크게 웃었다.

``내가 해냈어.''

엘사가 속삭였다.

``고마워, 안나야.''

안나는 씩 미소를 지었다.

``다 여왕님이 한 거죠.''

대답하려는 찰나에 안나는 자리에서 일어났다. 망토를 고쳐 두르며 팔을 비비고 있었다.

``괜찮니?''

엘사가 물었다.

``그럼요! 그냥 갑자기 추워져서요.''

안나는 손에 입김을 불며 제자리에서 뛰었다.

``이상해라. 전까진 안 추웠는데.''

``돌아가야겠어.''

엘사가 말했다.

``올라프는 어쩌고요?''

엘사는 돌아서서 이 못 움직이는 눈사람을 향해 미소를 지었다.

``여기 남아 있는 걸 싫어하진 않을 거야. 성 안은 더 따뜻하잖아.''

``녹고 싶지도 않을 거예요.''

엘사는 올라프를 남겨두고 안나를 성으로 데려갔다.

둘 다 안나의 흰 머리에 색이 돌아오고 있는 것은 알아채지 못했다.

