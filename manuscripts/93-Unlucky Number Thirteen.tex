

\chapter[외전3. 불운의 수 십삼][외전 3\hspace*{.5em}불운의 수 십삼]{외전 3 \ 불운의 수 십삼}



\begin{quote}

\small 24장의 앞 이야기입니다. 한스의 배경과 동기를 다루고 있습니다.\sourceatright{역자}

\end{quote} %force indent

``왜 다들 우리를 싫어하죠, 어머니?''

한스는 고작 여섯 살이었지만, 이미 주변의 모든 사람이 보내는 경멸의 시선을 느낄 수 있었다. 자신이 기억하는 한 한스와 어머니는 이 관용의 벼랑 끝에서 살아왔다. 많은 사건은 한스에게 냉혹함을 가르쳐주었다. 자기 형제의 멸시를 견디는 것, 자기 아버지에게 무정하게 무시당하는 것, 심지어는 이들에게 아무런 힘도 없는 것을 아는 하인들에게도 냉대받는 것도. 여섯 살이 되었을 때, 한스는 서던 제도가 자신이 있어야 할 곳이 아님을 알았다.

``그런 식으로 생각하지 마렴.''

아냐가 중얼거리듯 말했다. 항상 이런 식으로, 부드러우면서도 주저하는 목소리로 말하곤 했다. 자신에게 시선이 쏠리지 않게 더는 목소리를 높이지 않으려는 듯이.

``저도 다 안다고요. 다들 제가 쓸모없다고 생각해요. 하지만 전 그 사람들보다 낫다고요.''

한스가 말했다. 신랄한 말투를 감추지는 못하고 있었다. 한스는 자기 어머니의 조용한 모습에 화가 났다. 겨우 이런 신세가 될 이유가 없다. 겨우 이런 신세가 전혀 아니었다.

``네가 쓸모없다고 생각하는 사람은 없단다.''

아냐는 계속 말했다. 침대 옆자리를 톡톡 두드리고 있었다. 한스는 움직이려 하지도 않았다.

``네 형들은 그냥 놀리는 것뿐이야. 다 지나갈 거야, 정말로.''

``전 형들 말한 적 없어요. 엄마가 말한 거죠.''

한스가 말했다. 아냐는 이 말에 움찔했다.

``아무도 절 안 좋아하는 거 다 알아요. 제가 아버지 혈통이 아니래요. 저 보고 서자래요. 이런 취급 이상을 바라는 게 그렇게 잘못된 건가요?''

아냐는 무릎을 꿇고 앉아 자기 아들과 눈높이를 맞추었다. 이 꾸밈 없이 애정 어린 눈에 한스는 바로 자신의 응어리가 부끄러워졌다. 이런 때에 한스는 자신이 자신의 어머니와 다르다는 것을 알았다. 순진한 어린아이의 흔적이 남아있어도, 한스는 이와 반대되는 모습을 볼 수 있었다. 자신의 어머니는 자신이 보기에 가장 다정하고 이기심 없는 사람이었지만, 한스 자신은?

``한스야, 사랑한다.''

아냐가 말했다.

``저도요, 어머니.''

한스가 나지막이 말했다.

아냐는 미소를 짓고 한스의 머리를 쓸어주었다. 이마에 떨어진 머리카락을 정리해주고 있었다.

``아, 한스야. 다시 엄마로 불러주면 안 되겠니? 예전엔 그러더니.''

``하지만 아버지께서 허락 안 하실 거예요. 예의 없는 거잖아요.''

``어머니란 말은 좀 쌀쌀하잖니. 마지막으로 엄마란 말을 듣고 싶구나. 부탁 좀 들어줄 수 있겠니?''

아냐가 속삭였다.

``사랑해요, 엄마.''

한스는 천천히 말했다.

``무슨 일이에요—''

``쓸모없는 게 아니란 거 잊지 마렴. 넌 아주 쓸모 많은 사람이야. ''

아냐는 말을 마치고 바로 한스를 껴안았다. 한스는 뭐라 말을 꺼낼 수 없었다. 아냐는 떨고 있었다. 자기 아들을 꼭 껴안고 있어도 한스는 떠는 것을 느낄 수 있었다. 그리고 무섭게도, 자신의 어머니가 말하면서 흐느끼는 것을 들었다.

``넌 왕이 될 수 있어. 꼭 될 수 있단다. 내가 짐을 지우지만 않으면.''

``대체 무슨 일이에요?''

한스는 다시 물었다. 자신의 어머니를 껴안고 어깨에 얼굴을 파묻고 있었다. 한스의 어머니는 지금까지 자신의 곁에 있어 주었다. 이제는 자신이 어머니의 곁에 있을 때다.

``아무 일도 없어. 이번엔 진짜 아무 일도 없어.''

아냐가 말했다. 말을 끝낸 다음 다시 자세를 바로 하고 손등으로 눈물을 닦고 있었다. 그러고는 한스의 머리를 톡톡 두드리고 엷게 미소를 지어 보였다.

``이 엄마는 더 많은 걸 해줄 수 있단다, 한스야.''

이날 이후로 한스는 자신의 어머니를 볼 수 없었다.

\textbreak

서던 제도의 아냐는 외곽 지역 농부 집안의 딸로 태어났다. 최하층 집안은 아니었지만, 최하층에 매우, 매우 가까웠다. 아냐의 가족은 가난했지만, 집이 없는 정도는 아니었다. 아냐는 어린 두 자매와 부모와 함께 허름한 집에서 자랐다. 이런 환경에서도 집안은 밝은 분위기와 웃음으로 가득했다. 그리고 물려 입은 해진 옷은 따뜻하지는 않을지라도, 아냐의 사랑으로 가득한 마음은 그 어떤 매서운 겨울도 이겨낼 수 있었다. 삶은 단순했다. 사치스럽지도 않고 편안했다.

이 모든 것은 열여섯의 나이에 서던 제도 성의 시녀가 되었을 때 바뀌었다. 아냐는 어렸지만 야심적이지 않았다. 이 반짝이는 성의 비천한 시녀 이상이 되는 꿈을 즐기지도 않았다. 그저 더 많은 돈을 벌어 자신의 늙은 부모가 편안하게 휴식을 취하고 자신의 동생들이 더 나은 삶을 살 수 있게 할 생각이었다. 손쉽게 받아들여질 수 있었다.

한 번 바라보는 것으로 충분했다.

아냐는 겉으로는 아름답지 않았다. 이국적인 아름다움이나 명문 귀족의 황홀함과는 거리가 멀었다. 아냐의 머리는 휘날리는 금발이나 반들반들한 적발이 아닌 흔한 갈색 머리였다.\footnote{역주: 서양에서는 갈색 머리가 매우 흔함. 흔히 생각하는 금발에 파란 눈은 북유럽 외에는 흔치 않아 `이국적'으로 보는 경우가 있음.} 흘끗 보아서는 평범하다고도 할 수 있을 것이다. 그리고 주근깨까지도 있었다. 하지만 아냐는 자세히 볼수록 더욱 아름다워 보이는 부류였다. 아냐는 곱살한 데가 있었다. 표정은 상냥하고 부드러워 보기 좋았다. 바라보면 안심이 되기까지도 했다. 하지만 아냐는 항상 분위기가 있었다. 말씨는 부드럽고, 미소를 지을 때에는 항상 조심스러워하며 부끄러워했다. 동그란 눈은 어떠한 비밀도 감추지 않았다. 매우 간단히 다른 이들의 믿음을 샀다.

그러니 성 안에서 가장 중요한 이의 눈길을 끌게 되어 모두의 부러움을 산 것은 놀라운 일이 아니었다.

일솜씨가 좋고 머리도 좋아 아냐는 성에 들어오고 오래지 않아 왕비와, 심지어는 왕의 소실들에게도 귀여움을 받았다. 하지만 마르쿠스 왕이 좋아하게 되자 아냐는 바로 경쟁 상대가 되었고, 같이 있기 꺼려지게 되었다. 조용한 태도는 똑같은 전략을 사용한 여인들과 똑같이 보였고, 부드러운 말씨에는 감미로운 유혹이라는 수식이 붙었다. 마르쿠스는 아냐가 계속 자신의 시중을 들게 했고, 아냐는 마르쿠스가 업무를 보는 동안 차를 우려주었다. 모든 뜬소문에도 이들 사이에는 아무런 일도 벌어지지 않았다.

불가피하게 일이 벌어지기 전까지는.

아냐는 한편으로는 왕의 관심을 오랫동안 끌 수 없다는 것을 알고 있었지만, 더 어리고 순진한 다른 한편은 늦은 밤까지 탄원서를 읽고 제멋대로 퍼져 나가는 본인의 왕국을 다스리려 애쓰는, 이 일에 열심인 남자를 좋아했다. 아냐는 몇 주동안 지켜보아 왔다. 얼마나 열심히 일하는지 직접 보아왔다. 그리고 마르쿠스는 멋진 남자였다. 중년의 나이지만 여전히 한창때였고, 강하고 잘생겼다. 그러니 자신에게 온 호의를 보이자, 아냐는 사랑을 동반한 불합리한 생각에 시달렸다.

아냐는 어떠한 직함도 받지 못하는 것을 개의치 않았다. 아냐는 하인도 주인도 아닌 겨우 시녀일 뿐이었다. 다른 시녀들보다 지위는 높았지만, 정식인 것은 아니었다. 오직 더는 소홀히 대해지지 않도록 암묵적인 동의만이 있을 뿐이었다. 아냐는 자신이 시중을 드는 여인들에게 미움을 받았고 동료들에게도 따돌림을 받았다. 전자는 아냐의 뻔뻔함을 경멸했고, 후자는 두려움과 새 지위에 대한 부러움에 아냐 앞에 고개를 숙였다. 그래도 아냐는 여전히 위협이 되지 않아 아무런 직위 없이 홀로 남았다.

직위가 생기기 전까지는.

아냐의 임신은 온 성을 아연실색게 했다. 해산일이 가까워져서야—소실들의 갖은 노력에도 애석하게도 유산하지 않은 것이 분명해져서야—마르쿠스는 아냐와 결혼해주었다. 하지만 다른 일들이 혼례를 서둘러 끝내고, 사소한 일이 되게 했다. 황태자 구스타프의 반역이 겨우 평정되어 구스타프의 아내와 태어나지 않은 아들은 바즈로 사라졌고, 에드문드 왕자는 태어난 지 얼마 안 된 데다가 이름 없는 산모는 먼 곳으로 보내졌고, 게다가 이틀 뒤에는 한스 왕자가 열두 번째 왕자보다 덜한 축하를 받으며 태어났다. 열세 번째 왕자 한스가 치여 살리라는 것은 누가 보아도 뻔하디뻔한 일이었다.

``이게 내 아이인가?''

마르쿠스가 물었다. 충격적인 빨간머리의 모습에 얼굴을 찡그리고 있었다.

아냐는 마르쿠스가 믿을 만한 대답을 할 수 없었다. 그리고 이 순간에 아냐는 마르쿠스의 진정한 모습을 보았다. 아냐는 자신이 마르쿠스를 사랑했다고 생각했다. 갑작스레 이 군벌의 심판이 서린 고압적인 용모를 바라보니, 아냐는 다른 사람을 보고 있다는 생각이 들었다.

\textbreak

한스는 당연히도 존재가 희미해졌다. 어떻게 이러지 않을 수가 있을까. 위로만 열두 명에, 평민 혈통이라는 낙인에, 적출이 아닐 가능성까지도 있는데 말이다. 그리고 한스의 형제들은 자신들이 더 우위라는 것을 보이기를 매우 좋아했다. 지나가며 마주칠 때마다, 굳이 한스와 말을 섞으려고 하는 왕자들은 자신들이 한스를 어떻게 보는지를 일깨워 주곤 했다.

``남들이 네 어머니 그렇게 기억 못 하는 게 참으로 유감이네. 아니면 뭔가라도 될 기회가 있었을 텐데, 한스.''

토비아스가 말했다. 자신의 머리카락을 어깨 뒤로 넘기고 있었다.

한스는 얼굴을 찡그렸지만 어떠한 반박도 하지 않았다. 고개를 숙인 채 자신의 어머니와 같이 쓰는 방으로 돌아갈 뿐이었다. 하지만 짧은 다리를 아무리 움직여도 벗어날 수는 없었다. 열두 살 많은 것으로 으스대는 것에 보폭도 더 크니, 토비아스는 한스를 손쉽게 당해낼 수 있었고, 계속 높은 콧소리로 말을 이었다.

``난 네가 좋아, 정말로.''

토비아스가 말했다. 한스의 걸음이 빨라지는 것을 보고 작게 웃고 있었다.

``그러니까 충고 하나 하지. 네 어머니와 의절해.''

한스는 그대로 멈추어 선 채 토비아스를 올려다보았다. 인상을 쓰는 것을 참을 수 없었다. 자기 형제의 말이 얼마나 큰 영향을 주고 있는지를 얼굴에 드러내는 것은 실수지만, 평정심을 유지하는 것은 나중에야 배운 것이었다. 여섯 살 때는 여전히 생각을 숨기지 않고 있었다.

``뭐래.''

나지막하고 효과도 없는 대답이었지만, 한스는 자신의 어머니와 의절한다는 것이 얼마나 믿을 수 없는 말인지 말로 설명할 수 없었다. 사울을 제외한 모든 다른 왕자들은 보모의 손에 자랐다. 한스는 생모의 손에 자랐고, 다른 양육 방식은 상상할 수 없었다. 겨우 신체적으로만 편안하게 하는 이 끔찍한 이들의 차갑고 비인간적인 손길을 생각만 해도 한스는 욕지기가 날 지경이었다. 이들이 어머니의 사랑을 대신할 수 있다는 것처럼. 한스는 자신의 형제가 거의 가엾어 보이기까지 했다.

``말조심해. 또 알바르한테 예의범절 배우고 싶진 않잖아.''

토비아스가 말했다. 한스가 움찔하자 코웃음을 치고 있었다.

``꼭 발작이라도 하는 것 같네.''

아주 틀린 말은 아니었다. 알바르 생각에 한스는 숨고 싶어졌다. 알바르는 한스를 괴롭히는 습관이 있었다. 적출이 아니라는 소문을 확신하고 있고, 평민의 기운은 무엇이든 바로잡을 요량으로 항상 진정한 남자는 어떻게 행동해야 하는지 말하곤 했다. 다른 왕자들이 한스를—본인의 말로는—`괴롭히는' 것을 참고 넘어가지는 않았지만, 본인은 자신도 모르게 한스를 무서워하게 했다.

``앞으로 조심할게.''

한스가 중얼거렸다.

``이제 그냥 가게 해주면—''

``야, 봐봐. 우리의 인기인이 나타났어.''

토비아스가 말했다. 막 복도에 나타난 익숙한 얼굴을 보고 고개를 끄덕이고 있었다. 토비아스는 높은 목소리로 외쳤다.

``사울! 이거 반가운 얼굴이구만. 아직도 아버지의 총애를 받고 다니나? 아니면 이젠 에드문드를 좋아하시나?''

사울은 이들을 지나쳤다. 평소처럼 리드와 올리버가 따라붙었다. 이 삼인조는 멈추어서 인사했다. 사울은 평소대로 정중하게 질문을 회피했다.

``아버지께서는 우리가 지닌 장점으로 우리 모두를 좋아하십니다. 좋은 아침입니다, 토비아스 형님.''

올리버와 리드도 마찬가지로 인사했다. 다만 토비아스에게만 인사하고 있었다. 셋 다 한스에게는 눈길조차 주지 않았다.

``봤지, 한스? 저게 바로 예절 바르단 거야. 그리고 몇 살이지, 사울?''

사울이 대답하려 입을 열자 토비아스는 손을 올리고 나이를 맞히려 했다.

``일곱? 여덟?''

``\ldots열하나입니다.''

사울이 말했다.

토비아스는 사울을 조심스럽게 바라본 다음, 다들 들을 수는 있지만 들으라고 하지 않은 것으로 생각할 정도로 조용한 목소리로 중얼거렸다.

``그러면 밥 좀 더 먹고 다녀야겠네.''

그러고는 더 큰 목소리로 말했다.

``열하나! 잘 크고 있네. 수업은 어떤가? 한스는 잘 따라가고 있고?''

``잘하고 있어.''

한스가 말했다. 그리고 조금이나마 기대에 찬 목소리로 덧붙였다.

``사울 형이 가장 잘하지.''

사울은 눈 하나 깜빡이지 않고 계속 토비아스를 바라보았다. 올리버가 한스를 바라보려는 듯 움찔거리자, 사울은 올리버에게 손짓했고, 올리버는 가만히 있었다. 계속 친절한 미소를 띤 채 사울이 대답했다.

``수업은 잘 듣고 있습니다. 식사 때 보죠, 형님.''

한스는 말없이 있었다. 이 삼인조는 바로 떠나갔다. 마치 한스를 전혀 보지 못한 듯했다. 이들은 한스에 관한 말 한마디도 없이, 혹은 한스가 존재하지도 않는다는 듯 갈 길을 갔다. 한스는 이들이 가는 것을 지켜보았다. 마음이 동요하고 있었다. 토비아스가 괴롭히는 것이나 알바르가 못살게 구는 것은 참을 수 있었다. 가장 괴로운 것은 솔직한 불호가 아니라 무관심이었다.

``꼭 투명인간인 것처럼 한다니까. 사울이 요리사나 마구간 일손들한테도 인사하고 다니는 거 알아? 그리고 너한테는, 자기 형제한테는—진짠진 모르지만—아주 조용히 있네.''

토비아스는 상처에 소금을 뿌리는 재주가 있었다. 널리 소문난 성숙함에도 사울은 여전히 아이였고, 사울의 어머니인 왕비는 건강에 해가 될 정도로 아냐를 싫어했다. 물론, 아냐는 한때 왕비의 시녀였다. 그리고 장남 구스타프가 왕의 총애를 잃자 왕비의 모든 희망은 차남 사울에게 옮겨갔다. 한스는 경쟁 상대였다. 한술 더 떠 자신의 발을 씻겨주고 닦아주던 이와의 경쟁이었다. 용납할 수 없는 일이었다. 사울은 자신의 어머니를 보고 배웠을 뿐이었다.

``꼭 나보다 형을 더 좋아하는 것처럼.''

한스가 중얼거렸다. 토비아스가 잠깐이라도 굳은 채 서 있는 것을 보자 기분이 좋아졌다.

``하지만 네가 더 만만하지.''

토비아스가 말했다.

더는 말없이 있기 힘들었지만, 한스는 자신의 어머니가 대립각을 세우는 것을 싫어하는 것을 알기에 고개를 숙인 채 지나갔다. 다행히도 토비아스는 가는 동안 조용히 있었지만, 한스는 방 앞에 도착해도 토비아스가 여전히 옆에 있는 것에 화가 났다. 토비아스가 문고리를 향해 고개를 돌리자, 한스는 토비아스를 쏘아보며 말했다.

``어머니께서 안 보고 싶어 할 거야.''

``하지만 지난번엔 정말 멋진 대화를 나눴는걸.''

토비아스가 말했다.

한스는 토비아스가 자신의 어머니를 울게 한 이야기를 꺼내지 않기로 했다. 구스타프는 이런 옹졸한 짓을 하지 않았지만, 파비안과 토비아스는 이 평민 출신 여인의 천한 배경을 헐뜯는 이야기를 하며 매우 큰 즐거움을 느꼈다. 토비아스는 특히 한스에 관해 보고하는 것을 매우 좋아했다. 형제들이 피해 다닌다는 둥, 아냐 자신은 초대받지 못하고 한스는 형식적으로만 초대받는 만찬 때에 자신의 아버지가 약점을 찌른다는 둥 이야기를 했다.

``딴 사람이나 귀찮게 해, 형이랑 참고 얘기할 사람이나 있으면.''

한스가 말했다. 도서실에나 가서 박혀 있으라, 한스는 생각했다. 자신의 어머니는 토비아스의 모진 말에 괴롭힘을 받을 이유가 없었다.

``알았어.''

토비아스가 말했다. 항복의 표시로 손을 들어 보이고 있었다.

``하지만 내 말 생각해 보라고, 한스. 아냐는 너한텐 짐일 뿐이야. 그리고 본인도 잘 알고 있을 테고.''

``내 어머니라고.''

한스가 말했다. 말투는 점점 거칠어졌지만, 토비아스는 어깨를 으쓱일 뿐이었다.

``내가 필요하면 어디로 오면 되는지 알지?''

토비아스는 말을 마치고 떠나갔다.

토비아스가 모퉁이를 돌아 사라진 후에야 한스는 방문을 열었다. 여전히 낮이었으니 방 안은 밝아야겠지만, 아침에 일어났을 때처럼 모든 커튼이 쳐져 있었다. 밖에 나간 것일까. 아냐는 밖에 나가는 일이 잘 없었다. 온 성이 보내는 야유를 피해 안에 있는 것을 선호했다. 한스는 커튼을 걷고 밖을 바라보았다. 아름다운 날이었다. 서던 제도의 날씨가 완벽하지 않은 날은 없지마는 이날은 특히 화창하고 맑은 하늘에 구름 몇 조각만이 뙤약볕만을 적당히 가리고 있었다.

뭐, 날씨를 즐기러 간 것일지도 모르겠다. 곧 돌아오리라.

한스는 그대로 기다렸다. 한 시간, 두 시간, 세 시간. 그리고 자다 일어나서도 계속. 그러나 밤이 되어서도 한스의 어머니는 돌아오지 않았다.

하인이 생색내는 듯하게 만찬 초대를 알리자 한스는 애가 타기 시작했다. 방을 거의 떠나지 않은 것은 차치하고서라도, 한스의 어머니는 한스를 만찬에 보내기 전에 항상 단정히 해주곤 했다. 지금쯤이면 이미 돌아와 있어야 할 것이다. 한스는 식사를 거르겠다고 하며 하인을 돌려보냈다. 자신의 무례함으로 벌을 받을 것을 알고 있었지만, 깨어 기다려야 했다.

다시 시간이 지났다. 한 시간, 두 시간, 그러고는 세 시간.

한스는 계속 깨어 있으려 하며 소리 없이 밤새 방 안에 있었다. 그러고는 불청객처럼 잠이 찾아왔다.

\textbreak

한스는 다음 날 아침에 빈방에서 일어났다. 담요를 걷어치우고, 한스는 밖으로 뛰쳐나갔다. 좌우를 둘러보아도 아무도 없자, 한스는 아무도 오지 않는 성의 이 먼 구석으로 쫓겨난 것으로 욕을 내뱉었다. 삼 분 동안 복도를 달려가다가 처음 보는 사람에게 한스는 말을 걸었다. 전에 한스의 어머니와 같이 일했던 사람이었다.

``제 어머니 보셨나요?''

한스가 물었다.

``누굴 말하는지 모르겠구나.''

``제 어머니를 모른다는—''

이 하녀가 물러나려 하자 한스는 말문이 막혔다. 하지만 충격이 가시자, 한스는 달려가 앞길을 막은 다음 말했다.

``거짓말 마요! 모를 리가 없잖아요—''

``네 어머니는 여기서 산 적이 없어. 알아듣겠니?''

이 비밀스러운 말을 끝으로, 이 하녀는 떠나갔다.

누구를 만나든 모두 똑같은 반응이었다. 모두 한스의 어머니가 누구인지 모른다고 대답했다. 한스가 자신의 방에서 혼자 살아왔다고 말하고, 어머니가 누구든 한스를 직접 기른 적이 없다고 말했다. 같은 대답을 들을 때마다, 한스는 점점 미칠 것만 같았다. 물론 한스의 어머니는 이곳에 살았다. 대체 무슨 장난인가. 파비안의 짓이 틀림없다. 유괴까지 한 것일지도 모른다.

한스는 도서실을 지나쳐 가려다 멈추어 섰다. 돌아선 다음 문을 열어젖히자, 예상대로 토비아스가 먼지 쌓인 커다란 책에 빠져있는 모습이 보였다. 한스는 토비아스를 향해 돌진하고는 손에 들고 있는 책을 쳐서 떨어뜨렸다. 책이 책상에 부딪혀 빈방 안에 울려 퍼졌다. 책꽂이에 꽂힌 책들이 흔들렸다.

토비아스는 한숨을 내쉬었다.

``대체 왜 그러는데?''

``우리 엄마 어딨어?''

한스가 물었다.

무슨 이유인지, 토비아스는 자리에서 바로 일어나 빈 도서실을 둘러본 다음 한스에게 시선을 돌렸다.

``다른 사람한텐 물어봤어?''

``하녀들한테만—''

``다행이네. 다른 사람 말고 나한테 오길 잘한 거야. 아버지 귀에 들어갔으면 큰일 났을걸.''

토비아스가 말했다. 다시 의자에 주저앉고 있었다. 하지만 토비아스의 말과는 달리, 한스는 형제 중 누구를 봐도 단박에 가서 물어보았을 것이다.

``엄마 어딨는지 알고 싶다고!''

한스가 외쳤다.

``이젠 아무도 그 사람 얘기 안 할 거야. 너도 똑같이 하는 게 좋을걸.''

토비아스가 말했다. 한스가 질문을 반복하자, 토비아스는 한숨을 쉬고 고개를 저었다.

``알아봤자 좋을 거 없다고.''

``알고 싶다고.''

한스가 서둘러 말했다. 토비아스는 일부러 천천히 책을 덮고 책에 쌓인 먼지를 털어내고 있었다. 한스가 초조하게 기다리는 동안, 토비아스는 책을 책꽂이에 다시 꽂고 있었다. 마침내 한스가 말했다.

``말해!''

토비아스는 자리에 앉아 한스의 눈높이에서 한스를 바라보고 있었다.

``정말로?''

``응!''

``알았어. 따라와.''

한스가 따라잡기 힘들 정도로 빠르게 토비아스는 도서실을 나왔다. 한스는 토비아스를 따라갔다. 반은 의심에 차 있었지만, 반은 희망에 차 있었다. 토비아스를 믿어도 되는지 의심이 들었지만 달리 방법이 없었다. 성을 나와 마구간에 이르는 길에 들어서자 한스는 더욱 화가 날 뿐이었지만, 토비아스는 아무런 말도 꺼내지 않았고, 한스가 부르자 발걸음을 더욱 빠르게 했다.

이들은 마구간을 지나 숲으로 갔다. 아무도 오지 않은 듯한 곳으로 더욱 깊이 들어가더니 곧 나무가 햇빛을 가려 무성한 잎의 틈새로만 빛이 조금씩 들어왔다. 풀들이 크게 자라 있어 한스는 풀의 가시에 베이지 않으려 조심하고 있었지만, 서두르는 탓에 그럴 수 없었다. 다리에 따끔한 느낌이 여러 번 왔지만 어렴풋한 느낌일 뿐이었다.

마침내 이들은 멈추어 섰다.

``다 왔어.''

토비아스가 말했다.

그저 공터일 뿐이었다.

``아무것도 없잖아!''

한스는 주변을 둘러보았다. 장난질에 화가 났지만, 더 시간을 낭비하고 싶지 않았다. 한스가 떠나기 전에, 토비아스는 한스의 손목을 잡아 공터로 다시 돌려세웠다.

``봐봐!''

토비아스가 말했다. 나무 한 그루를 가리키고 있었다.

그저 평범한 나무였다.

하지만 한스는 자세히 들여다보고 있었다. 토비아스가 정확히 가리키는 것을 보자 입 안이 마르고 숨이 멈추고 심장도 멎는 듯했다. 한스가 보고 있는 것은 공터가 아니었다. 나무도 아니었다. 이 아무도 없는 숲 속에 있는 그 무엇도 아니었다.

가지에는 올가미가 걸려있었다.

``말도 안 돼.''

한스가 말했다. 눈을 감고 있었지만, 방금 본 광경은 이미 머릿속에 새겨져 버렸다. 더욱 선명히 보일 뿐이었다. 한스는 다시 눈을 뜨고 토비아스에게 분노에 찬 시선을 돌렸다. 토비아스—같이 있고, 두려움 없이 화를 쏟아낼 수 있는 이 말이다.

``말도 안 된다고!''

한스는 토비아스에게 달려들었지만, 옆으로 내팽개쳐질 뿐이었다. 등이 땅에 부딪혔다. 팔로 몸을 받쳐, 토비아스가 일으켜 주려 하는 참에 한스는 바닥에서 일어섰다.

``네 어머니는 너 때문에 돌아가신 거라고.''

토비아스는 빈정거리는 말투로 말했다.

`이런 취급 이상을 바라는 게 그렇게 잘못된 건가요?'

`이 엄마는 더 많은 걸 해줄 수 있단다, 한스야.'

한스는 바닥에 무릎을 꿇고 밧줄을 바라보았다. 자신의 어머니가—

정말로 한스 자신 때문에, 자신이 한 말 때문일까? 하지만 한스는 정말로 더 큰 것을 원하지 않았다. 한스가 필요한 것은 자신의 어머니뿐이었다. 왜 한스를 위해 이런 선택을 한 것일까\ldots? 한스는 결코 자신의 어머니를 두고 다른 것을 선택할 이유가 없었다. 한스의 어머니는 이를 알았어야 했다. 그래야만 했다.

하지만 한스의 잘못은 아니다.

그리고 한스의 어머니의 잘못도 아니다.

``들어 봐.''

한스가 생각에 잠겨있자 토비아스가 말했다. 

``아버지가 다시는 네 어머니 얘기 안 하도록 명령했어. 그리고 네가\ldots\,적자인지 아닌지 얘기도. 널 인정해줬다 생각하고 넘어가자고. 괜히 헛짓하지 말고. 알아들었어?''

``\ldots인정해주셨다고?''

한스가 중얼거렸다.

``인정해주셨어.''

토비아스가 말했다.

한스는 일어서서 눈을 감았다. 이제는 올가미의 모습이 아른거리지 않았다. 이후에 이 일을 생각할 때, 한스는 얼마나 이러고 서 있었는지 떠올릴 수가 없었다. 다시 눈을 뜨자, 한스는 토비아스를 바라보고는 자신의 장소인 성으로 시선을 옮겼다. 그리고 이 희생으로 얻어내야 할 왕좌를.

``난 열세 번째 왕자야. 그리고 아버지의 아들이고.''

한스가 말했다.

불운의 수 십삼.

한스의 잘못도, 어머니의 잘못도 아니다. 열두 왕자를 탓할 일이다. 한스에게는 형제 열둘이 있다. 한스는 한 명씩 한 명씩, 이들에게 쓴맛을 보여줄 것이다.

한 명씩 한 명씩, 아무도 남지 않을 때까지.

