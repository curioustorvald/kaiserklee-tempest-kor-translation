

\chapter[15장  엘사의 고뇌][15장\hspace*{.5em}엘사의 고뇌]{15장 \ 엘사의 고뇌}



안나는 깨어나서 눈을 깜빡여 눈앞의 검은 점들을 없애려 했다. 머리는 한 무리 병사들이 진격해오는 듯 아파졌다. 목은 아프고 입 안는 바싹 말랐다. 그래도 겨우 침을 삼킬 수 있었다. 술은 영 좋지 않았다. 그래도 최소한 춥지는 않았다\ldots

꽤 많이 본 외투가 어깨를 덮고 있기 때문이었다. 안나는 외투를 벗었다. 한스가 입던 것임을 알아챘다. 그러고는 자신의 상태를 확인하고 여전히 옷이 멀쩡히 있는 것을 알기 전까지 거의 허둥대고 있었다. 이무 일도 일어나지 않았다. 일단 진정이 되니 모든 것이 기억났다. 이들은 마시고 또 마셨다. 그러다가 한스는 그만 마시라고 말했다. 현명한 말이었을 것이다. 안나는 오래지 않아 쓰러졌다.

그렇다면 한스는 어디에 있을까? 안나는 주방을 둘러보았지만 한스의 흔적은 없었다. 안나는 한스가 자신을 버리고 간 것이 다소 실망스러웠지만, 자신이 쓰러지고 나서 떠난 것으로 생각했다. 이제는 술을 입에 대지 않으리라. 안나는 탁자 끝으로 가 일어선 다음 머릿속을 비우려 고개를 흔들었다. 더 어지러워질 뿐이었다. 안나는 고개를 아래로 떨구었다. 인제야 촛대 아래에 쪽지가 끼워져있는 것을 발견했다.

안나는 쪽지를 집어 천천히 펼쳤다.\begin{letter}

같이 있을 수 없게 되어서 참으로 유감이네요, 안나 공주님. 정말 걱정되는군요\ldots\,공주님과 정말로 사랑에 빠지지는 않을까 하고요.

걱정은 마세요. 아직은 그렇지도 않으니까요. 공주님의 마음이 다른 사람에게 가 있다는 건 압니다. 그냥 공주님의 진심 어린 조언에 크게 감명을 받은 것뿐이에요. 이제부턴 다른 남자분을 감명받게 하지 말아주십시오. 결국은 그 남자분이 공주님을 좋아하게 될 테니까요. 전 이미 너무 멀리 갔습니다. 이런 데에 마음을 팔릴 새가 없다고요. 그래도 감사드립니다.

끝으로, 조언을 좀 드리고자 합니다. 듣지 않으실 건 알지만요. 마음으로 사랑하는 게 그냥 해야만 하는 거라고 말씀하셨죠. 그건 소중하긴 하지만, 다시는 돌아올 수 없는 길이기도 합니다. 특히 저도 완전히 이해 못 하는 엘사 여왕님의 경우는요. 그래도 그렇다고 멈추지는 않으시겠죠.

\end{letter}

``그게 그렇게 뻔했나? 내가\ldots''

안나는 한숨을 쉬고 쪽지를 구겼다.

대체 무슨 말도 안 되는 소리를 한 것인가. 모두가 알고 있었다. 그리고 솔직히 말하자면, 안나 자신도 엘사에게 정확히 어떤 마음이 드는지를 알고 있었다. 마음으로 사랑하는 것은 해야만 하는 것이다. 어디에서 나온 말이었을까? 안나는 엘사를 싫어했다. 그러고는 엘사 때문에 혼란스러워졌고, 소리를 지르고, 상처를 받았다. 그래도 안나는 포기하고 싶은 생각이 들지 않았다.

그저\ldots\,이렇게 되어야만 했다.

안나는 비틀거리며 주방을 빠져나왔다. 어지러움이 숙취 때문인지, 혹은 엘사와 사랑에 빠졌다는 깨달음 때문인지는 알지 못하고 있었다. 엘사 여왕, 아렌델을 정복했으면서도 자신의 상처를 처치해주고, 위험에서 보호해주고, 마음을 써주는 여왕 말이다.

안나는 다시 엘사를 봐야 했다.

``벌써 너무 늦었어\ldots''

사람 없는 주방에서, 안나는 창밖을 내다보았다. 구름 없는 하늘에 달이 밝게 빛나고 있었다. 한스와 같이 있기로 한 때는 저녁녘이었다. 지금은 아주 늦은 시간이 되었을 것이다.

``어디 계시려나?''

다시 일을 바로잡아야 했다.

안나는 방으로 돌아가 조심스럽게 문을 열었다. 둔탁하게 끼익 하는 소리가 났다. 안나는 눈살을 찌푸렸다. 안을 들여다보자, 방 안에는 엘사가 있던 흔적도 없었다. 아주 놀랍지는 않았다. 다음은 집무실을 가볼 차례겠지만, 안나는 그곳에도 없으리라는 느낌이 들었다. 이 난리 후에는 말이다. 안나에게는 확실한 데가 있었다. 하지만 사생활 침해가 아닐까?

절박한 때에는 앞뒤를 가리지 않게 마련이다. 안나는 눈을 감고 마음속의 빛이 쏟아지는 모퉁이에 이르렀다. 전에 딱 한 번 해본 일이었다. 지난번에는 그냥 물러났다. 이번에는 더 안쪽으로 들어갔다. 연결이 넓어져 더욱더 많은 낯선 생각이 자신의 마음으로 쏟아지는 것이 느껴졌다. 매우 혼란스러웠다. 반대편에는 혼란과 심란함이 소용돌이를 이루고 있었다. 시작도 끝도 없이 뒤죽박죽되어 있었다. 모두 엘사에게서 온 것이다.

안나는 한편으로는 정말로 되리라고는 생각지 못했다. 혼란스러운 생각을 지나 발코니에 서 있는 엘사의 모습이 흐릿하게 나타났다.

안나는 어지러워질 것을 예상하며 물러났다. 하지만 그렇게 되지 않고 숙취로 인한 두통이 사라지고, 어느 때보다도 더욱 단호해졌다. 안나는 대현관으로 달려가 나선계단을 올라갔다. 탑으로 가던 길과 정확히 같은 길을 따라가고 있는 것을 알아채자 걸음이 느려지기는 했지만. 기시감이다. 그래도 안나는 멈추지 않았다. 다시 걸음을 재촉하면서, 안나는 문을 열어젖히고 발코니로 달려갔다.

\textbreak

이날 밤은 매우 추웠다. 평소보다도 더욱 추웠다. 그리고 어찌 된 일인지 한데 모여 서슬과 같이 느껴졌다. 바람을 막으려 팔을 올려도 맹렬한 바람 때문에 눈물이 맺혔다. 안나는 태풍 안으로 걸어 들어가는 듯했다. 하지만 정말로 그런 것인지도 모르겠다. 안나는 태풍의 눈으로 가고 있었다.

하지만 추위가 느껴지지 않았다. 앞에 있을 것만을 생각하는 동안에는. 그저 몇 걸음 거리에 엘사가 서 있었다. 등을 보인 채 난간에 손을 올려놓고 있었다. 앞에는 죽은 나무의 숲이 펼쳐져 있었다. 울창했을 나무들은 앙상한 가지만이 남아있었다. 기괴한 모습으로 휘어지고 비틀어져 있었다. 바람에 휘어지고 눈과 부딪혀 고통스러운 모습을 보이고 있었다. 이런 밖에서 살아남아 있는 것은 아무것도 없었다.

``뭔가 느낌이 온 것 같았어.''

엘사가 중얼거렸다. 안나가 다가가도 뒤돌아보지 않고 있었다. 바람은 천천히 산들바람으로 약해졌다.

``내 마음에 들어오려 하는 걸 느낄 수 있었어.''

``그건 정말 죄송해요. 하지만\ldots\,어디 계신지를 알아야 해서요.''

``괜찮았어. 따뜻한 느낌이었어.''

``옆에 서도 되나요?''

안나가 물었다. 엘사는 고개를 끄덕였다. 여전히 뒤돌아보지 않고 있었다. 안나는 돌아서서 난간에 기대어 엘사는 밖을 바라보고 있고, 안나 자신은 성을 향하고 있었다. 아마 겁을 내는 것일지도 모르겠다. 벽을 바라보고 있으니 말이다. 하지만 엘사의 목소리를 듣기만 하는 것도 견딜 수가 없었다. 안나에게 엘사—위엄차고, 당당하고, 아름답고, 손끝에서 끝없이 얼음을 뿜어내는 끔찍이 강력한 이—는 두려우면서도 감탄하게 되는 존재였다. 그저 평범한 인간이 아닌 폭풍 자체이고, 여신이었다.

본인이 아무리 강력할지라도 엘사는 심란해 하고 있었고, 안나는 마음 깊은 곳에서 고통을 느끼고 있었다.

``왜 여기 나와계신 거예요?''

안나는 부드러운 말투로 물었다.

``모르겠어. 여러 이유가 있지.''

엘사가 말했다.

``말해주실 거예요?''

``평소답지 않게 주저하는 것 같아.''

안나는 고개를 낮추고 발을 바닥에 비비고 있었다. 적당한 말이 떠오르지 않아 말없이 있었다. 자신의 느낌을 정확히 표현할 방법이 없었다. 두서없이 주절대는 데로 빠지지 않고서는.

``뭐, 이\ldots\,이젠 교훈을 얻었어요. 좀 전에 일 말예요. 제가—''

``그때 일은 정말 미안해. 너무 과했어. 다신 그럴 일 없을 거야.''

또 나왔다. 사과하는 버릇 말이다. 안나는 예상하고 있었다. 순전히 많이 들어보았기 때문은 아니었다. 엘사는 거리를 두려고 사과할 뿐이었다. 안나를 제외한 사람에게는 절대로 사과하지 않을 것이다. 가까이 올 수 있는 사람이 없기 때문이다. 안나는 이제 엘사가 같은 말을 할 때마다 자신을 떨어트려 놓으려던 것을 알아챘다.

``과한 게 아녜요. 제가 몰아붙인 거죠.''

``제어력을 잃지 않아야 했어.''

``그저 화났다는 게 제어력을 잃어버렸단 건 아녜요. 모든 사람은 다 가끔 그런다고요. 그러니까 화났다고 다른 것까지 무시해버리지 말라고요—''

안나는 깊게 숨을 들이마셨다. 안나는 다시 전철을 밟고 있었다. 다시 아무 생각 없이 밀어붙이고 있었다. 그러기에는 적절한 때와 장소가 아니었다.

``처음부터 말할게요. 알았죠? 제가 사과드려야 하는 거예요. 생각 없이 그랬으니까요.''

``사과 받았어.''

담담한 말투를 들으니, 안나는 이와 반대되는 소리를 들었으면 했다. 엘사는 화나 있지 않았다. 차라리 화난 목소리를 들으면 마음이 놓였을 것이다. 이렇게 단조로운 말투는 훨씬 심각했다. 엘사는 그저 더는 신경을 쓰지 않고 있었다. 다시 포기하고 있었다.

``왜 여기 있는진 말 안 하셨는데요.''

안나가 말했다. 안나의 시선은 탑 꼭대기까지 점점 위로 올라갔다. 안나의 마음은 탑 안에 있는 것들로 옮겨갔다.

``\ldots정말 완벽해.''

엘사는 고개를 들었다. 안나는 돌아서서 달 대신 엘사의 시선을 바라보았다. 창백하고 흠 없는 달은 아래에 있는 나무를 비추고 있었다. 죽은 껍질도 빛을 받아 은은한 하얀 빛으로 빛나고 있었다.

``거울에 관해서 물어봤지. 그리고 나도 궁금해했고. 어떻게 내가 거울과 연결돼 있을까? 그러고 보면, 거울하고 난 공통점이 아주 많이 있어. 둘 다 깨져있고 알아볼 수 없지. 거울은 달과 같은 아름다움을 꿈꿀까?''

``이렇게 멀리 있어서 완벽해 보이는 것뿐이에요.''

안나가 말했다.

``가까이에서 보면 모든 건 다 흠이 있다고요. 당연한 거예요.''

``\ldots당연한 거다.''

엘사가 말했다. 엘사는 고개를 아래로 낮추었다.

``자신이 누구인지처럼 간단한 걸 궁금해해 본 적이 있니?''

``아뇨.''

안나가 말했다.

``그럴 필요가 없겠지. 넌 아렌델의 안나 공주니까. 하지만 난\ldots''

엘사는 떨며 웃음을 내뱉었다. 마른 강둑에 있는 바싹 마른 갈대의 소리와 같았다.

``난 누구일까? 아렌델의 평민 엘사일까, 아니면 서던 제도의 엘사 여왕일까?''

안나는 그대로 굳었다.

``그러면 생각 안 해본 거네. 왜 안 물어보나 했어.''

엘사가 말했다. 이번에 나온 엘사의 웃음은 더 명확히 울렸다. 아침때처럼, 안나는 아름다운 목소리에 다시 놀랐다. 오직 반대로 말이다. 엘사와 같은 목소리는 말에 빛깔과 질감을 담을 수 있었다. 엘사의 입에서 이토록 씁쓸한 소리를 들으니 몹시 잘못되었다는 느낌이 들었다.

``난 반과 엘미라한테서 태어났어. 아렌델 왕국의 변두리에 사는 농부 가족이었지. 그러다가 서던 제도의 엘사 공주로 길러진 거야. 저주를 타고났고, 다른 어디에서도 안정을 찾을 수 없어서.''

``그래서 아렌델에 관해 그렇게 물어보시는 거예요?''

``아니야. 그런 썩어빠진 데는 신경도 안 쓰여. 거긴 내가 태어난 데고, 저주가 스스로 나타난 데일 뿐이야. 할 수 있다면, 그것도 완전히 지워버리고 싶어.''

엘사가 말했다.

다른 때면 안나는 아렌델을 변호하고, 자신의 고향이 썩어빠진 데로 불린 것에 화가 치밀었을 것이다. 그러나 무슨 일이 일어났는지를 알고 있으니\ldots

엘사는 당연히 아렌델을 싫어할 수밖에 없었다.

``그걸 저주로 생각하는 거예요?''

안나는 대신 이 질문을 던졌다.

``상관이 있니? 내가 아는 한, 이건 그럴 수도 있지. 왜 있는지 모르겠어.''

``이건 마법이라고요.''

안나는 강조하며 말했다. 안나는 손마디의 살이 터지도록 난간을 꽉 쥐었다.

``그냥 아름다워지게 두면 아름답게 된다고요. 숨길 것도, 부끄러워할 것도 없고요. 이건 여왕님 일부예요.''

``한땐 아름다웠지. 가능성에 빠져들기도 했고.''

엘사는 동의하며 말했다. 그러고는 말투가 채찍처럼 사나워졌다.

``하지만 넌 진정한 내가 누군지 잊고 있어.''

웅얼거리는 소리가 뒤에서 들려왔다. 안나는 엘사를 돌아보았다. 엘사의 검지 끝에 전기가 나와 있었다. 엘사는 검지를 앞으로 튕겼다. 벼락이 큰 소리를 내며 하늘을 갈랐다.

안나는 뒤돌아섰다.

``뭘 보이시려는 거예요?''

``난 파괴하고 목숨을 앗아가고, 그러면서도 후회하지 않아. 한편으로는 즐기기까지 하고. 이걸 이해할 수 있어, 안나야?''

엘사가 물었다. 말투는 무덤덤했지만, 안나가 느끼는 것을 가려줄 수 있는 것은 아무것도 없었다—자신에 차 있을 때에 흥분이 가져다주는 짜릿함, 자비에 매달려 두려워하는 이들을 볼 때에 힘이 주는 찬란함. 오랫동안 이는 달리 되어 왔다.

얼마 전까지는 이에 욕지기가 났을 것이다. 안나는 자신의 답이 하려던 말과 같으리라고는 생각지 못했다.

``네, 할 수 있어요.''

엘사는 잠깐 멈추었다. 그러고는

``어떻게?''

라고 말했다.

이들은 오랫동안 이 문제를 피해왔다. 이들 앞에 뻔히 드러나 있는 문제를 무슨 수를 써서든 무시해왔다. 엘사의 심장 조각이 자신의 심장에 박힌 채 깨어난 이후로, 이들은 단 한 번도 엘사의 기억에 관해 대화를 나눈 적이 없었다. 안나는 어떻게 대화를 꺼낼지 알지 못했다. 엘사는 기꺼이 잊히게 해왔다. 이제는 그럴 수 없었다.

``모든 걸 봤다고요.''

안나는 중얼거리는 목소리로 말했다.

``그러면 동정이네.''

엘사는 매섭게 말했다.

``필요 없어. 이제는 다른 사람한테 휘둘리는 나약한 아이가 아니라고. 나한테 상처를 줄 수 있는 건 없어.''

``동정이 아녜요. 거기 있었다고요. 여왕님이 느낀 걸 느꼈다고요. 지금도 그대로 느끼는걸요.''

안나가 말했다. 안나는 심장 위에 손을 올려놓았다. 손가락으로 심장이 고동치는 것을 느끼고 있었다. 안나는 모든 감정을 직접 겪는 것처럼 느낄 수 있었다. 두려움과 혐오와 절망이었다.

``아마도 완전히는 이해 못 할 거예요. 그렇게 해주시기 전까진 그러지도 못할 거고요. 전 시도하고 있다고요, 여왕님.''

``그러면 계속 시도해봐. 내가 어떤 느낌인지 말해보라고.''

엘사는 도발하며 말했다.

``그럴게요.''

``그리고 그게 틀린다면?''

이 반쯤 위협하는 말은 연기처럼, 그리고 장송곡처럼 공중에 머물렀다.

``틀릴 일 없을 거예요. 여왕님을 아니까요.''

안나가 말했다.

엘사는 아무 말도 하지 않았다. 안나는 계속 하라는 허락으로 받아들였다.

``처음에는 다르다는 걸 싫어한다고 생각했어요. 하지만 이건 틀렸죠. 여왕님은\ldots\,마법을 좋아하셨어요. 지금도 좋아하시고요.''

아무리 인정받기를 원했어도, 엘사는 자신의 `재능'을 버리는 것을 단 한 번도 고려하지 않았다. 안나는 알고 있었다. 어떻게 모를 수 있을까, 엘사가 자신의 능력에 감탄한 기억이 마음속에 새겨져 있는데 말이다. 이는 세상에서 가장 아름다운 것이었다. 그리고 항상 그래야 할 것이다.

엘사는 다시 아무 말도 하지 않았다. 안나는 계속했다.

``하지만 여왕님이 좋아한 건 힘이 아니었어요.''

엘사가 갈망한 것은 눈송이의 노래, 눈사람의 춤이었다. 절대로 전쟁터의 신음과 피에 젖은 얼음의 냄새가 아니었다.

``여왕님은 만드는 걸 원했죠, 파괴하는 게 아니라요. 그리고 이게 문제가 되는 거고요. 왜냐면\ldots''

안나는 말을 멈추었다. 아마도 상상이었을지도 모르겠지만, 안나는 심장이 멎는 듯했다. 난간에 살얼음이 퍼져 내려가는 것은 확실히 상상 속 모습이 아니었다. 안나는 살얼음이 발밑으로 다가오는 것을 볼 수 있었다.

``이젠 못하시니까요.''

안나는 난간을 꽉 쥐었다. 얼음이 손 아래에서 부수어지는 것이 느껴졌다.

``문제 되는 건 없어.''

엘사가 말했다. 마침내 말을 꺼냈고, 말투가 매우 조절되어 있어서 안나가 얼마나 근접했는지를 보였다.

``네 말이 아마도 한땐 맞는 말이었겠지만, 이제는 통제권을 쥐고 있어. 이 힘을 위해서라면 아이 같고 순진한 소망은 희생할 수 있다고—''

``그러면 아침엔 왜 도망친 거예요?''

안나가 물었다.

엘사는 굳어졌다.

``안 도망갔어.''

안나는 슬프게 미소를 지었다.

``항상 도망치고 있다고요, 여왕님.''

``무엇에서?''

``여왕님 자신이요.''

안나가 말했다. 안나는 심장이 고동치는 것, 정곡을 제대로 찔려 화가 밀려오는 것을 무시할 수 있었다. 엘사의 감정이지, 자신의 감정이 아님을 알았기 때문이다.

``여왕님은 그저 기억에서 도망치려고 감정을 억눌러 왔다고요. 이게 그렇게 인정하기 힘드세요?''

안나는 엘사가 크게 숨을 내쉬는 것을 들었다.

``그냥 저한텐 좀 솔직해지세요, 제발요.''

안나가 속삭였다.

``\ldots알았어.''

엘사가 말했다. 이제는 급하게 말하고 있었다. 궁지에 몰리고 있었다. 숨을 쉴 필요가 없어도 숨이 가빠졌다.

``나한텐 설계하는 거밖에 없었어, 안나야. 내가 아마도, 아마도 괴물이 아니라고 생각하게 해준 거였다고. 그러고는 직접 망쳐놨어. 내가 버렸다고. 옳은 선택을 해야만 했어.''

``이건 여왕님 잘못이 아녜요.''

``정말 아니라고? 내가 선택했어. 후회할 이유가 없다고.''

``여왕님이 선택한 게 아녜요. 아마 선택했다고 생각하시겠지만, 선택할 수가 없는 거였다고요.''

안나는 엘사가 질문을 던지고 싶어하는 것을 느낄 수 있었지만, 옆길로 샐 수 없었다. 이것이 이길 수 없는 싸움이라는 것을 알고 있었기 때문이다, 아직은.

``여왕님은 진정한 모습을 보이지 않고 있다고요.''

``\ldots그게 무슨 말이야?''

엘사는 예의 싫증으로 한숨을 내쉬었다. 마음 안에 새겨져 있는 것 말이다. 안나는 엘사가 대화를 그만 하고 싶다는 것을 알고 있었다. 그렇게는 두지 않을 참이었다.

``동정할 게 없다는 말 맞아요. 여왕님께 일어난 일에도—안 일어났으면 하지만요—전 동정 안 해요. 이 일에 지금의 여왕님이 됐고, 이건 다 여왕님의 한 부분이라고요. 아무 일도 없던 척하기 전까지는요.''

안나는 어떻게 대화가 이렇게 되었는지 알지 못했다. 안나는 사과하려 했지, 나무라려던 것은 아니었다. 그럼에도 엘사가 기꺼이 본인의 한 부분을 파괴했다는 것에 격분하고 있었다. 그러나 엘사에게 격분하는 것은 아니었다. 안나는 엘사를 탓하지 않았다. 엘사 때문에 화가 났지만, 엘사에게 화가 난 것은 아니었다.

엘사는 이를 세게 악물었다. 안나는 이가 갈리는 작은 소리를 들을 수 있었다.

``그게 잘못된 거니?''

안나는 간단히 아니라고 말할 수 있었다. 엘사가 원한 대답이고, 엘사를 몰아붙이지도 않을 대답이었다. 다시 둘 사이가 원래대로 될 대답이었다. 다시 전대로 돌아갈 수 있었다. 그리고 엘사가 한 일도 이해할 수 있는 일이었다. `거의' 옳은 일이었다. 지금의 자신이 되게 한 것이 매우 고통스러운 것인데 다른 사람이 되려 하는 것이 그토록 잘못된 것일까.

``네. 여왕님은\ldots''

`겨우 이 정도가 아니었다고요. 여왕님의 마법 이상이었다고요. 선택할 수 있었으니까요. 하지만 지금은 스스로 병기가 되려 하잖아요. 이건 여왕님이 원한 게 아녜요. 이건 마법이 아니라고요. 이건 여왕님이 아녜요.'

말을 멈추고 그저 생각하고자 하는 충동이었다. 생각으로 자신의 말과 느낌을 전달하고자 하는 충동이었다. 말로는 충분하지 않았다. 말은 약했다. 감정과 인상, 추상적인 생각과 감각, `느낌'—이것들이 엘사에게 보낸 것들이었다.

`진정한 모습을 보이세요. 느끼세요. 감추지 말라고요.'

엘사는 몸을 앞으로 숙이고는 떨면서 깊게 숨을 들이마셨다. 알아들었다는 표시를 전혀 하지 않았지만, 안나는 알아들은 것을 알았다. 이것으로 충분했다. 엘사는 오랫동안 말을 꺼내지 않았다. 그러다가—

``너를 죽이라 하셨어. 이제 두 번째야.''

둘 사이에 무거운 침묵이 내려앉았다. 안나는 자신의 숨소리만을 들을 수 있었다. 이마저도 공허 안에 들어선 듯했다. 누가 한 말인지는 물을 필요가 없었다. 엘사는 오직 한 사람에게만 명령을 받았다. 안나는 그저 단순히 물었다.

``왜요?''

``네가 훼방을 놓고 있으니까.''

``제가요?''

``그래.''

``또 뭐라고 하셨어요?''

``네가 모든 걸 엉망으로 할 거래. 날 망가뜨릴 거래.''

``제가요?''

``응.''

엘사는 급히 말했다.

``다시 날 산산이 부수었어. 아주 완전히. 평화를 발견해도 너 때문에 다시 심란해졌다고.''

안나는 턱을 들어 목을 보이고 자신의 운명을 기다렸다. 그러나 도발과 믿음에서 나온 행동이었다.

``그러면 절 죽이실 거예요?''

``안 그럴 거야.''

엘사가 말했다. 안나는 심장이 멎는 듯했다. 이번에는 자신의 심장이었다. 강렬한 말투 때문이었다.

``누가 뭐라 해도 안 그럴 거야.''

``그걸 즐긴다고 하셨잖아요.''

안나가 말했다.

``왜 저한텐 말이 다르죠?''

``그런 거 아냐. 난 그런 말\ldots''

엘사는 침을 삼켰다.

``넌 너잖아.''

``그게 다예요?''

``무슨 말을 하라는 건데?''

안나는 고개를 저었다.

``딱히 뭔 말을 하라는 게 아녜요. 그냥 어떤 느낌인지를 말해달라는 거죠. 아직도 그럴 수 있다는 걸 아니까요. 말로 할 수 없으면, 최소한 부정하지만은 말아 주세요.''

``왜—''

``빈말은 안 할게요. 마음에 있는 말만, 해야 하는 말만 할게요.''

안나는 눈을 감고 숨을 깊게 들이마셨다.

``사\ldots\,사랑해요.''

엘사는 대답이 없었다. 안나는 예상하고 있어서 태연히 미소를 지었다.

``아마도 바보 같은 걸지도 모르겠는데, 정말이에요. 감정을 잘못 이해한 거면 죄송해요. 전\ldots''

안나는 웃음을 터뜨렸다.

``아녜요.''

엘사에게 해줄 수 있는 것은 다 해주었다. 이제는 엘사가 결정을 내릴 때이다. 안나는 성 안으로 걸어가기 시작했다.

엘사가 팔을 잡아 뒤로 끌기 전까지는.

``왜 날 압박해야만 한 거야?''

엘사가 물었다. 처음으로 목소리가 떨렸다.

``나\ldots\,난 힘들게 부정해왔단 말이야. 널 정말 떠나보내고 싶었어. 너한테 상처를 주기도 싫었다고. 처음 본 날 이후로 내 마음은\ldots\,어디에 밀어놓거나 무시할 수가 없었어. 너 때문에 난\ldots''

엘사는 팔을 놓아주었다.

전에는 최대한 시선을 피하려 했다. 이제는 그럴 수 없었다. 안나는 엘사를 돌아보았다. 빨개진 눈에 눈물이 고여 흘러내리고 있었지만, 아무런 소리도 내지 않고 있었다. 닦아내지도 않고 있었다. 지금까지 쌓아온 방어가 사라졌다. 눈빛은 여리게 되었지만, 온화하고 부드러워졌다. 안나는 조용히 눈물을 닦아주었다. 엄지가 부드럽게 피부 위에서 움직였다.

``너한테 기대고 싶어. 절대로 누구에게도 의지하지 말라고 배웠지. 남에게 의존하면 약해진다고도 했고. 하지만\ldots''

엘사는 눈물을 흘리는 채로 미소를 지었다.

``하지만 널 보니까 내가 다른 사람이 된 줄 알았어. 뭘 하든 더는 그냥 아닌 체할 수가 없었어. 내가 누군지도 모르고, 또—''

안나는 끌어안아 주었다. 엘사는 안나에게 매달려 어깨에 머리를 올려놓았다. 마침내 조용히 훌쩍이기 시작했다. 안나는 온몸이 떨리는 것을 느꼈다. 엘사를 더 세게 끌어안아 주었다.

``괜찮아요. 더 말 안 해도 돼요.''

안나는 조용히 말했다.

``무슨 일이 있어도\ldots''

`여왕님은 제 여왕님이에요.'

