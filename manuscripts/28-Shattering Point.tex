

\chapter[28장  파괴점][28장\hspace*{.5em}파괴점]{28장 \ 파괴점}



이러고 있을 시간이 없었다.

``그렇잖아도 성가신 일들이 많다.''

엘사가 말했다. 머리가 점점 지끈거려지고 있었다.

``내가 관심도 없는 일을 계속 알려주지 마라. 무엇을 원하는지는 이미 말한 것으로 알고 있다. 다음부터는 그리 하도록 하라.''

사울은 고개를 숙인 채 말했다.

``죄송합니다. 일이 되어 가는 형편을 알기를 원하시리라고 생각했습니다.''

``한스에 관해서?''

엘사는 얼굴을 찡그렸다.

``죽지 않고 몸이 성하게 하라는 게 그리도 어렵던가?''

``살아있게 하는 건 쉽습니다만, 약을 거절하는 통에 몸이 성하게 하는 게 쉽지 않습니다. 억지로 먹이려고 생각도 해 보았지만, 그런 권한을 받는 게 더 나으리라고 생각했습니다.''

이 완우한 것. 엘사에게 대항하려는 음모를 꾀할 때나 지금이나 한스는 똑같이 성가신 존재다. 다만 이제는 적이 아닌 골칫거리니 지금이 더 성가실지도 모르겠다. 최소한 전에는 자신만을 지키면 되었다. 한스를 돌볼 일도 없었다. 억지로 먹이는 것에 동의하고 그냥 내버려둘 수도 있었지만, 안나가 이를 알아내면 뭐라고 할지를 알기에 다시 생각할 수밖에 없었다. 무슨 이유인지는 모르겠지만, 안나는 이 열세 번째 왕자를 좋아하고 있었다. 한스의 어떤 점을 좋아하는지는 전혀 이해하지 못하겠지만, 이런 사실 하나만으로 한스에게 관용을 약간 베풀어줄 수는 있으리라.

``내가 가 보겠다.''

엘사가 말했다. 말을 마치자마자 머리 오른쪽을 꿰뚫는 듯한 고통이 찾아왔다. 엘사는 자신도 모르게 손으로 눈을 가리고 이를 악문 채 잇소리를 냈다.

``그러기엔 몸이 안 좋으십니다.''

사울이 말했다. 앞으로 가 옆에 서서 진정하게 해 주려는 듯 손을 뻗었다. 물론 정말로 손을 대기 전에 멈추었다. 그 누구도 손을 댈 수 없다. 그 누구도, 안나를 제외하고는. 사울은 물러났다.

``약 드시는 거 잊지 않고 계시는지요? 다 떨어졌다면 더 처방해드릴 수도—''

``아니, 더 강한 것이 필요하다.''

의지력을 쥐어짜 내어 엘사는 얼굴에서 손을 떼고 흐트러뜨린 머리카락을 뒤로 넘겼다. 어떤 고통이었든 이제는 약하게 욱신거리고 있었고, 고통스럽기는 하지만 참을 수 있는 정도가 되었다.

``지금 드시는 것도 보통 처방하는 것보다 몇 배는 강한 겁니다. 다른 사람이었으면 이미 죽었을 겁니다.''

``그런 건 상관없다. 무엇을 주든 이 몸이 내성이 생기면서 효과가 약해질 것이다.''

엘사가 말했다. 마법이 계속 주는 재생 효과 덕분에 죽음의 낭떠러지로 기울어져도 계속 살 수 있었다. 심장은 짓이겨진 채로 말이다. 좋은 점은 독에 면역이 생기는 것이다. 하지만 약에도 똑같이 면역이 생기는 흠이 있었다. 엘사는 손을 저어 대꾸하는 것을 막은 다음 말했다.

``가 보아라. 나는 한스에게 가 보겠다.''

이 말이 끝나고 사울은 밖으로 나갔고, 몇 분 동안 쉰 후에 엘사도 집무실을 나왔다. 엘사는 왕자들이 사는 곁채 가까이에서 지낸 적이 전혀 없었지만, 성의 다른 부분처럼 길을 잘 알고 있었다. 한스의 방에 도착했을 때 방문은 닫혀있었다. 사람을 미리 보내 자신이 올 것을 알려 준비할 수 있도록 하는 것이 옳은 절차겠지만, 엘사는 시간에 쫓기고 있었고, 한스 본인도 별 신경을 쓰지 않을 것이다, 지금 이런 상태에서는. 그래도 예절은 중요하다. 엘사는 문을 두드리고 기다렸다.

``가거라.''

한스가 말했다.

엘사는 손을 휘저었다. 압축된 찬 공기가 문을 밀쳐 열었다. 경첩이 완전히 뜯어져 쓸모없는 나무판자는 벽으로 날아갔다. 한스는 침대에서 뒤척이며 바로 앉았지만, 바로 다리를 감싸 쥐었다. 갑작스러운 고통이 우거지상으로 잘 드러났다. 엘사는 방 안으로 걸어 들어가 주변을 흘끗 보았다. 이번이 처음 와 보는 것이다. 벽에는 아무것도 없었다. 완전히 새하얬다. 한스는 방을 전혀 꾸미지 않았다. 엘사는 자신의 방이 떠올랐다.

다만 방은 난장판이었다. 그릇과 병 조각이 바닥에 널려있었다. 거기에다 원래 들어있던 내용물에서 나는 고약한 약 냄새가 방 안에 스며들고 있었다. 양탄자와 멀리 있는 벽까지도 더럽히고 있었다. 엘사는 한스가 내던진 것으로 생각하고 있었다. 솔직히 이럴 힘이 있다는 것에 놀라고 있었다.

``인사할 필요 없다.''

엘사가 말했다. 한스는 이를 악물었다. 둘 다 그럴 수 없는 것을 알고 있었다.

``약을 거절한다는 말을 들었다.''

``여왕님께서 상관하실 게 아닙니다.''

한스가 말했다.

``그럴지도. 하지만 상관할 것이 아닌 일에 자네가 간섭한 적이 없는 건 아니지.''

엘사는 깨어진 조각을 피하며 방 안을 돌아다니다 한스의 침대 옆으로 갔다. 손을 젓자 얼음으로 된 의자가 나타났다. 의자에 앉고 엘사는 말했다.

``그렇다면 내 비위에 맞추어보아라.''

한스는 입술을 꽉 다물었다. 하지만 항상 그랬듯 엘사의 마법을 불편해한 것은 아니었다. 엘사는 그저 자신이 사라지기를 바라고 있다고 생각했다. 재미있는 일이다. 앞에 있는 것만으로 한스는 매우 거슬렸다. 이제는 처지가 뒤바뀌었으니, 원치 않는 손님이 되어 주인을 불쾌하게 하는 것이 은근히 즐거운 맛이 났다. 엘사는 왜 한스가 이를 매우 즐겼는지 알 수 있었다.

``직접 하시는 게 나을 겁니다. 운이 좋다면 제가 더 오래 이러고 있지는 않을 겁니다.''

``그리도 삶을 버리고 싶느냐?''

``그렇습니다.''

한스는 잠깐 엘사를 바라보았다. 하지만 눈이 마주치자 바로 시선을 돌렸다. 고개를 숙인 채 한스는 중얼거리듯이 말했다.

``저를 죽게 놔두는 걸 좋아하실 줄로 알았습니다.''

``그냥 좋은 정도가 아니었겠지. 하지만 마르쿠스의 자식이니 죽게 내버려 두지 않을 것이다.''

``그러니까 제 아버지 때문에 저를 살려두겠다는 겁니까?''

한스는 코웃음을 쳤다.

``그렇다면 죽게 놔둬도 됩니다. 신경도 안 쓰실 거라고요. 제가 죽는 걸 신경 쓸 사람은 없을 겁니다. 아무도\ldots\,안 기억할 거라고요.''

``자네의 넋두리까지 들을 시간은 없다.''

엘사가 말했다. 한스의 턱이 실룩거리고 있었다.

``해야만 한다면 억지로 먹이는 것도 마다하지 않을 것이다. 내 확실한 허락 없이 죽을 수 있다고는 생각하지 마라.''

``절 막는 게 당키나 한다고 생각하십니까?''

한스가 소리쳤다.

``제가 버리는 삶이 어떤 건지나 아십니까? 애초부터 비참했던 데다가 이제는 완전히 잊혀버릴 것을 제외하고 말입니다.''

엘사는 얼굴을 찡그렸다.

``여기서 비참한 건 자네일 뿐이다.''

``여왕님은 전혀 당치도 않습니다. 모든 걸 가지고 계시지만, 그걸로 괴로워 본 적은 없으시다고요!''

한스가 말했다.

``마법을 가지고 태어나시고, 여기로 데려와 져서 여왕으로 길러졌죠. 하지만 저는 어떻습니까? 저는 형님들한테 쓰레기 취급이나 당하고, 아버지께서는 일회용으로 취급하셨습니다. 어머니께서 저를 위해 목숨을 내어주셨어도 전 완전히 밀려났고, 제게 와야 할 건 여왕님께 갔다고요! 어머니께서는 삶을 내주셨다고요, 대가도 없이 말입니다! 여왕님께서는 자격도—''

엘사는 웃음을 터뜨렸다.

``정말 한심하구나.''

엘사가 말했다. 한스가 고개를 홱 돌려 노려만 보고 있는 동안에도 계속 웃고 있었다. 한스와 눈이 마주치자 웃음이 잦아들었다. 불쾌함에 입꼬리가 아래로 처지고 있었다.

``자네를 좋아하지는 않지만, 얕잡아 본 적도 없다. 최소한 자네의 교활함은 높이 사 왔다. 하지만 지금은 눈물 나는 이야기로 동정을 구걸하는 한심한 인간으로밖에 보이지 않는다. 이건 정말 애처로운 짓이다, 한스.''

한스는 주먹을 꽉 쥔 채 숨을 몰아쉬었다. 잠깐 눈을 질끈 감고는 다시 팍 떴다. 눈매에 서린 분노를 보았지만 엘사는 전혀 주의하지 않았다. 자신도 모르게 비웃고 있는 것을 알아챘다. 본인이 불평할 주제가 된다고 생각한다면 엘사는 기꺼이 자신의 삶을 살게 해줄 참이었다.

``애처로운 짓이라고요?''

엘사는 한스가 버둥대는 것을 보았다. 걸을 수 없다는 것을 잊은 채 자리에서 일어서서 손찌검하려 한 듯했지만, 앉은 자리에서 노려볼 뿐이었다.

``여왕님 생각을 제가 신경이나 쓰는 줄 아십니까? 그것도 평민 피가 흐르는 여왕님을—''

``이런 말이 나올 줄 알았지.''

엘사는 한스의 말을 자르며 말했다.

``그렇게 불평을 늘어놓으면서도 이 서던 제도의 왕자라는 것을 크게 즐기고 있다, 그것이 주는 힘을 놓을 생각도 없는 것처럼. 정말로 원하는 것이 대관절 무엇이냐? 슬픈 삶 이야기를 해서 무엇을 얻기를 바라느냐? 이런 것들로 모든 참견, 음모, 자네의 형제를 죽음으로 몰아넣은 것을 정당화할 수 있다고 생각하면\ldots''

엘사는 세게 멱살을 잡아 한스를 가까이 끌어당겼다.

``자네는 완전히 틀렸다.''

엘사는 한스를 세게 밀쳤다. 한스는 뒤로 넘어졌다. 약한 분노는 흩어지고 공포와 혼란이 자리했다. 갑자기 엘사는 구스타프가 한 말이 이해가 갔다. 다만 자신이 아닌 한스에 관한 것이었다. 파괴된 다리가 지금의 한스를 만든 것도, 망가진 몸이 한스를 가두고 있는 것도 아니었다.

바로 부정이었다.

``계속해 보아라. 계속 네 어머니의 복수를 한다고 해 보아라.''

엘사가 말했다. 한스는 움찔했다.

``그분을 만나 본 적은 없다. 지나가는 말로나 들어봤을 뿐이지. 그래도 자네가 그분을 걸고 하는 행동을 허락할 리가 없을 것이다. 자네가, 그것도 그분의 아들인 자네가 그렇지 않다고 생각할 수가 있겠느냐? 권력을 향한 욕망이 아닌 다른 이유로 왕좌를 원했다고 생각하느냐? 항상 자네 자신을 위한 것이었지.''

``아니, 제 어머니를 위한 것이었습니다. 모든 게 다 어머니를 위한—''

``기억도 못 하는구나. 그분도 평민 혈통이었다.''

한스는 입을 벌렸지만, 아무런 말도 하지 않았다. 잠시 후 다시 입이 닫혔다. 침을 삼키며 목이 불룩거리는 것이 보였다. 한스는 시선을 돌린 채 낮은 목소리로 말했다.

``가 주십시오. 그냥 가 주세요. 전 이미 이렇게 돼 버렸다고요. 죽어버리는 게 살아있는 것보다 나을 겁니다.''

``아직도 죽기를 바란다면 말리지 않겠다. 어려운 일은 아니지. 하지만 지금 죽어버리면 정말로 의미 없는 삶을 산 셈이 되겠지.''

한스는 말없이 침대 위에 드러누웠다.

눈길 하나 주지 않고 엘사는 한스의 방을 나왔다. 망가진 문을 보고는 보이지 않게 한숨을 쉬었다. 문을 고칠 누군가를 찾아야 할 것이다. 어쨌든 사울에게 잘못이 있다. 처방해 준 약이 효과는 있지만, 신경을 곤두서게 하고 앞뒤를 가리지 않게 하니 말이다. 사울이 문을 고치게 하는 것이 적당한 벌이리라. 최소한 건축 재능이 버려지지는 않을 것이다.

엘사는 문밖으로 나갔다.

안나가 옆에 서 있을 줄은 전혀 생각지 못했다.

안나는 두 손으로 문간을 쥐고 있었다. 눈이 마주쳤다. 엘사는 눈을 끔뻑였다. 안나는 엷은 미소를 띤 채 어깨를 으쓱였다. 아무 말 없이 엘사는 안나를 따라 복도를 나왔다. 이들은 계속 말없이 방으로 갔지만, 엘사는 안나를 주의 깊게 보고 있었다. 무언가가 달랐다. 항상 용감하기는 했지만, 자신감이 있지는 않았다. 지금과는 달랐다. 안나는 자신만만함으로 불안함을 잘 숨기기는 했지마는 엘사는 안나가 본인을 너무 낮잡아보는 것을 알았다. 이토록 확고히 자신감에 차 있던 적이 없었다. 어깨도 쫙 펴져 있었다, 전장에라도 들어서는 것처럼.

방에 도착하자 안나는 문을 닫았다.

\textbreak

너무나도 조용했다, 엘사에게도.

특히나 조용한 것은 아니었지만, 요즘 둘이 있을 때와는 비교도 되지 않았다. 엘사는 무슨 말을 해야 할지 갈피를 잡지 못하고 있었고, 안나는 대화를 끌어내려는 것을 기어이 포기한 듯했다. 하지만 약간은 서로를 이해하고 있었다. 둘 다 마음속에 많은 것들을 담아두고 있었다. 말로는 전할 수 없을 정도로 많은 것들을 말이다. 이제 침묵은 숨이 막힐 정도였다. 안나가 말을 하고는 싶어 하지만, 말을 하지 않으리라는 것을, 할 수 없다는 것을 엘사는 알고 있었다. 그리고 이것이야말로 진짜 차이점이었다.

마침내 엘사가 입을 떼었다.

``괜찮은 거니?''

``네.''

안나가 말했다. 하지만 긴장은 어깨를 떠날 줄을 모르고 있었다.

``네. 괜찮아요. 그냥\ldots\,말을 좀 나누고 싶었어요. 그러니까, 안 바쁘시면요. 괜찮죠?''

``물론이지.''

엘사는 다시 말이 없어진 채 안나의 말을 기다렸지만, 자기 생각과는 달리 이 빨간 머리는 아무런 말도 하지 않았다. 지금까지 알던 안나와는 달랐다, 지난 몇 날은 제하고 말이다. 전혀 짜증이 나지 않게 하면서도 계속 말하고 또 말할 수 있던 것이 안나였다. 이러면서도 전혀 지치지 않는다는 것을 인정하지 않을 수가 없었다. 침묵은 안나답지 않았다, 특히 본인의 말마따나 말을 하고 싶어 할 때는—

``또 계속 무시하실 거예요?''

안나는 날카로운 질문을 던졌다.

``\ldots뭐?''

엘사는 고개를 저었다. 몹시 당황해 더 다듬은 말을 할 수도, 몹시 놀라 품위 없는 대답을 고칠 수도 없었다.

``무슨 말인지 모르겠는데.''

``아뇨, 그 말이 아니라\ldots\,그게 그러니까\ldots\,됐어요.''

안나는 말을 끊고 푹 한숨을 쉬었다.

``처음부터 다시 말할게요. 머릿속에서 준비까지 했는데도 왜 계속 이상하게 하는질—''

``준비까지 할 건 없어. 아무거나 말해도 돼.''

``\ldots알았어요.''

엘사의 말에 용기가 솟았는지, 혹은 자신의 말도 엘사의 말만큼이나 힘이 없는 것을 알아서인지 모르겠다. 안나는 침대로 가서 가장자리에 앉은 다음 숨을 깊게 들이마시고 입술을 깨문 다음 옆자리를 톡톡 쳤다. 당황한 채 엘사는 자리에 가 앉았다. 안나는 여전히 엘사를 똑바로 바라보지 못하고 있었지만, 엘사가 앉자마자 안도의 한숨을 내쉬었다.

``그냥 요즘 많이 말을 안 한 것 같아요.''

안나가 말했다. 반대편 벽을 계속 응시하면서 담요를 만지작거리고 있었다. 그래도 최소한 더듬거리지는 않고 있었다. 나지막하면서도 매우 분명한 목소리로, 안나는 덧붙였다.

``그리고 여왕님이 그리웠어요.''

``더는 나랑 있기 싫어하는 줄로 알았어.''

안나는 고개를 홱 돌렸다. 아니라고 말하기도 훨씬 전에 이미 자신이 잘못 생각하고 있던 것을 알고 있다는 것에 완전히 당황해 눈을 동그랗게 떴다. 이런 모습에도 엘사는 전혀 마음이 풀리지 않았다. 외려 있는 줄도 몰랐던 분노의 도화선에 불이 붙었다.

``그런 게 아니었어요.''

안나가 말했다.

``그러면 대체 뭔데?''

엘사가 물었다. 분노가 치미는 것을 참고 있었지만, 마음속의 울화는 계속 담아두기에는 너무 컸다. 제어력이 부족해서는 아니었다. 영원히 이런 식으로 담아둘 수 있었다. 삶의 절반 이상을 이렇게 보내왔다. 감정을 담아 두고, 그 어느 것도 느끼지 않으며 살아왔다. 하지만 좋은 일인지 나쁜 일인지 안나만이 자신을 화나게 할 수 있었다.

``말도 없고, 계속 피하고. 이유나 좀 알고 싶다고, 안나야. 내가 생각하던 게 아니면 대체 뭐 때문인데?''

``바로 여왕님이 듣고 싶지 않아 했으니까 말없이 있었다고요. 여왕님이 절 보기 싫어했으니까 피해 왔고요.''

엘사는 말을 하려 입을 열었지만, 안나는 무시하고 계속 말했다.

``제가 계속 시도하고 있던 거 한 번이라도 눈치는 채셨어요? 단 한 번도 절 모르는 체한 적이 없다고 할 수 있으세요?''

오랜 시간이 지났다, 너무나 지났다. 하지만 엘사는 둘 다 자신만의 시간이 필요하다고 굳게 생각하고 있었다. 둘 사이의 의견 차이가 불 보듯 뻔하게 드러나니 거리를 두는 것이 신중한 행동인 것처럼 보였다. 그리고 안나가 평소에 하던 대로 집무실에 있는 자신을 보러 오지를 않자 안나도 같은 생각을 하고 있다고 생각했다. 서던 제도로 돌아온 지 하루가 지난 날이었다. 엘사는 종일 집무실 안에서 기다리고 있었다. 분주히 자기 자신을 업무 속에 파묻으려 했지만 집중할 수가 없었다. 초조하게 문이 활짝 열리기를 기다리고 있었지만, 아무런 낌새도 없이 하루가 지나갔다.

엘사는 이미 전보다도 훨씬 더 안나를 무시하고 있는 것이 아닌가 하고 걱정하고 있었다.

확실히 이를 피하려 했다가 반대로 되어버린 것은 분명했다.

``돌아오셨을 때부터 절 무시하셨죠.''

안나는 말을 준비했다고 했다. 안나의 눈에 분노의 눈물이 고이자 엘사는 안나가 준비했던 말 대신 다른 말을 하는 것을 알 수 있었다. 안나는 눈을 깜빡여 눈물을 들어가게 하려 했지만, 뜻대로 되지 않았다. 목소리가 떨리고 갈라졌다.

``계속 시도하고—''

``난 안 그랬던 줄 알아? 항상 기다리고 있었는데 넌 다른 사람에게만 갔다고.''

엘사는 자리에서 일어서 침대에서, 안나에게서 떨어졌다. 다시 돌아오려 하지 않았다.

``에드문드, 알렉—둘은 이해해, 괜찮아. 그런데 한스는? 한스는 내겐 문젯거리일 뿐이었다고. 그리고 듣기론 널 거의 죽을 뻔하게—''

``절 살려주셨다고요!''

``애초에 거기 있지를 말았어야지!''

엘사는 홱 돌아서서 안나에게 성큼성큼 걸어갔다. 자신의 마법이 튀어나오기 일보 직전인 것을 알고 있었지만, 이 빨간 머리는 전혀 기죽지 않았다. 이런 상황에서도 전혀 두려움이 없었다.

``그러면 위험해질 질도 없고, 안전하게 잘 지낼 수 있었을 거—''

``사울 왕자님 말대로 절 가둬서요?''

``그렇게라도 해야 하면!''

``왜 필요할 땐 제 편이 안 돼주는 건데요!''

엘사는 그대로 굳었다. 안나도 힘이 부치는 듯했다. 마지막 말이 천둥과 같이 울렸지만, 눈물도 같이 소리 없이 떨어졌다. 안나에게는 자신이 필요했다. 이를 알고 있었고, 더는 부정할 수 없었다. 다른 사람의 탓을 할 수도 없았다. 그리고 약속을 했지마는 안나를 지키지 못했다. 돌아오고 나서도 안나를 지키는 일은 조금도 하지 않았다. 안나가 평소대로이기를 바라고 있었다.

``미안해.''

엘사는 가까이 다가가 머뭇거리며 손을 내밀었다. 안나는 막지 않았다. 엘사는 천천히 눈물을 닦아주었다. 안나 옆에 다시 앉고 엘사는 말했다.

``냉담하게 굴어서 미안해.''

``여왕님 잘못이 아녜요. 둘 다 그냥\ldots\,서로가 먼저 뭘 하기를 기다린 거죠.''

안나는 헛기침을 했다. 엘사는 고개를 떨구었다. 안나가 다리를 흔들고 있던 것을 알아챘다. 부츠의 코가 살짝 들려있었다.

``너까지 그렇게 되지는 않았을 거야.''

``그런 게 아녜요. 우리 둘 다한테 그런 게 있는 거죠.''

``겁이 나서 그랬어.''

엘사가 말했다. 안나가 이상할 눈으로 보지 않을 최선의 대답이었을 것이다. 안나도 알 필요가 있었다.

``항상 겁이 나, 한 가지 것에만. 너를 절대로 잃고 싶지 않아. 이걸 다 무시하면, 말을 꺼내지 않으면 괜찮을 줄 알았어.''

``그\ldots\,그럴 줄 몰랐어요.''

엘사는 흠 하는 소리를 냈다. 다시 침묵이 깔렸다. 이제 안나는 손가락으로 침대보를 톡톡 두드리고 있었다. 엘사는 손을 무릎 사이에 넣은 채 앉아있었지만, 눈은 위아래로 규칙적으로 움직이는 안나의 손을 향해 있었다. 안나의 손을 마지막으로 잡아본 지가 며칠 전인 것이, 한 주가 넘은 것이 떠올랐다. 엘사는 몸을 돌렸다. 자신의 모습이 보였다. 자신도 모르게 손이 오므라들었다. 엘사는 시선을 피한 채 눈을 질끈 감았다. 손을 잡고 싶었다. 그 무엇보다도 이를 원했다. 하지만 지난 한 주 동안 둘 사이가—

따뜻한 것이 손마디를 스쳤다. 엘사는 반사적으로 손을 폈다.

안나는 움찔하며 물러났다.

``죄송해요! 먼저 물어봤어야 했는데. 난 정말 바보야—''

엘사는 손을 뻗어 안나의 손을 잡았다.

``당연히 그렇지.''

엘사가 말했다. 손마디로 안나의 이마를 가볍게 치고 있었다. 이 빨간 머리가 당황한 채 삐죽거리며 이마를 문지르자 엘사는 미소를 지었다.

``싫어하지 않는다는 거 알잖아.''

그러면서도 엘사는 누가 더 큰 바보인지 알고 있었다. 이 온기가 매우 그리웠다.

안나의 당황한 표정은 천천히 미소로 바뀌었다. 안나는 남는 손으로 엘사의 허리를 감싼 채 어깨에 자신의 머리를 대었다. 기쁨에 작은 미소를 지은 채 얼굴을 파묻고 있었다. 엘사는 안나의 머리카락을 쓸어내리고 조심스럽게 정수리에 입을 맞추었다. 하지만 목을 무는 느낌에 거의 놀랄 뻔했다. 엘사는 안나를 떼어내고 쏘아보았지만, 장난스러운 미소와 반달 모양이 된 눈이 보일 뿐이었다.

``절 탓하면 안 되죠.''

``그렇다고 봐.''

``정말로 그리웠어요.''

``나도 그랬어.''

둘 사이의 모든 장벽이 녹아 없어졌다. 칠흑과 같은 바닷속에서 수면으로 올라온 것과 같았다. 다시 한시름 놓을 수가 있었다. 머릿속을 뒤덮던 모든 좌절감, 자신을 괴롭히던 모든 의심—이 모든 것이 갑자기 연기와 같이 아무것도 아닌 것처럼 느껴졌다. 바로 앞에서 자신의 향기를 맡으며 커다란 미소를 짓고 있는 안나에 비하면 이 모든 것은 없는 것이나 마찬가지였다. 이 순간에 달리 생각해 볼 거리가 있을 수나 있을까.

``지난주 동안 연습용 말뚝 백여 개를 부순 거 아니?''

엘사는 나지막한 목소리로 말했다. 안나의 빨간 머리에 혼자 옥에 티처럼 있는 흰 머리칼을 만지작거리고 있었다.

``와. 정말 쌓인 게 많았나 보네요.''

``마음을 돌릴 거리가 필요했어. 이걸로 사상자를 만들면 네가 싫어할 테니까 이 말뚝이 대신 받아주게 했지.''

``가엾어라.''

``흔적도 안 남을 정도였지.''

엘사가 말했다. 안나는 코웃음을 쳤다.

``정말이야. 이젠 전기가 더 쉽게 나와. 그리고 어쩔 땐 제어도 못 할 정도로 벼락이 정말 세차게 나오고. 두통이 심해질수록 더 큰 힘을 낼 수 있는 것 같아.''

``두통이 더 심해진다고요?''

안나는 뒤로 물러난 다음 엘사를 자세히 바라보았다.

``괜찮은 거예요? 죄송해요. 물어보기라도 했어야—''

``위즐턴에서 돌아온 뒤로부터\ldots''

엘사는 건성으로 안나가 가까이 오지 못하게 했지만, 결국은 가까이 오게 했다. 자신의 관자놀이를 문지르는 안나의 손가락에서부터 온몸으로 온기가 퍼지는 것이 느껴졌다. 단순한 안나의 손길이 그 어떤 약보다도 효과가 있었다.

``걱정은 안 해도 돼. 그래도 고마워.''

``너무 무리하시는 거 같아요. 그러면 집무실에서 정확히 뭘 하시던 거예요?''

``확산 차단. 외곽 지역하고 창고에서 일어난 일을 처리하는 것 말이야.''

엘사가 말했다. 안나가 움직임을 멈추자 생각 없이 그대로 말한 자신을 나무라고 있었다.

``주변이 점점 동요되고 있었어.''

``아.''

안나는 눈을 끔뻑이며 물러났다. 엘사가 고개를 기울인 채 바라보는 동안 자신의 머리를 쓸어넘기고 있었다. 이상하다. 안나가 이러는 유일한 때는\ldots

``좀 불편한 일이 있는 것 같은데.''

``아무것도 아녜요. 그냥 드레스를 보고 있었어요.''

안나는 과장스럽게 익숙지 않은 옷을 바라보는 모습을 보였다. 하지만 얇은 천이 시선을 끈 것이 아니었다.

``이거 무슨 색이라고 할 거예요?''

``\ldots파랑?''

엘사는 자신의 드레스를 바라보고는 살짝 얼굴을 찡그렸다. 어깨를 으쓱이는 엘사 자신만의 방법이다.

``아뇨. 누가 봐도 연파란색이에요.''

안나가 말했다. 다시 엘사를 바라보고는 얼굴을 일그러뜨린 채 말했다.

``근데 진짜로, `파랑'이라고밖에 못하겠어요?''

``내가 만든 거라니까.''

``그러고도 무슨 색인지도 모르고요.''

안나는 고개를 저으며 말했다. 엘사는 거의 눈알을 굴릴 뻔했다.

``알았어요. 하나만 더 물어볼게요. 제 머리 색은—''

``빨강.''

엘사가 말했다. 안나가 더 동떨어져 있고 애매한 색으로 대꾸하기 전에 덧붙였다.

``다른 할 말이 있는 것 같은데.''

안나는 바로 무표정으로 돌아갔다. 엘사는 이 짧은 대화를 끊어버린 것을 후회했지만, 더 이성적인 한편은, 생각 대부분을 차지하는 이 한편은 피할 수 없는 일이 있다고 말해주고 있었다. 가끔은 마음의 일은 냉담한 논리와 조심스러운 셈으로는 따질 수 없다고 나무라는, 안나와 매우 비슷한 소리가 들려왔다. 엘사는 이를 따를 수 있기를 바랐다. 하지만 도로 머리로만 생각하는 것으로 돌아가곤 했다. 오랫동안 이 이성은 몸에 배어 거의 천성이나 다름없었다. 안나를 만나기 전부터 엘사는 이와 함께 해왔다. 손쉽고 익숙하고, 가끔은 안나보다도 더 익숙할 때도 있었다.

단지 또 다른 가면일 뿐이었다. 엘사는 기꺼이 이 가면을 써 왔다.

\textbreak

말을 꺼내기를 바랐을까?

자신의 습관이 다시 돌아오는 것도 모른 채, 안나는 입술을 깨물며 머리카락을 귀 뒤로 쓸어넘겼다. 엘사는 생각할 시간을 줄 정도의 배려는 해주었지만, 엘사의 표정없는 얼굴과 공허한 눈빛에 안나는 더욱더 긴장되었다. 갑자기 눈앞의 엘사가 바로 같은 자리에서 자신의 머리카락을 만지작거리던 엘사와 다른 사람인 것처럼 보였다. 이런 갑작스러운 태도 변화를 눈앞에서 본 적이 없었다면 다른 사람이 정말 똑같이 엘사를 흉내를 내고 있다고 생각했을 것이다. 하지만 이는 아니다. 바로 이런 엘사가 안나가 처음 만난 엘사였으니 말이다.

그리고 이는 성물을 만든 결과다.

안나는 그저 어떻게 말을 끄집어내야 할지 모르고 있었다.

``외곽 지역에서 무슨 일이 일어나고 있는데요?''

안나가 물었다.

``불평가들.''

엘사가 말했다. 매우 주의 깊게 중립적인 말투에서 안나는 엘사가 불쾌해 하는 것을 잡아내었다.

``알다시피 이 마법사가 취향이 특히 예민한 건 아니었잖아. 소문이 퍼져서 평민들이 정의를 찾고 있어.''

``\ldots가족들이 다 절망에 빠졌을 거예요.''

안나는 잡혀간 사람들을 구하려 이 무모한 일에 뛰어들었다. 결국은 아무도 구하지 못했다. 구할 사람이 아무도 없었다, 그 난도질 된 시신과 버려진 시체들의 한복판에서는. 떠올리는 것만으로도 몸서리가 났다. 이 사람들이 얼마나 알고 있었을까. 그 차갑고 어두운 곳에서 무슨 일이 일어났는지를 알면 엄청난 소동이 빚어질 것이다. 그러면서도 이들은 알 권리가 있었다.

``이제 어떡할 거예요?''

안나가 물었다. 엘사는 아무 말도 하지 않았다. 왜 그러느냐는 눈빛으로 바라만 볼 뿐이었다. 안나가 덧붙였다.

``해명해야 하지 않겠어요?''

``무슨 말을 하라는 건데?''

``뭐, 진실이요.''

``아무리 진실을 말해줘도 만족하지는 못할걸. 내가 간과한 점이기는 해. 그래도 진실을 말하면 불평가만 더 늘 거야. 이해하지도 못할 거고.''

``가 보긴 했어요? 그 사람들 말예요, 납치당하고 고통에 시달리는 사람들이요. 가족들한테 무슨 일이 일어났는지 말해주고, 최소한 적당히 장례라도—''

``장례라고? 죽은 몸뚱이도 못 꺼내는데 장례는 어떻게 해? 폭발 때문에 지하실이 완전히 무너졌다고. 안에 있던 것은 다 파괴되었고. 시원하게 잘 정리된 거지.''

이 성의 없는 한 마디 한 마디에 안나는 점점 신경이 곤두섰다. 엘사의 말 때문은 아니었다. 끔찍한 말이기는 하지만, 이 수많은 희생자의 가족에게 자신들의 소중한 사람이 실험 대상이 되었다고 말하는 것이 참상을 그대로 전하는 것보다 더욱 나쁘다는 것을 안나는 알 수 있었다. 그리고 희생자들의 시신을 수습하는 것은 정말로 불가능할 것이다, 경솔한 일일 수도 있다, 이들이 당한 일을 생각해 본다면 말이다. 하지만 엘사는\ldots

엘사는 이들의 고통에는 전혀 신경을 쓰지 않는다는 듯한 말투였다. 외곽 지역에 사는 사람들에 대해, 이 `평민'들에 대해 멸시를 거의 숨기지도 않은 말투로 말할 뿐이었고, 희생자에 대해서도 희생자나, 심지어는 사람도 아닌 양 말할 뿐이었다. 게다가 `시신'이나 `시체'도 아니라 `몸뚱이'라는 말을 쓰고 있었다, 이들이 정말 인간도 아니라는 것처럼.

``그러니까 모든 걸 숨기기만 할 거란 말예요?''

안나가 물었다.

``네가 생각하는 그런 게 아니야.''

엘사는 한숨을 쉬며 말했다.

``이게 정말 최선의 대처야. 모두 시간이 지나면 잊힐 것이고, 소란 없이 계속 살아갈 수도 있을 거야.''

``사람이 죽었다고요. 아무 잘못도 없는 사람들이 말예요. 한 분은 바깥 탐험을 좋아하는 딸 얘기에, 다른 한 분은 자기 집을 지나가면서 노래를 부르던 아이 얘기를 했고—''

``무슨 말을 하려고?''

``진짜 사람들이 죽었단 거예요. 그냥 평민이나 시체가 아니라, 진짜 사람이요. 그냥 숨기는 건 말이 안 돼요. 여왕님이 신경 쓴다는 거 안다니까요. 안 그러면 생필품을 배급해줄 리도—''

``안나야, 난 신경 안 써.''

``\ldots뭐라고요?''

안나는 자신의 귀를 의심할 수밖에 없었다. 잘못 들은 것이 틀림없다. 군주의 임무가 민중들을 인도하고 보호하는 것이라고 수도 없이 들어 왔고, 억지로 듣게 된 모든 수업에서 들은 것도 항상 도리에 맞는 것들이었다. 그런데 엘사가 자신이 통치하는 사람들을, 자신이 보호해야 하는 사람들을 단지 `신경 안 쓴다'고 하는 것을 듣는 것은?

하지만 곰곰이 생각해 보니, 안나는 진작 알았어야 했다는 생각이 들었다. 엘사가 신경을 썼으면 이런 참담한 생활 여건을 그냥 두지 않았을 것이다. 바로 코앞에서 사람들이 실종되는 것을 놓치지도 못했을 것이다. 이들이 아무것도 아니라고 생각해 아무것도 하지 않을 리가 없었다.

``고통을 덜어줄 수도 없어. 한때는 무언가를 해줄 수 있는 줄 알았지. 이들의 삶을 더 낫게 해줄 수 있다고 생각했다고. 하지만 할 수 있는 게 아무것도 없는 것이 현실이야. 삶 자체가 지독한 장난이라니까. 그리고 이걸 알게 됐지, 이걸 깨달았다고, 사람들은 신경을 쓰니까 상처를 받는 거야. 난 신경을 안 쓰기로—''

``어떻게 그냥 신경을 안 써버릴 수가 있어요?''

안나는 나지막이 말했다.

``그게 쉬웠으니까.''

``그야 아무런 마음도 없이 무정해졌으니까요.''

너무 늦었다고 깨달은 순간에 안나는 자신이 무슨 말을 했는지를 깨달았다. 이미 말을 해 버린 다음에 말이다. 결코 입 밖으로 내려 한 것이 아니었다. 뜻풀이 이상의 의미를 의도한 것도 결코 아니었다. 하지만 이미 늦었다. 엘사는 크게 충격을 받은 채 말없이 안나를 바라보았다. 안나는 아무런 대답도 할 수 없었다. 그 어떤 말도 상황을 나아지게 할 수 없었다.

``죄\ldots\,죄송해요. 가 봐야겠어요.''

안나는 도망쳐 나왔다.

엘사는 전혀 안나를 쫓아가지 않았지만, 안나는 방에서 뛰쳐나와 멀리, 갈 곳 없이 멀리, 자신이 가한 이 회복할 수 없는 공격을 생각하지 않아도 되는 그 어떤 곳으로라도 도망쳤다. 전혀 말하려 한 것이 아니었다. 이런 식으로 말하려 한 것이—

\textbf{하지만 그렇게 말해 버렸죠}

`하지만 그렇게 말해 버렸어.'

안나는 항상 엘사가 마법사와 비슷하다고 생각해 왔다. 하지만 더욱 무서운 것이 있다면? 안나는 이 마법사가 본인의 심장이 산산이 없어지고 있다고 말한 것이, 이것이 감정과 생명을 서서히 앗아가고 있다는 것이 떠올랐다. 엘사도 똑같은 과정을 겪고 있었다. 이미 시작되었다. 안나는 그 누구보다도 잘 알고 있었다. 엘사는 매우 심하게 망가져 있었다. 조각 하나가 안나에게 올 정도로 심장이 산산이 부서져 있었다.

\textbf{점점 저처럼 될 겁니다}

`점점 그 사람처럼 될 거야.'

\textbf{그러고는 죽게 되죠}

엄청난 공포가 밀려와 안나는 제대로 서 있을 수도 없었다. 숨이 막히게 하는 것들이 모든 생각에 들러붙고 숨까지도 막는 듯했다. 안나는 벽에 기대어 털썩 주저앉았다. 온갖 생각을 하면서 숨을 들이쉬려 헉헉거리고 있었다. 무엇이든 해야 했지만 할 수 있는 것이 없었다. 어찌할 길이 없었다. 이미 너무 늦어 버렸다—

\textbf{무엇이든 해야 해}

무엇이든 해야 했다.

\textbf{다시 한 번 가 봐}

다시 한 번 가 보아야 했다.

\textbf{뭘 해야 하는지 알잖아, 안나야}

성물이 있는 곳으로 가서 모든 것을 바로잡아야 했다.

