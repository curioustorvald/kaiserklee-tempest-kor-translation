

\chapter[2장  몰아치는 찬 바람][2장\hspace*{.5em}몰아치는 찬 바람]{2장 \ 몰아치는 찬 바람}



안나는 자신의 침실을 오랫동안 훑어보았다. 안나는 이곳을 그리워할 것이다. 물론 성은 안나의 아버지가 성문을 닫을 때부터 약간 적적했다. 이야기를 나눌 상대는 갑옷 장식과 그림밖에 없었지마는 안나는 그리워할 것이다. 갑옷 장식과 그림은 친구와 다름없었다.

``꼭 가야 할 필요는 없다니까, 안나야.''

왕이 다시 말했다. 안나는 이것이 천 번째라고 생각하고 있었다. 뭐, 사실 안나는 열 몇 번이 지나자 몇 번인지 잊어버렸다. 그래도 너무 과장한 것은 아니었을 것이다. 왕이 안나의 옷가지를 짐가방에서 빼는 동안, 안나는 계속 방 안의 모든 물건을 짐가방에 넣었다.

``전 그래야 할 것 같아요.''

안나가 말했다. 안나는 드레스 두 벌을 집어 들고 앞뒤로 흔들어댔다. 하나를 자기 앞에 대고 한쪽 눈을 감고 바라보고 있었다. 흰색이 뚱뚱해 보였을까?

``어떤 게 더 나으려나\ldots''

``안나야, 엘사 여왕이 꼭 따라야 할 필요는 없다고 말했잖니.''

왕이 말했다. 그러나 아무리 안나만큼 순진한 사람이어도 이것이 뻔뻔한 거짓말인 것을 알 수 있었다. 마치 엘사 여왕이 말한 `정복된 왕국'이 무엇이든 거절할 수 있는 것처럼.

``싫다면 안 가도 된단다. 네가 꼭—''

``그, 전 그냥 둘 다 가져갈래요.''

안나가 중얼거렸다. 그러고는 드레스 두 벌을 가방에 넣었다. 자신의 아버지가 끙하는 소리를 내는 것을 듣자, 안나는 기운차게 미소를 지었다.

``괜찮을 거예요.''

왕은 머리를 흔들고 안나가 커다란 옷장을 향해 돌아서려는 찰나에 안나의 팔을 잡았다. 뒤를 돌아보자, 안나는 처음으로 자기 아버지의 입가와 눈가가 걱정으로 주름이 지는 것을 보았다.

``안나야, 제발 좀.''

안나는 아버지의 목소리가 떨리는 것을 들었다. 그러자 잠깐 의지가 흔들렸다. 안나의 어머니가 병으로 죽은 후로, 왕에게는 안나밖에 남지 않았다. 왕은 안나가 해를 입지 않게 성문을 닫아두기까지 했다. 그래도 안나는 이것이 자기 아버지의 짐을 덜어주는 유일한 방법이라는 것을 알고 있었다. 안나가 없다면 왕은 안나와 나라를 둘 다 신경을 쓰느라 주의가 흐트러질 일도 없다.  모든 일이 더 나아질 것이다. 그리고 엘사 여왕은 안나가 자신과 함께 오면 더 많은 관용을 베풀겠다고 말했다. 웬일인지 안나는 엘사 여왕이 약속을 지키리라고 생각했다.

``괜찮을 거예요.''

안나는 부드러운 목소리로 다시 한 번 말했다. 그러고는 아버지의 손에 자신의 손을 올려놓으면서 앞에서는 말할 수 없는 것들을 전달하려 했다.

`더는 절 신경 쓸 필요 없어요. 제가 다른 데 가 있을게요.'

왕은 천천히 고개를 끄덕였다.

``그렇다면 막지는 않으마.''

안나는 자신의 아버지가 천천히 숨을 내쉬는 것을 보고 미소를 지었다. 그러고는 계속 짐을 꾸렸다. 그러는 동안 안나는 혁명이나 승리나 이런저런 극적인 생각으로 즐거워하고 있었다.

``누가 알아요? 아마도 기회가 있을 수도 있겠죠. 그 뭐냐, 거기를 안에서부터 무너뜨리는 거 말예요—''

``네가 그런 이야기 좋아한단 건 알지만, 제발 잔 다르크처럼은 되지 마라.''

왕은 화가 난 듯한 목소리로 말했다.

``난 네가 무사하기만을 바랄 뿐이야. 그 밖의 것을 바라는 게 아니고.''

``그 그림 걸어놓지 말아야 했네요, 아빠.''

안나가 말했다. 보이지 않게 어깨를 으쓱이려 했다. 물론 그러지는 못했다.

``잔 다르크는 제 우상이라고요.''

``그건 알고 있지, 안나야.''

``알아요.''

안나는 웃으며 말했다. 안나의 아버지는 항상 안나의 생각을 알고 있었고, 안나도 이를 알고 있었기 때문이다. 그러나 잠깐의 전율은 곧 서서히 사라졌고, 안나의 목소리는 낮아졌다.

``아빤\ldots\,항상 잘 알고 계시죠.''

왕은 아무 말도 하지 않았다. 안나는 짐을 꾸리는 것을 멈추었다. 자신의 드레스를 손마디가 하얘질 정도로 꽉 쥐고 있는 것을 깨닫자 손이 멈추었다.

안나는 갑옷 장식을 그리워할 것이다. 벽에 걸린 그림을 그리워할 것이다. 그리고 안나는 생각하지 않으려 하고 있지만, 자신의 아버지를 그리워할 것을 알고 있었다. 안나의 무모함도 더는 이를 가릴 수 없었다. 안나는 손에 들고 있던 것을 내팽개치고 자신의 아버지에게 가 마지막으로 꼭 안아주었다.

``아빠가 그리울 거예요.''

안나가 말했다. 아버지의 어깨에 얼굴을 파묻고 있었다. 안나의 아버지는 항상 한결같았다.

``정말로, 정말로 그리울 거예요, 아빠.''

왕은 목이 메어 겨우 이 말을 할 수밖에 없었다.

``몸조심하렴, 안나야.''

안나는 이를 확신하지 못하고 있었다.

\textbreak

밤새워 뒤척이다 선잠이 들자마자 안나는 바로 깨어나게 되었다. 경비병은 안나를 아침 일찍 부두로 데려갔다. 안나가 어릴 때 `하늘이 깨어났다'고 하던 때였다. 달리 말하면, 안나에게는 너무 이른 시간이었다.

일출을 보는 것이 싫다는 이야기는 아니다. 주황빛과 쪽빛이 아렌델의 잔잔한 바다에 펼쳐졌고, 추상적인 형상은 멀리 매여있는 배의 모습으로 바뀌었다.

안나는 이 정경에 마음이 흔들릴 뿐이었다. 그리고 한동안은 계속 두려워하다 자신의 임무도 져 버릴 지경이었다. 아렌델은 안나에게 집과 다름없었다. 항상 그럴 것이다. 아렌델에서 평생 살 수 있다면 무엇이든지 할 수 있었다. 창문으로 들어오는 햇빛도 안나에게는 아름다웠다. 이곳에 있는 동안 안나는 눈을 크게 뜨고 햇빛을 받는 도시의 모습을 최대한 보아 두려 했다. 영원히 이곳을 떠나기 전에 마음 가득 새겨두려 했다.

그러나 그럴 수 없었다.

짐짝처럼 여기저기 떠밀린 다음에, 안나는 널빤지가 내려와 있는 범선 앞에 서 있는 가벼운 갑옷을 입은 남자에게 밀쳐졌다. 이 남자는 안나의 손에 들린 가방을 쳐다본 다음 험악한 표정으로 가방을 낚아챘다.

``이봐요! 돌려줘요!''

안나는 가방을 잡아당겼지만, 이 냄새 나고 수염이 덥수룩한 군인은 안나의 갸날픈 팔이 내는 힘보다 더 세게 잡아당겼다. 안나가 조심스럽게 꾸린—급하게 막 던져 넣은—짐은 순식간에 다시 녹은 피오르로 던져졌다. 안나는 이 남자를 돌아보았다. 그러나 항상 행사하려 하던 권력은 더는 남아있지 않았다.

``이게 휴가라고 생각하는 것이냐?''

이미 작고 둥근 눈이 더욱 가늘게 뜨였다. 안나는 머릿속으로 이 남자가 돼지 같다고 생각했다.

``그쪽네 엘사 여왕이 필요한 건 다 가지고 올 수 있다 했다고요.''

안나는 성을 내었다. 자신의 가방이 가라앉아 피오르가 자신의 물건을 삼키는 것을 지켜보고 있었다. 안나에게 마지막 남은 위안거리가 사라져버렸다. 안나는 깊게 숨을 들이마신 다음 계속 나아가기로 했다. 여기서 질 수는 없었다. 헤쳐나가야만 했다.

이 남자의 거친 웃음소리는 불행하고 거친 현실을 다시 일깨워주었다. 그래도 최소한 안나의 불안감은 없어졌다.

``엘사라고? 엘사는 섭정일 뿐이야. 진짜 여왕이 아니라고.''

안나는 볼에 침이 튀자 얼굴을 찌푸렸다. 이 남자가 다시 웃는 동안 안나는 침을 닦아내었다.

``뭐 차이가 있나요?''

안나가 물었다. 이 명백히 무례한 말을 믿을 수 없는 듯했다. 아렌델의 모든 이는 분명히 엘사 여왕을 두려워했다. 그런데 이 남자가 이런 식으로 말한다니?

``많지.''

이 남자가 말했다. 안나가 더 말하기 전에, 이 남자는 창의 뭉툭한 끝으로 안나를 밀쳤다. 안나는 물에 빠질 뻔했다.

``이제 대화는 끝이다. 올라타.''

안나는 어렴풋이 보이는 범선을 바라보았다. 그러고는 먼 거리에 있는 반짝이는 얼음 배를 바라보았다. 얼음 배는 떠오르는 해와 어우러져 더욱 아름다워 보였다. 이상하게도 전과는 달라 보였다. 더 작아져 있고 더 수수해 보였다. 그러나 안나는 지금 모습이 더 낫다고 생각했다. 덜 고압적이었고, 정말로 예술작품과 같았다. 그러나 안나는 곧 언짢은 기분이 들었다. 저 배를 만든 예술가가 바로\ldots

``저 배가 아니고요?''

그래도 안나는 묻지 않을 수 없었다.

이 남자는 안나를 노려보았다.

``오직 엘사만이 저 배를 탄다. 그리고 저걸 탈만큼 멍청한 놈이 있어도 다른 사람은 저 저주받은 배에 타지 못하게 한단 말이다. 이제 빨리 움직여.''

안나의 등에 또다시 강한 충격이 왔다. 안나는 신음을 내지 않으려 입술을 깨물었다. 정말 아팠다. 특히 지금까지 손에 물 한 방울 묻히지 않고 살아온 이에게는. 안나는 자신의 부드러운 피부에 멍이 드는 것을 느낄 수 있었다. 그래도 안나는 이 남자를 인상을 쓴 채 쏘아본 다음 배 안으로 갔다.

안나가 보기에 배 안은 꽤 분주했다. 안나는 남자들이 튀어나와 항해를 위한 보급품을 싣는 것을 보았다. 안나는 배가 어디서 왔는지, 이들이 무엇을 하는지 전혀 알지 못했다. 어쨌든 거의 엘사 여왕이 혼자 전쟁에서 이겨버렸으니 말이다. 안나는 일꾼들이 기중기를 조작해 이런저런 것들이 든 나무 상자를 배로 올리고, 다른 일꾼들은 상자를 이곳저곳으로 옮기는 이 아수라장을 구경하고 있었다. 마치 환경이 열악한 공장을 보는 것과 같았다.

안나가 보고 있는 동안, 멀리 있는 도르래에서 끽소리가 났다. 무언가가 지렛대에 걸렸다. 이 기중기를 조작하는 금발 머리 남자아이는 빠르게 대처하지 못했다. 높이 매달려있는 상자가 아래로 떨어졌다. 상자는 큰 소리를 내며 깨어졌다. 안나는 고함을 치며 앞으로 달려나갔다. 상자 안에는 재목과 의류와 식품과 같은 각종 생필품이 들어있었다. 안나는 도안을 알아볼 수 있었다. 모두 아렌델에서 만들어진 것이었다. 다가오는 겨울을 나는 데 필요한 것들이었다.

``저리 가.''

조금 전의 그 남자가 말했다. 그러고는 안나의 무릎 뒤를 또다시 강타했다. 안나는 앞으로 고꾸라져 갑판에 턱이 세게 부딪혔다. 그러나 안나는 분노 때문에 통증을 느낄 수 없었다.

안나는 격분 속에 쓰라림과 또다시 얻어맞지 않으려는 상식도 잊고 자리에서 일어나 달려들었다.

``이게 다 뭐죠?''

``뭐겠어? 아렌델이 바쳐야 할 공물이지.''

이 남자가 말했다. 안나는 다시 서둘러 주변을 둘러보았다. 항구에 정박해 있는 배는 모두 언저리에까지 꽉 차 있었다. 이것들이 모두 공물이었을까? 하지만 엘사 여왕의 말은\ldots

말을 못 할 정도로 놀라지는 않았지만, 안나는 속았다는 것으로 말을 꺼낼 수 없었다. 이 남자는 안나를 끌고 갔다. 안나의 위팔에 빨갛게 손가락 모양 자국이 생겼다. 안나를 가두어놓거나 짐칸에 몰아넣으려는 것 같았다. 그러나 이들은 조금 전에 이상이 생긴 기중기를 지나가고 있었다. 안나가 지나가자마자 또 다른 상자가 높은 곳에서 떨어졌고, 금발 머리 아이는 놀라 투덜거리는 소리를 냈다. 다행히도 이 상자는 깨어지지 않았다.

안나를 끌고 가던 남자는 이 금발 머리 아이를 돌아보았다. 거의 게거품을 물고 있었다.

``잘 좀 해봐, 녀석아! 언제까지 깨먹기만 할 거야?''

안나는 이 아이에게 미안한 생각이 들었다. 그러고는 이 아이가 서던 제도에서 온 것을 깨닫고 바로 경멸하는 눈빛으로 바라보았다.

그러고는 곁눈으로 계속 바라보았다.

``죄송해요.''

이 금발머리가 말했다. 사과의 말을 하면서 여전히 뻔뻔스럽게 미소를 지으며 엄지를 치켜들고 있었다. 그런 다음 나중에 생각났다는 투로 덧붙였다.

``사령관님.''

안나는 좋지 않은 기분에도 웃지 않을 수 없었다. 이런 기분에도 안나는 이 금발머리의 웃음이 옮겨지는 것을 알아차렸다. 그리고 이 사령관은 안나가 웃음을 참는 것을 듣고 안나에게 돌아섰다.

``너도 이 자리에 있으니까 이 난장판을 치우도록.''

그러고는 돌아서서 점잔 빼며 걸어나갔다.

``이건 좀 심한데요.''

이 금발머리가 말했다. 안나는 공물이 든 상자를 바라보았다. 이런 일을 하자니 어색했다. 안나가 들어본 것 중 가장 무거운 것은 아마도 식료품 저장실에 있는 초콜릿을 꺼내기 위한 의자일 것이다.

``어쨌든.''

안나는 중얼거렸다. 안나는 한쪽 끝에 웅크리고 앉아 상자를 잡았다. 그러고는 용감한 영웅처럼 상자를 최대한 잡아당겼다.

``꼭 몸에서 돌 빼내려는 트롤 같은데요.''

이 금발머리가 말했다. 아마 이는 안나가 이날 처음 들어 본 악의 없는 말일 것이다.

안나는 깊게 숨을 들이마셨다.

``그럼 좀 도와주지 그래요, 금발머리?''

``전 크리스토프예요.''

이 금발머리는, 혹은 크리스토프는 황송하게도 마침내 반대쪽으로 가 이들은 상자를 들어 올렸다. 거의 크리스토프가 들고 있었지만, 안나는 최대한 애를 썼다.

천천히 상자를 옮기면서, 안나는 비꼬는 말을 하지 않을 수 없었다.

``그래서, 저 돼지가 사령관이라고요? 정말 슬프네.''

``돼지라고요?''

크리스토프는 매우 크게 웃어 상자를 거의 떨어뜨릴 뻔했다. 안나는 필사적으로 균형을 잡으려 하면서 크리스토프를 쏘아보았다. 크리스토프는 그저 어깨를 으쓱였다.

``네. 뭐, 짖는 개는 물지 않는다 하잖아요. 안데르센 사령관은 그 짖는 개나 마찬가지고요. 사실, 짖기보단 코웃음을 더 치지만요. 아마도요.''

크리스토프는 살짝 미소를 지어 보였다. 휴전을 제안하는 크리스토프만의 방식이었다.

안나는 크리스토프에게 미소를 지어 보였다. 친절한 표정을 보자 불안감이 누그러졌다.

``그쪽은 좀 괜찮네요.''

``아, 어\ldots\,고마워요.''

크리스토프는 목덜미를 긁적였다.

``왕족 사람한테 칭찬을 듣는 건 정말 오랜만이거든요. 그, 안나 공주님. 어\ldots\,전하?''

``이젠 저한텐 안 맞는 것 같아요. 그죠?''

안나는 옮기고 있는 상자를 바라보며 말했다. 그러자 갑판의 물웅덩이에 거의 미끄러질 뻔했다. 크리스토프는 이를 핑계 삼아 농땡이를 부리려 했지만, 안나는 바로 균형을 잡았다.

``에쿠, 미안.''

비틀거리며 몇 번을 더 가다 서다 한 다음, 이들은 마침내 상자를 계단으로 가져간 다음 짐칸으로 옮겼다. 짐칸은 물이 듣고 무언가로 안 좋은 냄새가 나는 어둡고 습한 곳이었다. 안나는 냄새에 코를 찡그렸다. 그러나 한 치의 도움도 되지 않았다.

``쥐예요.''

크리스토프가 말했다. 안나가 움찔하자 실실거리며 웃고 있었다.

``농담하는 거죠?''

``아뇨. 그냥 몇 마리밖에 없어요. 배는 원래 다 그런 법이라고요.''

크리스토프가 옆에 서 있던 나무통에 앉자 안나는 주저하며 옆에 앉았다. 지금까지는 어떤 쥐도 보이지 않았지만, 위에는 분명히 꽤 있을 것이다. 크리스토프는 이상하리만치 소극적이고 조용히 있었다. 한동안은 둘 다 아무 말 없이 있었다. 안나는 매달려있는 석유등을 툭툭 건드렸다. 등은 앞뒤로 흔들리며 어둠 속에 빛을 비추고 있었다.

``요령을 배웠죠.''

크리스토프가 말했다.

``너무 오래 있다는 생각이 들기 전까지 한 삼 분 동안은 여기서 잠깐 쉴 수 있어요.''

크리스토프의 목소리가 어두워져 안나는 크리스토프가 고통스러운 경험을 통해 배운 것을 알 수 있었다.

``오랫동안 해 온 거예요?''

안나는 조용히 물었다. 삐걱거리는 소리가 나지 않게 등을 제자리에 고정하고 있었다.

크리스토프는 어깨를 으쓱이고는 신발로 바닥을 비볐다. 바닥에 낀 때를 문지르고 있었다. 안나는 어렴풋이 이 때가 살아있다고 생각했다. 마치 스스로 움직이는 것처럼 보였다. 크리스토프는 한숨을 쉬고 발을 흔들어 때를 떨어냈다.

``그렇진 않아요. 전 평소엔 성의 마구간에서 일하거든요. 마구간 소년인 거죠.''

``어, 그러면 말에 관해선 잘 알겠네요.''

``꼭\ldots\,그렇진 않죠.''

안나가 얼굴을 찡그리자 크리스토프는 표정을 일그러트리며 미소를 지었다.

``뭔 소린지 곧 알게 될 거예요. 한 몇 년 전부터인가요? 아무도 서던 제도가 어떤진 정말로 모르나 보죠?''

서던 제도는 원래는 어떤 방문객도 다 받아주었지만, 몇 년 전부터 갑자기 항구를 폐쇄하고 아무도 접근할 수 없게 온 섬을 해군으로 둘러쌌다. 이상한 일이었다. 그러나 아렌델의 왕이나 다른 누구도 서던 제도의 고립을 대수롭지 않게 여겼다. 이들은 그 누구도 공격하지 않았고, 서던 제도는 여전히 연락을 유지했다. 안나는 이것이 엘사 여왕과 관련이 있다고 생각했다.

``엘사 여왕.''

안나가 중얼거렸다.

``화나신 것 같네요.''

크리스토프가 말했다.

``화가 나?''

안나는 자리에서 일어나 불쾌한 표정으로 크리스토프를 노려보았다. 지금까지 쌓인 분노가 터져 나왔다.

``아렌델을 속였다고요! 내가 같이 가면 관용을 베풀겠다고 해 놓고, 뭐, 내가 생각해도 말이 안 되긴 한데, 그래 놓고도 이렇게나 가져가 버리는 거냐고요?''

안나는 크리스토프가 앉아 있는 나무통을 발로 찼다.

크리스토프는 나무통을 고정하고 손을 내밀었다.

``진정하세요, 전하. 아, 그리고 이제 슬슬 올라가 봐야겠네요.''

크리스토프가 앞서가자, 안나는 마지못해 따라갔다. 크리스토프는 말을 계속 이어 나갔다.

``솔직히 말할게요. 이보다 더 심해질 수도 있었다고요.''

``어떻게 더 심해져요?''

``제가 보기에 최소한 아렌델의 그 누구도 굶주리진 않겠네요.''

안나는 인상을 썼다. 그러나 크리스토프의 말이 맞을 수도 있다. 아렌델은 이 정도는 없어도 그럭저럭 잘 견뎌낼 수 있을 것이다. 모두가 조금씩 아껴 쓴다면 말이다.

``그리고\ldots\,아주 최소한\ldots\,사슬에 매여서 끌려가는 사람도 없잖아요.''

크리스토프는 계속 말했다. 안나는 거의 계단에서 넘어질 뻔했다. 아직도 노예제도가 있는 것인가. 크리스토프는 승강구를 열어주고 헛기침했다. 알아챌 수 없는 감정이 눈빛에 뒤섞여있었다.

``그러니까, 공주님은 제외겠지만, 그래도요.''

``기분 나아지게 해 줘서 정말 고맙네요. 이젠 그냥 빠져 죽고 싶을 지경이야.''

안나가 말했다.

``천만에요.''

다시 기중기로 돌아가면서, 이들은 다른 짐을 싣는 것을 준비했다. 안나는 생각에 잠겨 시선을 아래로 내린 채 밧줄을 잡아당겼다. 크리스토프가 주의를 시키자 자신은 공중에 손을 휘두르고 있었고, 크리스토프가 모든 작업을 다 하고 있었다는 것을 깨달았다.

``저, 저기 좀 봐요.''

``또 뭐가 중요하다고—''

안나는 크리스토프의 손가락이 가리키는 곳을 보고 바로 조용해졌다. 엘사 여왕이 얼음 배의 뱃머리에 서 있었다. 이 장엄한 배도 엘사 여왕의 아름다움 앞에서는 볼품없었다. 인정하기는 싫었지만, 그만큼 안나는 눈앞에 펼쳐진 완벽한 모습에 놀라 숨이 멎는 듯했다. 폭풍이나 눈보라에 둘러싸이지 않아, 안나가 자신이 얼마나 위험했는지를 일깨워주는 힘이 드러나 있지 않아, 엘사 여왕은 그저 불행해 보였다. 위엄 있고 차분하고 우아한 모습이었지만, 불행해 보였다.

안나의 시선이 자신에게 향한 것을 알아챈 듯, 엘사 여왕은 옆으로 돌아서서 바로 안나를 바라보았다. 안나는 바로 눈을 내리깔았다. 그러나 용기를 모아 다시 올려다보자, 안나는 엘사 여왕이 `아마도' 살짝 미소를 짓는 것을 본 듯했다. 입꼬리가 살짝 올라가 있었다. 안나는 짐칸의 냄새 때문에 헛것을 본 것이 아닌가 하고 다시 보려 했지만, 엘사 여왕은 배 안으로 들어가버렸다.

``잠깐만, 저거 얼음으로 돼 있잖아. 안에 방은 있으려나?''

크리스토프는 말을 하려 입을 열었지만, 입을 벌린 채 가만히 있었다. 크리스토프는 미간을 찡그렸다.

``괜찮은 질문이네요\ldots\,그런데 그건 나중에 생각해보죠. 일단은 다시 일하자고요.''

크리스토프는 배의 반대편 끝을 바라보고 있었다. 안나는 크리스토프가 보고 있는 곳을 바라보았다. 안나는 공포에 질렸다. 감독관이 일꾼 두어 명을 채찍질하고 있었다. 피가 나도록 등에 긴 홈을 파고 있었다. 이 일꾼들이 속도를 올려도 채찍질은 멈추지 않았다. 안나는 이들이 크리스토프가 말한 노예라는 것을 깨달았다. 전쟁 포로들이 갑판에서 강제 노역을 당하고 있었다.

그리고 안나는 바로 불쾌한 표정을 짓고 있는 크리스토프를 바라보았다. 크리스토프가 이들의 운명을 어떻게 잘, 어떻게 직접 알고 있는지를 깨달았다. 크리스토프는 서던 제도 출신이 아니었다. 안나는 동정심이 생겼다. 그러나 크리스토프에게 해줄 말이 떠오르지 않았다. 안나는 그저 시선을 돌릴 수밖에 없었다.

안나는 자기 자신도 제대로 신경을 쓸 수 없었다.

\textbreak

본인의 관점에서 안나는 버릇없이 자란 아이가 아니었다. 그렇다 하더라도 안나는 여전히 공주였고, 육체노동에는 전혀 익숙하지 않았다. 크리스토프가 도와줄 때는 괜찮았다. 그러나 안나가 부르길 이 돼지—안데르센—는 더 고약한 사람들 사이에 던져 놓아 안나를 더 괴롭히기로 했다. 아무도 안나가 많은 일을 할 수 있으리라고 기대하지 않았지만, 확실히 안나가 애를 쓰는 것을 보고 즐거워하고 있었다.

크리스토프와 같이 있는 동안 끌어모은 긍정적인 생각은 오래가지 않았다. 안나는 노새처럼 짐을 날랐고, 손바닥이 빨개지고 상처가 날 때까지 거친 밧줄을 잡아당겼다. 안나는 단 한 번도 불평하지 않았다.

겨우 몇 시간밖에 되지 않았지만, 안나에게는 영원과 같았다. 안나를 위로해주는 유일한 것은 아렌델의 민중들이 자신을 보기에는 이른 시간이라는 사실이었다. 안나의 자존심 때문은 아니었다. 안나는 자신들의 공주가 이런 식으로 강제 노역을 당하는 것을 보면 어떤 생각을 하게 될지를 걱정하고 있었다. 손의 상처에서 피가 흘러나왔고, 안데르센에게 맞은 다리 뒤쪽에 멍이 생기고 있었다. 그리고 탈수증에 시달리고 있다고 생각했지만, 안나는 계속 헤쳐나갔다.

긍정은 안나의 몇 안 되는 힘이었다.

감옥에 갇히기 전까지는.

``이봐요! 절 그렇게—''

승강구가 닫히자 안나의 항의는 부질없는 것이 되었다. 안나는 우리와 같은 곳에 갇혀 어둠 속에 홀로 남겨졌다. 조금 전에 가지고 놀던 등만이 안을 밝혀주었다. 등불은 약했고, 거의 꺼져가고 있었다. 안나는 그래도 이 상태가 더 낫다고 생각했다. 조금이라도 더 밝았으면 바닥에 쌓인 때를 보게 될 테니 말이다. 안나는 때가 자신에게 달라붙는 것을 겨우 참고 있었다. 불결한 기름때가 피부에 스며드는 듯했다.

배가 움직이자 상황은 더 나빠질 뿐이었다. 안나는 멀미를 하지 않으려 애를 썼지만, 이리저리 흔들리는 배에서 그렇게 하려는 의지력은 남아있지 않았다. 신선한 공기 없이 어둠 속에 갇혀서, 안나는 언제라도 토하리라고 생각했다. 그리고 한 가지 생각이 떠올라 마음이 내려앉는 듯했다.

집에서 떨어지기 전에 아렌델을 마지막으로 볼 수도 없었다.

``언제까지 이러고 있어야 해?''

안나가 중얼거렸다.

안나는 누군가가 대답하리라고는 예상하지 못했다.

안나는 높게 끽끽대는 소리에 정신이 퍼뜩 들었다. 감옥 구석에서 눈 한 쌍이 반짝거리는 것이 보였다. 그리고 또 다른, 또 다른 쥐가 보였다. 점점 공황에 빠지는 것을 억누르려 하면서 호흡이 빨라졌다. 그러나 시선을 돌려 자신을 가두고 있는 창살을 보자 폐소공포증에 더욱 허둥댈 뿐이었다.

안나는 쥐들에게서 떨어지려 했다. 그러나 무언가가 발목을 잡아당겨 다리가 걸렸다. 안나는 잊고 있었다. 다리에 족쇄가 채워져 있었다. 이 쩔그럭거리는 소리에 쥐들은 소리를 내기 시작했다. 안나는 창살로 달려들었다.

``꺼내 줘요!''

안나가 소리쳤다. 우리를 최대한 세게 치고 있었다. 그러나 소리는 감옥 안에 거의 퍼지지도 못해 승강구를 지나 누군가 들을 수 있게 올라가지도 못했다.

``꺼내 줘요\ldots\,제발\ldots''

구석으로 돌아가면서 마침내, 마침내 눈물이 떨어졌다. 점점 추워지자 안나는 팔짱을 낀 채 웅크렸다.

`집에 가고 싶어.'

