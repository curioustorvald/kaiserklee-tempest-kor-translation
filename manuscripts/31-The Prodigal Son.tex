

\chapter[31장  방탕아][31장\hspace*{.5em}방탕아]{31장 \ 방탕아}



안나는 문을 두드렸다. 한 번, 세 번, 다시 한 번.

평소대로 에드문드는 들어오라고 말했다. 안나는 새로 단 문을 살며시 열며 열두 번째 왕자의 방으로 들어섰다. 에드문드는 여전히 침대 신세였고, 한쪽 눈에는 반창고가 붙어 얼굴의 사 분의 일이 거즈에 가려져 있었다. 머리카락 절반은 반창고에 눌려 있었고, 나머지는 고슴도치처럼 이리저리 뻗쳐 있었다. 안나를 보자 에드문드는 입꼬리를 올리며 손을 흔들어 보였다. 전과는 비할 수 없이 힘없는 미소와 손짓이었지마는 안나는 에드문드를 보게 되어 반가웠다. 에드문드에게 웃어 보이고 나서 안나는 곁으로 가 자리에 앉았다.

``오늘 아주 좋아 보이시네요.''

에드문드는 킬킬거렸다.

``얼굴이 가려져서 그렇다고 하시려고요?''

``그럼요! 확실히 나아진 점이죠.''

``말도 안 돼.''

에드문드가 중얼거렸다.

``뭐, 지금은 정신이 돌아왔으니, 제 얼굴 얘기 말고 뭐 하고 싶은 말 있어요?''

안나는 마법사 일이 지나자마자 에드문드를 만나 왔다. 에드문드가 방문을 받아도 괜찮겠다고 사울이 판단한 후부터였다. 안나는 에드문드가 얼마나 관계되어 있었는지 전혀 묻지 않았고, 엘사와 안나 사이에도 에드문드를 심문하지 않겠다는 말 없는 약속이 생겼다. 안나는 확신했다—엘사는 심문하기를 원했고, 한편으로는 심문을 반드시 해야 한다는 것을 알았지만, 처음 보러 갔을 때의 에드문드는 겨우겨우 정신이 들어 있었고 두 번째에도 크게 나아지지는 않았다. 지금에야 얼굴에 혈색이 돌아와 있었다.

안나는 에드문드가 안정을 취하는 동안 방안을 둘러보며 시간을 보냈다. 이때서야 안나는 처음 와보는 것을 깨달았다. 방은 생각한 것보다 깔끔했다. 에드문드의 덤벙대는 성격으로 보았을 때 안나는 방이 매우 지저분하리라고 생각했지만, 사실은 정반대였다. 에드문드의 방은 티 하나 없이 깔끔했다. 시녀나 하인의 작품으로는 보이지 않았다. 레일에 사다리까지 달린 삼중 책꽂이를 완벽하게 정렬하기가 쉽지 않았을 것이다. 안나는 책꽂이 안의 온갖 언어로 쓰인 수많은 책을 보고는 깜짝 놀랐다. 영어, 스페인어, 프랑스어, 그리스어, 이탈리아어, 독일어, 루마니아어, 러시아어, 라틴어에 한문, 그리고 알아볼 수 없는 다른 여러 언어로 된 책들이었다. 가장 이상한 점은 언어 교재는 한 권도 없다는 것이었다. 모두 진짜 문학 작품이었다.

``정말 이 언어에 다 유창한 거예요?''

안나가 물었다.

``십삼 개 국어요.''

에드문드가 대답했다. 놀라워하는 안나의 표정을 바라보며 에드문드는 웃음을 터뜨리고는 자신의 머리 옆을 톡톡 치며 말했다.

``어렵진 않았어요. 원래 기억력이 좋기도 하고 좋아하는 일엔 집중력도 강하니까요.''

``아, 그랬죠. 완전기억력이 있었죠.''

``맞아요. 그런데 그런 말 한 적 없는 거 같은데요.''

에드문드가 말했다. 한쪽으로 고개를 기울인 채 뽐내듯 씩 웃고 있었다.

``뭐, 언젠간 기억나겠죠.''

``사실 그게\ldots''

안나는 바로 어깨를 으쓱였다. 성물 이야기는 하지 않는 것이 최선이리라.

``아녜요. 그, 여왕님도 똑같아요.''

``그러실 줄 알았어요.''

책꽂이를 제외한 방 전체를 가리키면서 에드문드는 덧붙였다.

``또 다른 거 안 보이세요? 책꽂이 말고요.''

사실 방 안의 모든 것이 조금씩은 흥미로웠다. 모든 왕자의 성격이 한데 모인 것 같았다. 수가 많아 어지러이 있는 듯 보여도 가지런히 배치되어 있었다. 한쪽 벽에는 그림 세 점이 걸려 있었다. 모두 성경 속 말을 묘사한 그림이었다. 가운데에는 천사 미카엘과 용이 그려진 특히 커다란 그림이 있었다. 스테판의 것이다. 그림 앞에는 그랜드 피아노와 베토벤의 월광 소나타 악보가 있었다. 라파엘의 것이다. 침대 옆 걸이에는 장검과 단검이 있었다. 알렉와 알바르의 것이다. 사울도 한쪽 구석에 있는 찻잔 세트로 나타내어 있었다. 안나는 에드문드에게 포도주에 관한 지식이 있는 것을 알았다. 리드를 위한 것이다.

``사실 이 중에 흥미로운 게 하나도 없어요.''

에드문드가 말했다. 안나가 왜 그러느냐는 눈으로 바라보자 관심이 없는 듯 어깨를 으쓱였다.

``그, 다른 사람을 보고 따라 하는 게 더 쉽잖아요. 그래서 하는 거하고 좋아하는 걸 따라 했어요. 이러면 찾을 수도 있겠다고 생각했죠\ldots\,제 것이 뭔지를요.''

그러나 다른 것을 떠나, 가장 눈에 띄는 것은 구스타프의 것이었다. 다른 것은 밋밋할 정도로 완벽히 놓여 있었지만, 체스판만큼은 창틀 위에 비뚤게 놓여 있었다. 말도 옮겨져 있었다. 체스판을 놓을 탁자 같은 것도 없었다. 제대로 된 앉을 자리도 없었다. 안나는 창틀에 걸터앉아 체스를 두는 구스타프와 에드문드를 떠올릴 수 있었다.

``\ldots그럼 저건요?''

안나가 물었다.

에드문드는 놀란 듯 창문을 바라보았다. 안나는 잠깐은 에드문드가 답을 하지 않으리라고 생각했다. 최소한 에드문드가 작게 입꼬리를 올리기 전까지는.

``그건\ldots\,재밌죠. 사실 체스 두는 건 좋아했어요. 그, 잘 둬본 적은 없지만, 그래도 재밌었어요. 형님께선 너무 집중하고 있다고, 자신을 자유롭게 해야 한다고 말했죠. 그런데\ldots\,저는 그게 안 되는 거 같아요. 형님께서 가르쳐 주시는 건 그냥\ldots\,전혀 이해를 못 했어요. 체스만이 아니라 하시는 일도요.''

``두 분이 정말 가까웠죠.''

안나가 중얼거렸다.

예배실에서의 구스타프를 떠올리자 안나는 이를 의심했다. 그렇지마는 둘의 대화는 마지막에 구스타프가 에드문드에게 한 말과는 전혀 맞지 않는다고 생각했다. 그리고 특히 지금 에드문드의 눈이 그랬다. 예배실에서의 구스타프와 똑같았다. 안나는 아직은 이해하지 못하는 듯했지만, 확실히 무언가 연결고리가 있었다.

``그랬죠, 안 그런가요?''

에드문드는 움직임을 멈추고 손을 무릎에 올려놓았지만, 손은 계속 떨리고 시선은 아래로 떨어졌다. 오른손을 쥐었다 폈다 하고 있었다.

``네, 정말 그랬죠.''

안나는 아무 말도 하지 않았다. 에드문드는 더는 안나에게 말하고 있지 않았다.

``\ldots가끔은 다른 생각이 들어요. 형님의 지시에 신경질을 내고 형님을 원망하기도 했죠. 그래도 결국엔 얼마나 챙겨주시는지를 알게 됐어요.''

에드문드는 굵고 깊게 끙하는 소리를 냈다. 화가 사라지자 손 안에 얼굴을 파묻으며 끝에 가서는 거의 짐승과 같은 소리가 났다. 에드문드는 괴로운 듯 보였지만, 안나는 부상 때문에 그러는 것이 아님을 알고 있었다.

``그냥 이해가 안 돼요. 형님께서 하신 일이 그냥 말이 안 된다고요. 전 형님께서 그럴 줄은 전혀\ldots''

``이젠 다 괜찮을 거예요.''

안나는 나지막이 말했다.

``말하고 싶으면—''

에드문드는 고개를 치켜들었다. 눈빛은 금세 강렬해졌다.

``제 방 얘기하러 오신 거 아니죠, 공주님?''

``죄\ldots\,죄송해요. 캐물으려 한 건—''

``아뇨, 이해해요.''

에드문드는 깊게 숨을 들이마셨다. 눈 깜짝할 새에 태도가 바로 돌아왔다. 솔직히 말해 갑작스러운 변화에 안심이 되기보다는 놀랐다. 조금 전까지의 모습을 보지 않았으면 안나는 보기에 진짜 같은 미소에 속아 넘어갔을 것이다. 진심으로 미소를 짓는 것처럼 입꼬리가 올라가 있었다. 한쪽으로 약간 기울어 있었다. 거기에다 눈도 반달 모양이 되어 있었다. 너무나도 진짜 같았다. 에드문드는 안나가 주저하는 것을 알아챈 모양이다. 에드문드가 이렇게 물었기 때문이다.

``무슨 일 있나요?''

``아뇨.''

안나는 서둘러 대답했다.

``죄\ldots\,죄송해요. 갑자기 뭔 말을 해야 할지 몰라서요.''

``그거죠. 아무래도 그냥 제가 말을 해야겠네요. 하지만 알아채셨겠죠. 전 항상 형님이\ldots\,뭘 하는지 알고 있었어요.''

에드문드는 표정 변화 없이 말을 마쳤다. 중간에 멈추지도, 더듬지도 않았다. 준비된, 기계적인 말이었다.

``아니면 그렇게는 생각했죠. 공주님이 마법사 얘기를 해 주기 전까지는 제대로는 몰랐죠. 그 뒤론 형님께 말하러 갔어요. 그때가 알렉 형님이 제가 사라진 걸 본 때였죠. 구스타프 형님이 절 끌고 들어간 거예요. 그러고 나서 나중엔 성물에 가는 동안 예배실에서 기다리고 있으라고 했어요. 전 형님께 항상 아무도 다치게 하지 말라고 설득하려 했지만\ldots\,죄송하게 됐네요, 공주님.''

십 초 동안 침묵이 이어진 끝에야 안나는 에드문드가 말을 끝낸 것을 알았다. 에드문드가 말을 이을 것으로 생각했지마는 몰아붙이지 않기로 했다. 대화는 끝났고, 어쨌든 안나는 필요한 것을 얻었다.

``시도는 하신 거네요.''

안나가 말했다.

``\ldots네, 정말 그랬죠.''

할 법한 말도 많이 있었다. 안나는 진작 말을 하지 않은 것으로 한소리 했을 수도 있었다. 위로의 말을 해줄 수도 있었다. 하지만 할 법한 다른 말은 밖으로 나오지기 않았다. 결국에는 한동안 에드문드 옆에 앉아 있었다. 에드문드는 계속 손을 쥐었다 폈다 하고 있었다.

그때 에드문드가 갑작스레 정적을 깨고 말했다.

``같이 어디 좀 가실까요?''

에드문드가 침대에서 나와 다리를 옆쪽으로 휘두르자 안나는 서둘러 자리에서 일어났다.

``잠깐만요! 정말로 움직여도 괜찮은 거예요?''

``괜찮을 거예요.''

에드문드는 문을 향해 몇 걸음 걸어간 다음 제자리에서 뒤로 돌아서서 팔을 벌려 보였다.

``자, 완벽하잖아요.''

``대체 어디를 가시려고\ldots''

에드문드의 뜻을 알아채자 안나는 짧게 숨을 들이켰다.

``구스타프 왕자님 방에 가려고 제가 필요한 거네요.''

구스타프가 죽은 뒤로 엘사는 방을 봉인해 놓았지만, 안나는 당연히 아무렇지도 않게 들어갈 수 있었다. 하지만 엘사가 누구도 방에 발을 들여놓기를 원하지 않는 것을 알고 있었다.

``하지만 왕자님—''

``부탁이에요. 꼭 가 봐야 해요.''

에드문드는 침울한 목소리로 말했다. 팔을 늘어뜨려 놓은 채 안나를 뚫어지라 쳐다보고 있었다.

``이런 말 하게 될 줄 몰랐지만, 이거 이상을 바라는 게 아녜요. 제발요, 공주님.''

안나는 거절할 수 없었다.

\textbreak

모든 것이 빠르게 펼쳐졌다.

엘사는 마르쿠스에게 보고해야 하는 것을 알고 있었다. 이미 동면을 방해받지 않게 하는 데에 실패했다. 마법사가 한 번도 아니고 두 번이나 잠입하게 두었다. 그리고 자신도 거울에 반사된 공격으로 거의 온 탑을 무너뜨릴 뻔했다. 전투 중 엘사는 마르쿠스보다는 안나를 보호하는 데에 치중해 마르쿠스를 위험에 내던졌다. 이미 방어가 매우 강력하니 파괴될 틈도 없다고 할 수도 있었지만, 자신이 안나를 우선으로 생각한 것을 알고 있었다. 엘사는 마르쿠스가 자신의 결정을 달가워하지 않으리라고 생각했다. 지금 보고하는 것이 최선이리라.

하지만 너무 지쳐있었다.

마법사 일이 끝나고 한 주가 채 되지 않았다. 엘사는 마법사의 계략을 나머지 왕자들에게 밝히고, 알바르와 토비아스와 리드의 죽음을 마법사에게 연결하고, 구스타프의 소식과 함께 마법사의 정체를 밝혔다. 수가 훨씬 준 왕자 아홉 명이—에드문드는 여전히 침대 신세였다—회의를 위해 엘사 앞에 서 있었다. 이들은 엘사의 예상대로 반응했다.

완전히 아수라장이었다.

항상 구스타프를 지지하던 파비안은 원래 통치자가 되어야 할 첫째 왕자를 없애기 위해 엘사가 모든 일을 꾸며냈다고 외쳤다. 항상 자신의 형제들에게 반대의 목소리를 내는 라파엘은 본인만의 짜증이 날 정도로 차분한 목소리로 대꾸했다. 이 둘의 말싸움과 나머지 왕자들의 목소리가 섞여 있는 속에서 키루스는 혼란을 더했고, 알렉은 재치가 부족하니 거의 도움을 주지 못하고 있었다. 엘사는 점점 머리가 지끈거렸다. 한때는 마법으로 이들을 모두 입을 다물게 했겠지만, 지금은 왜인지 모르게 성가시다는 생각도 들지 않을 정도로 신경을 쓰지 않고 있었다.

다행히도 사울은 더 중요한 것을 들며 끼어들었고, 파비안은 조용해졌다. 구스타프와 혈연관계인 것 때문인지도 모르겠지만, 사울은 구스타프의 빈자리를 대신하고 있는 듯했다. 키루스와 파비안은 이제 지시를 받으려 사울을 바라보고 있었고,  사울은 매우 노련하게 상황을 정리하고 엘사에게 따로 고개를 끄덕여 보였다.

나중에 엘사는 따로 사울에게 말했다.

``가끔은 왜 마르쿠스께서 자네에게 통치권을 주지 않으신 것인지 의아할 때가 있네. 자네는 나보다 왕자들을 훨씬 잘 통제하는데.''

``제가 그러는 데에 신경을 쓰기 때문입니다. 여왕님도 바라신다면 똑같이 하실 수 있습니다.''

사울이 말했다.

이들은 엘사의 집무실에 있었다. 구스타프의 장례를 준비하기 위해 서류를 읽고 있는 동안 사울은 두통을 덜어주려 차를 타고 있었다. 엘사는 자신이 왜 이러고 있는지 전혀 알지 못했다. 에드문드의 감상에 물이 든 것일지도 모르겠다. 어느 정도는 마법사의 처지에 공감해서일지도 모르겠다. 아니면 생각 이상으로 구스타프를 좋게 생각해서일지도 모르겠다. 사실 엘사는 아직도 둘이 같은 사람으로 보이지 않았다. 이 장례를 마법사를 위한 것으로는 보지 않았다. 오직 구스타프를 위한 것이었다.

``자네의 형제들을 죽인 것으로 내게 치를 떨지는 않는가?''

사울은 잠깐 행동을 멈춘 다음 다시 차를 마저 탔다. 끓는 물을 화로에 올려놓은 사기 주전자에 부어 찻잎을 우려내고 있었다.

``이미 답을 아시리라고 생각합니다.''

``구스타프의 죽음으로 성에서 더 호의를 받게 됐으니\ldots''

사울이 아니라고 말하려 돌아서자 엘사는 기쁜 듯한 기색이 전혀 없이 웃음을 터뜨렸다.

``미안하게 됐네.''

``사실이기는 합니다.''

사울은 인정하는 말투로 말했다.

``다른 이들에게 영향력이 더 커졌습니다. 이제는 라파엘도 제 말을 듣습니다. 최소한 이들을 통제하는 데 도움을 드릴 수는 있겠습니다.''

``고맙네.''

엘사는 사울의 제안을 받아들이고 톡 쏘는 맛이 나는 쓴 차를 바로 한 모금 마셨다. 거친 맛에 진정이 되었다. 초콜릿의 단맛보다 웬일인지 더 생생했다. 입 안에 오래 남았다.

``알렉이 아직 문제가 될지도 모르겠습니다.''

엘사가 눈썹을 추켜세우자 사울은 한숨을 내쉬었다.

``능력과 특히 용기는 인정하지만, 동시에 권위를 무시하는 게 있습니다. 벌써 잊으셨을 리가 없겠죠. 여왕님의 지배에 도전한 것이 오래전 일이 아닙니다.''

``알렉이 안나를 잘 따른다. 이것으로 충분히 갈음하고도 남는다.''

엘사가 말했다.

안나에게는 만나는 사람들의 이면을 끄집어낼 수 있는 이상한 능력이 있었다. 알렉이 처음은 아니었다. 가장 극적이지도 않았다. 엘사는 자신을 싫어하는 알렉의 이면에 좋은 사람이 되고자 하는 마음이 있는 것을 알고는 있었다. 그러나 이제는 한스가 배급을 돕고 있는 것이야말로 기적이었다. 한때는 복수심에 빠져 다른 사람은 눈에도 들어오지 않던 한스가 이제는 마을로 가는 길뿐만 아니라 마을 사람들에게 묻는 설문 조사에도 발 벗고 나섰다. 자기 자신의 변화를 생각해 보아도 엘사는 안나의 영향을 전혀 의심하지 않았다.

사울은 얼굴을 찡그렸다.

``하지만 그분이 여왕님의 우선 사항인 것이 확실합니까?''

``내가 아니면 누가 된다는 것인가?''

``무례한 말이라면 죄송합니다만\ldots''

엘사가 계속하라고 손짓하자 사울은 헛기침을 했다.

``두 분 사이의 마찰을 최근에 알아챘습니다. 우선 사항에 변화가 생긴 것은 아니신지요?''

``안나를 의심한 적 없다.''

엘사가 말했다. 찻잔을 옆에 내려놓고 사울의 말에 집중하고 있었다.

``곧 원래대로 돌아올 것이다.''

``어찌 되었든 안나 공주는 우리의 일원이 아닙니다. 아렌델의 공주잖습니까.''

사울은 강조하며 말했다. 지쳐 방어적인 태도가 약해진 것인지 두려움이 삼킬 수 없을 정도로 커진 것인지 엘사는 안나의 고향 말에 움찔했다. 엘사의 두려움을 읽기라도 한 듯 사울이 덧붙였다.

``여왕님의 소유가 아니라고요.''

``이제 됐네.''

엘사는 자리에서 일어나 창문으로 갔다. 자신의 안식처인 하얀 눈밭을 바라보고 있었다. 아렌델도, 서던 제도의 것도 아니었다. 눈밭은 오직 자신만의 것이었다.

안나와는 다른 것이었다. 엘사는 안나를 가지려 하지도, 가지기를 원하지도 않았지만, 둘 사이의 거리는 벌어졌고, 엘사는 마음속 깊은 곳에서부터 안나가 이곳에서 행복해하지 않는 것을 알고 있었다. 엘사는 집무실에서만 시간 대부분을 보냈고, 안나는 에드문드나 한스나 알렉에게 가 있었다. 엘사는 안나와 같이 있는 짧은 시간 모두를 소중히 했다. 식사 때나 밤 때 말이다. 하지만 같이 이야기하자 약속하고도 여태 그러는 일이 없었다. 이 약속은 이들의 뇌리를 떠나지 않았다. 모르고 지나갈 수 없는, 목을 겨누는 칼과 같았다. 자신이 항상 알고 있는 것을 인정할 용기가 엘사에게 없었기 때문이었다.

둘 다 행복하지 않았다.

엘사는 돌아보지도 않고 손짓으로 사울을 물러가게 했다. 창밖의 순수하고 깨끗한 눈밭에 자신의 괴로움을 묻어두며 지금은 만족을 느끼고 있었다.

\textbreak

짧은 거리였다.

안나는 구스타프의 방을 막는 얼음판에 손을 가져다 대었다. 지금까지 그래왔듯이 벽은 유리가 깨어지듯 산산조각이 났다. 지금쯤이면 익숙해질 법도 하지만, 안나는 여전히 처음처럼 놀라고 있었다. 사실상 엘사가 자신을 받아준다는 증거였고, 지금 둘의 사이에도 약간은 안심이 되었다.

``자, 이제 도착\ldots\,했네요.''

에드문드가 자신을 밀치고 가 문간에 바로 서자 안나는 말을 멈추었다. 손으로 문틀을 꽉 쥔 채 방 안을 둘러보고 있었다. 에드문드는 숨을 멈춘 채 전혀 움직임 없이 서 있었다. 마침내 시선이 방바닥으로 떨어졌다. 엘사가 부순 검은색 나무문의 잔해가 있었다. 깊게 숨을 들이마시고 에드문드는 파편을 지나 안으로 들어갔다.

안나도 따라갔다.

전에 마지막으로 왔을 때는 잠깐만 볼 수 있었다. 호사스러운 문 너머는 소박해 보일 정도로 수수했다. 검은색과 흰색으로 된 바닥을 제외하고는 사치스러운 것이 없었다. 다른 왕자의 쓸데없이 큰 침대와 달리 구스타프의 침대에는 보통 평민들이 쓸 법한 불편하고 얇은 요가 올려져 있었다. 침대의 발 앞에 보관함이 있었지만, 수수한 나무로 되어 있었다. 똑같이 수수한 탁자에는 칼리그라피용 선지\footnote{宣紙. 서화에 쓰이는 종이.}가 흩어져 있었다. 하지만 이상하게도 종이학이 같이 올려져 있었다.

하지만 방에는 이보다도 더 놀라운 것이 있었다. 안나는 전에도 알아챘지만, 지금에야 먼 벽에 걸려 있는 초상화를 가까이에서 제대로 볼 수가 있었다. 작은 제단 위에 더 많은 종이학과 함께 놓여 있었다. 종이꽃으로 만든 꽃다발도 같이 올려져 있었다. 안나는 가까이 다가갔다. 그림 속 아름다운 여인의 모습에 놀라고 있었다. 그림에는 무언가 친숙한 것이 있었다. 그림 속 여인은 고운 자태와 기품이 서려 있었다. 옷은 평민이 입을 법한 옷이었다. 이목구비는 반듯하고 긴 백금발은 얼굴 양옆으로 내려와 있었다. 눈빛은 부드러웠고, 반달 모양으로 눈웃음을 짓고 있었다.

``첼리나예요.''

에드문드가 말해주었다. 제단 앞으로 다가오고는 무릎을 꿇고 있었다. 안나는 당황한 채 바라보았다. 에드문드는 이마를 땅에 댄 다음 다시 일어섰다.

``형님의 아내분이셨죠.''

``아.''

이제 이해가 갔다. 안나는 이 여인이 젊어 보이는 것에 어리둥절했지만, 이제는 첼리나가 이십 년 전에 유배된 것이 떠올랐다.

``형님께선 방에 들어올 때마다 항상 절을 하게 시키셨어요.''

에드문드가 말했다. 하지만 미소를 짓고 있었고, 조금도 짜증이 난다는 기색이 없었다.

``뭐, 전 괜찮은 거 같아요. 왠지는 모르겠지만 보는 게 좋아서요.''

``그\ldots''

안나는 초상화를 바라보고는 다시 에드문드를 바라보았다. 인제야 왜 그림 속 여인이 친숙하게 보이는지를 알아챘다.

``볼 때마다 왕자님이 생각나는데요.''

``정말요?''

에드문드는 웃음을 터뜨렸다.

``사실, 네. 정말요.''

머리 색이 다른 것을 제외하고, 에드문드는 많은 것이 닮아 있었다. 흰 피부, 홀쭉한 코, 이마가 약간 넓고 아래로 갈수록 좁아지는 얼굴, 그리고 뾰족한 턱도. 하지만 가장 큰 것은 눈이 닮은 것이었다. 첼리나의 눈 색깔은 잿빛이 섞인 파란색이고 에드문드는 검은색이었지만, 눈매가 정확히 똑같았다. 몰려 있고, 눈꺼풀이 처져 있고, 끝이 뾰족한 타원형이었다. 심지어는 웃을 때의 모양도 똑같았다.

기괴한 일이었다.

하지만 에드문드는 답을 하지 않고 자리를 떴다. 안나가 첼리나의 초상화에 빠져들어 있는 동안 에드문드는 어떻게든 보관함의 자물쇠를 열어 안을 뒤적거리고 있었다. 에드문드는 이미 사용된 칼리그라피 용지 다발을 꺼내어 옆에 놓았다. 눈을 가늘게 뜬 채 보관함 안을 쳐다보고 있었다. 안나도 안을 들여다보았다. 에드문드의 이름이 적힌 커다란 봉투가 있었다. 천천히, 에드문드는 봉투를 꺼냈다. 안나는 봉투를 쳐다보는 에드문드를 당황한 채 바라보았다.

``안 열어보실 거예요?''

안나가 물었다.

``그래야겠죠. 그런데 뭔가 안 보고 싶기도 하네요.''

에드문드는 납 봉인이 보이게 손 위에서 봉투를 뒤집었다.

``봐야 할 거 같나요?''

``왕자님 거잖아요. 그냥 어디 버려둬도 되고 읽어 봐도 되죠. 왕자님한테 달린 거예요. 하지만\ldots\,바란다면 여기 있을게요.''

에드문드는 고개를 끄덕이고는 바로 봉인을 뜯었다.

종이 여러 장이 쏟아졌다.\begin{letter}

에드문드에게

나 자신이 바라는 만큼 널 잘 안다면 지금은 내 죽음 이후에 답을 찾기 위해 여기 왔겠지. 지금도 이해하지 못하겠지. 알 수 없는 것에 분노하며, 내가 답을 주리라 생각하며 방 안에 있겠지. 하지만 그건 넘어가고서, 네가 무엇을 할지 예견하지도, 무엇을 바라지도 않는다는 것을 털어놓으마. 여기서 무엇을 찾아내든 무엇을 얻어갈지 나는 알지 못한다. 그리고 솔직히 그것이 두렵고. 난 희망을 크게 믿은 적이 없지. 항상 나를 배신해 왔으니까.

여기 도착해서 첼리나에게 절을 했는지 묻고 싶구나. 이것이 머릿속에 새겨져 있기를 바란다. 머리가 아니면 마음이나 넋에. 첼리나의 얼굴을 그림으로는 봐도 실제로는 본 적이 없겠지. 이게 내 걱정거리이기도 하고. 그래도 너를 너무 못 믿는 것 같기도 하구나. 한때는 너무 믿던 것처럼. 그래도 네게 답해줘야 할 게 있지. 내가 해야 할 일이고, 내가 소홀히 한 것이기도 하고, 아마도 이게 바로\ldots

말이 새고 있구나.

구스타프라는 남자의 이야기를 해주지. 이 구스타프의 삶이 어떻게 시작되고 끝났는지도.

한때 구스타프는 서던 제도의 황태자였다. 정의는 이 구스타프의 신념이었고 권력은 당연한 권리였지. 그림자의 능력을 지니고 태어난 이 구스타프가 내리는 징벌은 자신이 경멸하던 세상의 티끌을 씻어낼 정도로 강력했다. 하지만 구스타프의 마음은 차가웠지. 구스타프의 삶은 고립되고 결핍돼 있었다. 구스타프의 세상은 흑 아니면 백이었고. 구스타프는 자존심 강하고 거만했다. 그리고 아주 독선적인 탓에 자신의 무정한 정의가 바로 구원의 길이라고 믿었지. 사실은 무모한 판단이었지만. 구스타프는 눈이 멀어 있었다. 구스타프의 힘은 바람에 날리는 티끌과 같을 뿐이었고, 구스타프의 믿음도 마찬가지로 약했지.

그리고 구스타프는 자신을 이해하지 못했다.

자주 우리는 선과 악의 개념을 착각하고 있다. 하지만 도덕이라는 게 흑과 백으로 쉽게 나눌 수 있는 것이면 악은 존재하지도 않았겠지. 그렇게 된다면 악은 쉽게 알아볼 수 있게 될 테고, 빛에 의해 끌려 나와 온 세상이 경멸하게 될 수도 있겠지. 확실히 이런 세상이야말로 이상향이지만, 선악은 이런 게 아니지. 모든 사람은 절대 진리라는 환영을 좇는 법이다. 더 높은 진리를 따라 구원의 길에 들어서게 되도록 말이지. 절대 진리를 향해 발걸음을 서두르는 동안 우리는 자신을 잊고 정체성을 버리게 되지. 이게 바로 이 구스타프가 빠진 함정이다. 그리고 힘을 지닌 이는 그 무엇에도 멈추지 않고 다른 이를 똑같은 무간지옥으로 밀어 넣는다.

확실히 구스타프는 자신이 좋은 일을 하고 있다고 믿었다. 사회는 질서가 어느 정도 강제되어야 한다고 요구했고, 이 일을 도맡게 된 이가 바로 구스타프였지. 하지만 구스타프는 사람을 사람답게 하는 한 가지 진리를 잊고 있었다. 사람은 이기적일 수가 있다. 그리고 이건 잘못된 것이 아니지. 사실 나는 사람은 이기심을 가져야 한다고 생각한다. 우리의 역량이 닿는 데는 자신의 마음을 들여다보고, 소망을 이해하는 것까지이다. 그리고 이것이 우리가 완수할 수 있는 가장 좋은 일이다.

나는 사람이 행복을 최우선으로 갈망한다고 생각했다. 그때 한 여인이 행복은 다른 이들 위에 올라서는 것에서 생겨나는 것이 아님을 가르쳐주었지. 나는 그때부터 황태자 구스타프라는 것을 벗어 던졌다.

그 여인을 처음 만났을 때 나는 덧없는 것들을 좇고 있었지.

그때의 나는 매우 어렸다. 잘해봐야 열넷이었지. 그때는 아직 바지를 짓기 전이었다. 나의 정의를 고집하기 전이었지. 무정해지기도 전이었고. 물론 그때도 벌써 거만했지. 그리고 어리석게도 무적이라고 믿고 있었고. 나는 홀로 소위 반역자를 쫓다 부상을 당했다. 다쳐서 오도 가도 못하게 되었고, 이대로 끝인 줄 알았지. 그때 파란 눈의 소녀가 나를 구해주었다. 나는 이름을 묻지 않았지. 하루도 같이 있지 않고 나는 도망쳤다. 그리고 다시 쓰러졌고. 그 소녀는 다시 나를 찾아왔다.

``계속 널 찾을 거야. 다시 네가 일어설 수 있을 때까지 기다릴 거고. 알았지?''

난 남이 시키는 건 듣지 않지만\ldots\,왠지 이 소녀의 말은 듣게 되었다.

그래서 거기서 나날을 보냈다. 쉬면서 이 소녀가 종이학을 접는 것을 배우고 바라보았다. 나는 뭐하러 이런 하찮은 걸 하느냐고 했지만, 이 소녀는 하찮은 것에 빠져야만 지혜의 때가 온다고 답했지. 마음속의 분노를 삭여줄지도 모른다고도 했고. 그래서 난 쉬면서 하찮은 인간들의 하찮을 삶을 지켜보았다. 이들을 지켜봤지마는, 나는 평안하지 못한데 왜 다른 이들이 행복하게 보이는지는 이해하지 못했지. 소녀는 마음이 세상에서 가장 강력한 것이라 말했고, 마음을 연다면 행복을 가져다준다고 했다. 떠날 때쯤에도 나는 여전히 이해하지 못했다. 그러고도 이름을 묻지 않았고.

그리고 나는 이곳으로 돌아왔다. 돌아오자 나는 칭송을 받았지. 그제야 이해할 것도 같았다. 매우 두려워 나는 더 무정해졌고, 강압적인 왕자가 되었다.

나는 바지를 만들었다. 아주 많은 사람을 거기에 집어넣었지. 죄가 있건 없건. 하느님의 자손들을 깨끗하게 하기 위한 교도소, 내가 만든 것 중 가장 위대한 것이었지만, 동시에 고문이 벌어지는 미궁이기도 했다. 나는 끝도 없이 이를 괴로워했지. 아버지의 광기를 발견한 것도 바로 이곳이었다. 아버지는 내게 모든 희망을 걸었고, 나는 아버지를 사랑하고 존경했다. 하지만 내 교도소를 바꿔놓은 모습은 참을 수 없었지. 착각 속에서 나는 사람들을 보호하고 있다고 생각했다. 당신이 흑마법에 깊게 빠져들어 인간성이 없어져 가는 것을 봤을 때 나는 깨어났다. 전혀 옳은 일이 아니었지.

아버지께서 언제 발견했는지는 모르겠지만, 거울은 이상한 것이었다. 어떤 이름도, 의도도 없었지. 하지만 거울이 담을 수 있는 엄청난 힘에 아버지는 조각만으로도 완전히 빠져들었다. 가끔은 이 지각 없는 물건을 탓하기도 했지만, 곧 아버지의 광기가 당신의 욕심과 두려움에서 비롯한 것임을 깨달았지. 아버지는 죽음을 원치 않으셨지. 모든 것을 정복하고 세상이 끝나는 날까지, 가능하면 세상이 끝나도 지배하기를 원하셨지. 아버지는 정말로 거울을 제어하고 싶어 모든 것을 하셨다.

한 해가 지나서야 나는 그 파란 눈 소녀가 말한 것을 아버지께서 이해하고 있다는 것을 깨달았다. 정말 역설적이었지. 거울을 움직일 수 있는 건 바로 마음이었다. 나를 움직이게 해 주는 것도 바로 마음이었지.

절망에 빠져 있는 동안 나는 나도 모르게 그곳으로 돌아가 소녀를 다시 찾았다. 몇 년이 지나지도 않았지만, 그 소녀는 더욱 현명해졌고 나는 더욱 무지해졌지. 소녀는 다시 나를 환영했다. 내게 종이학을 주었지. 나는 그 학을 받아 들었다. 시간이 지나니 분노가 정말로 가라앉았다. 나는 진리를 이해하기 시작했다. 나 자신의 마음보다 더 큰 진리는 없다는 것을 말이다. 그동안 나는 나 자신의 감정을 무시해 왔다. 내 안에는 잔혹함만이 있다고 여겼지. 나는 전에는 전혀 알지 못한 것으로 마음의 공백을 채우고자 했다.

나는 마음이 바뀐 채 다시 집으로 돌아갔지만, 여전히 총애를 받고 있었고, 나의 요청은 받아들여졌다. 바지는 파괴되었다. 나는 바뀌기 시작했다. 더 부드러워졌고, 다른 이의 악행을 진압하기보다는 선행을 베풀었지. 나는 자주 첼리나에게 가곤 했다. 물론 그냥 넘어가게 되지는 않았지. 아버지는 그만 하라고 명령했지만, 나는 더는 당신의 소중한 아들이 될 수 없었다. 나는 첼리나에게 마음을 고백했다. 우리는 같이 성으로 돌아갔고, 아버지는 분노를 삼켰다. 아직은 나를 대체할 수 없는 노릇이었으니까.

첼리나와 함께 한 삼 년은 좋은 나날이었다. 이 편지에 칼리그라피가 같이 들어 있을 것이다, 에드문드. 첼리나가 글을 잘 알지 못한다는 것을 알고 나는 글을 가르쳐주었다. 정말 심각했지.

\end{letter}

``정말이네요.''

종이를 훑어보며 에드문드가 중얼거렸다. 안나는 휘갈겨 쓴 편지를 집어 들었다.

``형님께서 써보라고 할 때 제가 쓰던 것과 정말 비슷해요.''

``점점 나아지셨어요.''

안나가 말했다. 마지막 장을 들어 흔들고 있었다. 첼리나는 정말로 최소한 구스타프 정도로는 매우 아름답게 써내었다. 어떤 이름이 쓰여 있었다.

``봐요, 오든이라고 써 있어요.''\begin{letter}

그동안 성에서의 내 힘은 곁으로 사람들이 모여들면서 점점 커졌다. 정말로 반역을 꿈꾼 건 아니었지. 변화를 바랐을 뿐이다. 아버지가 당신의 오류를 알아채시고 힘을 추구하는 그 광기 어린 짓을 그만두기를 원했지만, 아버지는 거울에 빠진 채 움직일 줄을 몰랐지. 그래도 아버지는 참으셨다. 내가 당신의 아들이어서, 당신의 마법을 물려받은 유일한 아들이어서 그랬겠지. 그래도 아버지는 항상 안달을 내셨다. 나를 이용할 수 없게 될 수도 있었으니까. 정말로 나를 사랑했는데 내가 업신여겼다고 생각하신 것인지도 모르겠다. 당신이 완전히 달라지셨으니 말이다. 이미 마음 상해 있는 분의 마음에 소금을 뿌린 셈이었지.

그러고는 갑자기 내가 버려질 수도 있게 되었다.

모든 패가 한순간에 무너졌지.

아버지는 새로운 누군가를 찾아내셨다. 거울에 더욱 적합한 사람이었지. 적합하기보다는 완벽한 정도였다, 바로 그 목적을 위해 태어난 것처럼. 이 아이의 운명은 매우 가혹했다. 순전히 우연으로 그 재능과 저주가 주어진 것이니 말이다. 할 수 있으면 똑같은 고통을 겪지 못하게 해 주려 했지만, 나는 이기적이었고, 마지막 타격이 오기 전까지는 아무것도 하지 않았다.

오든 때문이었다. 갑자기 유순한 손자가 나타났지. 아버지나 나와 같은 재능을 지닌 손자가 말이다. 아버지는 나를 버리고 오든을 취하셨고, 나는 내 아들이 나처럼 망가지게 둘 수 없었다. 오든과 다른 이들을 위해 나는 반역을 일으켰다. 그리고 패했지. 단순히 아버지를 끝장냈으면 이 모든 광기가 사라졌을지도 모르지. 하지만 내 결심은 꺾이었다. 아버지는 이를 그대로 갚았고 나는 그냥 방치되었다. 가장 잔인한 벌이지. 첼리나는 바지로 유배되어 내가 만들어낸 무언가에 죽음을 맞았지. 그리고 오든은\ldots

오든은 그대로 남겨졌다.

오든은 다른 이름으로 항상 성 안에 있었지.

\end{letter}

안나는 고개를 치켜들었다.

``하지만 왕자님이—''

``첼리나하고 오든하고 둘 다 바지로 보내진 줄 알았죠! 이건 대체\ldots''

에드문드는 첼리나의 초상화를 향해 고개를 홱 들었다. 눈의 반창고를 뜯어내어 핏빛이 된 눈을 드러내고는 공포에 눈을 동그랗게 떴다. 처음으로 자신과 첼리나의 닮은 점을 알아본 것이다.

``형님이\ldots\,형님은\ldots''\begin{letter}

나는 오든이 내가 자기의 아버지인 것을 모르는 채, 자기가 아버지라고 부르는 사람이 내 아들을 앗아가고 자기 어머니를 죽인 것을 모르는 채 크는 것을 지켜보았다. 나는 지켜봤으면서도 아무런 말도 하지 않았다. 할 수가 없었다. 그나마 남은 약간의 안전을 망칠 수는 없었다. 오든의 삶을 망가뜨릴 수는 없었다. 그러면서도 책임을 지려 하지 않는 겁쟁이였지. 이미 한 번 실패한 주제에 또 실패할 수는 없으니 말이다. 그래서 난 말없이 있었다. 지켜만 보았지. 그리고 또다시, 더 심하게 실패했고.

지금쯤이면 알아챘겠구나.

이제야 네게 해야 할 일을 했구나. 여느 아버지가 하는 것처럼\ldots\,자기 아들에게 이야기를 들려주었지.

너무 늦지 않았기를 바랄 뿐이다. 네가 뭔가를 배워갈 수 있게 이런 이야기를 했다, 내가 네 어머니에게 배운 것처럼 말이다. 네 이름은 오든이었다. 하지만 이제는 에드문드지. 이것도 괜찮은 것이고. 이제 모든 것을 알게 됐을 테지만, 너는 너일 뿐이다. 네가 원하는 누구든 될 수 있지. 나는 너무 늦었다. 너도 그렇게 되지 않기를 바란다.

잘 선택해라, 에드문드.

잘 가거라\ldots

오든.

\end{letter}

에드문드가 편지를 구기자 안나는 화들짝 놀라 뒤로 물러났다. 편지를 든 손을 꽉 쥐면서 목에 핏발이 섰다. 그러다가는 고함을 지르며 편지를 갈기갈기 찢었다. 갑작스레 일어서면서 칼리그라피 용지가 바닥에 흩어졌다. 보관함을 발로 차 버리고는 알아들을 수 없는 말을 외쳤다. 안나는 눈을 감은 채 움찔했지만, 에드문드는 안나는 안중에도 없는 채 방 안을 마구 헤집어 놓았다. 눈에 보이는 모든 것을 박살을 내고 있었다. 안나는 에드문드가 첼리나의 초상화에 다가가는 것을 떨면서 바라보았다. 아니다, 자기 어머니의 초상화이다.

``이건\ldots\,다\ldots\,거짓이야!''

에드문드는 제단을 쾅하고 내려쳤다. 제단에 비치는 자신의 얼굴을 바라보고 있었다. 이제 닮은 점이 보이는지 안나는 궁금해했다. 에드문드는 이에 더 격분하는 듯했다. 원초적인 분노에 악을 쓰며 초상화를 찢고는 있는 힘껏 내팽개쳤다. 벽에 날아가 부딪치는 동시에 에드문드는 반동으로 바닥에 주저앉았다.

와지끈하는 소리만이 정적을 깨었다.

그리고 에드문드의 분노는 마침내 가라앉았다. 주저앉은 채 에드문드는 거칠고 불안정하게 숨을 몰아쉬었다. 얕다가도 깊게 숨을 들이내쉬고 있었다. 그러고는 몸을 웅크린 채 손바닥에 얼굴을 파묻었다.

안나는 조심스럽게 다가가서는 옆에 무릎을 꿇고 앉았다. 손을 뻗자 에드문드는 움찔하며 물러났다. 안나는

``정말 죄송해요.''

하며 물러났다

``\ldots눈치채야 했어요. 눈치채야 했다고요.''

``이런 걸 어떻게 눈치챌 수 있었겠어요.''

``뻔한 거였다고요. 공주님도 닮은 걸 보셨잖아요.''

``닮았다는 건 아무것도 아닌 거예요. 그런 건 아무도 안 떠올린 다고요.''

``공주님은 이해 못 해요!''

에드문드는 고개를 홱 들고 중얼대었다.

``알았어야 했다고요. 제 잘못이라고요! 아직도 모르겠어요? 형님은 죽을 일이 없었다고요. 처음부터 알고 있어야 했다고요. 그러면 최소한 이렇게 되지는—''

``구스타프 왕자님이 벌인 일을 싫어할 순 있어요.''

안나는 부드럽게 에드문드의 어깨에 손을 올려놓았다. 에드문드는 안나를 바라보았다. 기가 죽은 눈은 답을 간절히 바라는 빛이었다.

``말 안 해준 걸 싫어할 수 있어요. 구스타프 왕자님이 말씀하신 대로요. 선택할 수 있어요. 구스타프 왕자님을 좋아하면서도 벌인 일을 싫어할 수 있는 거예요, 왕자님.''

에드문드는 웃으면서 슬픔의 눈물을 쏟고 있었다.

``그럴 수만 있다면 그렇게 할 거예요, 공주님. 형님을 좋아할 거라고요. 말만 해줬으면 그랬을 거예요—''

안나는 에드문드를 끌어안아 주었다. 에드문드는 마침내 울음을 터뜨렸다.

눈물에는 분노와 괴로움이 갈마들어 있었다.

